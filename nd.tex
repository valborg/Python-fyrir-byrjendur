\chapterimage{chapters9.png} % Chapter heading image

\chapter{N-dir}\index{Ndir}\label{k:ndir}
Nú ætlum við að kynnast nýrri týpu, hún heitir \textbf{n-d} (lesist ennd) eða n-und (e. tuple) \footnote{\href{http://stæ.is/os}{Skoðið endilega orðasafn stærðfræðifélagssins}}.
Lykilorð þessarar týpu er \textbf{tuple}.
Nafnið er komið frá hugmyndinni um tvenndir og þrenndir, nema við vitum ekki hversu mörg stök er verið að hópa saman, þau gætu verið af n fjölda svo við köllum týpuna n-d eða n-und (þá frá tvíund og þríund).
Líklega eina orðið í íslensku sem inniheldur ekki sérhljóða.
Hér eftir verður týpan kölluð \emph{nd}, þó margir noti n-und.
 
Hún líkist listum að því leitinu til að margar af sömu aðgerðum sem má gera á lista má gera á ndir.
Hún líkist strengjum því að hún er óbreytanleg.
Það má ekki bæta við, breyta eða taka út stök eftir að ndin er skilgreind.

Ástæður fyrir því að nota ndir í stað lista er sú að það getur verið hagkvæmara, ndir nota ekki eins mikið minni, eða okkur er umhugað um gagnaheilindi, og svo sjáum við í kafla \ref{k:föll} um hvernig megi fá eina nd í stað margra skilagilda.

\section{Skilgreining}\index{Skilgreining n-da}
Við skilgreinum nd með svigum.
Athugið að hingað til höfum við notað sviga til að aðgreina segðir og það er vandmeðfarið að átta sig á því hvenær er sviginn stærðfræðilegur (þ.e. einungis fyrir forritarann til að aðgreina samhengi) og hins vegar skilgreining á gögnum af týpunni nd.
Aðgreiningin verður augljós þegar við áttum okkur á því að til þess að skilgreina nd þá þurfum við, líkt og með lista, að aðgreina stökin innan ndinnar með kommum.
Sjáum í kóðabút \ref{lst:ndir-kynntar} hvernig má skilgreina ndir og hvernig svigar gera það ekki nema við notum kommur.
\todo{vísa í kóðabút úr segðum eða tölum þar sem segðir eru aðgreindar með svigum til útskýringar, mögulega er todo að búa það til}
Þar sjáum við einnig að það eru einungis tvær aðferðir til fyrir týpuna, \texttt{.count()} og \texttt{.index()}.
Hvernig má það vera að týpan líkist listum þegar það eru bara til tvær aðferðir?
Var ekki verið að taka fram að það mætti gera margt það sama?
Jú, aðgerðir og aðferðir er ekki það sama.
Við getum ítrað í gegnum nd, við getum skeytt einni nd aftan við aðra (fáum þá nýja nd en breytum henni ekki), við getum náð í hluta úr ndinni (með hornklofum eins og hlutstrengi eða hluta úr lista).

Skoðum nú í kóðabút \ref{lst:ndir-kynntar} hvernig svigar geta annars vegar verið til að afmarka stærfræðilegan forgang og hins vegar til að skilgreina nýju týpuna okkar.
Tökum sérstaklega eftir notkuninni á innbyggða fallinu \texttt{type()} sem auðveldar okkur að skilja hvenær nd verður til.

\begin{lstlisting}[caption=Ndir skilgreindar, label=lst:ndir-kynntar]
a = (3+4)*2
print("a er af taginu", type(a))

b = (1)
print("b er af taginu", type(b))

c = () # þetta verður tóm nd
d = (1,) # þetta verður nd sem inniheldur eitt stak, athugið kommunotkunina
e = (1, 1, 2, 2, 5) # þetta verður nd sem inniheldur 5 stök
\end{lstlisting}
\lstset{style=uttak}
\begin{lstlisting}
a er af taginu <class 'int'>
b er af taginu <class 'int'>
c er <class 'tuple'> d er <class 'tuple'> og e er <class 'tuple'>
\end{lstlisting}
\lstset{style=venjulegt}

Nú gerum við ráð fyrir að eiga ennþá ndirnar \texttt{c,d} og \texttt{e} úr kóðabút \ref{lst:ndir-kynntar}, skoðum þá aðferðirnar tvær sem eru innbyggðar í kóðabút \ref{lst:ndir-adfr}, ásamt því hvað gerist þegar við skeytum einni nd aftan við aðra, hvernig ítrun með for lykkju er lík því sem við þekkjum með lista og að lokum hvernig megi sækja hlut-nd.

\begin{lstlisting}[caption=Ndir aðgerðir og aðferðir , label=lst:ndir-adfr]
print(d.index(1))
print(e.count(1)) 
print(e + d)
print()

for tala in e:
	print(tala)

# Athugum að við megum ekki breyta nd, svo eftirfarandi kóði veldur villu
# c[4] = 3

print(e[1:3])
\end{lstlisting}
\lstset{style=uttak}
\begin{lstlisting}
0
2
(1, 1, 2, 2, 5, 1)

1
1
2
2
5
(1, 2)
\end{lstlisting}
\lstset{style=venjulegt}



\section{Notkun}\index{Notkun nda}
Þar sem ndir eru óbreytanlegar er gagnlegt að nota þær til að halda utan um ástand sem við viljum ekki að sé hróflað við.
Segjum að það séu ákveðin tengsl á milli tveggja gilda og við viljum halda heilindum þeirra þá væri gott að nota nd.
Við getum líka notaðað þær til að spara minni þegar við þurfum litla lista sem þarf bara að nota tímabundið og breytast ekki.
Einnig geta þær nýst til að halda utan um breytur sem á svo að nota hverja í sínu lagi seinna.
Tökum eftir að vissulega megi útfæra fyrri þrjú, af þessum fjórum atriðum nefndum, með listum.

Skoðum kóðabút \ref{lst:nd-notkun} þar sem við sjáum dæmi um nd sem við viljum að haldist óbreytt þó hún sé skoðuð en við viljum getað úthlutað hverju staki í einhverja breytu (e. unpack) til að nota og skoða án þess að það hafi áhrif á ndina.
Við viljum ekki keyra aðferðir á borð við \texttt{.sort()} á ndina því að þá breytist hún, við viljum heldur ekki geta haft áhrif á einstaka sætisvísi (skoðið hvaða villu fæst við þá aðgerð með því að keyra línu 10 í kóðabút \ref{lst:ndir-adfr}), það sem við viljum er létt gagnagrind sem passar upp á gögnin.


\begin{lstlisting}[caption=Ndir notaðar fyrir það sem þær eru gagnlegar, label=lst:nd-notkun]
notanda_upplysingar = ("valborg", "rosalega gott lykilorð", "netfang@internet.is")

notandanafn = notanda_upplysingar[0]
lykilord = notenda_upplysingar[1]
netfang = notenda_upplysingar[2]

notandanafn, lykilord, netfang = notenda_upplysingar

print(notandanafn)
notandanafn = notandanafn.upper()
print(notenda_upplysingar)
\end{lstlisting}
\lstset{style=uttak}
\begin{lstlisting}
valborg
('valborg', 'rosalega gott lykilorð', 'netfang@internet.is')
\end{lstlisting}
\lstset{style=venjulegt}

Línur 3-5 og lína 7 eru jafngildar í kóðabút \ref{lst:nd-notkun}.
Þessi ,,afpökkun'' er læsileg og þægileg leið til að vinna með nd, við sjáum það svo betur þegar við skoðum skilagildi í kafla \ref{uk:skilagildi} hversu mikilvægt er að kunna á þetta.
Takið einnig eftir í úttakinu og þrátt fyrir að breytan \texttt{notandanafn} hafi verið uppfærð þá hafði það engin áhrif á ndina.
Reynið nú að uppfæra það gildi í ndinni, reynið að setja í staðinn fyrir fremsta stakið þessa nýju breytu og sjáið hvaða villu þið fáið.

Að sjálfsögðu er markmiðið okkar ekki enn sem komið er orðið að því að skrifa kóða í sem fæstum línum mögulegum, en það sem við viljum þó geta gert er að gera kóðann okkar eins læsilegan og mögulegt er með því að nota þær aðgerðir sem Python býður upp á.
Jafnvel þó að eini ávinningurinn er að við sjálf skiljum kóðann ennþá þegar við skoðum hann seinna.

%-------------------------------
\newpage
\section{Æfingar}
\begin{exercise}\label{nd1}
Búið til nd sem inniheldur 3 stök og setjið svo aftasta stakið í breytu.
\end{exercise}
\setboolean{firstanswerofthechapter}{true}
\begin{Answer}[ref={nd1}]
Hér vitum við að það eru þrjú stök í ndinni, svo að aftasta stakið hefur sætisnúmerið 2.
\begin{lstlisting}
nd = (1,2,3)
tala = nd[2]\end{lstlisting}
\end{Answer}
\setboolean{firstanswerofthechapter}{false}

\begin{exercise}\label{nd2}
Búið til nd sem inniheldur eingöngu tölur, ítrið í gegnum ndina og prentið út þær tölur sem eru stærri en 100.
\end{exercise}
\begin{Answer}[ref={nd2}]
Rifjum upp að til að ítra í gegnum hlut af gefinni stærð er best að nota for lykkju.
\begin{lstlisting}
talna_nd = (1,34,432,324,999,1,2,3,1,3,55,664,10000)
for tala in talna_nd:
	if tala > 100:
		print(tala)\end{lstlisting}
\end{Answer}

\begin{exercise}\label{nd3}
Búið til nd sem inniheldur tvær tölur, úthlutið svo þeim tveimur tölum í tvær breytur með afpökkun.
Geymið svo útkomuna úr því hvort að fremri talan sé stærri en sú seinni og búið til nýja nd þar sem útkomunni er skeytt aftan við upphaflegu ndina.
\end{exercise}
\begin{Answer}[ref={nd3}]
Við erum beðin um að búa til nd eð tveimur stökum, svo erum við beðin um að geyma útkomu sem þýðir að við þurfum breytu.
Sú breyta á að innihalda svarið við hvort að fyrri talan sé stærri en sú seinni svo það er sanngildi.
Að því loknu erum við beðin um að nota samskeytingu en útkoman er ekki nd svo við þurfum að setja hana í nd til að geta beytt samskeytingu.
	\begin{lstlisting}
nd = (1,2)
tala1, tala2 = nd
svar = tala1 > tala2
ny_nd = nd + (svar,)\end{lstlisting}
\end{Answer}

\begin{exercise}\label{nd4}
Búið til tóma nd.
Búið til lykkju sem keyrir fjórum sinnum og í hvert sinn spyr hún notandann um uppáhaldslitinn sinn.
Í hvert sinn sem notandinn er búinn að svara skal endurskilgreina ndina sem það sem hún var áður að viðskeyttu nýja svarinu.
Þegar lykkjan hefur lokið keyrslu sinni skulið þið prenta út ndina.
\end{exercise}
\begin{Answer}[ref={nd4}]
Rifjum upp  \texttt{input()} fallið úr kafla \ref{k:segðir}, einnig \texttt{range()} fallið úr kafla \ref{k:lykkjur}.
\begin{lstlisting}
nd = ()
for i in range(4):
	svar = input("hver er uppáhalds liturinn þinn?")
	nd = nd + (svar,)
print(nd)\end{lstlisting}
\end{Answer}