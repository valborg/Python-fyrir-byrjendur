\chapterimage{chapter_head_2.pdf} % Chapter heading image

\chapter{Lykkjur}\index{Lykkjur}\label{k:lykkjur}

Í Python eru til tvær tegundir af lykkjum, þær heita \textbf{for} og \textbf{while}.
Við ætlum að kynnast því til hvers þær eru ætlaðar og hvers þær eru megnugar, í hvaða tilfellum á að nota hvora fyrir sig og lykilorð sem gera notkun þeirra öflugri.
Byrjum á því að skoða til hvers ,,að lykkja'' og hvað það eiginlega þýðir.
Það að nota lykkju þýðir að skrifa forritsbút sem keyrir endurtekið.

Nöfnin á for og while verða ekki þýdd sérstaklega í þessari bók, en við segjum t.a.m. ,,að gera eitthvað á meðan'' eða ,,gera eitthvað fyrir hvert stak''.

Tökum dæmi úr daglegu lífi; ef við viljum framkvæma einhverja aðgerð eins og að vaska upp búum við til reglu eins og að setja fyrst upp uppþvottahanska, láta vatnið renna og stafla öllu sem er óhreint við hliðina á vaskinum. 
Svo viljum við endurtaka aðgerðina að þrífa hvern hlut sem er öðru megin við vaskinn, þar til þeir eru allir komnir hreinir hinu megin.
Endurtekningin þarna er að taka upp hvern óhreinan hlut og þrífa hann.
Þá gætum við sagt að fyrir hvern hlut sem er hægra megin, viljum við þrífa hann og setja svo vinstra megin (fer eftir því hvernig vaskurinn snýr) og hætta þegar engir hlutir eru eftir hægra megin.
Þetta ferli að framkvæma sömu aðgerð á stök í mengi er einmitt það sem for lykkja getur gert.

Tökum annað dæmi úr daglegu lífi; ef við ætlum að bíða eftir einhverjum og framkvæma svo einhverja aðgerð þegar viðkomandi kemur þá myndum við væntanlega bíða þangað til að viðkomandi kemur.
Svo á meðan viðkomandi er ekki enn kominn þá höldum við áfram að bíða.
En þar sem við erum ekki tölvur þá myndum við ekki bíða endalaust, við myndum gefast upp.
Þetta ferli að halda áfram að framkvæma einhverja aðgerð þangað til að eitthvað skilyrði á ekki við er það sem while lykkja getur gert.

Til þess að keyra kóða endurtekið án þess að afrita og líma eða handvirkt keyra hann oft, þá notum við lykkjur.
Lykkjur eru kóðabútar sem keyrist endurtekið eftir ákveðnum reglum, þær lykkjur sem eru til í Python eru for lykkjur og while lykkur.
For lykkjur keyra fyrir hvert stak í ítranlegum hlut eða hverja tölu á bili (keyra ákveðið oft, í mesta lagi).
While lykkjur keyra á meðan skilyrðið fyrir keyrslu þeirra er satt (geta keyrt að ,,eilífu'').

\comment{
	
	%%%%%%%%%%%%%%%% comment byrjar
	
\section{Gagnleg lykilorð}\index{lykilord}\label{uk:lykkjulykilorð}
Áður en lengra er haldið í hvernig á að beita lykkjum er ágætt að nefna nokkur grunn lykilorð sem hjálpa okkur gríðarlega.
Þau eru \textbf{pass}, \textbf{continue}, \textbf{break} og \textbf{in}.

Það sem þessi lykilorð gera er má sjá í kóðabút \ref{lst:lykkjur-for-lykilorð}, sjá einnig kóðabúta \ref{lst:lykkjur-for} og \ref{lst:lykkjur-while}:
\begin{itemize}
	\item pass er lykilorð sem gerir ekkert, tölvan heldur áfram keyrslu sinni eins og ekkert hafi verið gert, nema að þarna er kóði sem er rétt inndreginn og gerir það að verkum að tölvan kvartar ekki yfir því að hafa búist við einhverju inndregnu en fengið ekkert.
	Þetta notum við þegar við erum ekki viss hvað á að vera í lykkjunni og við setjum þetta orð inn svo að við getum haldið áfram með annað sem átti að forrita.
	pass er gagnlegt sem staðhaldari (e. placeholder) þegar við erum ekki viss hvernig á að halda áfram en verðum að setja eitthvað því að annars fengjum við málskipunar villu (e. syntax error).
	Þetta lykilorð má nota annarsstaðar en í lykkjum og er einnig gagnlegt sem staðhaldari í föllum.
	\item continue er lykilorð sem lætur vélina stoppa þar sem hún er í lykkjunni, hunsa allt sem kemur á eftir því og fara efst í lykkjuna.
	Continue er gagnlegt þegar kemur að því að það er bara ákveðin virkni sem á að framkvæma undir vissum aðstæðum og við viljum ekki að vélin geri allar aðgerðir sem koma fram í lykkjunni okkar.
	Þetta lykilorð má einungis nota inni í lykkjum.
	\item break hættir keyrslu lykkjunnar, ólíkt continue þá förum við alfarið út úr lykkjunni þegar kallað er í þetta lykilorð og keyrir vélin næst kóða sem er ekki inndreginn undir lykkjunni.
	Þetta lykilorð má einungis nota inni í lykkjum.
\end{itemize}

Án þess að fara meira út í hvernig kóðinn fyrir þessi lykilorð virka þá er þess virði að nefna að þau eru ekki nauðsynleg í hverri lykkju sem við forritum hér eftir, þau eru gagnleg þegar þau eiga við og við þurfum að átta okkur á hvernær svo er.

Skoðum nú lykilorðið in áður en lengra er haldið til þess að öðlast dýpri skilning á því hvernig for lykkjan virkar.



Þegar þetta orð er notað í for lykkjum er þó ekki verið að setja fram segð heldur er verið að úthluta einhverri hlaupandi breytu tilteknu gildi úr ítranlegum hlut.
Það að hlutur sé ítranlegur þýðir að við getum horft á hann stak fyrir stak, skoðað eitt gildi úr honum í einu.
Eins og strengur hefur vísa þá getum við horft á hvert tákn fyrir sig með því að rúlla í gegnum vísana frá 0 og út í enda (eða einhverri annarri röð).
Listar eru einnig ítranlegir þar sem stökin í listum hafa vísa og því má horfa á hvert stak fyrir sig í heild sinni, hvort sem það er annar listi eða ein stök tala.
Heiltölur, fleytitölur og sanngildi eru ekki ítranleg og því ekki hægt að rúlla í gegnum þau með for lykkju.
Hægt er að komast framhjá því með því að kasta þeim í strengi.



%% búið að nota þetta !

\section{Lykkjur}\index{Lykkjur}
Til þess að keyra kóða endurtekið án þess að afrita og líma eða handvirkt keyra hann oft, þá notum við lykkjur.
Lykkjur eru kóðabútar sem keyrist endurtekið eftir ákveðnum reglum, þær lykkjur sem eru til í Python eru for lykkjur og while lykkur.
For lykkjur keyra fyrir hvert stak í ítranlegum hlut eða hverja tölu á bili (keyra ákveðið oft, í mesta lagi).
While lykkjur keyra á meðan skilyrðið fyrir keyrslu þeirra er satt (geta keyrt að ,,eilífu'').

%%%%%% comment endar
}

\section{For}\index{For lykkjur}
For lykkkjur nota lykilorðið \textbf{for} ásamt lykilorðinu \textbf{in}.
Það sem in gerir þegar það er notað eitt og sér er að spyrja hvort að \textit{eitthvað} sé ,,í'' \textit{einhverju öðru} (sjá kóðabút \ref{lst:lykkjur-in}) en í sem hluti af for lykkju þá er það in sem úthlutar lykkjunni næsta staki úr menginu til að skoða.
Þannig að þetta býr til segð sem skilar sanngildi eða einu tilteknu tákni eða staki úr hlut.
Nú skulum við líta á kóðabút \ref{lst:lykkjur-for} til að átta okkur á því hvernig lykkjan er notuð, hvernig við beitum inndrætti til að skilgreina stef lykkjunnar (það sem tilheyrir henni) og hvernig skilyrðissetningar bætast við þetta.

\begin{lstlisting}[caption=Lykilorðið in, label=lst:lykkjur-in]
print('er táknið a í strengum Valborg?')
print("a" in "Valborg")

print('er táknið x í Valborg?')
print("x" in "Valborg")
\end{lstlisting}
\lstset{style=uttak}
\begin{lstlisting}
er táknið a í strengum Valborg?
True
er táknið x í Valborg?
False
\end{lstlisting}
\lstset{style=venjulegt}

Eins og sést í kóðabút \ref{lst:lykkjur-in} þá virkar in nokkuð svipað því sem orðið í gerir í setningu.
Ekki gleyma þessu lykilorði við lykkjugerðina.
Sjá ítarefni í lok kaflans um önnur lykilorð sem gagnast við forritun á lykkjum.
Í næstu kóðabútum verður sýnd grunnvirkni for-lykkjunnar.

\begin{lstlisting}[caption=For lykkjur kynntar, label=lst:lykkjur-for-kynning]
# við byrjum á að skilgreina streng
strengur = "Valborg"
print(strengur, "til viðmiðunar")

for stafur in strengur:
	print(stafur)
	
print()
print('lykkjan er búin')
\end{lstlisting}
\lstset{style=uttak}
\begin{lstlisting}
Valborg til viðmiðunar
V
a
l
b
o
r
g

lykkjan er búin
\end{lstlisting}

Í kóðabút \ref{lst:lykkjur-for-kynning} hvernig rúllað er í gegnum strenginn \texttt{Valborg} með breytunni \texttt{stafur}.
Sú breyta er búin til í línu 6 þegar lykkjan er búin til, það þurfti ekki að skilgreina hana áður.
Það er vegna þess að breytan er skilgreind inni í lykkjunni fyrir okkur, hún er áfram aðgengileg en er ósköp gagnslaus eftir keyrsluna svo okkur er alveg sama um hana, hún er svokölluð tímabundin (e. temporary) breyta sem hættir að skipta máli eftir notkun innan lykkjunnar.

Allt það sem tilheyrir svo lykkjunni eða á að gerast í hverri keyrslu hennar er inndregið undir henni.
Línur 9 og 10 eru ekki hluti af stefi lykkjunnar og keyrast því ekki nema einu sinni, eins og lína 3 prentast bara einu sinni.

Það sem kemur á úttakið úr línu 7, breytan \texttt{stafur}, er hvert tákn fyrir sig í þeirri röð sem það kemur fyrir í strengnum sem verið er að ítra í gegnum.
Lesið yfir þennan kóðabút og gerið tilraunir á eigin spýtur til að átta ykkur á því hvað það er sem er að gerast.

Hvað gerist ef inndrátturinn breytist?
Hvað prentast þá út?
Hvað gerist ef eitthvað annað orð er sett í staðinn fyrir \texttt{stafur}?
En \texttt{strengur}?
Haldið áfram að gera tilraunir með þetta þangað til að þið áttið ykkur betur á því hvernig þetta hangir saman.

Sjáum nú hvernig megi flétta skilyrðissetningar inn í þetta.

\lstset{style=venjulegt}
\begin{lstlisting}[caption=For lykkja og skilyrðissetningar, label=lst:lykkjur-for-skil]
for stafur in strengur:
	# við vitum að það er a í Valborg svo þetta mun einhvern tímann gerast
	if(stafur == 'a'):
		print(stafur)
\end{lstlisting}
\lstset{style=uttak}
\begin{lstlisting}
a
\end{lstlisting}

Í kóðabútum \ref{lst:lykkjur-for-kynning} og \ref{lst:lykkjur-for-skil} vorum við að vinna með sama strenginn, en í fyrra skiptið prentaðist hann allur út en í seinna skiptið fengum við bara eitt stakt tákn út.
Munurinn er sá að í kóðabút \ref{lst:lykkjur-for-skil} þá þegar við vorum komin með táknið í hendurnar vildum við gera eitthvað við það svo við spurðum er þetta tákn jafngilt \texttt{a}, einungis í því tilfelli vildum við prenta eitthvað út.
Við ákváðum að prenta út táknið sem við vorum að skoða sem er geymt í breytunni \texttt{stafur} en hefðu hæglega getað gert eitthvað annað.

Við sjáum einnig að þegar við settum inn skilyrðissetningu þá bættist við annar inndráttur, það er vegna þess að inndráttarnotkunin breytist ekki sama hvar við erum að nota kóða sem krefst inndráttar heldur dregst kóðinn bara lengst til hægri eftir því sem við förum innar. 
Því getur verið ágætt að takmarka hreiðrun til þess að kóðinn sé sem læsilegastur.

Gerið nú tilraunir til að átta ykkur betur á því hvernig megi skoða ítranlegan hlut en framkvæma einungis aðgerð ef eitthvað skilyrði á við.
Hvað gerist ef við breytum skilyrðissetningunni?
En ef við breytum því sem er undir henni?
Hvað gerist ef við bætum við \textit{annars} klausu?
Má setja eitthvað inn í stef lykkjunnar á eftir skilyrðissetningunni?

Næst skoðum við annan ítranlegan hlut með for lykkju, það er listi.

\lstset{style=venjulegt}	
	
\begin{lstlisting}[caption=For lykkja með lista, label=lst:lykkjur-for-listi]
listinn_minn = [0, "strengur", [0, 1, 2]]

for x in listinn_minn:
	# nú hleypur x í gegnum stökin í breytunni listinn_minn
	print(x)

\end{lstlisting}
\lstset{style=uttak}
\begin{lstlisting}	
0
strengur
[0, 1, 2]
\end{lstlisting}



\lstset{style=venjulegt}

Tökum eftir í kóðabút \ref{lst:lykkjur-for-listi} þá fær x það gildi sem er næst í röðinni í listanum, og það skiptir ekki máli af hvaða týpu gögnin eru.
Listinn í línu 1 inniheldur gögn af þremur týpum, breytan \texttt{x} kippir sér ekkert upp við það og birtir gögnin í þeirri röð sem þau bárust.

Prófið ykkur nú aðeins áfram og reynið í staðinn fyrir \texttt{listinn\_minn} að setja einhvern annan lista, sem er ekki geymdur í breytu.
Prófið nú að setja inn skilyrðissetningu þarna og nota \texttt{type()} með skilyrðissetningu til að prenta einungis þau x sem eru strengir.

For lykkjur eru því helst gagnlegar þegar við vitum hversu oft við viljum að lykkjan keyri, því til stuðnings ætlum við að skoða innbyggða fallið \textbf{\texttt{range()}} sem gefur okkur hlut af tölum á ákveðnu bili.
Fallið \texttt{range()} tekur við sömu viðföngum eins og hornklofarnir\footnote{það eru í raun ekki viðföng, í "strengur"[1:5:2] eru 1, 5, og 2 ekki viðvöng beinlínis heldur vísar.} þegar við sóttum nokkur tákn upp úr streng eða lista, þau eru þó aðgreind með kommum\footnote{Við höfum áður séð viðföng notuð í falli eins og print() fallið, þar sem við getum prentað margt út svo lengi sem við setjum kommur á milli þess.}.

Viðföngin í \texttt{range(a,b,c)} fallið eru heilartölur og þeim er raðað svona:

\begin{enumerate}
	\item \textbf{a}: talan sem á að byrja að nota (hér má sleppa því að setja þetta inn því að sjálfgefið gildi er 0)
	\item \textbf{b}: talan sem á að hætta fyrir framan (þetta verður að setja inn, því að þetta er aðalatriðið)
	\item \textbf{c}: tala sem segir til um skrefastærðina (sjálfgefið gildi er 1 og þessu má sleppa), ef skrefastærð er tekin með þarf að velja upphafsstað (annars myndast tvíræðni)
\end{enumerate}

Ef við viljum leysa verkefni sem felst í því að finna sléttar tölur frá 0 og upp að 1000 hvernig myndum við fara að því?
En ef við viljum vera viss um að eitthvað gerist ákveðið oft?
Skoðum kóðabút \ref{lst:lykkjur-range-kynnt} þar sem fyrra verkefnið er leyst á tvo mismunandi vegu.

\begin{lstlisting}[caption=range() fallið, label=lst:lykkjur-range-kynnt]
for tala in range(5):
	if(tala%2 == 0):
		print(tala)		

print()

for tala in range(0,5,2):
	print(tala)

print()

for x in range(3):
	print('bílalúgudýraspítali', x)
\end{lstlisting}		
		
\lstset{style=uttak}
\begin{lstlisting}
0
2
4

0
2
4

bílalúgudýraspítali 0
bílalúgudýraspítali 1
bílalúgudýraspítali 2
\end{lstlisting}

Aðalatriðið sem þarf að hafa í huga þarna í kóðabút \ref{lst:lykkjur-range-kynnt} er að for lykkjan hleypur í gegnum lista af tölum svo að við vitum alltaf hvar við erum stödd og við vitum hvað við keyrum lykkjuna oft.
Sjáum í línu 13 þá er x prentað og á úttakinu (línur 9-11) sést að það er hlaupandi númer sem byrjar í 0 og hættir í 2 sem er talan fyrir framan 3 og þar vildum við hætta.
Nú getið þið breytt þessum lykkjum til að skoða til dæmis sléttar tölur undir 1000 eða tölur deilanlegar með 17 undir 100.

En við þurfum ekki endilega að byrja í núll, sjáum í kóðabút \ref{lst:lykkjur-range-meira} hvernig hægt er að velja afmarkaðra talnabil og svo sjáum við í kóðabút \ref{lst:lykkjur-range-mest} að hægt er að telja afturábak.

Allt er þetta þó spurning um að finna það sem hentar því verkefni sem við erum að reyna að leysa.
Næstu sýnidæmi eru meira til þess fallin að sýna virkni \texttt{range()} fallsins og for lykkja yfirhöfuð án þess þó að vera að leysa einhver flókin vandamál.
Eftir að hafa séð þetta, gætuð þið prófað ykkur áfram og náð þannig tökum á þessu.

Það sem skiptir mestu máli til að ná ákveðinni leikni er að prófa sig áfram, gera tilraunir og þora að mistakast.

\lstset{style=venjulegt}
\begin{lstlisting}[caption=range() fallið, label=lst:lykkjur-range-meira]
for tala in range(10, 20):
	if(tala%3 == 0):
		print(tala)
\end{lstlisting}

\lstset{style=uttak}
\begin{lstlisting}
12
15
18
\end{lstlisting}

Eins og áður kom fram þá þarf að taka fram upphafspunkt ef nota á skrefastærð svo að í línu 1 í kóðabút \ref{lst:lykkjur-range-mest} fer ekki milli mála að byrja á fyrir framan töluna 10 og enda fyrir framan töluna 20.
Svo er athugað hvort að hver og ein tala sem er á því bili sé deilanleg með þremur ef svo er er talan prentuð út.


\lstset{style=venjulegt}
\begin{lstlisting}[caption=range() fallið, label=lst:lykkjur-range-mest]
for tala in range(100, 70, -1):
	if(str(tala)[-1] == '9'):
		print(tala, 'endar á 9')
\end{lstlisting}

\lstset{style=uttak}
\begin{lstlisting}
99 endar á 9
89 endar á 9
79 endar á 9
\end{lstlisting}

Nú vitum við að ef okkur vantar eitthvað ákveðið oft, eða við vitum nákvæmlega hversu oft við viljum að lykkjan keyri þá getum við beitt range() fallinu.
Nú skulum við skoða hvernig lykkju lykilorðin sem við skoðuðum í undirkaflanum \nameref{uk:lykkjulykilorð} (\ref{uk:lykkjulykilorð}) nýtast með for lykkjum í kóðabút \ref{lst:lykkjur-for-lykilorð}.
\lstset{style=venjulegt}
\begin{lstlisting}[caption=Lykilorðin pass\, continue og break notuð með for lykkju, label=lst:lykkjur-for-lykilorð]
# við viljum búa til lykkju sem á að rúlla í gegnum 1000 tölur en við erum ekki viss hvað á að gera innan í lykkjunni
for x in range(1000):
	pass
# hér fáum við ekkert úttak en jafnframt enga villu.

# við viljum búa til streng úr stórum tölum í einhverjum lista, tölum sem eru hærri en 50
storar_tolur = ""
for x in [1, 2, 52, 9, 53, 2]:
	if (x < 50):
		continue
		print(x)
	storar_tolur += str(x)

print(storar_tolur)
# þetta skilar okkur úttakinu
# 5253

# við viljum búa til lykkju sem ítrar í gegnum lista af tölum en ef hún sér töluna 13 þá hættir hún keyrslu
listi_af_tolum = [1,5,7,9,13,15,17,18]
for tala in listi_af_tolum:
	if(tala == 13):
		break
		print(tala)
	else:
		print(tala)

# þetta skilar okkur úttakinu
# 1
# 5 
# 7
# 9
\end{lstlisting}

Tökum eftir að í kóðabút \ref{lst:lykkjur-for-lykilorð} þá er tilfellið sem notað er til að sýna continue kannski ekki mjög nytsamlegt þar sem auðvelt væri að gera bara if (x > 50) : storar\_tolur += str(x) en þar sem við erum að reyna að kynnast notkun lykilorðsins þá er þetta ágætis dæmi.
Það sem gerist er að ef x < 50 er satt þá förum við inn í skilyrðissetninguna og þar sem við förum inn í hana þá rekumst við á orðið continue og um leið og það er lesið þá förum við efst í lykkjuna og úthlutum x nýju staki, print(x) gerist ekki og því er eina úttakið okkar það sem við prentum eftir að keyrslu lykkjunnar lýkur.
Við sjáum að lykkjunni er lokið því að print(storar\_tolur) kemur fram í sama inndrætti og for lykilorðið, sem þýðir að kallið print(storar\_tolur) tilheyrir ekki lykkjunni, er ekki undir henni.

Þegar við notum break gerist það sama, um leið og break er notað þá hættir keyrsla lykkjunnar, breytan tala fær ekki nýtt gildi og hún endar með sem talan 13.
Að sama skapi þá prentum við ekki út 13 því að break gerist fyrir ofan print() kallið.
Þetta lykilorð getur reynst ómetanlegt þegar við skoðum while lykkjur.

\section{While}\index{While lykkjur}
While lykkjur nota lykilorðið \textbf{while} og keyra ,,á meðan'' eitthvað skilyrði er enn satt.
Þær eru helst ganglegar þegar við vitum ekki hvað við viljum að lykkjan keyri lengi eða þegar við viljum að hún keyri endalaust nema annað sé tekið fram (t.d. með break).

Skilyrðið fyrir keyrslunnar er metið sem sanngildi, annað hvort með sanngildinu sjálfu eða segð sem skilar sanngildi.
Þá gefst okkur tækifæri á að forrita lausn á vanda eins og ,,ef það er enginn eftir í stofunni á að slökkva ljósið'' og forritið keyrir á meðan ,,einhver er eftir í stofunni''.
Þarna þurfum við ekki að gera annað en að fylgjast með aðstæðum.
While lykkjur eru vandmeðfarnar og harla líklegt að lenda í því að skrifa lykkju sem keyrir endalaust við fyrstu notkun.
Þær eru jafnframt öflugar til að leysa ýmsan vanda sem krefst þess að aðstæður hverju sinni séu skoðaðar.

Skoðum kóðabút \ref{lst:lykkjur-while} til þess að sjá hvernig má auðveldlega lenda í vandræðum við gerð slíkra lykkja og hvernig uppsetning þeirra lítur út.


\begin{lstlisting}[caption=while lykkjur, label=lst:lykkjur-while]
# Grunnuppbyggingin kynnt:
while(True):
	# inndreginn kóði sem tilheyrir lykkjunni - stef lykkjunnar
	pass
	
# þessi lykkja keyrir að eilífu vegna þess að skilyrðið fyrir henni er True og ekkert breytir því í stefi hennar

while(True):
	# nú ætlum við að reyna að komast út
	break
	
# við komumst út úr lykkjunni, hún keyrði einu sinni og hætti strax keyrslu, ekki mjög gagnleg lykkja en hún keyrði allavega ekki að eilífu

while(False):
	print('þetta mun aldrei prentast því að stef lykkjunnar mun aldrei keyrast')

# prófum núna að gera segð sem skilyrði

x = 5
while(x > 1):
	# Við munum keyra stef lykkjunnar að minnsta kosti einu sinni þar sem segðin er sönn í fyrstu keyrslu, 5 er vissulega stærri en 1
	# en nú verðum við að gera eitthvað sem hefur áhrif á x þannig að lykkjan okkar eigi möguleika á að hætta, það er við verðum að gera það að verkum að x > 1 verði einhvern tímann ósatt.
	if(x > 1):
		print("talan er", x, "og hún er enn stærri en 1")
		x -= 1
		
# úttakið verður:
# talan er 5 og hún er enn stærri en 1
# talan er 4 og hún er enn stærri en 1
# talan er 3 og hún er enn stærri en 1
# talan er 2 og hún er enn stærri en 1

# Tökum aðeins sértækara dæmi, sem byggir á sauðakóða en ekki Python kóða sem hægt er að keyra. 
# Til þess að sjá fyrir okkur betur hvað er að gerast.

fjoldi_i_stofu = 5
while(True):
	if(fjoldi_i_stofu == 0):
		slökkva ljós
		break
		
	# við komumst bara hingað ef það er einhver í stofunni svo við ætlum að telja fjölda í stofu aftur
	fjoldi_i_stofu = talning
	
# Stef þessarar lykkju byggir á því að tölvan viti hvað slökkva ljós og talning þýðir, en þar sem ætlunin var ekki að gera kóða sem virkar heldur sýna lesanda virknina fær lykkjan að vera í friði svona.


\end{lstlisting}

Annað sem má gera við while lykkjur er að koma fram við þær sem skilyrðissetningu sem má fá else klausu aftan við sig sem keyrist þegar skilyrði lykkjunnar verður ósatt.

\begin{lstlisting}[caption=Ap nota else með while, label=lst:lykkjur-while-else]
# Við viljum áfram skoða það hvenær á að slökkva ljósin og við getum gert það svona:
while(fjoldi_i_stofu > 0):
	fjoldi_i_stofu = talning
else:
	slökkva ljósin
	
# nú slökkvast ljósin þegar fjoldi_i_stofu er orðinn 0.

# skoðum dæmi sem hægt er að keyra

x = 5
while(x > 1):
	print("talan er", x, "sem er stærra en 1")
	x -= 1 
else:
	print("nú er talan orðin 1 því 1 er ekki stærri en 1 -->", x)
	
# þetta skilar okkur úttakinu
# talan er 5 sem er stærra en 1
# talan er 4 sem er stærra en 1
# talan er 3 sem er stærra en 1
# talan er 2 sem er stærra en 1
# nú er talan orðin 1 því 1 er ekki stærri en 1 --> 1
\end{lstlisting}
