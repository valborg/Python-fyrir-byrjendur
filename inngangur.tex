\chapterimage{chapter_head_2.pdf} % Chapter heading image

\chapter{Inngangur}

\section{Tilgangur bókarinnar}\index{Tilgangur Python}

Þessi bók fjallar um þau undirstöðu atriði sem þarf að kynna til að ná tökum á forritun í Python. 
Höfundi finnst mikilvægt að kenna námsefnið með íslenskum hugtökum þar sem ætlunin er að nota hana í kennslu í íslenskum framhaldsskólum. 
Ef nemendur ætla að leggja fyrir sig tölvunarfræði í framhaldssnámi er nauðsynlegt að búa yfir ríkulegu íðorðasafni, þess þá heldur ef nemandi hyggst framfleyta fræðunum. 
Hugtök verða þó líka sett fram á ensku því lesandi gæti óskað að fletta upp ítarefni sem meira er til af á netinu á ensku en íslensku.

Það er algengur misskilningur að forritarar kunni rosalega mörg forritunarmál, eins og fólk sem getur talað mörg tungumál, eða að það að kunna rosalega mörg mál geri þig að góðum forritara.
Þvert á móti.
Að sýna hæfni og leikni í einu máli er auðveldlega yfirfæranlegt á önnur mál sé þess þörf.
Þess vegna er spurningin ,,hvað kanntu mörg forritunarmál?'' út í hött.
Ekki aðeins eru tungumál og forritunarmál gerólík, forritunarmál eru formleg mál og þekking á einu hlutbundnu máli er nær því að vera jafn frábrugðið öðru í grunninn eins og mállýskur innan tungumála.
Nær væri að spyrja hvort viðkomandi hafi meiri áhuga á framenda eða bakenda forritun, hvað er skemmtilegasta reikniritið sem viðkomandi hefur útfært eða hvert er það forritunarmál sem viðkomandi grípur oftast í.

Einnig er það algengur misskilningur að það fyrsta sem fólk gerir er að búa til tölvuleik.
Það þarf mikla undirstöðu kunnáttu til þess að geta búið til tölvuleiki, alveg eins og áður en hafist er handa við að skrifa bók þarf að læra stafrófið.
Þessi grunnvinna finnst mörgum vera leiðigjörn.
Að mati höfundar er það vegna þess að við erum svo vön því að nota tölvur dagsdaglega, svo fræðigreinin sem tæknin byggir á hlýtur líka að vera okkur kunnug ekki satt?
Nei, alveg eins og dýralækningar eru okkur ekki augljósar við að eiga gæludýr og pípulagnir heldur ekki við að eiga klósett.
Innan tölvunnar eru ákveðnar grunneiningar sem eru notandanum ekki augljósar, af góðri ástæðu, það væri hrikalegt ef við þyrftum öll að vera píparar til þess að geta notað klósett.
Þó þessi samlíking hafi verið heldur gróf þá sýnir hún að það eru svo margir hlutar sem eru okkur huldir að við hreinlega vitum ekki hvað við vitum ekki.
Því er nauðsynlegt að læra grunninn vel og fara rólega yfir hann svo þegar við ætlum að fara að afrita og líma kóða frá síðum eins og stackoverflow þá vitum við allavega hvað sá kóði gerir (nokkurn veginn).

\section{Hvers vegna Python?}\index{Hvers vegna Python?}

Ástæður þess að Python er gott mál til þess að byrja á að skoða eru eftirfarandi: \footnote{Strax í þessum texta koma fyrir hugtök sem verða skýrð betur seinna, ekki missa kjarkinn.}.

\begin{enumerate}
	\item \textit{Málskipanin} er mjög svipuð mannlegu máli svo það er auðvelt að læra hvernig eigi að ,,tala'' við tölvuna.
	\item  Python er \textit{kvikt tagað} forritunarmál, það þýðir að notandinn þarf ekki að gefa upp hvers konar \textit{gagnatýpur} er unnið með. 
	Þetta gerir það að verkum að notandinn þarf ekki að læra urmull af lykilorðum áður en byrjað er að forrita.
	\item  Python er ekki alveg \textit{hlutbundið} forritunarmál, sem gerir það að verkum að notandinn þarf ekki að læra hvernig á að beita hlutbundinni forritun fyrr en góð undirstaða er þegar komin.
	\item Python er frítt og aðgengilegt öllum helstu stýrikerfum og einnig er hægt að forrita yfir netið í vafra og því óþarfi fyrir notandann að setja nokkuð upp sé þess óskað.
	\item Python er mikið notað, algengt mál svo það er praktískt að hafa undirstöðu skilning á því.
	\item Nefnt í höfuð á Monty Pyhton grínhópsins.
\end{enumerate}

\section{Uppsetning}\index{Uppsetning}
Uppbygging bókarinnar er þannig að fyrri hlutinn snýr að því að kynna lesandann fyrir grunnvirkni Python; málskipan, lykilhugtök og lykilorð, gagnatýpur, lykkjur og föll.
Seinni hlutinn snýr svo að því að beita þekkingu úr fyrri hlutanum í hlutbundinni forritun, þar eru kynntir til sögunnar klasar og aðferðir sem lesandinn útfærir upp á eigin spýtur.
Í lok hvers kafla eru svo verkefni til að reyna á leikni lesandans, lausnir við þeim öllum má finna í lausnarhluta aftast í bókinni.

Ekki er búist við neinni fyrri kunnáttu við lestur þessarar bókar, hún á að geta staðið fyrir sínu án þess að lesandinn búi yfir nokkurri þekkingu á sviði tölvunarfræða eða forritunar.
Ef slík þekking er fyrir hendi gæti lesandanum þótt ágætt að fara hratt í gegnum fyrri hluta bókarinnar og einbeita sér að verkefnum úr seinni hlutanum.
Í gegnum bókina fylgjum við svo þremur verkefnum sem verða þyngri og flóknari eftir því sem fleiri hugtök eru kynnt til sögunnar. \todo{þremur??}

Víðsvegar um bókina má finna númeraða kóðabúta.
Ástæðan fyrir því er að þessum kóðabútum er auðveldara að viðhalda heldur en skjáskotum úr vinnubókum og því er heldur vísað í bækur sem eru aðgengilegar lesendum og frumstæðari framsetning ræður heldur ríkjum hér.

\begin{lstlisting}[caption=Kóðabútar kynntir til sögunnar]
	# Svona líta kóðabútar út
	# kóðann má allan afrita og keyra til að sjá þá virkni sem verið er að kynna
\end{lstlisting}
\lstset{style=uttak}
\begin{lstlisting}
	# Svona lítur svarið út þegar kóðabúturinn fyrir ofan er keyrður
\end{lstlisting}
\lstset{style=venjulegt}

Einnig eru á nokrum stöðum stuttar efnisgreinar af ítarefni sem er ekki nauðsynlegt að hafa fullan skilning á en þó gott að skoða, sérlega fyrir þá lesendur sem vilja leggja frekara nám í tölvunarfræðum fyrir sig.
Þær líta svona út:

\begin{valBox}
	Þennan texta má leiða hjá sér við flýtilestur en gott að hafa skilning á ef lesandi vill ná góðum tökum á efninu.
\end{valBox}

\section{Að keyra kóða}\index{Að keyra kóða}\label{uk:keyra-koda}

Það fyrsta sem nemendur vilja yfirleitt gera er að byrja að skrifa sinn eigin kóða. 
Áður en við komumst svo langt þarf að útskýra hvernig það er gert. 
Þessi kennslubók byggir á notkun Jupyter Notebooks með hjálp Anaconda hugbúnaðarins, sem er öflugt pakkakerfi og tólakista sem hefur upp á mikið meira en bara Jupyter að bjóða. 
Hægt er að nálgast Anaconda á \href{www.anaconda.com}{anaconda.com}.
Hægt er að nota Jupyter án þess að ná í Anaconda með síðum eins og \href{www.cocalc.com}{cocalc.com}. 
Einnig er hægt að keyra kóða á netinu í gegnum síður eins og \href{www.repl.it}{repl.it}, nota ritla (eins og notepad eða sublime) til að keyra .py skrár í skipanalínu, eða nota þyngri umhverfi eins og pycharm sem eru sérhönnuð fyrir hugbúnaðarþróun. 
Hér er gert ráð fyrir Juptyter umhverfinu og verður bókin öll miðuð að því.


Þessari bók fylgja einnig nokkrar vinnubækur úr Jupyter sem lesandinn getur nýtt sér. 
Hér á mynd \todo{mynd af tómri bók} sést hvernig tóm Jupyter vinnubók lítur út. 
\todo{útskýra hvernig hún virkar?} 
Virkninni er skipt upp í sellur og keyrsluröð sellanna skiptir máli, við sjáum seinna mikilvægi þess að geta skipt upp kóða svona og hvers vegna þetta umhverfi er þægilegt til að byrja í. 
En hver sella hefur aðgang að svokölluðu skilgreiningarsvæði vinnubókarinnar en er þó sín eigin eining, því má keyra eina sellu í einu án þess að keyra allan kóðann í vinnubókinni.

Hér væri réttast að skoða Vinnubók 1 sem fylgdi þessari bók. 
\todo{hér væri gott að taka fyrir dæmi úr vinnubók sem á að fylgja}
\todo{segja miklu meira um hvernig á að nota vinnubækur yfirhöfuð}

En við keyrslu á kóða þarf einnig að hafa í huga að tölvan gerir nákvæmlega það sem við segjum henni að gera og ekkert annað.
Og þá komum við niður á stórt vandamál, að tölvur eru mjög bókstaflegar og vitlausar.
Þær skortir allt vit, þær reyna ekki að hafa vit fyrir þér. 
Þær gera nákvæmlega það sem þú baðst um.
Nákvæmlega eins og þú baðst um það.

Þannig að ef ég ætlaði að segja tölvu að smyrja handa mér hnetusmjörs og sultu samloku þá þyrfti ég að segja vélinni að gera eftirfarandi í nákvæmlega þessari röð:
\vspace{0.4cm}
\begin{enumerate}
	\item taka fram hníf
	\item taka fram tvær brauðsneiðar
	\item opna hnetusmjörið
	\item setja beitta endann ofan í hnetusmjörið þannig að hann nái upp 50gr af hnetusmjöri
	\item setja hnetusmjörið sem er á hnífnum á miðja brauðsneiðina
	\item nota hnífinn til þess að smyrja hnetusmjörinu á þá hlið sem hnetusmjörið er nú þegar á, og enga aðra
	\item taka fram skeið
	\item opna sultuna
	\item setja kúpta enda skeiðarinnar ofan í sultukrukkuna 
	\item taka skeiðina upp úr sultukrukkunni með kúfaða skeið af sultu
	\item setja sultuna á hina brauðsneiðina
	\item nota skeiðina til að smyrja sultunni yfir þá hlið brauðsneiðarinnar sem sultan er á og enga aðra hlið
	\item  setja brauðsneiðarnar saman þannig að hnetusmjörið og sultan snertist og hornin mætast öll. 
\end{enumerate} 
\vspace{0.4cm}
Takið eftir að hér er gert ráð fyrir þó nokkru og ef vélin kann ekki nú þegar skil á eftirfarandi, mun þetta klúðrast: 
\vspace{0.2cm}
\begin{enumerate}
	\item taka fram
	\item hnífur
	\item opna
	\item mæla 50 gr
	\item smyrja
	\item hlið á brauðsneið
	\item miðja á brauðsneið
	\item skeið
	\item kúfað
\end{enumerate} 
\vspace{0.2cm}

\todo{setja inn mynd af brauðinu með osti og skinku á milli}

Þessi útskýring á samlokugerð kann að vera alveg ofboðslega óþarflega nákvæm þá er ekki víst að úr þessu verði nokkur samloka.
Þetta könnumst við öll við, að tölvur gera það sem þeim er sagt, ekki það sem við viljum.

Helsta verkefni forritara er að búta niður verkefni í svo litla hluta að hægt er að útskýra þá fyrir tölvu.
Ekki búast við því að setjast niður við fyrsta verkefni og ætlast svo til að búa til tölvuleik eða hakka banka.
Forritun er einnig frábrugðin þeirri venjulegu tölvunotkun sem þú hefur vanist dagsdaglega.
Þar ertu ekki að gefa tölvunni þínar eigin skipanir heldur ertu að beita skipunum sem aðrir forritara hafa samið og sett upp í hugbúnaðinn sem þú ert að nota.

%------------------------------------------------

\section{Málskipan}\index{Málskipan}

\textbf{Málskipan} (e. syntax) er safn þeirra reglna sem við þurftum til þess að skrifa kóða sem tölvan skilur, svo að hann þýðist í vélamál, þær reglur sem við þurfum að fara eftir þegar við forritum, þær reglur sem forritunarmálið býst við að við förum eftir. 
Ef við brjótum þessar reglur fáum við villu, og einhver algengasta villa sem hægt er að fá er málskipanarvilla (e. syntax error). 
Python er frábrugðið öðrum forritunarmálum á þann hátt að málskipanin krefst þess að kóðinn sé settur upp á ákveðinn hátt. 
Líkja má því við að þurfa ekki að hafa greinarmerki í huga þegar við ljúkum setningum heldur setjum við orðin okkar á réttan stað í samræðum.

\subsection{Uppsetning á kóða}\index{Uppsetning á kóða}
Þessi kóðabútur er þannig uppsettur að allar línur byrja jafnlangt til vinstri, eins og hver setning í töluðu máli stendur hver lína fyrir sínu, ein og sér.
\begin{lstlisting}[caption=Réttur Python kóði, label=lst:inng-kóðadæmi]
# Réttur Python kóði sem keyrist
4 + 8
5 + 6
breyta = 9 * 2
\end{lstlisting}

Þessi næsti kóðabútur hinsvegar er ekki nógu vel uppsettur, þar eru ,,setningar'' sem virðast hanga undir öðrum og vera þeim háðar.

\begin{lstlisting}[caption=Rangur Python kóði sem veldur villu, label=lst:inng-malskipanarvilla]
# Illa skrifaður Python kóði sem keyrist ekki
4 + 8. # punkturinn veldur málskipanarvillu
	5 + 6 # inndrátturinn hér er rangur
breyta = 9 * 2 * # málskipanarvilla fæst hér því síðasta táknið er á röngum stað
\end{lstlisting}

Svona inndrætti er einungis beitt ef lína á beinlínis að hanga undir línunni að ofan og tilheyrir henni. 
Þess vegna þarf að huga að því hvernig kóði er uppsettur. 
Í öðrum málum eru notuð greinamerki til að segja tölvunni að lína sé búin og að aðrar línur eigi að heyra undir eitthvað ákveðið samhengi en ekki í Python, þar er treyst á að forritarinn setji kóðann upp á máta sem hægt er að sjá að sé réttur. 
Dæmi um hvernig línur geta verið aðgreindar í öðrum málum:

\begin{lstlisting}[language=Java , caption=Dæmi um annað mál sem er strangt tagað og með greinamerkjum]
// Dæmi um kóða í forritunarmálinu Java
int i = 7;
i + 5;

// Þetta myndi líka ganga í Java en ekki í Python
int i = 7; i + 5;
\end{lstlisting}

\begin{lstlisting}[language=Lisp, caption=Dæmi um annað mál sem byggir á afmörkuðu samhengi en með greinamerkjum]
; Lisp
(setq x 10)
(setq y 34.567)

(print x)
(print y)
\end{lstlisting}

Í þessum tveimur frábrugðnu málum sem voru tekin sem dæmi var óþarfi að setja kóðann í mismunandi línur, því greinamerkin væru nóg til að aðgreina hverja línu fyrir sig. 
Hins vegar er það góð venja að skrifa kóða sem er læsilegur öðru fólki. Í Java eru greinarmerkin semikommur (;) en í Lisp eru línur og samhengi afmörkuð með svigum. 
Python byggist hinsvegar á því að forritarinn stilli öllu upp rétt með réttum inndrætti. 

\subsection{Gagnatýpur og lykilorð}\index{Gagnatýpur og lykilorð}

Í Python eru nokkrar grunn gagnatýpur sem við munum kynnast í þessari bók. 
Ástæðan fyrir því að þær eru kallað grunntýpur er sú að þær fylgja með Python uppsetningunni og notandinn getur beitt þeim í samræmi við það sem þær eru færar um, sem má skoða í skjölun Python \href{https://www.python.org/doc/}{https://www.python.org/doc/}. 
Týpa eða tag er hugtak sem þýðir að hlutur sé af einhverri ákveðinni tegund sem má framkvæma ákveðnar aðgerðir á, í þessari bók verða týpur ýmist kallarðar það eða tög. 
Lesandi þekkir muninn á orðum og tölum úr daglegu tali og veit að hægt er að framkvæma mismunandi aðgerðir á þessum mismunandi týpum, eins og hægt er að skipta út hástöfum fyrir lágstafi í orðum en ekki tölum og hægt er að hefja tölur í veldi en ekki orð. 
Að sama skapi eru til aðgreinanlegar týpur sem tölvan kann skil á og leyfir ákveðnar aðgerðir á.
Í fyrri hluta þessarar bókar verða gerð skil á tveimur talnatýpum (heiltölum og fleytitölum), strengjum, listum, sanngildum, orðabókum (einnig kallaðar hakkatöflur), nd-um og mengjum.

Lykilorð eru orð sem eru frátekin og birtast þau græn í Jupyter vinnubók. 
Hver gagnatýpa hefur eitt lykilorð og eru einnig nokkur innbyggð föll í Python, sem við kynnumst fljótlega, með frátekin orð. 
Forðast skal að yfirskrifa þessi lykilorð, en gerist það þá er auðvelt að laga það í Jupyter. 
Hver vinnubók hefur sinn kjarna til að vinna á og það eina sem þarf að gera í aðstæðum þar sem innbyggt orð er allt í einu farið að þýða eitthvað annað þá dugir að endurræsa kjarnann.
Kjarninn í vinnubókinni er hvaða túlk eða þýðanda er verið að nota til þess að láta tölvuna skilja kóðann.
Í okkar tilfelli erum við að nota Python 3.