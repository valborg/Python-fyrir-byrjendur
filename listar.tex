\chapterimage{chapter_head_2.pdf} % Chapter heading image

\chapter{Listar}\index{Listar}\label{k:listar}
Listar eru gagnagrindur, sem þýðir að þeir geta geymt fyrir okkur hin ýmsu gögn og gert okkur þau aðgengileg á ákveðinn máta.
Listar eru skilgreindir með hornklofum [ ] og er lykilorðið þeirra \textbf{list}.

\chapter{Listar skilgreindir}\index{Listar skilgreindir}\label{uk:listar-skilgreindir}
Listar geyma, í ákveðinni röð, þau gögn sem við viljum geyma sem mega vera af hvaða týpu sem er.
Gögnin sem eru sett inn í listann eru kölluð stök og röðin sem þau eru í eru aðgengileg eftir vísum eða sætisnúmerum alveg eins og strengir.
Stökin eru aðgreind með kommum.
Þær týpur sem við höfum séð hingað til eru heiltölur, fleytitölur, strengir og listar.
Allt eru þetta möguleg stök í lista.

\begin{lstlisting}[caption=Listar skilgreindir, label=lst:listar-skilgreindir]
# Fyrsti listinn okkar er tómur
listinn_minn = []

# þegar við skilgreinum lista aðgreinum við stökin með kommum
nyr_listi = ["núllta stakið", 1, 2, 3.0, "fjórða stakið", [5]]
\end{lstlisting}

Í kóðabút \ref{lst:listar-skilgreindir} sjáum við að við erum með 6 stök í listanum nyr\_listi sem er skilgreindur í línu 6 \todo{passa að breyta ekki kóðabút til að þessi vísun haldist}.
Fremsta stakið er strengur, næstu þrjú eru tölur, síðan kemur annar strengur og síðasta stakið í sæti 5 er listi.
Sá listi inniheldur eitt stak sem er þá í núllta vísi í þessum innri lista.

Ef við hugsum okkur töflureikni eins og Calc eða Excel þá getum við ímyndað okkur að ein lína sé eins og einn listi, hver dálkur er stak í listanum og ein röð er listinn sem heldur utan um þau.
Þá getum við líka ímyndað okkur að ef við erum með margar raðir séu þær geymdar á einni örk eða einu skjali.
Sjáum hvernig það myndi líta út í kóðabút \ref{lst:listar-arkir}

\begin{lstlisting}[caption=Listar af listum, label=lst:listar-arkir]
# Ef við ættum skjal í töflureikni sem héldi utan um allt starfsfólk í fyrirtæki gæti hausinn á því litið svona út:
# Nafn Tölvupóstur Deild Símanúmer 

# Svo er hver röð fyrir neðan það útfyllt með upplýsingum um einhvað tiltekið starfsman, t.d.:
# Jóna Jónsdóttir jona@fyrirtaeki.is Póstur 4445555

# Ef þetta væri útfært í Python með listum væri það gert svona:

starfsfolk = [["Jóna Jónsdóttir", "jona@fyrirtaeki.is", "Póstur", "4445555"],
			  ["Kristján Kristjánsson", "kristjan@fyrirtaki.is","Laun","4445589"],
			  ["Halldóra Halldórudóttir", "halldora@fyrirtaeki", "Skrifstofa", "4445500"]]

\end{lstlisting}

Við tökum eftir því að listinn starfsfolk í kóðabút \ref{lst:listar-arkir} í línu 9 inniheldur þrjá aðra lista, og þeir eru aðgreindir með kommum alveg eins og stökin inni í hverjum innri lista fyrir sig eru einnig aðgreind með kommum.
Einnig tökum við eftir því að hér sjáum við í fyrsta sinn inndrátt, það er í raun bara aukalegt bil sem vélin hunsar við að skilgreina breytuna starfsfolk og er því fyrir okkur til að geta lesið kóðann auðveldlegar.
Þetta er ekki eins og inndrátturinn sem við munum sjá og beita í næsta kafla (\nameref{k:segðir}
 
\section{Að vinna með gögn}\index{Að vinna með gögn}\label{uk:gagnavinnsla-listar}

\section{Gagnlegar aðferðir á lista}\index{Aðferðir á lista}\label{uk:aðferðir-listar}