
\chapter{Tölur og breytur}\index{Tölur og breytur}
Í þessum kafla ætlum við að hefjast handa við að forrita. 
Það fyrsta sem við ætlum að gera er að kynnast talnatýpum og keyra kóða eins og við værum að nota reiknivél. 
Við könnumst við reiknivélar og hvernig þær afgreiða röð aðgerða. 
Nú viljum við sannreyna að þær reikniaðgerðir sem við þekkjum séu til í Python og að þegar við keyrum kóðann okkar þá verði útkoman sú sama og við áttum von á. 
Við viljum líka geta geymt útkomuna okkar til að nota aftur seinna, til þess þurfum við breytur (e. variables).

\section{Tölur - talnatýpur}\index{Tölur - talnatýpur}
Í Pyhton eru í grunninn tvær týpur af tölum (en til eru tvær týpur af hvorri fyrir sig, sem snýr meira að minnisnotkun og er út fyrir svið þessarar bókar). 
Þær eru:

\begin{itemize}
	\item \textbf{Heiltölur} - tölur sem eru ekki með neinum aukastaf. 
	Á ensku eru þessar tölur kallaðar integers og er lykilorð þeirra því \textbf{int}.
	\item \textbf{Fleytitölur} - tölur sem eru með aukastaf, sem er fyrir aftan punkt (ekki kommu, fleytitölur eru oft kallaðar kommutölur á íslensku). 
	Á ensku eru þessar tölur kallaðar floating point numbers og er því lykilorðið þeirra \textbf{float}.
\end{itemize}

\begin{lstlisting}[caption=Heiltölur og fleytitölur]
# Heiltölur, enginn aukastafur
42
100000
-139

# Fleytitölur, aukastafur/ir fyrir aftan punkt
4.0
3.1415926
-100.98
\end{lstlisting}

\section{Reikniaðgerðir og tákn}\index{Reikniaðgerðir og tákn}
Grunn reikniaðgerðir eru nokkrar sem við könnumst við úr grunnskóla en aðrar eru framandi og við skulum skoða aðeins betur.
\todo{tala um reikniaðgerðirnar hérna}

Í eftirfarandi dæmum er vert að draga fram nokkur atriði sem eru ekki augljós byrjanda. 
Það fyrsta er að myllumerkið (\#) þýðir að allt sem kemur fyrir aftan það er \textit{athugasemd}, athugasemdir eru engöngu til að gera kóða læsilegri fyrir fólk, þær eru hunsaðar af tölvunni þegar hún breytir kóðanum í eitthvað sem hún skilur. 
Einnig eru þarna bil á milli talna og tákna, það er líka til að gera kóðan læsilegri, bilin mega bara ekki vera fremst í línunni enn sem komið er.

\begin{lstlisting}[caption=Reikniaðgerðir]
# Samlagning framkvæmd með + 
# Þegar eftirfarandi kóði er keyrður ætti útkonan að vera 10
6 + 4 

# Frádráttur framkvæmdur með -
# Þegar eftirfarandi kóði er keyrður ætti útkonan að vera 10
14 - 4 

# Margföldun framkvæmd með * 
# Þegar eftirfarandi kóði er keyrður ætti útkonan að vera 10
10 * 2 

# Deiling framkvæmd með / 
# Athugið að þetta er fleytitöludeiling sem skilar nákvæmu svari
# Þegar eftirfarandi kóði er keyrður ætti útkonan að vera 10.0
60 / 6 

# Heiltöludeiling framkvæmd með //
# Athugið að þessi deiling er frábrugðin þeirri sem þið kannist við
# Hér viljum við vita hversu oft, heil tala, ein tala gengur upp í aðra og okkur er sama um afganginn
# Þegar eftirfarandi kóði er keyrður ætti svarið að vera 10
177 // 17

# Veldishafning framkvæmd með **
# Hér er mikilvægt, eins og með deilinguna, að hafa í huga hvor talan kemur á undan.
# Fyrst kemur talan sem hefja á í veldi og svo kemur talan sem er veldisvísirinn
# Þegar eftirfarandi kóði er keyrður ætti svarið að vera 9
3 ** 2

# Leifareikningur framkvæmdur með % (e. modulus)
# Þetta er eitthvað alveg nýtt og framandi, en þó ekki óskiljanlegt
# Það sem þetta reiknar er hversu mikil leif eða afgangur er eftir þegar heiltöludeilingu er beitt.
# Þegar eftirfarandi kóði er keyrður ætti svarið að vera 7
177 % 17

\end{lstlisting}

Í öllum þessum dæmum var verið að vinna með heiltölur, þó var útkoman úr deilingunni (stundum kölluð fullkomin deiling) fleytitala. 
Hvað gerist ef þessir sömu útreikningar eru gerðir með fleytitölum? 
Ef við myndum skipta út hverri tölu fyrir sig og setja í staðinn sömu tölu með .0 fyrir aftan þá yrðu útkomurnar þær sömu nema fleytitölur. 
En hvað gerist ef við breytum aðeins fyrri tölunni en ekki seinni tölunni?
Þá ertu að nota ólíkar týpur og slíkt er vandmeðfarið, en í þessu tilviki er það í lagi þar sem Python gerir þá ráð fyrir að það sé í lagi að reikna með fleytitölum og framkvæmir reikninginn eins og þú hafir verið að beita fleytitölum og niðurstaðan verður þá að sjálfsögðu fleytitala.

\section{Breytur}\index{Breytur}
Nú höfum við séð hvernig má keyra kóða einfaldlega eins og í reiknivél.
Höldum okkur við samlíkinguna um reiknivélina til að útskýra breytur.
Á hefbundinni reiknivél sem notuð er í stærðfræðitíma í framhaldsskóla er takki sem á stendur ANS.
Það stendur fyrir answer og ef ýtt er á hann getur vélin geymt síðasta gildið sem hún gaf sem svar og unnið svo með það til að gefa næsta svar.
Flottari vélar geta svo geymt nokkuð mörg svör en það er útfyrir gagnsemi þessarar samlíkingu.
Þegar ýtt er á þennan takka er minnisvæði í reiknivélinni tekið frá og skrifað er í það gildi, sem er svo sótt þegar ANS er notað í útreikningi.
Að sama skapi má láta Python úthluta minnissvæði í tölvunni fyrir þær breytur sem þið viljið geyma.
Munurinn er sá að þið nefnið sjálf hvað minnisvæðið er merkt sem, eruð ekki bundin við að nota ANS og að þið eruð svo gott sem með óteljandi minnissvæði.

Að gefa minnissvæði merkingu og gildi er gert með \textit{gildisveitingu}.
Gildisveiting þýðir að nú er einhver ákveðinn merkimiði kominn með eitthvað til að geyma.
Sjáum einfalt dæmi um þetta.

\begin{lstlisting}[caption=Breytur kynntar]
# Hér er ég að fara að búa til breytu sem heitir val
val = 5

# Þegar ég keyri línuna fyrir ofan segi ég vélinni að hafa aðgengilegt minnisvæði sem ég get notað með því að skrifa orðið val, og settu í það svæði gildið 5.

# Svo ég er að veita breytunni val gildið 5, þess vegar er það kallað gildisveiting.

# Svo get ég notað breytuna mína
# þegar þetta er keyrt fæst svarið 10
val + 5
\end{lstlisting}

Ef þú prófar þig áfram við að búa til breytur gætir þú rekist á svolítið sem hefur ekki gerst áður í vinnubók, að þegar sella inniheldur eingöngu gildisveitingu og er keyrð þá ,,gerist ekkert''.
Þetta finnst mörgum mjög skrýtið því þau vilja fá einhverja útkomu.
En útkoman er sú að þú sagðir vélinni að geyma þetta, þú sagðir henni ekki að gera neitt annað.

Breytur eru skilgreindar vinstra megin við jafnaðarmerki í Python.
Eins og það væri lesið, val fær gildið 5.
Það væri lítið vit í því að hafa það öfugt, 5 er núna jafngilt val.
Það sem við værum þá að segja tölvunni að í hver sinn sem hún vill nota heiltöluna fimm þá á hún að hætta við að nota töluna sjálfa og í staðinn vísa eingöngu í það sem er í minnissvæði merktu val.
Það er alls ekki það sem við viljum.

\begin{lstlisting}[caption=Dæmi um gildisvetingar\, réttar og rangar]
# Hér er ég að fara að búa til breytu sem heitir val
val = 5

# Hér er ég ekki að búa til breytu sem heitir val heldur er ég að segja að talan fimm er ekki lengur til sem ehiltala heldur gæti hún vísað í hvað sem er sem er geymt í minnisvæði merktu val
5 = val

# Hér bý ég til breytu sem heitir heiltala sem fær gildið 0
heiltala = 0

# Hér yfirskrifa ég lykilorðið fyrir týpuna heiltala og læt það innihalda 0
int = 0
# þetta er harðbannað og ef þetta gerist er ekki nóg að þurrka þetta út og keyra aftur, nú þarf að endurræsa kjarna vinnubókarinnar.
\end{lstlisting}

Nú þegar við höfum séð hvernig má skilgreina breytu viljum við vita hvernig á að nota þessa breytu.
Ef við snúum okkur aftur að reiknivélasamlíkingunni um ANS takkann þá ætti eftirfarandi kóðabútur að geta sýnt með eðlislægum hætti hvernig breytur nýtast.

\begin{lstlisting}[caption=Að nota breytu]
# Hér framkvæmi ég útreikning sem ég geymi í breytunni ANS
ANS = 5**2 + (4+8.9)**2

# Segjum að þetta hafi verið endapunkturinn í löngu algebrudæmi og nú veit ég hvað y er, og get þá nýtt það til að finna x eins og verða vill svo oft í stærðfræði að x sé týnt. Gefum okkur að x = 3 * y og því fæst
x = 3 * ANS

# Nú ef við viljum reikna eitthvað út með x eigum við það til í minnissvæði merktu x með réttu gildi.
Til dæmis með því að búa til breytu fyrir hálft x.
halft_x = x/2


# Nú langar okkur til að vera viss um að við séum við vitrænt svar svo við biðjum tölvuna um að segja okkur hvað er geymt í breytunni x.
print(x)


\end{lstlisting}

% Að því sögðu þá þurfum við að skoða breytur nokkuð betur áður en við förum að beita þeim á skilvirkan hátt.

Takið eftir að þarna er notuð ný framsetning sem við höfum ekki séð áður, þarna stendur print með svigum fyrir aftan og inni í svigunum er breytan okkar.
Ef þessi kóðabútur er keyrður þá kemur á \textit{staðalúttak} 
\footnote{Þann stað sem texti myndi prentast þegar forritið er notað, hvort sem það er á skjá eða beint á pappír úr prentara eða eitthvað allt annað. Kannski verður úttaki varpað beint inn í heilann á forriturum einhvern tíma?} 
það gildi sem breytan \textbf{x} inniheldur.
