\chapterimage{chapter_head_2.pdf} % Chapter heading image

\chapter{Segðir, skilyrðissetningar og Boolean gildi}\label{k:segðir}

Kóða má skipta í segðir (e. expressions) og yrðingar (e. statements).
Segðir eru línur þar sem eitthvað er metið sem gildi, ef við líkjum því við tungumál væru það setningar þar sem einhver niðurstaða fæst eins og ,,er rigning''? 
Yrðingar eru línur þar sem eitthvað er sett fram sem þarf ekki að meta, í tungumáli væru það setningar á borð við ,,mér er kalt''

Þessi skipting er ekkert sérlega merkileg að svö stöddu en í þessum kafla ætlum við að einbeita okkur að því að meta útkomu og fá í hendurnar svör sem við getum svo gert eitthvað við.

Til þess að gera það þurfum við að læra á nýja týpu sem heitir Boolean og hefur lykilorðið \textbf{bool}.
Hún er frábrugðin þeim týpum sem við höfum séð hingað til því að það eru eingöngu tvö möguleg gildi sem Boolean getur verið, \textbf{True} og \textbf{False} sem þýðast sem 1 og 0, satt og ósatt og eru upprunin úr búlískri algebru \footnote{Ekki verður farið yfir búlíska algebru af neinu ráði í þessari bók en þau fræði eru gífurlega góður grunnur til að skilja betur hvernig segðir og rökvirkjar virka og því hvetur höfundur til að lesandi fletti allavega upp wikipedia greininni.} (e. Boolean algebra). 
Nú er það flestum kunnug staðreynd að tölvur vinna með 0 1 í grunninn, en hvernig það er notað í æðri forritunarmálum (e. high level programming languages) er ekki eins augljóst.

Í þessum kafla verður farið yfir búlísk gildi, samanburð (e. comparison) og samanburðarvirkja (e. comparison operands), rökvirkja (e. logical operands) og svo skilyrðissetningar (e. conditional statements).

\section{Boolean gildi}\index{Boolean gildi}
\section{Segðir}\index{Segðir}
\subsection{Samanburður}\index{Samanurður}
\subsection{Rökvirkjar}\index{Rökvirkjar}
\section{Skilyrðissetningar}\index{Skilyrðissetningar}
\subsection{if}\index{if}
\subsection{else}\index{else}
\subsection{elif}\index{elif}