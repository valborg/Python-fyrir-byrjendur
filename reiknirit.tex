\chapterimage{chapter_head_2.pdf} % Chapter heading image

\chapter{Reiknirit}\index{Reiknitir}\label{k:reiknirit}
Reiknirit (e. algorythm) er forritsbútur sem sinnir sérhæfðum útreikningi.
Dularfyllra er það ekki.
Reiknirit sinna því ákveðnum tilgangi og eru þau oft í grunninn stærðfræðlegs eðlis.

Dæmi um reiknirit sem við höfum séð áður í þessari bók væri útfærsla á fjarlægð milli lesta og að setja nýja lestarstöð inn á leið lestar í kóðabút \ref{lst:klasar-aðferðir-lestar}.

Ástæðan fyrir því að nauðsyn þykir að kynna reiknirit í bók sem þessari er að ef nemendur hafa áhuga á að kynna sér tölvunarfræði í framhaldssnámi er gott að hafa fengið nasasjón af því hvað felst í að beita reikniritum og útfæra þau.
Margir nemendur hefja nám í tölvunarfræði með ýmsar forhugmyndir sem eiga sér sumar ekki stoð í raunveruleikanum.
Þessi kafli og sá næsti fjalla um þau atriði sem leikmenn átta sig ekki endilega á að séu stór hluti af tölvunarfræðum og hugbúnaðarþróun.
Stærðfræði og samvinna.
Þessi kafli er um stærðfræðilegu hliðina og næsti um samvinnuna.

\comment{
\section{Reiknirit sem við höfum séð}\index{Reiknirit sem við höfum séð}\label{uk:reiknirit-okkar}
omg omg omg
\begin{lstlisting}[caption=Við höfum séð eftirfarandi reiknirit, label=lst:reiknirit-okkar]
# kóði
\end{lstlisting}
}

\section{Þekkt reiknirit}\index{Þekkt reiknirit}\label{uk:reiknirit-þekkt}
Það sem við ætlum að skoða í þessum kafla eru tvö ákveðin reiknirit, bæði mjög þekkt og svo hugmyndin um endurkvæmni.


\subsection{Endurkvæmni}\index{Endurkvæmni}\label{uk:reiknirit-endurkvæmni}
Endurkvæmni (e. recursion) er sú virkni forrits að vísa í sig sjálft.
Þið þekkið eflaust listaverk sem virka eins og skynvillur, þar sem manneskja getur labbað í hring upp stiga en endað á sama stað því stiginn fer í raun í hring.
Eða þið hafið séð ykkur sjálf í spegli þar sem var annar spegill fyrir aftan og þið sáuð ótal spegilmyndir raðast af ykkur.

Í forritun heitir það endurkvæmni þegar við látum forrit vísa í sig sjálft, nota sinn eigin kóða.
Gott dæmi um hvernig megi beita endurkvæmni til að fá skilmerkilega niðurstöðu er að útfæra fall sem reiknar fyrir okkur einhverja gildi í fibonacci röðinni.
En áður en við gerum það skulum við skoða enn einfaldara dæmi þar sem við erum með fall sem kallar í sig sjálft og gerir ekkert annað en það.
Í kóðabút \ref{lst:reiknirit-endurkvæmni1} sjáum við einfalda útgáfu af endurkvæmni, þar sem hugmyndin er í raun kynnt án þess að fallið sé neitt gagnlegt.
Það eina sem fallið gerir er að athuga hvort að talan sé stærri en núll og ef hún er það þá kallar fallið í sig sjálft með einu lægra gildi, annars ef talan er ekki stærri en núll prentast ,,þú kannt á endurkvæmni''.
Hveru oft ætli það prentist ef við setjum inn töluna 5?

\begin{lstlisting}[caption=Endurkvæmni - einfalt, label=lst:reiknirit-endurkvæmni1]
def endurkvæmt_fall(tala):
	if(tala > 0):
		# á meðan talan er hærri en 0 þá köllum við aftur í fallið
		# en við köllum í það af gildi einu lægra
		return endurkvæmt_fall(tala-1)
	else:
		# hér erum við komin niður í 0 og prentum eftirfarandi texta
		print('við kunnum endurkvæmni')
		
endurkvæmt_fall(5)
# úttakið verður
# 'við kunnum endurkvæmni'
\end{lstlisting}

Við sjáum í kóðabút \ref{lst:reiknirit-endurkvæmni1} að þó að við kölluðum í fallið með tölunni 5 þá fengum við bara einu sinni út strenginn ,,við kunnum endurkvæmni''.
Það er vegna þess að við kölluðum í fallið fyrir fimm og það sem fallið gerir fyrir okkur er að klára endurkvæmnina fyrir það kall, það verða ekki til fjögur önnur köll.
Heldur verður fimm að fjórum sem skilar okkur svo niðurstöðunni fyrir þrjá og svo koll af kolli þar til við erum komin niður í núll og þá hættir forritið.

Það sem þetta reiknirit okkar gerir ekki er að skila einhverri niðurstöðu til baka, en nú skulum við skoða tvö þannig reiknirit sem eru endurkvæm.
Annað þeirra reiknar summu af einhverri tölu og öllum jákvæðum tölum lægri en henni, svo talan 5 gæfi útkomuna 5 + 4 + 3 + 2 + 1 sem er 15 og fyrir 100 væri það 100 + 99 + ... + 1 sem gæfi 5050.
Hitt reiknar n-tu töluna í fibonacci rununni.

\subsection{Helmingunarleit}\index{Helmingunarleit}\label{uk:reiknirit-helmingunarleit}
Byrjum á að skoða eitthvert þekktasta reiknirit sem til er. 
Helmingunarleit að tölu á bili. 
Hugsum okkur að við séum með raðaðan lista af tölum og við viljum finna eina tiltekna tölu. 
Ef við ættum að skoða hverja einustu tölu í listanum til að finna hana þá tæki það mjög langan tíma.
Eða allavega fyrir okkur sem manneskjur, en allur tímasparnaður er góður.
Því að aðgerðin ,,að skoða spil'' kostar einhvern tíma og því færri þannig aðgerðir sem við getum gert því hraðara er reikniritið okkar.


\subsection{Bubble sort}\index{Bubble sort}\label{uk:reiknirit-bubble}
omg

