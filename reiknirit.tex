\chapterimage{chapter_head_2.pdf} % Chapter heading image

\chapter{Reiknirit}\index{Reiknitir}\label{k:reiknirit}
Reiknirit (e. algorithm) er forritsbútur sem sinnir sérhæfðum útreikningi.
Dularfyllra er það ekki.
Reiknirit sinna því ákveðnum tilgangi og eru þau oft í grunninn stærðfræðlegs eðlis.

Dæmi um reiknirit sem við höfum séð áður í þessari bók væri útfærsla á fjarlægð milli lesta og að setja nýja lestarstöð inn á leið lestar í kóðabút \ref{lst:klasar-aðferðir-lestar}.

Ástæðan fyrir því að nauðsyn þykir að kynna reiknirit í bók sem þessari er að ef nemendur hafa áhuga á að kynna sér tölvunarfræði í framhaldssnámi er gott að hafa fengið nasasjón af því hvað felst í að beita reikniritum og útfæra þau.
Margir nemendur hefja nám í tölvunarfræði með ýmsar forhugmyndir sem eiga sér sumar ekki stoð í raunveruleikanum.
Þessi kafli og sá næsti fjalla um þau atriði sem leikmenn átta sig ekki endilega á að séu stór hluti af tölvunarfræðum og hugbúnaðarþróun.
Stærðfræði og samvinna.
Þessi kafli er um stærðfræðilegu hliðina og næsti um samvinnuna.

\comment{
\section{Reiknirit sem við höfum séð}\index{Reiknirit sem við höfum séð}\label{uk:reiknirit-okkar}
omg omg omg
\begin{lstlisting}[caption=Við höfum séð eftirfarandi reiknirit, label=lst:reiknirit-okkar]
# kóði
\end{lstlisting}
}

\section{Þekkt reiknirit}\index{Þekkt reiknirit}\label{uk:reiknirit-þekkt}
Það sem við ætlum að skoða í þessum kafla eru nokkur grundvallar reiknirit, og hugmyndin um endurkvæmni.


\subsection{Röðun}\index{Röðun}\label{uk:reiknirit-röðun}
Áður en við getum farið að leita ætlum við að raða.
Ef við ímyndum okkur að nokkur spil úr spilastokk séu fyrir framan okkur, þau snúa upp svo við sjáum hvaða spil þetta eru en við sjáum einnig að þau eru ekki í réttri röð.
Sú röð gæti verið hver sem er, en gerum ráð fyrir, til einföldunar, að þau séu ekki í vaxandi röð með lægsta spilið vinstra megin og hæsta spilið hægra megin.
Hvað er það fyrsta sem við gerum?
Það er ekki eitt svar við því.
Hugsum okkur hér hvað sé fyrsta skrefið og svo næsta og koll af kolli.
Reynum svo að lýsa því, eins og með hnetusmjörssamlokuna í kafla \ref{uk:keyra-koda}, þannig að við gætum snúið baki við einhverjum spilum og lýst fyrir einhverri annarri manneskju hvernig hún gæti farið að því að raða spilunum með aðferðinni okkar.

Hér er sterklega mælt með því að þið reynið á þessa æfingu áður en lengra er haldið.


Ein tiltekin leið til að raða spilunum væri að bera saman fremstu spilin og skipta þeim ef það hærra er vinstra megin og gera það fyrir öll spilin út röðina, skoðum töflu \ref{tbl:insert} til að sjá hvernig það myndi ganga fyrir sig.
Þar erum við með einhverja ákveðna upphafsstöðu á spilunum og byrjum á að skoða fremstu tvö spilin, þá ,,svissum'' við þeim til að hærra spilið sé hægra megin.
Í næstu atrennu ætlum við þá að skoða spilið sem er næst í röðinni og við ,,svissum'' því þangað til að það er annað hvort úti í enda eða orðið hærra en það sem er vinstra megin.
Þá vegna þess að við erum bara með fjögur spil skoðum við síðasta spilið og látum það berast niður spilaröðina þar til það passar.


\begin{table}
\begin{center}\begin{tabular}{!{\vrule width 2pt}c!{\vrule width 2pt}P|P|P|P!{\vrule width 2pt}}
	\noalign{\hrule height 2pt}
	upphafsstaðan &\textbf{\textcolor{ocre}{K}} & \textbf{\textcolor{ocre}{10}} & 4 & 8 \tabularnewline
	\cline{1-5}
	& 10 & K & \textbf{\textcolor{ocre}4} & 8 \tabularnewline
	\cline{2-5}
	& 4 & 10 & K & \textbf{\textcolor{ocre}8} \tabularnewline
	\cline{1-5}
	%\hline
	lokastaðan & 4 & 8 & 10 & K \tabularnewline
	\noalign{\hrule height 2pt}
\end{tabular}
\end{center}
\caption{Hér sést hvernig fjórum spilum er raðað í vaxandi röð. Í hverri stöðu er einungis verið að skoða takmarkaðan fjölda spila til að raða, þau spil eru sýnd með grænum lit í töflunni.}
\label{tbl:insert}
\end{table}

Reikniritið sem lýsir þessari röð aðgerða heitir \emph{insertion sort} eða innsetningarröðun og í kóðabút \ref{lst:reiknirit-insertion} ætlum við að skoða hvernig skiptingarnar eiga sér stað.
Á mynd \ref{tbl:insert-abstract} sést hvað er að gerast í einhverju tilteknu skrefi þar sem þarf að finna stað fyrir næsta stak sem á að raða, x.

\begin{table}
	%\begin{table}
		\begin{center}
			\begin{tabular}{c|c |c|c}
				\multicolumn{2}{c}{Búið að raða þessum hluta} &\multicolumn{2}{c}{ Á eftir að raða} \tabularnewline
				\noalign{\hrule height 2pt}
				minna eða jafnt x & stærra en x & x & ??? 
				\tabularnewline \noalign{\hrule height 1.5pt}
				\multicolumn{4}{c}{Í næsta ástandi er röðun spila orðin:}
				
				\tabularnewline
				\multicolumn{3}{c}{Búið að raða þessum hluta} &\multicolumn{1}{c}{ Á eftir að raða}  \tabularnewline
				\noalign{\hrule height 2pt}
				minna eða jafnt x& x & stærra en x  & ??? 
				\tabularnewline \noalign{\hrule height 1.5pt}
			\end{tabular}
		\end{center}
	%\end{table}
\caption{Hér sést hvernig staðan er þegar á að finna stað fyrir eitthvað x, sem gæti verið tala eða spil eða einhver annar hlutur. Athugið að annað hvort eða bæði hólfin gætu verið tóm.}
\label{tbl:insert-abstract}
\end{table}

Skoðum nú kóðann á bakvið þetta reiknirit.

\begin{lstlisting}[caption=Insertion sort reikniritið, label=lst:reiknirit-insertion]
def insertion_sort(A):
	#raðar í vaxandi röð
	i = 1
	while i < len(A):
		j = i
		while j > 0 and A[j-1] > A[j]:
			#skiptum stökunum ef þau eru ekki rétt röðuð
			aj = A[j]
			ajminus = A[j-1]
			A[j] = ajminus
			A[j-1] = aj
			j = j - 1
		i = i + 1
\end{lstlisting}

Nokkrar hugmyndir til að ná upp leikni í notkun á reikniritum er að:
\begin{itemize}
	\item Reyna að nota reikniritið fyrir einhvern lista af tölum.
	\item Reyna að finna leið til þess að gera línur 8-11 í einni línu, Python býður upp á það.
	\item Reyna að láta reikniritið raða í minnkandi röð.
\end{itemize}

Í kóðabút \ref{lst:reiknirit-bubble} sjáum við útfærslu á bubble sort.
Útfærslan felst í tveimur hreiðruðum for-lykkjum\footnote{sem er yfirleitt ekki góðs viti þegar kemur að tímaflækju, enda er hægt að gera ráð fyrir að tíminn sem það tekur að keyra bubble sort sé $n^2$, sem segir kannski ekkert fyrir óþjálfað auga en hægt er að treysta því að það er ekki ákjósanlegt.}.

Reikniritið virkar í grunninn þannig að það tekur við lista sem á að raða, það rúllar í gegnum listann frá upphafi og út í enda og ýtir stærsta stakinu út í enda.
Eins og loftbólur sem þrýstast upp á yfirborðið.
Þegar það er búið að rúlla einu sinni í gegnum listann er stærsta stakið komið út í enda og það stak er ekki skoðað aftur heldur álitið á sínum stað.
Þá er aftur rúllað í gegnum listann og stakið sem er þá stærst fer út í enda vinstra megin við stakið sem var stærst.

Þannig að ytri for lykkjan keyrir fyrir hvert stak í listanum, eða segir til um hversu oft þurfi að finna stærsta stakið, og innri lykkjan sér um samanburðinn og skiptingarnar.
Tökum sérstakleg eftir þar að við getum horft á næsta stak hægra megin þegar við erum að bera saman og það er vegna þess að við hættum fyrir framan aftasta stakið hverju sinni í innri lykkjunni.
%Ef við myndum bara beita breytunni \texttt{vinstri\_hlid} sem \texttt{n-i} þá myndum við fá vísisvillu því við værum að vísa út fyrir listann okkar en vegna þess að við lækkum okkur um 1 þar að auki þá er möguleiki að skoða \texttt{listi[j+1]} sem er næsta stak hægra megin við það stak sem við erum stödd í \texttt{listi[j]}.

\begin{lstlisting}[caption=Bubble sort reikniritið, label=lst:reiknirit-bubble]
	def bubblesort(listi):
	# n er þá fjöldi staka í listanum
	n = len(listi)
	
	# Förum í gegnum öll stökin
	for i in range(n):
		# vinstri hliðin er óröðuð 
		# í fyrstu ítrun er i 0 og vinstri_hlid er því jöfn n-1
		# sem er síðasta stakið í listanum
		# svo verður vinstri_hlid alltaf minni og minni
		# Því síðustu i stökin eru komin á sinn stað
		vinstri_hlid = n-i-1
		
		for j in range(0, vinstri_hlid):
			vinstra = listi[j]
			haegra = listi[j+1]
			# Hér förum við frá 0 upp í n-i-1
			# af því að viljum byrja úti í vinstri enda 
			# og við viljum geta skoðað næsta stak fyrir aftan
			# 
			# Svo skiptum við á stakinu við stakið hægra megin
			# ef stakið er stærra en það sem er hægra megin
			if vinstra > haegra:
			listi[j] = haegra
			listi[j+1] = vinstra
		
	return listi
\end{lstlisting}



\subsection{Helmingunarleit}\index{Helmingunarleit}\label{uk:reiknirit-helmingunarleit}
Nú þegar við kunnum að raða þá getum við farið að leita.
Skoðum nú eitthvert þekktasta reiknirit sem til er. 
Helmingunarleit að tölu á bili. 
Hugsum okkur að við séum með raðaðan lista af tölum og við viljum finna eina tiltekna tölu. 
Ef við ættum að skoða hverja einustu tölu í listanum til að finna hana þá tæki það mjög langan tíma.
Eða allavega fyrir okkur sem manneskjur, en allur tímasparnaður er góður.
Því að aðgerðin ,,að skoða spil'' kostar einhvern tíma og því færri þannig aðgerðir sem við getum gert því hraðara er reikniritið okkar\footnote{Tölvunarfræðingar eru mjög uppteknir af því hvað aðgerðir og reikningar taka mikinn tíma.
	Þetta er kallað tímaflækja (e. time complexity) og er tölvunarfræðingum mjög hugleikin.
	Tímaflækja helmingunarleitar er sérstaklega lág eða $log_2{n}$ vegna þess að reikniritið helmingar alltaf vandamálið (e. problem space).}.

Hugmyndin er sú, að verið sé að skoða í fyrstu allan listann sem er raðaður (mjög mikilvægt, annars virkar þetta engan veginn) og miðju gildið er skoðað, ef gildið sem við leitum að er stærra en miðjugildið þá eigum við að vera að leita þeim megin við miðjugildið sem stærri gildi eru.
Svo gerum við þetta endurtekið, helmingum alltaf vandamálið þar til við höfum annað hvort fundið gildið sem við leitum að eða bilið sem við erum að leita á inniheldur engin stök.


\begin{lstlisting}[caption=Helmingunarleit að tölu í röðuðum lista með lykkju, label=lst:reiknirit-helm-for]
	def helmingunarleit_med_lykkju(listi, x):
	# Þetta fall tekur við lista sem er raðaður í vaxandi röð
	# og gildi sem á að leita að í listanum
	# Þetta er gert með lykkju og án endurkvæmni
	
	# Fallið skilar sætisvísinum sem gildið er í eða -1 ef gildið er ekki í listanum
	
	# minnsti og hæsti vísirinn í listanum
	minnsti = 0
	haesti = len(listi)-1
	
	while minnsti <= haesti:
		# Þetta keyrist á meðan minnsti er enn minni eða jafn og haesti, það er við erum enn með bil til að leita á
	
		midjan = int((minnsti + haesti)/2)
		if listi[midjan] == x:
			# við fundum gildið og skilum vísinum sem það er í
			return midjan
		if listi[midjan] > x:
			haesti = midjan - 1
		else:
			minnsti = midjan + 1
	
	# lykkjan hætti keyrslu svo minnsti vísirinn er orðinn hærri en sá hæsti og þá vitum við að talan er ekki í listanum 
	# við skilum því tölunni -1 til að segja að það var enginn vísir sem x var í
	return -1
\end{lstlisting}

\subsection{Endurkvæmni}\index{Endurkvæmni}\label{uk:reiknirit-endurkvæmni}
Endurkvæmni (e. recursion) er sú virkni forrits að vísa í sig sjálft.
Þið þekkið eflaust listaverk sem virka eins og skynvillur, þar sem manneskja getur labbað í hring upp stiga en endað á sama stað því stiginn fer í raun í hring.
Eða þið hafið séð ykkur sjálf í spegli þar sem var annar spegill fyrir aftan og þið sáuð ótal spegilmyndir raðast af ykkur.

Gott dæmi um hvernig megi beita endurkvæmni til að fá skilmerkilega niðurstöðu er að útfæra fall sem reiknar fyrir okkur einhverja gildi í fibonacci röðinni.
En áður en við gerum það skulum við skoða enn einfaldara dæmi þar sem við erum með fall sem kallar í sig sjálft og gerir ekkert annað en það.
Í kóðabút \ref{lst:reiknirit-endurkvæmni1} sjáum við einfalda útgáfu af endurkvæmni, þar sem hugmyndin er í raun kynnt án þess að fallið sé neitt gagnlegt.
Það eina sem fallið gerir er að athuga hvort að tala sé stærri en núll og ef hún er það þá kallar fallið í sig sjálft með gildi sem er einum lægra, annars ef talan er ekki stærri en núll prentast ,,við kunnum á endurkvæmni''.
Hveru oft ætli það prentist ef við setjum inn töluna 5?

\begin{lstlisting}[caption=Endurkvæmni - einfalt, label=lst:reiknirit-endurkvæmni1]
def endurkvæmt_fall(tala):
	if(tala > 0):
		# á meðan talan er hærri en 0 þá köllum við aftur í fallið
		# en við köllum í það af gildi einu lægra
		return endurkvæmt_fall(tala-1)
	else:
		# hér erum við komin niður í 0 og prentum eftirfarandi texta
		print('við kunnum endurkvæmni')
		
endurkvæmt_fall(5)
\end{lstlisting}
\lstset{style=uttak}
\begin{lstlisting}
við kunnum endurkvæmni
\end{lstlisting}
\lstset{style=venjulegt}

Við sjáum í kóðabút \ref{lst:reiknirit-endurkvæmni1} að þó að við kölluðum í fallið með tölunni 5 þá fengum við bara einu sinni út strenginn ,,við kunnum endurkvæmni''.
Það er vegna þess að við kölluðum í fallið fyrir fimm og það sem fallið gerir fyrir okkur er að klára endurkvæmnina fyrir það kall, það verða ekki til fjögur önnur köll.
Heldur verður fimm að fjórum sem skilar okkur svo niðurstöðunni fyrir þrjá og svo koll af kolli þar til við erum komin niður í núll og þá hættir reikniritið keyrslu.
Í þessu tilfelli skilar það engu til baka upp fallakallið en klárast engu að síður þarna í línu 8 þegar það kemst þangað.

Það sem þetta reiknirit okkar gerir ekki er að skila einhverri niðurstöðu til baka, en nú skulum við skoða tvö þannig reiknirit sem eru endurkvæm í kóðabút \ref{lst:reiknirit-endurkvæmni2}.
Annað þeirra reiknar summu af einhverri tölu og öllum jákvæðum tölum lægri en henni, svo talan 5 gæfi útkomuna 5 + 4 + 3 + 2 + 1 sem er 15 og fyrir 100 væri það 100 + 99 + ... + 1 sem gæfi 5050.
Hitt reiknar n-tu töluna í fibonacci rununni.
Endurkvæmnin er því fólgin í því að upphaflega fallakallið með summa(100) skilar okkur 100 + summa(99), sem skilar okkur 100 + 99 + summa(98) og svo koll af kolli þar til talan 1 fæst, og við fáum útreikninginn 100 + 99 + ... + 1 sem skilar okkur 5050.

Endurkvæmnin í fib(4) er fólgin í því að skoða alltaf tvær summur í einu en þær skila sér í sama fallakallið.
Svo fib(4) skila því sem fib(2) + fib(3) skila, þau köll skila okkur annars vegar fib(0) + fib(1) og hins vegar fib(1) + fib(2).
Fyrri hlutinn sem fékkst úr fib(2), sem varð að fib(0) + fib(1) skilar sér sem 0 + 1 = 1.
Og seinni hlutinn er fib(1) + fib(2) og fib(1) skilar 1 og við vitum að fib(2) mun skila 1 svo þar stendur 1 + 1 = 2 svo við fáum upp til baka að fib(4) = 1 + 2 = 3.

\begin{lstlisting}[caption=Summa reiknuð endurkvæmt, label=lst:reiknirit-endurkvæmni-summa]
def summa(n):
	if n <= 1:
		return 1
	else:
		return n + summa(n-1)

summa(6)
\end{lstlisting}
\lstset{style=uttak}
\begin{lstlisting}
21
\end{lstlisting}
\lstset{style=venjulegt}

\begin{lstlisting}[caption=Fibonacci tölur reiknaðar endurkvæmt, label=lst:reiknirit-endurkvæmni-fib]
def fib(n):
	if n <= 1:
		return n
	else:
		return fib(n-2) + fib(n-1)

fib(4)
\end{lstlisting}
\lstset{style=uttak}
\begin{lstlisting}
3
\end{lstlisting}
\lstset{style=venjulegt}


Nú höfum við séð lykkju útfærslu á helmingunarleit sem keyrir á meðan við höfum bil til að leita á en nú skulum við skoða hvernig megi gera þetta endurkvæmt.

\begin{lstlisting}[caption=Helmingunarleit að tölu í röðuðum lista með endurkvæmni, label=lst:reiknirit-helm-end]
def helmingunarleit_med_endurkvaemni(listi, vinstri, haegri, x): 
	# Endurkvæmt fall sem skilar sætisvísinum sem x finnst í í listanum listi
	# Ef x er ekki að finna í listanum listi þá fæst -1
	
	# Athugum hvort að vinstri sé enn vinstra megin
	if haegri >= vinstri: 
	
		# Stillum miðjuna
		midjan = int((vinstri + haegri)/2)
		#midjan = math.floor(vinstri + (haegri - vinstri)/2)
		
		# Athugum hvort gildið sé í miðjustakinu
		if listi[midjan] == x: 
			return midjan 
		
		# Ef gildið er lægra en miðjan þá þurfum við að leita frá vinstri að miðju
		elif listi[midjan] > x: 
			return helmingunarleit_med_endurkvaemni(listi, vinstri, midjan-1, x) 
	
		# Annars hlýtur gildið að vera frá miðju að hægra gildi
		else: 
			return helmingunarleit_med_endurkvaemni(listi, midjan+1, haegri, x) 
	
	else: 
		# Nú er hægra orðið minna en vinstra svo að
		# við erum búin að leita af okkur allan grun
		# gildið er ekki í listanum
		return -1
\end{lstlisting}

%\subsection{Bubble sort}\index{Bubble sort}\label{uk:reiknirit-bubble}
%Nú þegar við höfum leyst það hvernig á að leita að tölu á bili ef bilið er raðað, þá er mikilvægt að skoða hvernig er eiginlega hægt að raða?

%Við höfum séð að listar, innbyggða gagnatagið í Python, hefur innbyggðu aðferðina sort().
%Það sem sort gerði var að breyta lista fyrir okkur þannig að hann væri raðaður í vaxandi röð.

%Nú viljum við hinsvegar átta okkur á því hvernig við getum útfært reiknirit sem raðar fyrir okkur stökum í lista.


%-------------------------------
\newpage
\section{Æfingar}
\begin{exercise}\label{rei1}
Búið til fall sem tekur við tveimur hnitum, x og y gildum.
Það sem fallið á að gera er að skila tölu byggt á því hversu margar hreyfingar hrókur, taflmaðurinn, þyrfti að fara til að komast úr einu hniti í annað.
Í boði eru þrír möguleikar, tvær, ein eða engin hreyfing.

Gerum ráð fyrir að öll möguleg rauntöluhnit séu lögleg fyrir þennan ofurhrók.
\end{exercise}
\setboolean{firstanswerofthechapter}{true}
\begin{Answer}[ref={rei1}]
	\begin{lstlisting}
def hnit(hni11, hnit2):
	# gert er ráð fyrir að hnit sé annað hvort nd eða listi
	if(hnit1 == hnit2):
		return 0
	if(hnit1[0] == hnit2[0] or hnit1[1] == hnit2[1]):
		return 1
	else:
		return 2\end{lstlisting}
\end{Answer}
\setboolean{firstanswerofthechapter}{false}

\begin{exercise}\label{rei2}
Við viljum geta sagt til um hvaða leið er styst og hver er lengst, gefið er ákveðin lestarleið í formi lista af tölum.
Tölurnar tákna hversu langt stoppin eru frá upphafstöðinni.
Þið eigið annars vegar að búa til fall sem tekur við lista og skilar stystu fjarlægð milli tveggja nærliggjandi stöðva og hins vegar eigið þið að búa til fall sem tekur við sambærilegum lista en skilar lengstu fjarlægð milli tveggja nærliggjandi stöðva.

Athugið að fyrir listann \texttt{[1,3,7,19]} væri svarið 2 fyrir fyrra fallið en 12 fyrir seinna fallið.
Athugið hér sérstaklega bubblesort og hvernig á að skoða tvö stök í einu.
\end{exercise}
\begin{Answer}[ref={rei2}]
Hér er mikið atriði að prófa allskonar mismunandi lista til að vera viss um að þið séuð með kórréttan kóða.
	\begin{lstlisting}
def minnsta_fjarl(listi):
	styst = abs(listi[1]-listi[0])
	for i in range(len(listi)-1):
		if(styst > abs(listi[i+1]-listi[i])):
			styst = abs(listi[i+1]-listi[i])
	return styst

def mest_fjarl(listi):
	mest = abs(listi[1]-listi[0])
	for i in range(len(listi)-1):
		if(mest < abs(listi[i+1]-listi[i])):
			mest = abs(listi[i+1]-listi[i])
	return mest

\end{lstlisting}
\end{Answer}