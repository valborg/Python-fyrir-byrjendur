\chapterimage{chapter_head_2.pdf} % Chapter heading image

\chapter{Listar}\index{Listar}\label{k:listar}
Listar eru gagnagrindur, sem þýðir að þeir geta geymt fyrir okkur hin ýmsu gögn og gert okkur þau aðgengileg á ákveðinn máta.
Listar eru skilgreindir með hornklofum [ ] og er lykilorðið þeirra \textbf{list}.

\section{Listar skilgreindir}\index{Listar skilgreindir}\label{uk:listar-skilgreindir}
Listar geyma, í ákveðinni röð, þau gögn sem við viljum geyma sem mega vera af hvaða týpu sem er.
Gögnin sem eru sett inn í listann eru kölluð stök og röðin sem þau eru í eru aðgengileg eftir vísum eða sætisnúmerum alveg eins og strengir.
Stökin eru aðgreind með kommum.
Þær týpur sem við höfum séð hingað til eru heiltölur, fleytitölur, strengir og listar.
Allt eru þetta möguleg stök í lista.

\begin{lstlisting}[caption=Listar skilgreindir, label=lst:listar-skilgreindir]
# Fyrsti listinn okkar er tómur
listinn_minn = []

# þegar við skilgreinum lista aðgreinum við stökin með kommum
nyr_listi = ["núllta stakið", 1, 2, 3.0, "fjórða stakið", [5]]
\end{lstlisting}
