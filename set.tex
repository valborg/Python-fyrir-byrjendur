\chapterimage{chapter_head_2.pdf} % Chapter heading image

\chapter{Mengi}\index{Orðabækur}\label{k:sett}
Mengi (eða sett) eru týpa sem geymir óraðað safn af gögnum án tvítekninga, þau eru ein af fjórum innbyggðum gagnagrindum í Python (listar, ndir, orðabækur eru hinar) og geta þau geymt gögn af hvaða týpu sem er.
Lykilorðið þeirra er set.
Mengi eru skilgreind með slaufusvigum og eru stök þeirra aðgreind með kommum, ólíkt orðabókum þá eru engin lykla og gildispör sem hanga saman með tvípunkti og því ruglast vélinn ekki á þessum tveimur týpum.
Eins og orðabækur eru óraðaðar, þá er ekki hægt að nota vísa til þess að segja hvar eitthvað stak er í mengi.

Mengi þessi eru eins og mengi sem við könnumst við í stærðfræði, þar sem hvert stak kemur þó aðeins fyrir einu sinni.
Við getum framkvæmt ýmsar stærðfræðilegar aðgerðir á þau ásamt hefðbundnum aðgerðum til að bæta við eða fjarlægja stök, hins vegar er ekki hægt að breyta staki sem er nú þegar komið í mengið.

Skoðum kóðabút \ref{lst:set-kynnt} til þess að sjá hvernig mengi eru skilgreind og hvernig megi nota lykilorðið til að búa til mengi fyrir okkur úr gögnum.

\begin{lstlisting}[caption=Mengi skilgreind, label=lst:set-kynnt]
# Fyrsta mengið okkar inniheldur nokkrar tölur
mengid_mitt = {1,2,3,4}
print(mengid_mitt)
# úttakið verður 
# {1, 2, 3, 4}

# en til þess að búa til tómt mengi þarf að nota lykilorðið
tomt_mengi = set()

# því að þetta er tóm orðabók:
ekki_mengi = {}
\end{lstlisting}

\section{Tvítekning}\index{Tvítekning}
Tvítekning í mengjum er ekki leyfileg og því ágætt að nota mengi til þess að fjarlægja tvítekningar úr gögnunum okkar.
Ef við tökum fyrir orðið 'halló' og gerum mengi úr því með set('halló') þá fengjum við mengi sem innihéldi 'h', 'a', 'l', og 'ó'.
Stafurinn 'l' kemur tvisvar fyrir í strengnum en hann kemur einu sinni fyrir í menginu af strengnum.
Sjáum kóðabút \ref{lst:set-duplicate} hvernig við fáum ekki út tvítekningar sama hvernig við reynum.
Takið eftir í línu 10 þar sem stafirnir koma í einhverri röð, sú röð er ekki heilög þar sem þetta er óraðað gagnatag og þessi röðun verður ekki endilega eins við aðra keyrslu.

\begin{lstlisting}[caption=Mengi skilgreind, label=lst:set-duplicate]
# Skilgreinum mengi með endurtekningum
mengid_mitt = {1,2,3,4, 1, 2, 3, 4}
print(mengid_mitt)
# úttakið verður 
# {1, 2, 3, 4}

# notum lykilorðið til að búa til mengi úr streng
print(set("Valborg Sturludóttir vinsamlegast"))
# úttakið verður
# {'b', 'i', 'ó', 'n', 'l', 'a', 'g', 'd', 'o', 'V', ' ', 'e', 'S', 'r', 's', 'u', 't', 'm', 'v'}
\end{lstlisting}

\section{Aðgerðir}\index{Aðgerðir}
Aðgerðir sem hægt er að gera á set er að bæta við staki, \textbf{add()}, fjarlægja stak, \textbf{remove()} og uppfæra mengið með mörgum stökum, \textbf{update()}.
Engin þessara aðferða gerir okkur kleyft að eiga tvö eins stök í menginu.
Tvítekning er ekki liðin, sama hvernig við reynum að komast framhjá henni.

Stærðfræðilegar aðgerðir sem hægt er að gera með mengi er að finna sniðmengi eða sammengi tveggja mengja, eða mengi sem inniheldur einungis stök sem eru ekki í báðum mengjunum sem verið er að sameina. Allt þrennt af þessu er hægt að gera með innbyggðum föllum sem taka við tveimur mengjum og skila einu mengi til baka eða aðferðum á mengi sem breyta þá menginu sem aðferðin var kölluð á með tilliti til mengisins í viðfanginu.

Þetta reynist vel við að vinna með gögn eins og símaskrár eða tölvupóstföng því við viljum ekki tvítekningar og þegar á að sameina símaskrár eða tölvupóstföng með ákveðnum reglum er gott að vita að hægt sé að beita þessari týpu.