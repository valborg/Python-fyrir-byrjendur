%%%%%%%%%%%%%%%%%%%%%%%%%%%%%%%%%%%%%%%%%
% The Legrand Orange Book
% LaTeX Template
% Version 2.4 (26/09/2018)
%
% This template was downloaded from:
% http://www.LaTeXTemplates.com
%
% Original author:
% Mathias Legrand (legrand.mathias@gmail.com) with modifications by:
% Vel (vel@latextemplates.com)
%
% License:
% CC BY-NC-SA 3.0 (http://creativecommons.org/licenses/by-nc-sa/3.0/)
%
% Compiling this template:
% This template uses biber for its bibliography and makeindex for its index.
% When you first open the template, compile it from the command line with the 
% commands below to make sure your LaTeX distribution is configured correctly:
%
% 1) pdflatex main
% 2) makeindex main.idx -s StyleInd.ist
% 3) biber main
% 4) pdflatex main x 2
%
% After this, when you wish to update the bibliography/index use the appropriate
% command above and make sure to compile with pdflatex several times 
% afterwards to propagate your changes to the document.
%
% This template also uses a number of packages which may need to be
% updated to the newest versions for the template to compile. It is strongly
% recommended you update your LaTeX distribution if you have any
% compilation errors.
%
% Important note:
% Chapter heading images should have a 2:1 width:height ratio,
% e.g. 920px width and 460px height.
%
%%%%%%%%%%%%%%%%%%%%%%%%%%%%%%%%%%%%%%%%%

%----------------------------------------------------------------------------------------
%	PACKAGES AND OTHER DOCUMENT CONFIGURATIONS
%----------------------------------------------------------------------------------------

\documentclass[11pt,fleqn]{book} % Default font size and left-justified equations

\input{structure.tex} % Insert the commands.tex file which contains the majority of the structure behind the template

%\hypersetup{pdftitle={Title},pdfauthor={Author}} % Uncomment and fill out to include PDF metadata for the author and title of the book

%----------------------------------------------------------------------------------------

\begin{document}

%----------------------------------------------------------------------------------------
%	TITLE PAGE
%----------------------------------------------------------------------------------------

\begingroup
\thispagestyle{empty} % Suppress headers and footers on the title page
\begin{tikzpicture}[remember picture,overlay]
\node[inner sep=0pt] (background) at (current page.center) {\includegraphics[width=\paperwidth]{background.pdf}};
\draw (current page.center) node [fill=ocre!30!white,fill opacity=0.6,text opacity=1,inner sep=1cm]{\Huge\centering\bfseries\sffamily\parbox[c][][t]{\paperwidth}{\centering Inngangur að forritun\\[15pt] % Book title
{\Large Python fyrir byrjendur}\\[20pt] % Subtitle
{\huge Valborg Sturludóttir}}}; % Author name
\end{tikzpicture}
\vfill
\endgroup

%----------------------------------------------------------------------------------------
%	COPYRIGHT PAGE
%----------------------------------------------------------------------------------------

\newpage
~\vfill
\thispagestyle{empty}

\noindent Copyright \copyright\ 2020 Valborg Sturludóttir\\ % Copyright notice

% \noindent \textsc{Published by Publisher}\\ % Publisher

% \noindent \textsc{book-website.com}\\ % URL

\noindent Licensed under the Creative Commons Attribution-NonCommercial 3.0 Unported License (the ``License''). You may not use this file except in compliance with the License. You may obtain a copy of the License at \url{http://creativecommons.org/licenses/by-nc/3.0}. Unless required by applicable law or agreed to in writing, software distributed under the License is distributed on an \textsc{``as is'' basis, without warranties or conditions of any kind}, either express or implied. See the License for the specific language governing permissions and limitations under the License.\\ % License information, replace this with your own license (if any)

\noindent \textit{First printing, Sept 2020} % Printing/edition date

%----------------------------------------------------------------------------------------
%	TABLE OF CONTENTS
%----------------------------------------------------------------------------------------

%\usechapterimagefalse % If you don't want to include a chapter image, use this to toggle images off - it can be enabled later with \usechapterimagetrue

\chapterimage{chapter_head_1.pdf} % Table of contents heading image

\pagestyle{empty} % Disable headers and footers for the following pages

\tableofcontents % Print the table of contents itself

\cleardoublepage % Forces the first chapter to start on an odd page so it's on the right side of the book

\pagestyle{fancy} % Enable headers and footers again

%----------------------------------------------------------------------------------------
%	PART
%----------------------------------------------------------------------------------------

\part{Fyrri hluti-\\ Grunnurinn}\index{Fyrri hluti test}
\todo{laga gæsalappir og vspace staðla}
\chapterimage{chapter_head_2.pdf} % Chapter heading image

\chapter{Inngangur}

\section{Tilgangur bókarinnar}\index{Tilgangur Python}

Þessi bók fjallar um þau undirstöðu atriði sem þarf að kynna til að ná tökum á forritun í Python. 
Höfundi finnst mikilvægt að kenna námsefnið með íslenskum hugtökum þar sem ætlunin er að nota hana í kennslu í íslenskum framhaldsskólum. 
Ef nemendur ætla að leggja fyrir sig tölvunarfræði í framhaldssnámi er nauðsynlegt að búa yfir ríkulegu íðorðasafni, þess þá heldur ef nemandi hyggst framfleyta fræðunum. 
Hugtök verða þó líka sett fram á ensku því lesandi gæti óskað að fletta upp íterefni sem meira er til af á netinu á ensku en íslensku.

Það er algengur misskilningur að forritarar kunni rosalega mörg forritunarmál, eins og fólk sem getur talað mörg tungumál, eða að það að kunna rosalega mörg mál geri þig að góðum forritara.
Þvert á móti.
Að sýna hæfni og leikni í einu máli er auðveldlega yfirfæranlegt á önnur mál sé þess þörf.
Þess vegna er spurningin ,,hvað kanntu mörg forritunarmál?'' út í hött.
Ekki aðeins eru tungumál og forritunarmál gerólík, forritunarmál eru formleg mál og þekking á einu hlutbundnu máli er nær því að vera jafn frábrugðið öðru í grunninn eins og málýskur innan tungumála.
Nær væri að spyrja hvort viðkomandi hafi meiri áhuga á framenda eða bakenda forritun, hvað er skemmtilegasta reikniritið sem viðkomandi hefur útfært eða hvert er það forritunarmál sem viðkomandi grípur oftast í.

Einnig er það algengur misskilningur að það fyrsta sem fólk gerir er að búa til tölvuleik.
Það þarf mikla undirstöðu kunnáttu til þess að geta búið til tölvuleiki, alveg eins og áður en hafist er handa við að skrifa bók þarf að læra stafrófið.
Þessi grunnvinna finnst mörgum vera leiðigjörn.
Að mati höfundar er það vegna þess að við erum svo vön því að nota tölvur dagsdaglega, svo fræðigreinin sem tæknin byggir á hlýtur líka að vera okkur kunnug ekki satt?
Nei, alveg eins og dýralækningar eru okkur ekki augljósar við að eiga gæludýr og pípulagnir heldur ekki við að eiga klósett.
Innan tölvunar eru ákveðnar grunneiningar sem eru notandanum ekki augljósar, af góðri ástæðu, það væri hrikalegt ef við þyrftum öll að vera píparar til þess að geta notað klósett.
Þó þessi samlíking hafi verið heldur gróf þá sýnir hún að það eru svo margir hlutar sem eru okkur huldir að við hreinlega vitum ekki hvað við vitum ekki.
Því er nauðsynlegt að læra grunninn vel og fara rólega yfir hann svo þegar við ætlum að fara að afrita og líma kóða frá síðum eins og stackoverflow þá vitum við allavega hvað sá kóði gerir (nokkurn veginn).

Uppbyggining er þannig að fyrri hlutinn snýr að því að kynna lesandann fyrir grunn virkni Python; málskipan, lykilhugtök og lykilorð, gagnatýpur, lykkjur og föll. 
Seinni hlutinn snýr svo að því að beita þekkingu úr fyrri hlutanum í hlutbundinni forritun. 
Þar eru kynntir til sögunnar klasar og aðferðir sem lesandinn útfærir upp á eigin spýtur. 
Ekki er búist við neinni fyrri kunnáttu við lestur þessarar bókar, hún á að geta staðið fyrir sínu án þess að lesandinn búi yfir nokkurri þekkingu á sviði tölvunarfræða eða forritunar. 
Ef slík þekking er fyrir hendi gæti lesandanum þótt ágætt að fara hratt í gegnum fyrri hluta bókarinnar og einbeita sér að verkefnum úr seinni hlutanum. 
Í gegnum bókin fylgjum við svo þremur verkefnum sem verða þyngri og flóknari eftir því sem fleiri hugtök eru kynnt til sögunnar. \todo{þremur??}

\section{Hvers vegna Python?}\index{Hvers vegna Python?}

Ástæður þess að Python er gott mál til þess að byrja á að skoða eru eftirfarandi: \footnote{Strax í þessum texta koma fyrir hugtök sem verða skýrð betur seinna, ekki missa kjarkinn.}.

\begin{enumerate}
	\item \textit{Málskipanin} er mjög svipuð mannlegu máli svo það er auðvelt að læra hvernig eigi að ,,tala'' við tölvuna.
	\item  Python er \textit{kvikt tagað} forritunarmál, það þýðir að notandinn þarf ekki að gefa upp hvers konar \textit{gagnatýpur} er unnið með. 
	Þetta gerir það að verkum að notandinn þarf ekki að læra urmull af lykilorðum áður en byrjað er að forrita.
	\item  Python er ekki alveg \textit{hlutbundið} forritunarmál, sem gerir það að verkum að notandinn þarf ekki að læra hvernig á að beita hlutbundinni forritun fyrr en góð undirstaða er þegar komin.
	\item Python er frítt og aðgengilegt öllum helstu stýrikerfum og einnig er hægt að forrita yfir netið í vafra og því óþarfi fyrir notandann að setja nokkuð upp sé þess óskað.
	\item Python er mikið notað, algengt mál svo það er praktískt að hafa undirstöðu skilning á því.
	\item Nefnt í höfuð á Monty Pyhton grínhópsins.
\end{enumerate}

\section{Uppsetning}\index{Uppsetning}
\todo{lýsa hvernig bókin er byggð upp}.

Víðsvegar um bókina, aðallega í upphafi \todo{er það?}, má finna númeraða kóðabúta sem eru ekki teknir úr vinnubók og eru því ekki eins litakóðaðir og þeir sem eru teknir úr vinnubókum.
Ástæðan fyrir því er að þessum kóðabútum er auðveldara að viðhalda heldur en skjáskotum úr vinnubókum og því er heldur vísað í bækur sem eru aðgengilegar lesendum og frumstæðari framsetning ræður heldur ríkjum hér.

\section{Að keyra kóða}\index{Að keyra kóða}\label{uk:keyra-koda}

Það fyrsta sem nemendur vilja yfirleitt gera er að byrja að skrifa sinn eigin kóða. 
Áður en við komumst svo langt þarf að útskýra hvernig það er gert. 
Þessi kennslubók byggir á notkun Jupyter Notebooks með hjálp Anaconda hugbúnaðarins, sem er öflugt pakkakerfi og tólakista sem hefur upp á mikið meira en bara Jupyter að bjóða. 
Hægt er að nálgast Anaconda á \href{www.anaconda.com}{anaconda.com}.
Hægt er að nota Jupyter án þess að ná í Anaconda með síðum eins og \href{www.cocalc.com}{cocalc.com}. 
Einnig er hægt að keyra kóða á netinu í gegnum síður eins og \href{www.repl.it}{repl.it}, nota ritla (eins og notepad eða sublime) til að keyra .py skrár í skipanalínu, eða nota þyngri umhverfi eins og pycharm sem eru sérhönnuð fyrir hugbúnaðarþróun. 
Hér er gert ráð fyrir Juptyter umhverfinu og verður bókin öll miðuð að því.


Þessari bók fylgja einnig nokkrar vinnubækur úr Jupyter sem lesandinn getur nýtt sér. 
Hér á mynd \todo{mynd af tómri bók} sést hvernig tóm Jupyter vinnubók lítur út. 
\todo{útskýra hvernig hún virkar?} 
Virkninni er skipt upp í sellur og keyrsluröð sellanna skiptir máli, við sjáum seinna mikilvægi þess að geta skipt upp kóða svona og hvers vegna þetta umhverfi er þægilegt til að byrja í. 
En hver sella hefur aðgang að svokölluðu skilgreiningarsvæði vinnubókarinnar en er þó sín eigin eining, því má keyra eina sellu í einu án þess að keyra allan kóðann í vinnubókinni.

Hér væri réttast að skoða Vinnubók 1 sem fylgdi þessari bók. 
\todo{hér væri gott að taka fyrir dæmi úr vinnubók sem á að fylgja}
\todo{segja miklu meira um hvernig á að nota vinnubækur yfirhöfuð}

En við keyrslu á kóða þarf einnig að hafa í huga að tölvan gerir nákvæmlega það sem við segjum henni að gera og ekker annað.
Og þá komum við niður á stórt vandamál, að tölvur eru mjög bókstaflegar og vitlausar.
Þær skortir allt vit, þær reyna ekki að hafa vit fyrir þér. 
Þær gera nákvæmlega það sem þú baðst um.
Nákvæmlega eins og þú baðst um það.

Þannig að ef ég ætlaði að segja tölvu að smyrja handa mér hnetusmjörs og sultu samloku þá þyrfti ég að segja vélinni að gera eftirfarandi í nákvæmlega þessari röð:
\vspace{0.4cm}
\begin{enumerate}
	\item taka fram hníf
	\item taka fram tvær brauðsneiðar
	\item opna hnetusmjörið
	\item setja beitta endann ofan í hnetusmjörið þannig að hann nái upp 50gr af hnetusmjöri
	\item setja hnetusmjörið sem er á hnífnum á miðja brauðsneiðina
	\item nota hnífinn til þess að smyrja hnetusmjörinu á þá hlið sem hnetusmjörið er nú þegar á, og enga aðra
	\item taka fram skeið
	\item opna sultuna
	\item setja kúpta enda skeiðarinnar ofan í sultukrukkuna 
	\item taka skeiðina upp úr sultukrukkunni með kúfaða skeið af sultu
	\item setja sultuna á hina brauðsneiðina
	\item nota skeiðina til að smyrja sultunni yfir þá hlið brauðsneiðarinnar sem sultan er á og enga aðra hlið
	\item  setja brauðsneiðarnar saman þannig að hnetusmjörið og sultan snertist og hornin mætast öll. 
\end{enumerate} 
\vspace{0.4cm}
Takið eftir að hér er gert ráð fyrir þó nokkur og ef véliin kann ekki nú þegar skil á: 
\vspace{0.2cm}
\begin{enumerate}
	\item taka fram
	\item hnífur
	\item opna
	\item mæla 50 gr
	\item smyrja
	\item hlið á brauðsneið
	\item miðja á brauðsneið
	\item skeið
	\item kúfað
\end{enumerate} 
\vspace{0.2cm}
Svo þó svo að þér hafi þótt þessi útskýring á samlokugerð alveg ofboðslega óþarflega nákvæm þá er ekki víst að úr þessu verði nokkur samloka.
Þetta könnumst við öll við, að tölvur gera það sem þeim er sagt, ekki það sem við viljum.

Helsta verkefni forritara er að búta niður verkefni í svo litla hluta að hægt er að útskýra þá fyrir tölvu.
Ekki búast við því að setjast niður við fyrsta verkefni og ætlast svo til að búa til tölvuleik eða hakka banka.
Forritun er einnig frábrugðin þeirri venjulegu tölvunotkun sem þú hefur vanist dagsdaglega.
Þar ertu ekki að gefa tölvunni þínar eigin skipanir heldur ertu að beita skipunum sem aðrir forritara hafa samið og sett upp í hugbúnaðinn sem þú ert að nota.

%------------------------------------------------

\section{Málskipan}\index{Málskipan}

Málskipan (e. syntax) er hugtak sem þýðir hvernig á að skrifa kóða svo að hann þýðist í vélamál sem tölvan skilur. 
Málskipan eru þær reglur sem við þurfum að fara eftir þegar við forritum, þær reglur sem forritunarmálið býst við að við förum eftir. 
Ef við brjótum þessar reglur fáum við villu, og einhver algengasta villa sem hægt er að fá er málskipunarvilla (e. syntax error). 
Python er frábrugðið öðrum forritunarmálum á þann hátt að málskipanin krefst þess að kóðinn sé settur upp á ákveðinn hátt. 
Líkja því má við að þurfa ekki að hafa greinamerki í huga þegar við ljúkum setningum heldur setjum við setningarnar okkar á réttan stað í samræðunum.

\subsection{Uppsetning á kóða}\index{Uppsetning á kóða}
Þessi kóðabútur er þannig uppsettur að allar línur byrja jafnlangt til vinstri, eins og hver setning í töluðu máli stendur hver lína fyrir sínu, ein og sér.
\begin{lstlisting}[caption=Réttur Python kóði]
# Réttur Python kóði sem keyrist
4 + 8
5 + 6
breyta = 9 * 2
\end{lstlisting}

Þessi næsti kóðabútur hinsvegar er ekki nógu vel uppsettur, þar eru ,,setningar'' sem virðast hanga undir öðrum og vera þeim háðar. 

\begin{lstlisting}[caption=Rangur Python kóði]
# Illa skrifaður Python kóði sem keyrist ekki
4 + 8
	5 + 6
breyta = 9 * 2
\end{lstlisting}

Svona inndrætti er einungis beitt ef lína á beinlínis að hanga undir línunni að ofan og tilheyrir henni. 
Þess vegna þarf að huga að því hvernig kóði er uppsettur. 
Í öðrum málum eru notuð greinamerki til að segja tölvunni að lína sé búin og að aðrar línur eigi að heyra undir eitthvað ákveðið samhengi en ekki í Python, þar er treyst á að forritarinn setji kóðann upp á máta sem hægt er að sjá að sé réttur. 
Dæmi um hvernig línur geta verið aðgreindar í öðrum málum:

\begin{lstlisting}[language=Java , caption=Dæmi um annað mál sem er strangt tagað og með greinamerkjum]
// Java
int i = 7;
i + 5;

// Þetta myndi líka ganga í Java en ekki í Python
int i = 7; i + 5;
\end{lstlisting}

\begin{lstlisting}[language=Lisp, caption=Dæmi um annað mál sem byggir á afmörkuðu samhengi en með greinamerkjum]
; Lisp
(setq x 10)
(setq y 34.567)

(print x)
(print y)
\end{lstlisting}

Í þessum tveimur frábrugnu málum sem voru tekin sem dæmi var óþarfi að setja kóðann í mismunandi línur, því greinamerkin væru nóg til að aðgreina hverja línu fyrir sig. 
Hins vegar er það góð venja að skrifa kóða sem er læsilegur öðru fólki. Í Java eru greinamerkin semikommur (;) en í Lisp eru línur og samhengi afmörkuð með svigum. 
Python byggist hinsvegar á því að forritarinn stilli öllu upp rétt með réttum inndrætti. 

\subsection{Gagnatýpur og lykilorð}\index{Gagnatýpur og lykilorð}

Í Python eru nokkrar grunn gagnatýpur sem við munum kynnast í þessari bók. 
Ástæðan fyrir því að þær eru kallað grunntýpur er sú að þær fylgja með Python uppsetningunni og notandinn getur beitt þeim í samræmi við það sem þær eru færar um, sem má skoða í skjölun Python \href{https://www.python.org/doc/}{https://www.python.org/doc/}. 
Týpa eða tag er hugtak sem þýðir að hlutur sé af einhverri ákveðinni tegund sem má framkvæma ákveðnar aðgerðir á, í þessari bók verða týpur ýmist kallarðar það eða tög. 
Lesandi þekkir muninn á orðum og tölum úr daglegu tali og veit að hægt er að framkvæma mismunandi aðgerðir á þessum mismunandi týpum, eins og hægt er að skipta út hástöfum fyrir lágstafi í orðum en ekki tölum og hægt er að hefja tölur í veldi en ekki orð. 
Að sama skapi eru til aðgreinanlegar týpur sem tölvan kann skil á og leyfir ákveðnar aðgerðir á.
Í fyrri hluta þessarar bókar verða gerð skil á tveimur talnatýpum (heiltölum og fleytitölum), strengjum, listum, orðabókum (einnig kallaðar hakkatöflur) og boolean gildum. 
Í seinni hlutanum bæstast svo við sett \todo{sett?}.

Lykilorð eru orð sem eru frátekin og birtast þau græn í Jupyter vinnubók. 
Hver gagnatýpa hefur eitt lykilorð og eru einnig nokkur innbyggð föll í Python, sem við kynnumst fljótlega, með frátekin orð. 
Forðast skal að yfirskrifa þessi lykilorð, en gerist það þá er auðvelt að laga það í Jupyter. 
Hver vinnubók hefur sinn kjarna til að vinna á og það eina sem þarf að gera í aðstæðum þar sem innbyggt orð er allt í einu farið að þýða eitthvað annað þá dugir að endurræsa kjarnann.
Kjarninn í vinnubókinni er hvaða túlk eða þýðanda er verið að nota til þess að láta tölvuna skilja kóðann.
Í okkar tilfelli erum við að nota Python 3.


\chapter{Tölur og breytur}\index{Tölur og breytur}
Í þessum kafla ætlum við að hefjast handa við að forrita. 
Það fyrsta sem við ætlum að gera er að kynnast talnatýpum og keyra kóða eins og við værum að nota reiknivél. 
Við könnumst við reiknivélar og hvernig þær afgreiða röð aðgerða. 
Nú viljum við sannreyna að þær reikniaðgerðir sem við þekkjum séu til í Python og að þegar við keyrum kóðann okkar þá verði útkoman sú sama og við áttum von á. 
Við viljum líka geta geymt útkomuna okkar til að nota aftur seinna, til þess þurfum við breytur (e. variables).

\section{Tölur - talnatýpur}\index{Tölur - talnatýpur}
Í Pyhton eru í grunninn tvær týpur af tölum (en til eru tvær týpur af hvorri fyrir sig, sem snýr meira að minnisnotkun og er út fyrir svið þessarar bókar). 
Þær eru:

\begin{itemize}
	\item \textbf{Heiltölur} - tölur sem eru ekki með neinum aukastaf. 
	Á ensku eru þessar tölur kallaðar integers og er lykilorð þeirra því \textbf{int}.
	\item \textbf{Fleytitölur} - tölur sem eru með aukastaf, sem er fyrir aftan punkt (ekki kommu, fleytitölur eru oft kallaðar kommutölur á íslensku). 
	Á ensku eru þessar tölur kallaðar floating point numbers og er því lykilorðið þeirra \textbf{float}.
\end{itemize}

\begin{lstlisting}[caption=Heiltölur og fleytitölur]
# Heiltölur, enginn aukastafur
42
100000
-139

# Fleytitölur, aukastafur/ir fyrir aftan punkt
4.0
3.1415926
-100.98
\end{lstlisting}

\section{Reikniaðgerðir og tákn}\index{Reikniaðgerðir og tákn}
Grunn reikniaðgerðir eru nokkrar sem við könnumst við úr grunnskóla en aðrar eru framandi og við skulum skoða aðeins betur.
\todo{tala um reikniaðgerðirnar hérna}

Í eftirfarandi dæmum er vert að draga fram nokkur atriði sem eru ekki augljós byrjanda. 
Það fyrsta er að myllumerkið (\#) þýðir að allt sem kemur fyrir aftan það er \textit{athugasemd}, athugasemdir eru engöngu til að gera kóða læsilegri fyrir fólk, þær eru hunsaðar af tölvunni þegar hún breytir kóðanum í eitthvað sem hún skilur.
Eins og sést í kóðabút \ref{lst:reiknadg} í línu merktri númer 20 er athugasemdin svo löng að hún birtist okkur sem tvær línur en hún er í keyrslu tölvunnar álitin ein heild.
Þess vegna þurfum við ekki að hafa áhyggjur af þessum inndrætti sem birtist, hann er í rauninni ekki til staðar þar sem þessi hluti textans er partur af annarri línu.
Einnig eru þarna bil á milli talna fremst í línu og tákna, það er líka til að gera kóðan læsilegri, bilin mega bara ekki vera fremst í línunni enn sem komið er.
Athugasemdir í kóða eru mjög mikilvægur hluti af skjölun kóða og ættu öll sem vilja tileinka sér forritun að venja sig á að skrifa athugasemdir.
Í fyrstu erum við ekki að skrifa flókinn kóða svo athugasemdirnar segja okkur ekki mikið, en þegar kóðinn er ekki augljós eða lausn á verkefni ekki augljós er gott að skrifa athugasemdir.
Flest allir kóðabútar eru skjalaðir með athugasemdum til að gera þá læsilegri því allur kóði í bókinni er skrifaður fyrir fólk til að skilja.
Kóði sem þið komið til með að skrifa seinna meir á einnig að vera ykkur sjálfum skiljanlegur þegar þið komið að honum seinna.

\begin{lstlisting}[caption=Reikniaðgerðir, label=lst:reiknadg]
# Samlagning framkvæmd með + 
# Þegar eftirfarandi kóði er keyrður ætti útkonan að vera 10
6 + 4 

# Frádráttur framkvæmdur með -
# Þegar eftirfarandi kóði er keyrður ætti útkonan að vera 10
14 - 4 

# Margföldun framkvæmd með * 
# Þegar eftirfarandi kóði er keyrður ætti útkonan að vera 10
10 * 2 

# Deiling framkvæmd með / 
# Athugið að þetta er fleytitöludeiling sem skilar nákvæmu svari
# Þegar eftirfarandi kóði er keyrður ætti útkonan að vera 10.0
60 / 6 

# Heiltöludeiling framkvæmd með //
# Athugið að þessi deiling er frábrugðin þeirri sem þið kannist við
# Hér viljum við vita hversu oft, heil tala, ein tala gengur upp í aðra og okkur er sama um afganginn
# Þegar eftirfarandi kóði er keyrður ætti svarið að vera 10
177 // 17

# Veldishafning framkvæmd með **
# Hér er mikilvægt, eins og með deilinguna, að hafa í huga hvor talan kemur á undan.
# Fyrst kemur talan sem hefja á í veldi og svo kemur talan sem er veldisvísirinn
# Þegar eftirfarandi kóði er keyrður ætti svarið að vera 9
3 ** 2

# Leifareikningur framkvæmdur með % (e. modulus)
# Þetta er eitthvað alveg nýtt og framandi, en þó ekki óskiljanlegt
# Það sem þetta reiknar er hversu mikil leif eða afgangur er eftir þegar heiltöludeilingu er beitt.
# Þegar eftirfarandi kóði er keyrður ætti svarið að vera 7
177 % 17

\end{lstlisting}

Í öllum þessum dæmum var verið að vinna með heiltölur, þó var útkoman úr deilingunni (stundum kölluð fullkomin deiling) fleytitala. 
Hvað gerist ef þessir sömu útreikningar eru gerðir með fleytitölum? 
Ef við myndum skipta út hverri tölu fyrir sig og setja í staðinn sömu tölu með .0 fyrir aftan þá yrðu útkomurnar þær sömu nema fleytitölur. 
En hvað gerist ef við breytum aðeins fyrri tölunni en ekki seinni tölunni?
Þá ertu að nota ólíkar týpur og slíkt er vandmeðfarið, en í þessu tilviki er það í lagi þar sem Python gerir þá ráð fyrir að það sé í lagi að reikna allt með fleytitölum og framkvæmir reikninginn eins og þú hafir verið að beita fleytitölum í hvívetna og niðurstaðan verður þá að sjálfsögðu fleytitala.

\section{Breytur}\index{Breytur}
Nú höfum við séð hvernig má keyra kóða einfaldlega eins og í reiknivél.
Höldum okkur við samlíkinguna um reiknivélina til að útskýra breytur.
Á hefbundinni reiknivél sem notuð er í stærðfræðitíma í framhaldsskóla er takki sem á stendur ANS.
Það stendur fyrir answer og ef ýtt er á hann getur vélin geymt síðasta gildið sem hún gaf sem svar og unnið svo með það til að gefa næsta svar.
Flottari vélar geta svo geymt nokkuð mörg svör en það er útfyrir gagnsemi þessarar samlíkingu.
Þegar ýtt er á þennan takka er minnisvæði í reiknivélinni tekið frá og skrifað er í það gildi, sem er svo sótt þegar ANS er notað í útreikningi.
Að sama skapi má láta Python úthluta minnissvæði í tölvunni fyrir þær breytur sem þið viljið geyma.
Munurinn er sá að þið nefnið sjálf hvað minnisvæðið er merkt sem, eruð ekki bundin við að nota ANS og að þið eruð svo gott sem með óteljandi minnissvæði.

Að gefa minnissvæði merkingu og gildi er gert með \textit{gildisveitingu}.
Gildisveiting þýðir að nú er einhver ákveðinn merkimiði kominn með eitthvað til að geyma.
Sjáum einfalt dæmi um þetta.

\begin{lstlisting}[caption=Breytur kynntar]
# Hér er ég að fara að búa til breytu sem heitir val
val = 5

# Þegar ég keyri línuna fyrir ofan segi ég vélinni að hafa aðgengilegt minnisvæði sem ég get notað með því að skrifa orðið val, og settu í það svæði gildið 5.

# Svo ég er að veita breytunni val gildið 5, þess vegar er það kallað gildisveiting.

# Svo get ég notað breytuna mína
# þegar þetta er keyrt fæst svarið 10
val + 5
\end{lstlisting}

Ef þú prófar þig áfram við að búa til breytur gætir þú rekist á svolítið sem hefur ekki gerst áður í vinnubók, að þegar sella inniheldur eingöngu gildisveitingu og er keyrð þá ,,gerist ekkert''.
Þetta finnst mörgum mjög skrýtið því þau vilja fá einhverja útkomu.
En útkoman er sú að þú sagðir vélinni að geyma þetta, þú sagðir henni ekki að gera neitt annað.

Breytur eru skilgreindar vinstra megin við jafnaðarmerki í Python.
Eins og það væri lesið, val fær gildið 5.
Það væri lítið vit í því að hafa það öfugt, 5 er núna jafngilt val.
Það sem við værum þá að segja tölvunni að í hver sinn sem hún vill nota heiltöluna fimm þá á hún að hætta við að nota töluna sjálfa og í staðinn vísa eingöngu í það sem er í minnissvæði merktu val.
Það er alls ekki það sem við viljum.

Nokkrar reglur í nafnavali á breytum, þetta vill vefjast fyrir sumum en lærist fljótlega:
\vspace{0.5cm}
\begin{enumerate}
	\item Kóðalitunin á breytuheitinu má ekki vera annað en venjulegi liturinn fyrir kóða, þannig að ef nafnið fær áherslumerkingu (annan lit) er það ekki löglegt breytuheiti. 
	Áherslulitunin í númeruðu kóðabútunum í þessari bók er marklaus því hún er mjög frumstæð.
	Dæmi um það sem fær áherslulitun eru frátekin lykilorð og tölustafir.
	\item Breytuheitið ætti ekki að innihalda séríslenskan staf (það er löglegt í jupyter vinnubókum en er hrikalega slæmur ávani því það er ekki löglegt allsstaðar).
	\item Breytuheitið má ekki innihalda bil.
\end{enumerate}
\vspace{0.5cm}
Nokkur tilmæli um breytunöfn með tilliti til nafnavenja í Python:
\vspace{0.5cm}
\begin{enumerate}
	\item Breytuheiti byrja á litlum staf.
	\item Ef það þarf að gera löng breytuheiti er venjan að nota snákaframsetningu (e. snake casing) sem felur í sér að gera niðurstrik á milli orða, dæmi \texttt{thetta\_er\_langt\_nafn\_a\_breytu}.
	Annars er til kamelframsetning (e. camel casing) sem felur í sér að annað hvert orð er með stórum staf, dæmi \texttt{thettaErLikaLangtBreytuheiti}.
	Hvort sem þið endið á að nota meira, haldið ykkur bara við annað þeirra.
	\item Breytuheiti eiga að vera lýsandi.
	Ef ég væri að reikna hliðar í þríhyrningi væri gott að eiga breyturnar \texttt{a}, \texttt{b} og \texttt{c}.
	En ef ég væri að búa til reiknirit sem býr til tölvuleikjapersónu af handahófi með því að velja tilviljanakennt nafn, aldur og starf þá væru breytuheitin \texttt{a}, \texttt{b} og \texttt{c} alveg glötuð því þegar ég kæmi aftur að kóðanum mínum myndi ég ekki hafa hugmynd um hvað a, b og c væru. 
	Betra væri að breyturnar hétu \texttt{nafn}, \texttt{aldur} og \texttt{starf}.
\end{enumerate}

\begin{lstlisting}[caption=Dæmi um gildisvetingar\, réttar og rangar]
# Hér er ég að fara að búa til breytu sem heitir val
val = 5

# Hér er ég ekki að búa til breytu sem heitir val heldur er ég að segja að talan fimm er ekki lengur til sem heiltala heldur gæti hún vísað í hvað sem er sem er geymt í minnisvæði merktu val, ólöglegt.
5 = val

# Hér bý ég til breytu sem heitir heiltala sem fær gildið 0
heiltala = 0

# Hér yfirskrifa ég lykilorðið fyrir týpuna heiltala og læt það innihalda 0
int = 0
# þetta er harðbannað og ef þetta gerist er ekki nóg að þurrka þetta út og keyra aftur, nú þarf að endurræsa kjarna vinnubókarinnar.

Gott nafn = 1.0
# Þetta er ekki bara bannað vegna bilsins á milli orðanna ,,Gott'' og ,,nafn'' heldur er það líka ljótt því að það byrjar á stórum staf

3_litlar_mys = 3
# má ekki byrja á tölustaf eða tákni

utreiknud_laun_eftir_skatt = 0.65 * laun
# frábært, lýsandi og gott breytuheiti (hér er þó gert ráð fyrir að vélin þekki breytuna laun)
\end{lstlisting}

Nú þegar við höfum séð hvernig má skilgreina breytu viljum við vita hvernig á að nota þessa breytu.
Ef við snúum okkur aftur að reiknivélasamlíkingunni um ANS takkann þá ætti kóðabútur \ref{lst:notabreytu} að geta sýnt með eðlislægum hætti hvernig breytur nýtast.
Fyrst segi ég vélinni hvað það er sem ANS vísar á, svo segi ég vélinni að mig langar til þess að búa til nýja breytu sem á að byggja á því sem ANS inniheldur.
Í þessum kóðabút er svo haldið áfram með þessa afleiddu breytu og önnur afleidd breyta búin til útfrá henni.
Það sem gerist svo í endann er sambærilegt við það að ýta á ,,='' takkann á reiknivélinni.
Takið eftir að þarna er notuð ný framsetning sem við höfum ekki séð áður, þarna stendur print með svigum fyrir aftan og inni í svigunum er breytan okkar.
Ef þessi kóðabútur er keyrður þá kemur á \textit{staðalúttak} 
\footnote{Þann stað sem texti myndi prentast þegar forritið er notað, hvort sem það er á skjá eða beint á pappír úr prentara eða eitthvað allt annað. Kannski verður úttaki varpað beint inn í heilann á forriturum einhvern tíma?} 
það gildi sem breytan \textbf{x} inniheldur.
Ef þar hefði staðið \texttt{print(halft\_x)} hefðum við fengið svarið sem er geymt í breytunni \texttt{print(halft\_x)}.

\begin{lstlisting}[caption=Að nota breytu, label=lst:notabreytu]
# Hér framkvæmi ég einhvern útreikning sem ég geymi í breytunni ANS
ANS = 5**2 + (4+8.9)**2

# Segjum að þetta hafi verið endapunkturinn í löngu algebrudæmi og nú veit ég hvað y er, og get þá nýtt það til að finna x eins og verða vill svo oft í stærðfræði að x sé týnt. Gefum okkur að x = 3 * y og því fæst
x = 3 * ANS

# Nú ef við viljum reikna eitthvað út með x eigum við það til í minnissvæði merktu x með réttu gildi. Til dæmis með því að búa til breytu fyrir hálft x.
halft_x = x/2


# Nú langar okkur til að vera viss um að við séum við vitrænt svar svo við biðjum tölvuna um að segja okkur hvað er geymt í breytunni x.
print(x)
a = "texti í streng"
'strengur'
"annar strengur"


\end{lstlisting}

Við megum beita print skipuninni óspart og hvetur höfundur til þess að lesandi venji sig á að skoða úttakið sitt í hverju þrepi áður en leitað er hjálpar til annarra.
Print er \textit{fall}, við skoðum föll svo nánar í kafla \todo{ref kafla um föll} en þangað til munum við kynnast nokkrum innbyggðum föllum.

Núna höfum við séð tvær týpur, heiltölur og fleytitölur.
Breyta getur innihaldið hvernig týpu sem er.
Þá þurfum við að athuga að hvað má gera við breyturnar okkar.
Hingað til höfum við eingöngu skoðað reikniaðgerðir sem eru framkvæmdar með kunnuglegum táknum, við höfum ekki verið að beita neinum innbyggðum aðferðum á tölurnar okkar. En við sjáum það gert í kafla \ref{k:strengir} þegar við skoðum hvernig megi vinna með texta.

% Að því sögðu þá þurfum við að skoða breytur nokkuð betur áður en við förum að beita þeim á skilvirkan hátt.



\chapter{Strengir}\index{Strengir}\label{k:strengir}
Til þess að geta sýnt og notað texta þarf gagnatýpu til að halda utan um hann. Í flestum forritunarmálum, og Python er ekki undantekning, eru gögn af þeirri týpu kölluð \textbf{strengir}.




\chapterimage{chapter_head_2.pdf} % Chapter heading image

\chapter{Listar}\index{Listar}\label{k:listar}
Listar eru gagnagrindur, sem þýðir að þeir geta geymt fyrir okkur hin ýmsu gögn og gert okkur þau aðgengileg á ákveðinn máta.
Listar eru skilgreindir með hornklofum [ ] og er lykilorðið þeirra \textbf{list}.

\section{Listar skilgreindir}\index{Listar skilgreindir}\label{uk:listar-skilgreindir}
Listar geyma, í ákveðinni röð, þau gögn sem við viljum geyma sem mega vera af hvaða týpu sem er.
Gögnin sem eru sett inn í listann eru kölluð stök og röðin sem þau eru í eru aðgengileg eftir vísum eða sætisnúmerum alveg eins og strengir.
Stökin eru aðgreind með kommum.
Þær týpur sem við höfum séð hingað til eru heiltölur, fleytitölur, strengir og listar.
Allt eru þetta möguleg stök í lista.

\begin{lstlisting}[caption=Listar skilgreindir, label=lst:listar-skilgreindir]
# Fyrsti listinn okkar er tómur
listinn_minn = []

# þegar við skilgreinum lista aðgreinum við stökin með kommum
nyr_listi = ["núllta stakið", 1, 2, 3.0, "fjórða stakið", [5]]
\end{lstlisting}

Í kóðabút \ref{lst:listar-skilgreindir} sjáum við að við erum með 6 stök í listanum nyr\_listi sem er skilgreindur í línu 6 \todo{passa að breyta ekki kóðabút til að þessi vísun haldist}.
Fremsta stakið er strengur, næstu þrjú eru tölur, síðan kemur annar strengur og síðasta stakið í sæti 5 er listi.
Sá listi inniheldur eitt stak sem er þá í núllta vísi í þessum innri lista.

Ef við hugsum okkur töflureikni eins og Calc eða Excel þá getum við ímyndað okkur að ein lína sé eins og einn listi, hver dálkur er stak í listanum og ein röð er listinn sem heldur utan um þau.
Þá getum við líka ímyndað okkur að ef við erum með margar raðir séu þær geymdar á einni örk eða einu skjali.
Sjáum hvernig það myndi líta út í kóðabút \ref{lst:listar-arkir}

\begin{lstlisting}[caption=Listar af listum, label=lst:listar-arkir]
# Ef við ættum skjal í töflureikni sem héldi utan um allt starfsfólk í fyrirtæki gæti hausinn á því litið svona út:
# Nafn Tölvupóstur Deild Símanúmer 

# Svo er hver röð fyrir neðan það útfyllt með upplýsingum um einhvað tiltekið starfsman, t.d.:
# Jóna Jónsdóttir jona@fyrirtaeki.is Póstur 4445555

# Ef þetta væri útfært í Python með listum væri það gert svona:

starfsfolk = [["Jóna Jónsdóttir", "jona@fyrirtaeki.is", "Póstur", "4445555"],
			  ["Kristján Kristjánsson", "kristjan@fyrirtaki.is","Laun","4445589"],
			  ["Halldóra Halldórudóttir", "halldora@fyrirtaeki", "Skrifstofa", "4445500"]]

\end{lstlisting}

Við tökum eftir því að listinn starfsfolk í kóðabút \ref{lst:listar-arkir} í línu 9 inniheldur þrjá aðra lista, og þeir eru aðgreindir með kommum alveg eins og stökin inni í hverjum innri lista fyrir sig eru einnig aðgreind með kommum.
Einnig tökum við eftir því að hér sjáum við í fyrsta sinn inndrátt, það er í raun bara aukalegt bil sem vélin hunsar við að skilgreina breytuna starfsfolk og er því fyrir okkur til að geta lesið kóðann auðveldlegar.
Þetta er ekki eins og inndrátturinn sem við munum sjá og beita í næsta kafla, \nameref{k:segðir}.
 
\section{Að vinna með gögn}\index{Að vinna með gögn}\label{uk:gagnavinnsla-listar}
Þegar við geymum gögn viljum við að þau séu aðgengileg og að við getum skoðað þau, breytt þeim og unnið með á máta sem hentar okkur.
Listar gera okkur kleyft að nálgast gögn eftir sætisvísum.
Við náum í gögn upp úr lista eftir sætisvísi alveg eins og við sóttum tiltekið tákn úr streng, með því að nota hornklofa og það sætisnúmer sem við vildum.
Hér þurfum við að athuga að við viljum ekki ruglast á því að skilgreina lista með hornklofum og að sækja gögn úr lista eða streng með hornklofum.
Í fyrra tilfellinu standa hornklofarnir einir og sér, þar sem við erum að skilgreina nýjan lista.
Í seinna tilfellinu standa hornklofarnir fyrir aftan þá breytu sem á að sækja gögn upp úr með ákveðnum sætisvísum.
Sjáum dæmi.

\begin{lstlisting}[caption=Listar af listum, label=lst:listar-gogn-sott]
# listi skilgreindur
[1,2,3]

# listi skilgreindur og geymdur í breytu
listinn = [1,2,3, "langur strengur sem hefur einnig sætisnúmer"]

# gögn sótt upp úr listanum
listinn[1] # skilar okkur tölunni 2

# Við getum líka sótt gögn upp úr þeim gögnum sem leyfa það
listinn[3][0] # skilar okkur stafnum l
\end{lstlisting}

Við sjáum í kóðabút \ref{lst:listar-gogn-sott} í línu 11 að þar erum við að keðja (e. to chain) hornklofana okkar.
Þetta megum við því að listinn[3] skilar okkur til baka strengnum "langur strengur sem hefur einnig sætisnúmer" og við megum sækja úr honum stak númer 0 sem er aðgengilegt með því að gera [0] fyrir aftan listinn[3].
Þetta gagnast okkur sérstaklega þegar við lítum aftur á starfsmanahaldið okkar hér á undan og viljum geta sótt gögn upp úr innri listum.
Sjáum hvernig við getum fengið upplýsingar sem eru skráðar um tiltekið starfsman úr listanum sem við geymdum í kóðabút \ref{lst:listar-arkir}.

\begin{lstlisting}[caption=Unnið með gögn úr lista, label=lst:listar-gagnanotkun]
print(starfsfolk[0])

# þetta skilar okkur 
["Jóna Jónsdóttir", "jona@fyrirtaeki.is", "Póstur", "4445555"]

print(starfsfolk[0][0])
# þetta skilar
"Jóna Jónsdóttir"
# þar sem nafnið hennar er 0 stakið í innri listanum
\end{lstlisting}

\subsection{Listar eru breytanlegir}
Nú allt í einu munum við að Jóna er ekki Jónsdóttir heldur Alfreðsdóttir og við þurfum að laga það, við þurfum ekki að skilgreina listann allann upp á nýtt (sem við hefðum þurft að gera ef við værum með streng) heldur þurfum við bara að setja nýtt gildi inn fyrir það sem heldur utan um nafnið hennar Jónu.

\begin{lstlisting}[caption=Unnið með gögn úr lista, label=lst:listar-gagnabreyting]
print(starfsfolk[0][0])

# við munum að þetta skilar 
"Jóna Jónsdóttir"
# en við munum að nafnið var óvart vitlaust skráð svo við breytum því

starfsfolk[0][0] = "Jóna Alfreðsdóttir"
# þetta skilar okkur engu því að við vorum hér að skilgreina eitthvað, segja tölvunni að geyma eitthvað
# en við erum búin að endurskilgreina starfsfolk listann og ef við köllum í hann núna sjáum við að hann er breyttur

print(starfsfolk) 
# þetta skilar 
[["Jóna Alfreðsdóttir", "jona@fyrirtaeki.is", "Póstur", "4445555"], ["Kristján Kristjánsson", "kristjan@fyrirtaki.is","Laun","4445589"], ["Halldóra Halldórudóttir", "halldora@fyrirtaeki", "Skrifstofa", "4445500"]]
\end{lstlisting}

\section{Gagnlegar aðferðir á lista}\index{Aðferðir á lista}\label{uk:aðferðir-listar}

Eins og tekið var fram í kaflanum um strengi þá er ekki ætlunin að fara yfir allar þær innbyggðu aðferðir sem til eru fyrir lista heldur draga fram nokkrar sem eru mjög gagnlegar til að auka skilning á notkun á aðferðum.

Gefum okkur að við eigum listann [0,2,1,3] sem er geymdur í breytunni listinn\_minn

\begin{itemize}
	\item listinn\_minn.pop() 
	\begin{itemize}
		\item það sem þetta gerir er að breyta listanum og skila staki.
		\item gildið sem það skilar er aftasta stakið úr listinn\_minn.
		\item 3 er gildið sem það skilar í okkar tilfelli svo listinn\_minn verður að [0,2,1].
		\item hægt er að geyma það með því að gera x = listinn\_minn.pop() og þá inniheldur x töluna 3.
		\item einnig er hægt að setja inn sætisnúmer sem viðfang og þá er stakið í því sæti fjarlægt og listinn dregst saman, sjá kóðabút \ref{lst:listar-pop}.
	\end{itemize}
	\item listinn\_minn.append(x)
	\begin{itemize}
		\item það sem þetta gerir er að breyta listanum þannig að búið er að bæta breytunni x aftast í listann.
		\item þessi aðferð skilar engu til baka til okkar svo það er ekkert vit í því að skrifa listi = listinn\_minn.append(4)
		\item segjum að x hafi verið stillt sem talan 4 þá lítur listinn núna svona út [0,2,1,3,4] (ef við gerum ráð fyrir að hafa ekki keyrt neinar aðrar aðferðir á hann).
		\item þessi aðferð verður að fá eitt viðfang og nákvæmlega eitt viðfang, sem er af hvaða gagnatýpu sem er, svo við gætum sett inn einn lista sem inniheldur 100.000 stök en það er nákvæmlega einn listi.
		\item sjá notkun í kóðabút \ref{lst:listar-append}
	\end{itemize}
	\item listinn\_minn.sort()
	\begin{itemize}
		\item það sem þetta gerir er að raða listanum í röð með samanburðarvirkjum (þeir verða kynntir í kafla \nameref{k:segðir}), og stökin í listanum þurfa þá að vera samanburðarhæf.
		\item aðferðin raðar listanum í röð frá lægsta gildi til hæsta gildis, það er okkur tamt þegar við skoðum talna lista en í því tilfelli að listinn innihaldi bara strengi þýðir það að listanum er raðað í stafrófsröð sem er skilgreind eftir því táknakerfi sem Python notar.
		\item listinn\_minn.sort myndi gera það að verkum að hann sé nú geymdur sem [0,1,2,3] (gerum ráð fyrir að við höfum ekki keyrt neinar aðrar aðferðir á hann).
		\item aðferðin skilar engu svo það er ekkert vit í því að gera x = listinn\_minn.sort()
		\item sjá notkun í kóðabút \ref{lst:listar-sort}.
	\end{itemize}
	
\end{itemize}

\chapterimage{chapter_head_2.pdf} % Chapter heading image

\chapter{Segðir, skilyrðissetningar og sanngildi}\label{k:segðir}
\todo{laga þessa kynningu}
Kóða má skipta í segðir (e. expressions) og yrðingar (e. statements).
Segðir eru línur þar sem eitthvað er metið sem gildi, ef við líkjum því við tungumál væru það setningar þar sem einhver niðurstaða fæst eins og ,,er rigning''? 
Yrðingar eru línur þar sem eitthvað er sett fram sem þarf ekki að meta, í tungumáli væru það setningar á borð við ,,mér er kalt''

Þessi skipting er ekkert sérlega merkileg að svö stöddu en í þessum kafla ætlum við að einbeita okkur að því að meta útkomu og fá í hendurnar svör sem við getum svo gert eitthvað við.

Til þess að gera það þurfum við að læra á nýja týpu sem heitir Boolean og hefur lykilorðið \textbf{bool}, boolean gildi eru kölluð búlsk gildi eða sanngildi.
Boolean týpan er frábrugðin þeim týpum sem við höfum séð hingað til því að það eru eingöngu tvö möguleg gildi sem Boolean getur verið, \textbf{True} og \textbf{False} sem þýðast sem 1 og 0, satt og ósatt og eru upprunin úr búlískri algebru \footnote{Ekki verður farið yfir búlíska algebru af neinu ráði í þessari bók en þau fræði eru gífurlega góður grunnur til að skilja betur hvernig segðir og rökvirkjar virka og því hvetur höfundur til að lesandi fletti allavega upp wikipedia greininni.} (e. Boolean algebra). 
Nú er það flestum kunnug staðreynd að tölvur vinna með 0 1 í grunninn, en hvernig það er notað í æðri forritunarmálum (e. high level programming languages) er ekki eins augljóst.

Í þessum kafla verður farið yfir búlísk gildi, samanburð (e. comparison) og samanburðarvirkja (e. comparison operators), rökvirkja (e. logical operators) og svo skilyrðissetningar (e. conditional statements).

\section{Sanngildi}\index{Sanngildi}

Eins og kom fram í inngangi kaflans eru búlsk gildi einungis tvö, True og False.
Hægt er að geyma þau í breytum eins og gögn af öðrum týpum sem við höfum séð.
Sanngildi eru einnig metin sem 1 eða 0, fyrir True annars vegar og False hinsevegar.

Vitandi að gildin geta verið 0 eða 1 (aldrei bæði í einu) þá er þess virði að nefna hérna sanntöflur.
Látum p vera yrðinguna ,,það er rigning'' og látum q vera yrðinguna ,,mér er kalt''.
Þá gætum við, með því að skoða mismunandi aðstæður, fengið rökrétt svar við t.d. spurningunni ,,er rigning og er mér kalt?'' sem við getum skrifað sem s1 og svo annarri spurningu sem er ,,er rigning eða er mér kalt?'' sem við getum kallað s2.

\begin{center}
\centering
\begin{table}
	\centering
\caption{Sanntafla}
\vspace{3pt}
\label{tbl:sanntafla}
\begin{tabular}{|c c|c|c|}
	% |c c|c| means that there are three columns in the table and% a vertical bar ’|’ will be printed on the left and right borders,
	% and between the second and the third columns.% The letter ’c’ means the value will be centered within the column,
	% letter ’l’, left-aligned, and ’r’, right-aligned.
	p & q & s1 & s2\\ 
	% Use & to separate the columns
	\hline  
	% Put a horizontal line between the table header and the rest.
	0 & 0 & 0 & 0\\
	0 & 1 & 0 & 1\\
	1 & 0 & 0 & 1\\
	1 & 1 & 1 & 1\\
	\end{tabular}

\end{table}
\end{center}
\todo{laga þetta ótrúlega mikla pláss sem er á milli textans og töflunnar}
Ef við horfum á töflu \ref{tbl:sanntafla} þá sjáum við að yrðingarnar okkar um rigningu og kulda eru uppsettar þannig að hver lína í töflunni er einstakt ástand, og allar mögulegar samsetningar koma fram \footnote{Fjöldi lína í sanntöflu byggir á fjölda yrðinga sem á að skoða. Ef það er bara ein yrðing þá er fjöldi lína 2, það er satt eða ósatt. Fjöldi lína er 2 í veldi fjölda staðhæfinga, eins og í töflu \ref{tbl:sanntafla} þá eru yrðingarnar tvær svo línurnar eru 2², og ef þær væru þrjár þá væri línufjöldinn 2³ og svo framvegis.}.
Báðar yrðingar eru ósannar í fyrstu línunni, svo eru þær sannar sitt á hvað, og í fjórðu línu eru þær báðar sannar.
Þá eru dálkarnir fyrir s1 og s2 svörin við spurningunum hér að ofan miðað við sanngildi yrðinganna í þeim tilteknu aðstæðum.
Í þeim aðstæðum þar sem er hvorki rigning né mér er kalt þá er svarið við báðum spurningum einnig neitandi (0).
Í þeim aðstæðum þar sem er bæði rigning og mér er kalt þá er svarið við báðum spurningum játandi (1).
Þannig að til þess að svarið við spurningu 1 sé játandi þá þarf mér bæði að vera kalt og það þarf að vera rigning, svo þegar yrðingarnar eru ekki sannar á sama tíma þá skiptir ekki máli hvor sé sönn því að önnur er ósönn og því er svarið neitandi.
En spurning 2 er þannig orðuð að það sé nóg að annað hvort sé mér kalt eða það sé rigning úti til þess að svarið sé játandi, svo þegar yrðingarnar eru sannar á víxl þá er svarið alltaf játandi.
\begin{lstlisting}[caption=Sanngildi geymd sem breytur, label=lst:bool-breytur]
test = True
# þetta geymir gögn af týpunni Bool

test = true
# þetta veldur villu þar sem nú er verið að biðja um að test innihaldi það sama og breytan true vísar á
# munurinn er í stóru og litlu t.

test = "True"
# þetta er ekki af týpunni Bool heldur er þetta strengur

test = False
# þetta geymir gögn af týpunni Bool
\end{lstlisting}

Akkúrat núna þurfum við bara að vita að týpan Boolean sé til og hvernig eigi að nota hana, með hástaf fremst.
Sjáum svo í seinni köflum hvernig hún gagnast okkur.


\section{Segðir}\index{Segðir}

Eins og kom fram í inngangi kaflans má líta svo á að segðir séu sá hluti af kóðans sem er metinn sem eitthvað gildi, eins og 4 + 5 er segð en x = 5 er yrðing.
Nú ætlum við þó að einblína á búlskar segðir, það er horfa á spurningar sem hafa svar sem er annað hvort satt eða ósatt.
Er rigning?
Þá horfum við út og sjáum að miðað við aðstæður þá er svarið annað hvort satt eða ósatt og það breytist eftir því hvenær við horfum.

\subsection{Samanburður}\index{Samanurður}
Hvað er samanburður?
Það er þegar eitthvað er metið miðað við eitthvað annað, eins og er þetta stærra en hitt?
Er þetta þyngra?
Er þetta jafngilt?

Nú þurfum við nýtt hugtak, við erum búin að kynnast reiknivirkjum eins og + og - í kafla \ref{k:tolur}.
Nýja hugtakið okkar eru \textbf{samanburðarvirkjar}.
Samanburðarvirkjar eru notaðir til að spyrja hvort að ákveðin tengsl gilda á milli einhverja tveggja hluta.
Eins og í daglegu tali þegar við segjum ,,er þetta epli stærra en þessi appelsína?'' og erum þannig að bera saman epli og appelsínur, samanburðarvirkjar eru til þess að gera slíka setningu formlega svo að tölva geti svarað spurningunni.

Samanburðarvirkjar eru nokkir í Python:
\begin{itemize}
	\item == þá er spurt hvort að hlutirnir sitt hvoru megin við virkjann séu jafngildir
	\item != þá er spurt hvort að hlutirnir sitt hvoru megin við virkjann séu ólíkir
	\item < þá er spurt hvort að það sem er vinstra megin sé strangt minna en það sem hægra megin (3 er ekki minna en 3 t.d.)
	\item > þá er spurt hvort að það sem er vinstra megin sé strangt stærra en það sem er hægra megin
	\item <= þá er spurt hvort að það sem er vinstra megin sé minna eða jafnt því sem er hægra megin
	\item >= þá er spurt hvort að það sem er vinstra megin sé stærra eða jafnt því sem er hægra megin
\end{itemize}

Skoðum kóðabút þar sem þessir samanburðarvirkjar eru nýttir til þess annars vegar að fá niðurstöður með tölur og hinsvegar strengi.
\begin{lstlisting}[caption=Samanburðarvirkjar, label=lst:bool-samanburður]
# nú viljum bera saman einhver gögn, búum okkur til breytur til að bera saman
strengur1 = "abc"
strengur2 = "bcd"
strengur3 = "3"
tala1 = 3
tala2 = 3.0
tala3 = 4

# nú erum við komin með nokkrar breytur til að gera prófanir á:

# byrjum á að skoða hvort að 3 sé jafngilt 4 eða tveir jafnlangir strengir séu jafngildir
tala1 == tala3 
# þetta skilar False

strengur1 == strengur2
# þetta skilar False

# en hvað með þetta?
tala1 == tala2
# þetta skilar True þar sem til að geta borið talnatýpur saman er þeim kastað í sambærileg gögn (skoðum kast í seinna í kaflanum)

strengur3 == tala1 
# þetta skilar False þar sem ekki er verið að vinna með eingöngu gögn af talnatýpum

# Allt það sem skilar okkur True með == skilar okkur False með != 
# og öfugt, það sem skilar False með == skilar okkur True með !=

# skoðum þá minna en og stærra en

strengur1 < strengur2
# þetta skilar okkur True þar sem strengur1 er framar í stafrófinu, ekki er verið að bera saman lengdina á strengjunum

tala1 < tala2
# þetta skilar okkur False þar sem tölurnar eru jafngildar, sáum það að hér að ofan, og önnur getur ekki verið bæði minni og jöfn á sama tíma

tala1 <= tala2
# þetta skilar okkur True þar sem spurt er hvort að tala1 sé annað hvort minni en eða jöfn hinni breytunni

# Það skiptir máli hvernig goggarnir snúa, a > b hér er spurt hvort a sé stærra en b, b < a, hér er spurt hvort b sé minna en a (sem er sama spurningin).
\end{lstlisting}

\subsection{Rökvirkjar}\index{Rökvirkjar}
Rökvirkjar (e. logical operators) í Python eru þrír, þeir eru \textbf{og}, \textbf{eða}  og \textbf{ekki} táknað með \texttt{and}, \texttt{or} og \texttt{not}.
Nöfnin þeirra eru lykilorð í Python eins og nöfnin á týpunum sem við höfum séð (\textbf{str}, \textbf{int}, \textbf{float}, \textbf{list}) en rökvirkjar eru ekki gögn af einhverri týpu heldur eru meira eins og reiknivirkjarnir (+, -, *, **, //, \%).
Það sem þessir virkjar gera fyrir okkur er að taka tvær búlskar segðir og egja okkur eitthvað um samsetningu þeirra.
Tökum dæmi; ,,Kaffið er heitt og það eru til sítrónur.'' 
Hægt er að meta hvort að kaffið sé heitt eða ekki, og fá þannig út sanngildi fyrir þá segð, það er hægt að gera það sama fyrir segðina um sítrónurnar.
En tökum eftir að á milli þessara tveggja segða er rökvirkinn \textit{og}, sem segir okkur að til þess að meta gildi allrar setningarinnar þurfa báðar segðirnar sitthvoru megin við rökvirkjann að vera sannar til þess að setningin í heild sinni skili sönnu annars er hún ósönn.
\vspace{10pt}
\begin{itemize}
	\item \textbf{and} til þess að segð með þessum rökvirkja sé sönn þurfa báðar hliðar að vera sannar, annars er hún ósönn
	\begin{itemize}
		\item Það má líta á \textit{og} rökvirkjann eins og margföldun, hann hefur forgang umfram \underline{eða}.
		\item Þar sem satt er 1 og ósatt 0 þá ef við margöldum með 0 fáum við alltaf 0 út.
		\item ,,það er heitt úti'' og ,,það er kalt úti'' myndi skila okkur ósönnu því ekki getur bæði verið satt.
		\item ,,það er heitt úti'' og ,,klukkan er fimm'' myndi skila okkur sönnu eftir aðstæðum.
	\end{itemize}
	\item \textbf{or} til þess að segð með þessum rökvirkja sé sönn þarf önnur hvor hliðin að vera sönn, annars er hún ósönn.
	\begin{itemize}
		\item Það má líta á \textit{eða} rökvirkjann eins og samlagningu.
		\item Þar sem satt er 1 og ósatt 0, þá þurfum við bara að sjá 1 einu sinni til þess að útkoman í heild sinni verði sönn.
		\item ,,það er heitt úti'' eða ,,það er kalt úti'' myndi skila okkur sönnu ef þetta væru þau einu tvö hitastig sem væru í boði.
		\item ,,það er heitt úti'' eða ,,klukkan er fimm'' myndi skila sönnu eftir aðstæðum.
	\end{itemize}
	\item \textbf{not} snýr við sanngildi segðar, not er ekki sett á milli segða heldur fyrir framan eins segð.
	\begin{itemize}
		\item Það má líta á rökvirkjan \textit{ekki} eins og mínus, hann snýr við sanngildi eins og formerki
		\item Ekki satt yrði ósatt, ekki ósatt yrði satt.
		\item \textbf{ekki} ,,það er heitt úti'' yrði að yrðingunni ,,það er ekki heitt úti''.
	\end{itemize}
\end{itemize}

\vspace{5pt}
\todo{formatting}
\begin{itarefni}
\textbf{Rökvirkjar sem reikniaðgerðir}\\
Til að halda áfram með þessa samlíkingu með margföldun, samlagningu og mínus skulum við skoða eftirfarandi reikningsdæmi:
1$\cdot$1$\cdot$1$\cdot$0 + 1$\cdot$0 + (-1).
Hér gerum við ráð fyrir að hver hluti af þessu reikningsdæmi sé yrðing sem búið er að meta sem sanna eða ósanna eftir þeim aðstæðum sem við erum í (kaffið er heitt, það er kalt úti og þess háttar).
Þegar við reiknum þetta dæmi sjáum við að margfaldað er með 0 í báðum þáttunum þar sem margföldun kemur fyrir svo útkoman í hvorum fyrir sig ætti að vera núll.
Þessi síðasti liður er okkur ekki eins eðlislægur en við munum að það eru bara til 0 eða 1 og mínus skiptir gildinu okkar svo við hljótum að enda með 0.
Þannig að við endum í 0 + 0 + 0 sem gefur okkur 0 og því er öll segðin metin sem ósönn.
\end{itarefni}


\section{Skilyrðissetningar}\index{Skilyrðissetningar}
Nú viljum við vita til hvers í ósköpunum við vorum eiginlega að leggja það á okkur að skilja hvenær eitthvað er satt eða ósatt.
Það er einmitt heilmikið tölvunarfræðilegt gagn í því að geta spurt svona já eða nei spurninga sem tölvan getur svarað.
Til dæmis viljum við geta framkvæmt einhverja aðgerð í forritinu okkar \textbf{ef} einhver skilyrði eru fyrir hendi.
Segjum að við séum með vekjaraklukku sem við forritum til að hringja þegar klukkan er orðin 8.
Þá viljum við geta spurt tölvuna hvort að það sé satt eða ósatt að klukkan sé orðin 8.
Ef klukkan er ekki orðin 8 viljum við ekki gera neitt, en ef hún er orðin átta þá viljum við að hún spili einhvern hljóm eða titri.
Við gætum líka verið að forrita einfaldan tölvuleik eins og hengimann, ef spilarinn er ekki búinn að giska á alla stafina í orðinu okkar viljum við geta beðið viðkomandi að spyrja aftur.
\textbf{Annars} viljum við að notandinn fái verðlaun fyrir að hafa giskað á rétt orð \footnote{Hérna er gert ráð fyrir að mega giska óendanlega oft rangt.}.
Einnig gætum við viljað gera eitthvað tiltekið þá og því aðeins að eitthvað annað var ósatt.
Ef við notum okkar eigin máltilfinningu til að leggja skilning í eftirfarandi setningu: Ef við eigum ekki mjólk vil ég kaupa mjólk, ef svo er ekki vil ég athuga hvort að við eigum kex og ef við eigum ekki kex vil ég kaupa það, annars fer ég ekkert í búðina.
Hér er aðaláherslan lögð á mjólkurstöðuna okkar, ef við eigum ekki mjólk viljum við laga það, en ef við eigum mjólk þá getum við gert eitthvað annað.

Þarna eru komnar aðstæður þar sem við athugum mólkurstöðuna og fyllum á ef þarf, en ef við eigum nóg af mjólk þá viljum við samt athuga hvort að við eigum nóg af safa því að við gætum þurft að fylla á þar.
Hér er kannski ekki augljóst en ef það vantar mjólk þá skiptir ekki máli hvort það vanti kex eða ekki, við förum í búðina og kaupum mjólk, við kaupum ekki kex.
Þetta skilst kannski frekar á flæðiriti sem sést á mynd \ref{fig:flæðirit}.
Flæðiritið líkir eftir uppsetningu á skilyrðissetningum þannig að það sem er inni í gænum þríhyrningum eru spurningar sem þarf að svara, bláu ferhyrningarnir eru svo niðurstöður sem fást í málið.

\tikzstyle{startstop} = [rectangle, rounded corners, minimum width=3cm, minimum height=1cm,text centered, draw=black, fill=red!30]
\tikzstyle{io} = [trapezium, trapezium left angle=70, trapezium right angle=110, minimum width=2cm, minimum height=1cm, text centered, draw=black, fill=blue!30]
\tikzstyle{process} = [rectangle, minimum width=3cm, minimum height=1cm, text centered, draw=black, fill=blue!30]
\tikzstyle{decision} = [diamond, minimum width=3cm, minimum height=3cm, text centered, draw=black, fill=green!30]
\tikzstyle{arrow} = [thick,->,>=stealth]

\vspace{5pt}
\begin{figure}[H]
	\centering
\begin{tikzpicture}[node distance=2cm]
\node (mjolk) [decision] {Engin mjólk};
\node (kex) [decision, below of=mjolk, yshift=-1.5cm] {Ekkert kex};
\node (bud1) [process, right of=mjolk, xshift=3cm] {Kaupa mjólk};
\node (bud2) [process, right of=kex, xshift=3cm] {Kaupa kex};
\node (bud3) [process, below of=kex, yshift=-0.5cm] {Vera heima};
\draw [arrow] (mjolk) -- node[anchor=west] {ósatt} (kex);
\draw [arrow] (mjolk) -- node[anchor=south] {satt} (bud1);
\draw [arrow] (kex) -- node[anchor=south] {satt} (bud2);
\draw [arrow] (kex) -- node[anchor=west] {ósatt} (bud3);
\end{tikzpicture}
\caption{Hér sést hvernig setningin: ,,Ef við eigum ekki mjólk vil ég kaupa mjólk, ef svo er ekki vil ég athuga hvort að við eigum kex og ef við eigum ekki kex vil ég kaupa það, annars fer ég ekkert í búðina.'' má setja fram sem fæðirit. Ef það er engin mjólk þá förum við og kaupum mjólk, en ef það er til mjólk þá athugum við hvort að það sé til kex og kaupum það ef það vantar, hins vegar ef við eigum bæði kex og mjólk er engin ástæða til að fara í búðina.}
\label{fig:flæðirit}
\end{figure}

\vspace{5pt}

Vegna þess að áherslan er lögð á ,,við eigum ekki mjólk'' þá er vitlegast að setja inn segð sem er með neitun.
Ef yrðingin m stendur fyrir setninguna ,, við eigum mjólk'' þá er yrðingin ekki m (not m) ,,við eigum ekki mjólk''.
Skoðum þetta í töflu \ref{tbl:sanntafla-kaffi}, sambærilegri þeirri sem við sáum áður (tafla \ref{tbl:sanntafla}), nema í staðinn fyrir p og q notum við yrðinguna ,,það er til mjólk''.
\begin{center}
	\centering
	\begin{table}[H]
		\centering
		\caption{Sanntafla með ákveðnum yrðingum}
		\vspace{3pt}
		\label{tbl:sanntafla-kaffi}
		\begin{tabular}{|c | c| c |}
			% |c c|c| means that there are three columns in the table and% a vertical bar ’|’ will be printed on the left and right borders,
			% and between the second and the third columns.% The letter ’c’ means the value will be centered within the column,
			% letter ’l’, left-aligned, and ’r’, right-aligned.
			m = það er til mjólk & ekki m = það er ekki til mjólk\\ 
			% Use & to separate the columns
			\hline  
			% Put a horizontal line between the table header and the rest.
			0 & 0 & Bæði ósatt, gengur ekki upp\\
			0 & 1 &\\
			1 & 0 &\\
			1 & 1 & Bæði satt á sama tíma, gengur ekki upp\\
			\hline
		\end{tabular}
		
	\end{table}
\end{center}
Við viljum að aðalatriðið komi fram í inngangspunktinum í skilyrðissetningunni okkar til að hún sé skýrt upp sett og skiljanleg, til þess gætum við þurft að nota neitun.
Við sjáum betur í næstu þremur undirköflum hvað ætti að fara á hvaða stað, en eins og með góðar nafnavenjur þegar við nefnum breyturnar okkar skulum við venja okkur á strax í upphafi að skilyrðissetningarnar okkar eru skýrar.

Nú höfum við séð í inngangi þessa undirkafla orðunum ef og annars slengt fram.
Við þekkjum þessi orð og skiljum hvernig á að nota þau í setningu til að kalla fram útkomu.
En það sem við þurfum að gera núna er að átta okkur á því að þessi orð eru mun formlegri í forritun heldur en í daglegu tali.
Eins og til dæmis: ,,Ertu ekki að hugsa um Jamie Lee Curtis?''
Í íslensku er hægt að svara þessari spurningu með ,,já ég er ekki að hugsa um hana'' eða ,,nei ég er ekki að hugsa um hana'' og bæði skilst, einnig er hægt að segja ,,jú ég er að hugsa um hana''.
Forritunarmál eru ekki tungumál, þau eru formleg og því er engin tvíræðni í boði.

Skoðum því nú hvað það þýðir að nota skilyrðissetningar (e. conditional statements) í Python með lykilorðunum \textbf{if - elif - else}, sem verða þó kynnt annarri röð.

\subsection{if}\index{if}
Fyrsta lykilorðið sem við tökum fyrir er \textbf{if}, þar sem ekki er hægt að búa til skilyrðissetningu án þess.
Og nú þurfum við að huga að því hvernig kóðinn okkar er uppsettur.
Það sem á að framkvæma undir ef setningunni/if yrðingunni er inndregið um fjögur bil eða einu sinni á ,,tab'' takkann \todo{íslenskt orð}.
Eina sem ræður því hvað fer mikið af kóða undir hverja yrðingu er hóf og skynsemi.
Við sjáum svo í kafla \ref{uk:hreiðrun} um hreiðrun hvers vegna það er mikilvægt að skilyrðissetningar séu skýrar.

Góð venja er að búa til skilyrðissetningar þar sem aðalvirknin á sér stað inni í if yrðingunni, þannig að segðin sem fer þar inn passi við hvað eigi að framkvæma.
Ef við tökum aftur dæmið um mjólkina og búðarferðina í mynd \ref{fig:flæðirit} og skoðum hvernig flæðiritið breytist eftir því hvernig við orðum skilyrðin.
Það er að við skoðum hvernig uppsetningin á flæðiritinu verður bjöguð ef við orðum spurninguna með játun en ekki neitun: Ef það er til mjólk vil ég athuga hvort það sé til kex ef svo er vil ég vera heima, annars kaupi ég kex ef það er til mjólk en ekki kex og annars kaupi ég mjólk ef það er ekki til mjólk.
Þetta er kannski ekki nógu flókin setning til þess að valda þeim hugrhrifum sem ætlast er til, en við sjáum að til þess að komast að þeim endapunkti sem aðaláherslan er á ,,vera heima'' þar sem hún er fyrsti endapunkturinn okkar þá þurfum við að fara í gegnum tvær spurningar.
Svo með því að orða spurninguna öðruvísi erum við búin að setja upp skilyrðissetninguna upp þannig að mólkurstaðan er núna ekki lengur í forgrunni, við virðumst frekar vera að reyna að halda okkur heima.

\begin{figure}[H]
	\centering
	\begin{tikzpicture}[node distance=2cm]
	\node (mjolk) [decision] {Mjólk er til};
	\node (kex) [decision, right of=mjolk, xshift=2cm] {Kex er til};
	\node (bud1) [process, below of=mjolk, yshift=-4cm] {Kaupa mjólk};
	\node (bud2) [process, below of=kex, yshift=-2cm] {Kaupa kex};
	\node (bud3) [process, right of=kex, xshift=2cm] {Vera heima};
	\draw [arrow] (mjolk) -- node[anchor=south] {satt} (kex);
	\draw [arrow] (mjolk) -- node[anchor=west] {ósatt} (bud1);
	\draw [arrow] (kex) -- node[anchor=west] {ósatt} (bud2);
	\draw [arrow] (kex) -- node[anchor=south] {satt} (bud3);
	\end{tikzpicture}
	\caption{Hér sést hvernig setningin: ,,Ef það er til mjólk vil ég athuga hvort það sé til kex ef svo er vil ég vera heima, annars kaupi ég kex ef það er til mjólk en ekki kex og annars kaupi ég mjólk ef það er ekki til mjólk.'' má setja fram sem fæðirit.
	Þetta veldur því að aðaláherslan virðist nú vera að komast að því hvort eigi að kaupa kex eða vera heima og mjólkurstaðan er athuguð fyrst af einhverri ástæðu.
	Setningin í heild er frekar ruglingsleg og hún kom mun betur út í flæðiritinu á mynd \ref{fig:flæðirit}.
	Þó áherslan sé önnur er niðurstaðan sú sama, það er á ábyrgð forritara að skrifa kóða sem er læsilegur og skiljanlegur.}
	\label{fig:flæðirit-neitun}
\end{figure}

Skoðum nú kóðabút \ref{lst:if} og hvað er átt við með réttum inndrætti, hér er einungis sýnt if setning ein og stök.
Við sjáum í næsta undirkafla hvað við getum gert ef við förum framhjá if setningunni okkar og viljum gera eitthvað í því tilfelli.

\begin{lstlisting}[caption=if notað, label=lst:if]
m = True # m er yrðingin hvort að til sé mjólk eða ekki, við gefum okkur í upphafi að til sé mjólk því er m upphafsstillt sem True

if(not m):
	# fara í búð
	print('við fórum í búðina og keyptum mjólk')
	
	# Við áttum mjólk svo þetta prentast ekki, við förum framhjá aðgerðunum sem eru undir if skipuninni
\end{lstlisting}

\subsection{else}\index{else}
Lykilorðið \textbf{else} má fylgja \textbf{if}, en það er ekki nauðsynlegt.
Hinsvegar verður að vera eitthvað ef til þess að það geti verið eitthvað annars.
Setningin ,,annars kaupi ég mjólk'' er ekki sérlega vitræn því að okkur vantar alveg fyrri hlutann.
Einnig er ekki sérlega vitrænt að segja ,,ég kaupi mjólk ef vantar annars kaupi ég kex annars kaupi ég te annars...''.
Því er einungis hægt að setja eitt annars við hvert ef, sjáum kóðabút \ref{lst:else}.
Sú klausa keyrist einungis þegar ef setningin sem hún hangir fyrir neðan keyrist ekki, og það eru einu skilyrðin.
Það þarf ekki að spyrja neinnar spurningar sem er metið sem boolean gildi til að keyra else, það mun alltaf keyrast þegar segðin í ef yrðingunni var ósönn.

\begin{lstlisting}[caption=else notað, label=lst:else]
m = True # m er yrðingin hvort að til sé mjólk eða ekki, við gefum okkur í upphafi að til sé mjólk því er m upphafsstillt sem True

if(not m):
	# fara í búð
	print('við fórum í búðina og keyptum mjólk')
else:
	# takið eftir að hér var ekkert skilyrði athugað
	# við förum hingað inn ef skilyrðið fyrir ofan var ekki keyrt
	print('vera heima')

# Við áttum mjólk svo við fáum út 'vera heima'

if(3 < 4):
	print("þrír er minna en fjórir")
else:
	print('ég fer ekki hingað inn, því 3 er vissulega minna en fjórir, en það er gott að vera við öllu búin')
\end{lstlisting}

Að geta sett svona annars-klausu er mikilvægt því að við viljum geta brugðist við ef upphaflega skilyrðið okkar er ósatt.
Við viljum geta tekið á öðrum tilfellum heldur en bara upphafsskilyrðinu okkar.

\subsection{elif}\index{elif}
En hvað ef við viljum geta tekið á einhverju sérstöku tilfelli, sem kemur einungis upp í ákveðnum aðstæðum?
Við viljum ekki bara grípa það að inngangspunkturinn okkar var ósannur heldur viljum við einnig skoða eitthvað fleira?
Þarna kemur setningin um mjólkina, kexið og búðarferðirnar aftur inn.
Þá getum við sagt ,,ef það er ekki til mjólk, fer ég í búð, \textbf{annars ef} það er ekki til kex fer ég í búð og kaupi kex, nú annars er engin ástæða til að fara í búðina og ég verð bara heima''.
Sjáum þetta forritað í kóðabút \ref{lst:elif}.

Skilyrðissetningar eru settar upp þannig að það verður að vera eitt \textbf{if} svo mega koma núll eða fleiri \textbf{elif} og að lokum má setja 0 eða 1 \textbf{else}.
Þetta er eins og málfræðilegur skilningur okkar er á tungumálinu, við megum hengja endalaust af annars ef klausum inn í setningarnar okkar, þær verða þá bara erfiðari að skilja.

\begin{lstlisting}[caption=elif notað, label=lst:elif]
m = True # m er yrðingin hvort að til sé mjólk eða ekki, við gefum okkur í upphafi að til sé mjólk því er m upphafsstillt sem True
k = False # k er yrðingin hvort að til sé kex eða ekki, við gefum okkar að í upphafi sé ekki til neitt kex og k er því False


if(not m):
	# fara í búð
	print('við fórum í búðina og keyptum mjólk')
elif(not k):
	# keyrist ef not m var metið sem False OG not k metið sem True
	print('við fórum í búðina og keyptum kex')
else:
	# takið eftir að hér var ekkert skilyrði athugað
	# við förum hingað inn ef skilyrðið fyrir ofan var ekki keyrt
	# tökum eftir að það skilyrði er nú elif setningin okkar
	print('vera heima')

# Við áttum mjólk en við áttum ekki kex, svo not k var metið sem True og við fáum svo við fáum 'við fórum í búðina og keyptum kex'

# Við förum ekki niður í else nema að allt fyrir ofan sé ósatt.

if(5 < 4):
	print("fimm er minna en fjórir")
elif(4 < 4):
	print("fjórir er minna en fjórir!")
elif(3 < 4):
	print("þrír er minna en fjórir")
elif(2 < 4):
	print("þetta prentast ekki því að ég get bara farið inn í eina klausu í hverri skilyrðissetningu")
	print("og það gerðist hérna fyrir ofan, svo þetta skilyrði verður ekki keyrt")
else:
	print('ég fer ekki hingað inn, því 3 er vissulega minna en fjórir, en það er gott að vera við öllu búin')
\end{lstlisting}

\subsection{Hreiðrun}\index{Hreiðrun}\label{uk:hreiðrun}
Hreiðrun (e. nesting) þýðir að setja endurtekið undir eitthvað annað, eins og babúska dúkkur eða að pakka gjöf inn í mörg lög af gjafapappír.
Í forritun þýðir hreiðrun að yrðing af einhverri tegund tilheyri og sé keyrð innan í yrðingu af sömu tegund.
Við getum hugsað þetta í samhengi við skilyrðissetningar að við séum með innri skilyrðissetningar sem þarf einnig að meta til þess að komast að niðurstöðu.
Skoðum þetta aftur í samhengi við mjólkurkaupin nema nú bætum við því við að við eigum bara ákveðið mikinn pening sjá kóðabút \ref{lst:nesting}.

\begin{lstlisting}[caption=Hreiðrun, label=lst:nesting]
m = True # m er yrðingin hvort að til sé mjólk eða ekki, við gefum okkur í upphafi að til sé mjólk því er m upphafsstillt sem True
k = False # k er yrðingin hvort að til sé kex eða ekki, við gefum okkar að í upphafi sé ekki til neitt kex og k er því False
p = 100 # p eru hversu mikinn pening við erum við á okkur.

if(not m):
	# fara í búð, ef við komumst hingað inn þá var m False og not m var þá True
	# nú sjáum við hvað þarf að gerast til þess að við getum keypt mjólkina ef við komumst hingað inn
	if(p > 200):
		print('við fórum í búðina og keyptum mjólk því við vorum með nógu mikinn pening')
	else:
		print('okkur vantaði mjólk en við vorum ekki með nógu mikinn pening')
elif(not k):
	# keyrist ef not m var metið sem False OG not k metið sem True
	if(p > 99):
		print('við fórum í búðina og keyptum kex því við vorum með nægan pening')
	else:
		print('okkur vantaði kex en við vorum ekki með nægan pening')
else:
	# takið eftir að hér var ekkert skilyrði athugað
	# við förum hingað inn ef skilyrðið fyrir ofan var ekki keyrt
	# tökum eftir að það skilyrði er nú elif setningin okkar
	print('vera heima og geyma allan peninginn')

# Við áttum mjólk sem var heppilegt, því að við hefðum ekki getað keypt hana
# Við áttum ekki kex, og svo heppilega vildi til að við vorum með nóg fyrir kexpakka og því fáum við niðurstöðuna:
# 'við fórum í búðina og keyptum kex því við vorum með nægan pening'
\end{lstlisting}

Hreiðrun er gagnleg því að við viljum skoða eitthvað innra skilyrði aðeins ef ytra skilyrðinu er mætt.

\section{Inntak}\index{Inntak}
Nú höfum við verið að skoða spurningar og svör við þeim sem við skráðum sjálf.
Það sem við viljum geta gert er að spyrja notandann að einhverju og geta gert eitthvað byggt á því svari.
Við þurfum að fá inntak (e. input) frá notandanum.

Þá lærum við um nýtt innbyggt fall í Python sem heitir \texttt{input()}.
Það sem input() gerir er að það tekur við streng sem það birtir á úttaki (e. output) þar sem notandinn sér það og getur sett eitthvað inn.
\todo{er ég ekki búin að tala um output?}
Að nota input() í skipanalínu gefur okkur nýja línu til að svara, að nota input() í Jupyter Notebooks gefur okkur lítinn glugga til að skrifa svarið okkar í fyrir neðan selluna þar sem input() skipunin er keyrð.
Skoðum kóðadæmi í kóðabút \ref{lst:input}, þar sem við geymum svarið frá notanda í breytu og viðfangið sem við setjum inn í fallið er strengur sem inniheldur spurninguna sem notandinn sér.
Vert er að athuga að allt sem er sett inn sem svar í þennan glugga er kastað í streng, það er breytan sem við notum til að taka á móti því sem input() fallið skilar er af týpunni strengur.
Ef við viljum geta spurt notandann um tölustafi þurfum við að kunna að kasta á milli gagnataga sjálf og við sjáum hvernig það er gert í kóðabút \ref{lst:kast}


\begin{lstlisting}[caption=input() fallið notað, label=lst:input]
svar = input('skrifaðu nafnið þitt')

# hér birtist gluggi fyrir okkur til að skrifa eitthvað inn í sem er merktur sem 'skrifaðu nafnið þitt'
# ég skrifa nafnið mitt og ýti á vendibil (e. enter) takkann og þá er svarið mitt komið í breytuna svar

print('halló', svar, 'mikið heitir þú fallegu nafni')

# Fyrir inntakið Valborg myndi nú prentast:
# 'halló Valborg mikið heitir þú fallegu nafni'
\end{lstlisting}

Þetta er mikilvægt fall fyrir okkur að skilja og nota á þessu stigi málsins, því að við verðum að átta okkur á því að þegar við forritum þá erum við miklu meira að vinna með breytunöfn heldur en gögn sem við getum horft á.
Í kóðabút \ref{lst:input} þá kemur hvergi fram í kóðanum að nafnið sé Valborg, og það getur verið hvað sem er, við prentum bara út það sem notandinn gaf okkur án þess að vera eitthvað að skoða hvað það er.
Oft vilja byrjendur horfa á gögnin sín og setja inn niðurstöður fyrir tölvuna, til dæmis verkefnið ,,búðu til breytu sem inniheldur nafn og prentaðu út breytuna ásamt strengnum "halló:"'' endar í kóða eins og sést í kóðabút \ref{lst:byrjendur}

\begin{lstlisting}[caption=Oft forðast byrjendur að nota breytur og treysta meira á að sjá hvað ætti að koma út, label=lst:byrjendur]
# Búðu til breytu sem inniheldur nafnið þitt:
nafn = 'Valborg'
# prentaðu hana út ásamt "halló:"
print('Halló: Valborg') # hér er ekkert verið að nota breytuna nafn heldur horfir forritarinn á kóðann og sér að tölvan eigi að prenta út 'Valborg' og setur það inn handvirkt.

# Annað dæmi væri að finna tákn sem er í miðjum streng
strengur = "þessi strengur hefur 29 tákn!"
# miðjan í þeim streng er því í tákni 14 sem tölvan getur náð í með len(strengur)//2
# og getur því fundið miðjutáknið með 
strengur[len(strengur)//2] # skilar okkur tákninu í vísi 14

# en hér er einnig hægt að taka nokkur skref sem er algengt að byrjendur geri
print(len(strengur)) # þetta skilar okkur 29
print(29/2) # þetta skilar okkur 14.5 eða print(29//2) sem skilar 14
print(strengur[14]) # þetta skilar okkur tákninu í miðjunni

\end{lstlisting}

Eins og sést í kóðabút \ref{lst:byrjendur} að til að komast að því að setja vísi 14 inn og fá táknið þarf að keyra fyrst línuna len(strengur) til þess að sjá þá tölu og svo þarf að deila þeirra tölu með tveimur til að finna miðjuna og svo þarf handvirkt að setja þá tölu inn eftir að hafa breytt henni í næstu heilu tölu.
Það sem er að gerast er ekkert rangt, það er hinsvegar mikil vannýting á því sem tölvan getur gert fyrir okkur og eykur vinnuna fyrir okkur sjálf umtalsvert því að það þarf að keyra hvert skref í kóðabútnum fyrir sig til að komast að því hvað eigi að gera í næsta skrefi, í stað þess að gera það í einni línu eins og í línu 10.

\subsection{Kastað á milli gagnataga}
Til þess að geta unnið með gögn eins og þá týpu sem við viljum þurfum við að læra að kasta á milli taga/týpna (e. typecasting).
Þetta þýðir að við látum vélina umrita gögnin okkar yfir í annað gagnatag, sem er einungis hægt ef að gögnin eru sambærileg týpunni sem á að kasta í.

Þá koma lykilorðin sem við höfum lært fyrir týpurnar okkar að gagni.
Við þekkjum núna strengi með lykilorðið \textbf{str}, heiltölur með lykilorðið \textbf{int}, fleytitölur með lykilorðið \textbf{float} og lista með lykilorðið \textbf{list}.
Þá notum við lykilorðið eins og fall og setjum inn í fallið sem viðfang það sem á að verða að því gagnatagi sem lykilorðið segir til um.
Sjáum í kóðabút \ref{lst:kast} hvernig á að fara að þessu.

Nú höfum við séð að input() fallið skilar alltaf til okkur gögnum af taginu/týpunni strengur.
En við viljum geta kannski unnið með inntakið frá notandanum sem tölu.
Ef að strengurinn inniheldur einungis tölur á bilinu 0-9 er hægt að geyma hann sem heiltölu eða fleytitölu, ef hann inniheldur einungis tölur á bilinu 0-9 og nákvæmlega einn punkt er hægt að geyma hann sem fleytitölu.


\begin{lstlisting}[caption=Hvernig á að kasta á milli gagnataga, label=lst:kast]
# Þessi strengur inniheldur bara tölur
talnastrengur = "123"

# og því má kasta honum í heiltölu eða fleytitölu
int(talnastrengur)
# þetta skilar okkur tölunni 123

float(talnastrengur)
# þetta skilar okkur tölunni 123.0

fleytitolustrengur = "123.0"

float(fleytitolustrengur) 
# þetta skilar okkur tölunni 123.0

int(fleytitolustrengur)
# þetta skilar okkur villu því að heiltölur eru ekki með punkti þó að þetta sé í grunninn sama tala og 123

# Nánast öllu má svo kasta í streng, þá er hverju tákni úthlutað sætisvísi og geymt eins og það er skrifað.
str(123)
# þetta skilar okkur "123"

str([1,2,3])
# þetta skilar okkur "[1,2,3]"

# Það sem gerist þegar við köstum í lista er að hverju ítranlegum hluta af gögnum er varpað í stak í listanum
# Skoðum betur í kafla um lykkjur hvað ítranlegt þýðir.

list("123.0")
# þetta skilar okkur ["1", "2", "3", ".", "0"]

list(123.0)
# Þetta skilar okkur villu því að tölur eru ekki ítranlegar, þær hafa ekki sætisnúmer (ekki eins og tölvan skilgreinir þau). 
\end{lstlisting}

Nú þar sem við vitum að við getum fengið streng í hendurnar frá notanda, vitandi það að við báðum um tölu, getum við leyft okkur að kasta strengnum í það talnatag sem okkur hentar.
Við sjáum svo í seinni hluta bókarinnar hvernig á að taka á mismunandi tilfellum og reyna á eitthvað sem gæti valdið villu án þess að það skemmi fyrir okkur, en núna ætlum við að láta sem að við getum treyst notendum til að gefa okkur inntak sem samræmist því sem við báðum um. \todo{vísa í kafla um try except}

Ástæðan fyrir því að vilja kasta á milli taga er til að geta beitt þeim aðgerðum og aðferðum sem eru í boði fyrir það gagnatag sem við sækjumst eftir að nota, til dæmis er ekki hægt að sækja þriðja tölustafinn í heiltölu en ef við köstum henni í streng getum við sótt táknið í sætisnúmeri 2 og fengið þannig þriðja tölustafinn.

\begin{itarefni}
\textbf{Dæmi um notkun á kasti milli taga}\\
Seinna munum við sjá gagnatýpuna mengi, ein gagnleg notkun á kasti milli taga væri að kasta lista í mengi til að losna við tvítekningar og kasta menginu svo í lista aftur.\\
\texttt{a = [1, 2, 2]}\\
\texttt{b = set(a)}\\
\texttt{a = list(b)}\\
Nú er listinn \texttt{a} orðinn að \texttt{[1,2]}, sjá meira um það í kafla \ref{k:sett}.
\end{itarefni}


\chapterimage{chapters5.png} % Chapter heading image

\chapter{Lykkjur}\index{Lykkjur}\label{k:lykkjur}
Til þess að keyra kóða endurtekið án þess að afrita og líma eða handvirkt keyra hann oft, þá notum við lykkjur.
Lykkjur eru kóðabútar sem keyrist endurtekið, eða ítrar, eftir ákveðnum reglum.
Þær lykkjur sem eru til í Python eru \textbf{for} lykkjur og \textbf{while} lykkjur.
For lykkjur keyra fyrir hvert stak í ítranlegum hlut eða hverja tölu á bili (keyra ákveðið oft, í mesta lagi).
While lykkjur keyra á meðan skilyrðið fyrir keyrslu þeirra er satt (geta keyrt að ,,eilífu'').
Nöfnin á for og while verða ekki þýdd sérstaklega í þessari bók, en við segjum t.a.m. ,,að gera eitthvað á meðan'' eða ,,gera eitthvað fyrir hvert stak''.
\begin{wrapfigure}{I}{0.15\textwidth} %i o r l 
	\begin{center}
		\includegraphics[scale=0.5]{doooodles_valborg-10.png}
	\end{center}
\end{wrapfigure}
Við ætlum að kynnast því til hvers þær eru ætlaðar og hvers þær eru megnugar, í hvaða tilfellum á að nota hvora fyrir sig og lykilorð sem gera notkun þeirra öflugri.
Byrjum á því að skoða til hvers ,,að lykkja'' og hvað það eiginlega þýðir.
Það að nota lykkju þýðir að skrifa forritsbút sem keyrir endurtekið.

Tökum dæmi úr daglegu lífi; ef við viljum framkvæma einhverja aðgerð eins og að vaska upp búum við til reglu eins og að setja fyrst upp uppþvottahanska, láta vatnið renna og stafla öllu sem er óhreint við hliðina á vaskinum. 
Svo viljum við endurtaka aðgerðina að þrífa hvern hlut sem er öðru megin við vaskinn, þar til þeir eru allir komnir hreinir hinu megin.
Endurtekningin þarna er að taka upp hvern óhreinan hlut og þrífa hann.
Þá gætum við sagt að fyrir hvern hlut sem er hægra megin, viljum við þrífa hann og setja svo vinstra megin (fer eftir því hvernig vaskurinn snýr) og hætta þegar engir hlutir eru eftir hægra megin.
Þetta ferli að framkvæma sömu aðgerð á stök í mengi er einmitt það sem for lykkja getur gert.

Tökum annað dæmi úr daglegu lífi; ef við ætlum að bíða eftir einhverjum og framkvæma svo einhverja aðgerð þegar viðkomandi kemur þá myndum við væntanlega bíða þangað til að viðkomandi kemur.
Svo á meðan viðkomandi er ekki enn kominn þá höldum við áfram að bíða.
En þar sem við erum ekki tölvur þá myndum við ekki bíða endalaust, við myndum gefast upp.
Þetta ferli að halda áfram að framkvæma einhverja aðgerð þangað til að eitthvað skilyrði á ekki við er það sem while lykkja getur gert.

\section{For}\index{For lykkjur}
For lykkkjur nota lykilorðið \textbf{for} ásamt lykilorðinu \textbf{in}.
Það sem in gerir þegar það er notað eitt og sér er að spyrja hvort að \textit{eitthvað} sé ,,í'' \textit{einhverju öðru} (sjá kóðabút \ref{lst:lykkjur-in}) en sem hluti af for lykkju þá er það in sem úthlutar lykkjunni næsta staki úr menginu til að skoða.
Þannig að þetta býr til segð sem skilar sanngildi eða einu tilteknu tákni eða staki úr hlut.
Nú skulum við líta á kóðabút \ref{lst:lykkjur-for} til að átta okkur á því hvernig lykkjan er notuð, hvernig við beitum inndrætti til að skilgreina stef lykkjunnar (það sem tilheyrir henni) og hvernig skilyrðissetningar bætast við þetta.

\begin{lstlisting}[caption=Lykilorðið in, label=lst:lykkjur-in]
print('er táknið a í strengum Valborg?')
print("a" in "Valborg")

print('er táknið x í Valborg?')
print("x" in "Valborg")
\end{lstlisting}
\lstset{style=uttak}
\begin{lstlisting}
er táknið a í strengum Valborg?
True
er táknið x í Valborg?
False
\end{lstlisting}
\lstset{style=venjulegt}

Eins og sést í kóðabút \ref{lst:lykkjur-in} þá virkar in nokkuð svipað því sem orðið í gerir í setningu.
Ekki gleyma þessu lykilorði við lykkjugerðina.
Sjá ítarefni í lok kaflans um önnur lykilorð sem gagnast við forritun á lykkjum.
Í næstu kóðabútum verður sýnd grunnvirkni for-lykkjunnar.
\vspace{1cm}
\begin{itarefni}
	\textbf{Nánar um \texttt{in} og vísa}\\
	Þegar þetta orð er notað í for lykkjum er þó ekki verið að setja fram segð heldur er verið að úthluta einhverri hlaupandi breytu tilteknu gildi úr ítranlegum hlut.
	Það að hlutur sé ítranlegur þýðir að við getum horft á hann stak fyrir stak, skoðað eitt gildi úr honum í einu.
	Eins og strengur hefur vísa þá getum við horft á hvert tákn fyrir sig með því að rúlla í gegnum vísana frá 0 og út í enda (eða einhverri annarri röð).
	Listar eru einnig ítranlegir þar sem stökin í listum hafa vísa og því má horfa á hvert stak fyrir sig í heild sinni, hvort sem það er annar listi eða ein stök tala.
	Heiltölur, fleytitölur og sanngildi eru ekki ítranleg.
\end{itarefni}
\todo{formatting}
\vspace{2cm}
%\begin{wrapfigure}{}{0.4\textwidth} %i o r l 
	\begin{center}
		\includegraphics[scale=1.3]{doooodles_valborg-07.png}
	\end{center}
%\end{wrapfigure}

\newpage \todo{formatting}

\begin{lstlisting}[caption=For lykkjur kynntar, label=lst:lykkjur-for-kynning]
# við byrjum á að skilgreina streng
strengur = "Valborg"
print(strengur, "til viðmiðunar")

for stafur in strengur:
	print(stafur)
	
print()
print('lykkjan er búin')
\end{lstlisting}
\lstset{style=uttak}
\begin{lstlisting}
Valborg til viðmiðunar
V
a
l
b
o
r
g

lykkjan er búin
\end{lstlisting}

Í kóðabút \ref{lst:lykkjur-for-kynning} hvernig rúllað er í gegnum strenginn \texttt{Valborg} með breytunni \texttt{stafur}.
Sú breyta er búin til í línu 5 þegar lykkjan er búin til, það þurfti ekki að skilgreina hana áður.
Það er vegna þess að breytan er skilgreind inni í lykkjunni fyrir okkur, hún er áfram aðgengileg en er ósköp gagnslaus eftir keyrsluna svo okkur er alveg sama um hana, hún er svokölluð \emph{tímabundin} (e. temporary) breyta sem hættir að skipta máli eftir notkun innan lykkjunnar.

Allt það sem tilheyrir svo lykkjunni eða á að gerast í hverri keyrslu hennar er inndregið undir henni.
Línur 9 og 10 eru ekki hluti af stefi lykkjunnar og keyrast því ekki nema einu sinni, eins og lína 3 prentast bara einu sinni.

Það sem prentast á úttakið úr línu 6, breytan \texttt{stafur}, er hvert tákn fyrir sig í þeirri röð sem það kemur fyrir í strengnum sem verið er að ítra í gegnum.
Lesið yfir þennan kóðabút og gerið tilraunir á eigin spýtur til að átta ykkur á því hvað það er sem er að gerast.

Hvað gerist ef inndrátturinn breytist?
Hvað prentast þá út?
Hvað gerist ef eitthvað annað orð er sett í staðinn fyrir \texttt{stafur}?
En \texttt{strengur}?
Haldið áfram að gera tilraunir með þetta þangað til að þið áttið ykkur betur á því hvernig þetta hangir saman.

Sjáum nú hvernig megi flétta skilyrðissetningar inn í þetta.

\lstset{style=venjulegt}
\begin{lstlisting}[caption=For lykkja og skilyrðissetningar, label=lst:lykkjur-for-skil]
for stafur in strengur:
	# við vitum að það er a í Valborg svo þetta mun einhvern tímann gerast
	if(stafur == 'a'):
		print(stafur)
\end{lstlisting}
\lstset{style=uttak}
\begin{lstlisting}
a
\end{lstlisting}

Í kóðabútum \ref{lst:lykkjur-for-kynning} og \ref{lst:lykkjur-for-skil} vorum við að vinna með sama strenginn, en í fyrra skiptið prentaðist hann allur út en í seinna skiptið fengum við bara eitt stakt tákn út.
Munurinn er sá að í kóðabút \ref{lst:lykkjur-for-skil} þá þegar við vorum komin með táknið í hendurnar vildum við gera eitthvað við það svo við spurðum er þetta tákn jafngilt \texttt{a}, einungis í því tilfelli vildum við prenta eitthvað út.
Við ákváðum að prenta út táknið sem við vorum að skoða sem er geymt í breytunni \texttt{stafur} en hefðu hæglega getað gert eitthvað annað.

Við sjáum einnig að þegar við settum inn skilyrðissetningu þá bættist við annar inndráttur, það er vegna þess að inndráttarnotkunin breytist ekki sama hvar við erum að nota kóða sem krefst inndráttar heldur dregst kóðinn bara lengst til hægri eftir því sem við förum innar. 
Því getur verið ágætt að takmarka hreiðrun til þess að kóðinn sé sem læsilegastur.

Gerið nú tilraunir til að átta ykkur betur á því hvernig megi skoða ítranlegan hlut en framkvæma einungis aðgerð ef eitthvað skilyrði á við.
Hvað gerist ef við breytum skilyrðissetningunni?
En ef við breytum því sem er undir henni?
Hvað gerist ef við bætum við \textit{annars} klausu?
Má setja eitthvað inn í stef lykkjunnar á eftir skilyrðissetningunni?

Næst skoðum við annan ítranlegan hlut með for lykkju, það er listi.

\lstset{style=venjulegt}	
	
\begin{lstlisting}[caption=For lykkja með lista, label=lst:lykkjur-for-listi]
listinn_minn = [0, "strengur", [0, 1, 2]]

for x in listinn_minn:
	# nú hleypur x í gegnum stökin í breytunni listinn_minn
	print(x)

\end{lstlisting}
\lstset{style=uttak}
\begin{lstlisting}	
0
strengur
[0, 1, 2]
\end{lstlisting}



\lstset{style=venjulegt}

Tökum eftir í kóðabút \ref{lst:lykkjur-for-listi} þá fær \texttt{x} það gildi sem er næst í röðinni í listanum, og það skiptir ekki máli af hvaða týpu gögnin eru.
Nafnið á breytunni er ekki lýsandi, ekki eins og \texttt{stafur} eða \texttt{strengur} í kóðabút \ref{lst:lykkjur-for-skil}, ástæðan fyrir því að nafnið x varð fyrir valinu er til að sýna lesendum að þetta er breytuheiti eins og hvert annað og lútir sömu venjum og við sáum í kafla \ref{k:tolur}.
Listinn í línu 1 inniheldur gögn af þremur týpum, breytan \texttt{x} kippir sér ekkert upp við það og birtir gögnin í þeirri röð sem þau bárust.

Prófið ykkur nú aðeins áfram og reynið í staðinn fyrir \texttt{listinn\_minn} að setja einhvern annan lista, sem er ekki geymdur í breytu.
Prófið nú að setja inn skilyrðissetningu þarna og nota \texttt{type()} með skilyrðissetningu til að prenta einungis þau x sem eru strengir.

For lykkjur eru því helst gagnlegar þegar við vitum hversu mörg stök lykkjan á mörgulega að skoða, því til stuðnings ætlum við að skoða innbyggða fallið \textbf{\texttt{range()}} sem gefur okkur hlut af tölum á ákveðnu bili.
Fallið \texttt{range()} tekur við sömu viðföngum eins og hornklofarnir\footnote{það eru í raun ekki viðföng, í "strengur"[1:5:2] eru 1, 5, og 2 ekki viðvöng beinlínis heldur vísar.} þegar við sóttum nokkur tákn upp úr streng eða lista, þau eru þó aðgreind með kommum\footnote{Við höfum áður séð viðföng notuð í falli eins og print() fallið, þar sem við getum prentað margt út svo lengi sem við setjum kommur á milli.}.

Viðföngin í \texttt{range(a,b,c)} fallið eru heilartölur og þeim er raðað svona:

\begin{enumerate}
	\item \textbf{a}: talan sem á að byrja að nota (hér má sleppa því að setja þetta inn því að sjálfgefið gildi er 0)
	\item \textbf{b}: talan sem á að hætta fyrir framan (þetta verður að setja inn, því að þetta er aðalatriðið)
	\item \textbf{c}: tala sem segir til um skrefastærðina (sjálfgefið gildi er 1 og þessu má sleppa), ef skrefastærð er tekin með þarf að velja upphafsstað (annars myndast tvíræðni)
\end{enumerate}

Ef við viljum leysa verkefni sem felst í því að finna odda tölur frá 0 og upp að 1000 hvernig myndum við fara að því?
En ef við viljum vera viss um að eitthvað gerist ákveðið oft?
Skoðum kóðabút \ref{lst:lykkjur-range-kynnt} þar sem fyrra verkefnið er leyst á tvo mismunandi vegu.

%\begin{wrapfigure}{}{0.4\textwidth} %i o r l 
\begin{center}
	\includegraphics[scale=0.4]{doooodles_valborg-02.png}
\end{center}
%\end{wrapfigure}

\begin{lstlisting}[caption=range() fallið kynnt með for lykkju, label=lst:lykkjur-range-kynnt]
for tala in range(6):
	if(tala%2 != 0):
		print(tala)		

print()

for tala in range(0,5,2):
	print(tala)

print()

for x in range(3):
	print('bílalúgudýraspítali', x)
\end{lstlisting}		
		
\lstset{style=uttak}
\begin{lstlisting}
1
3
5

0
2
4

bílalúgudýraspítali 0
bílalúgudýraspítali 1
bílalúgudýraspítali 2
\end{lstlisting}

Aðalatriðið sem þarf að hafa í huga þarna í kóðabút \ref{lst:lykkjur-range-kynnt} er að for lykkjan hleypur í gegnum lista af tölum svo að við vitum alltaf hvar við erum stödd og við vitum hvað við keyrum lykkjuna oft.
Sjáum í línu 13 þá er \texttt{x} prentað og á úttakinu (línur 9-11) sést að það er hlaupandi númer sem byrjar í 0 og hættir í 2 sem er talan fyrir framan 3 og þar vildum við hætta.
Nú getið þið breytt þessum lykkjum til að skoða til dæmis sléttar tölur undir 1000 eða tölur deilanlegar með 17 undir 100.

En við þurfum ekki endilega að byrja í núll, sjáum í kóðabút \ref{lst:lykkjur-range-meira} hvernig hægt er að velja afmarkaðra talnabil og svo sjáum við í kóðabút \ref{lst:lykkjur-range-mest} að hægt er að telja afturábak.

Allt er þetta þó spurning um að finna það sem hentar því verkefni sem við erum að reyna að leysa.
Næstu sýnidæmi eru meira til þess fallin að sýna virkni \texttt{range()} fallsins og for lykkja yfirhöfuð án þess þó að vera að leysa einhver flókin vandamál.
Eftir að hafa séð þetta, gætuð þið prófað ykkur áfram og náð þannig tökum á þessu.

Það sem skiptir mestu máli til að ná ákveðinni leikni er að prófa sig áfram, gera tilraunir og þora að mistakast.

\lstset{style=venjulegt}
\begin{lstlisting}[caption=for lykkja og range() fallið með skilyrðissetningu, label=lst:lykkjur-range-meira]
for tala in range(10, 20):
	if(tala%3 == 0):
		print(tala)
\end{lstlisting}

\lstset{style=uttak}
\begin{lstlisting}
12
15
18
\end{lstlisting}

Eins og áður kom fram þá þarf að taka fram upphafspunkt ef nota á skrefastærð svo að í línu 1 í kóðabút \ref{lst:lykkjur-range-mest} fer ekki milli mála að byrja á fyrir framan töluna 10 og enda fyrir aftan töluna 20.
Svo það sem gerist í skilyrðissetningunni er að tölunni er kastað í streng og spurt er hvort að síðasta táknið í strengnum sé talan 9.
Ef svo er þá prentum við út töluna ásamt textanum \texttt{endar á 9}, svo þegar keyrslu lykkjunnar lýkur fáum við að sjá hvað breytan \texttt{tala} inniheldur.


\lstset{style=venjulegt}
\begin{lstlisting}[caption=for lykkja range() fallið notað til að telja aftur á bak, label=lst:lykkjur-range-mest]
for tala in range(100, 70, -1):
	if(str(tala)[-1] == '9'):
		print(tala, 'endar á 9')
print('lykkjan er búin, hvað er tala?', tala)
\end{lstlisting}

\lstset{style=uttak}
\begin{lstlisting}
99 endar á 9
89 endar á 9
79 endar á 9
lykkjan er búin, hvað er tala? 71
\end{lstlisting}
\lstset{style=venjulegt}

\subsection{Gagnleg lykilorð}\index{lykilord}\label{uk:lykkjulykilorð}
	Áður en lengra er haldið í hvernig á að beita lykkjum er ágætt að nefna nokkur grunn lykilorð sem hjálpa okkur gríðarlega.
	Þau eru \textbf{pass}, \textbf{continue} og, \textbf{break}.

\lstset{style=venjulegt}
\begin{lstlisting}[caption=Lykilorðið pass notað með for lykkju, label=lst:lykkjur-for-pass]
for x in range(15):
	if(x % 3 != 0):
		pass
	else:
		print('þetta gerðist fyrir töluna', x)
		
\end{lstlisting}
\lstset{style=uttak}
\begin{lstlisting}
þetta gerðist fyrir töluna 0
þetta gerðist fyrir töluna 3
þetta gerðist fyrir töluna 6
þetta gerðist fyrir töluna 9
þetta gerðist fyrir töluna 12
\end{lstlisting}

\lstset{style=venjulegt}
\begin{lstlisting}[caption=Lykilorðið continue notað með for lykkju, label=lst:lykkjur-for-cont]
for x in [1, 2, 59, 9, 53, 2]:
	if (x < 50):
		continue
	print(x)
\end{lstlisting}
\lstset{style=uttak}
\begin{lstlisting}
59
53
\end{lstlisting}
\lstset{style=venjulegt}
\begin{lstlisting}[caption=Lykilorðið continue notað með for lykkju, label=lst:lykkjur-for-break]
listi_af_tolum = [1,5,7,9,13,15,17,18]
for tala in listi_af_tolum:
	if(tala == 13):
		print("það er þrettán í listanum")
		break
	elif(tala != 13):
		continue
	else:
		print("þetta prentast aldrei")
		
\end{lstlisting}
\lstset{style=uttak}
\begin{lstlisting}
"það er þrettán í listanum"
\end{lstlisting}
\lstset{style=venjulegt}
Í kóðabútum \ref{lst:lykkjur-for-pass}, \ref{lst:lykkjur-for-cont} og, \ref{lst:lykkjur-for-break} sjáum við að lykilorðin geta gefið okkur möguleika á að hætta keyrslu, nota bara hluta úr kóða eða gefa okkur kost á að nýta staðhaldara þegar við vitum ekki hvaða kóði á að koma þangað.
Án þess að fara meira út í hvernig kóðinn fyrir þessi lykilorð virka þá er þess virði að nefna að þau eru ekki nauðsynleg í hverri lykkju sem við forritum hér eftir, þau eru gagnleg þegar þau eiga við og við þurfum að átta okkur á hvernær svo er.


\begin{itarefni}
\textbf{Nánar um lykkju lykilorðin}\\	
\begin{itemize}[leftmargin=*]
\item \textbf{pass} er lykilorð sem gerir ekkert, tölvan heldur áfram keyrslu sinni eins og ekkert hafi verið gert, nema að þarna er kóði sem er rétt inndreginn og gerir það að verkum að tölvan kvartar ekki yfir því að hafa búist við einhverju inndregnu en fengið ekkert.
Þetta notum við þegar við erum ekki viss hvað á að vera í lykkjunni og við setjum þetta orð inn svo að við getum haldið áfram með annað sem átti að forrita.
pass er gagnlegt sem staðhaldari (e. placeholder) þegar við erum ekki viss hvernig á að halda áfram en verðum að setja eitthvað því að annars fengjum við málskipanar villu (e. syntax error).
Þetta lykilorð má nota annarsstaðar en í lykkjum og er einnig gagnlegt sem staðhaldari í föllum.
\item \textbf{continue} er lykilorð sem lætur vélina stoppa þar sem hún er í lykkjunni, hunsa allt sem kemur á eftir því og fara efst í lykkjuna.
Continue er gagnlegt þegar kemur að því að það er bara ákveðin virkni sem á að framkvæma undir vissum aðstæðum og við viljum ekki að vélin geri allar aðgerðir sem koma fram í lykkjunni okkar.
Þetta lykilorð má einungis nota inni í lykkjum.
\item \textbf{break} hættir keyrslu lykkjunnar, ólíkt continue þá förum við alfarið út úr lykkjunni þegar kallað er í þetta lykilorð og keyrir vélin næst kóða sem er ekki inndreginn undir lykkjunni.
Þetta lykilorð má einungis nota inni í lykkjum.		
Þetta lykilorð getur reynst ómetanlegt þegar við skoðum while lykkjur.
\end{itemize}

\end{itarefni}
\comment{
	
	%%%%%%%%%%%%%%%% comment byrjar
	

	
	Skoðum nú lykilorðið in áður en lengra er haldið til þess að öðlast dýpri skilning á því hvernig for lykkjan virkar.
	
	
	

	
	
	
	%% búið að nota þetta !
	
	\section{Lykkjur}\index{Lykkjur}
	Til þess að keyra kóða endurtekið án þess að afrita og líma eða handvirkt keyra hann oft, þá notum við lykkjur.
	Lykkjur eru kóðabútar sem keyrist endurtekið eftir ákveðnum reglum, þær lykkjur sem eru til í Python eru for lykkjur og while lykkur.
	For lykkjur keyra fyrir hvert stak í ítranlegum hlut eða hverja tölu á bili (keyra ákveðið oft, í mesta lagi).
	While lykkjur keyra á meðan skilyrðið fyrir keyrslu þeirra er satt (geta keyrt að ,,eilífu'').
	
	%%%%%% comment endar
}


\section{While}\index{While lykkjur}
While lykkjur nota lykilorðið \textbf{while} og keyra ,,á meðan'' eitthvað skilyrði er satt.
Þær eru helst ganglegar þegar við vitum ekki hvað við viljum að lykkjan keyri lengi eða þegar við viljum að hún keyri endalaust nema annað sé tekið fram (t.d. með break).

\begin{wrapfigure}{i}{0.2\textwidth} %i o r l 
	\begin{center}
		\includegraphics{doooodles_valborg-11.png}
	\end{center}
\end{wrapfigure}
Skilyrðið fyrir keyrslunni er metið sem sanngildi, annað hvort með sanngildinu sjálfu eða segð sem skilar sanngildi.
Þá gefst okkur tækifæri á að forrita lausn á vanda eins og ,,ef það er enginn eftir í stofunni á að slökkva ljósið'' og forritið keyrir á meðan ,,einhver er eftir í stofunni''.
Þarna þurfum við ekki að gera annað en að fylgjast með aðstæðum.
While lykkjur eru vandmeðfarnar og harla líklegt að lenda í því að skrifa lykkju sem keyrir endalaust við fyrstu notkun.
Þær eru jafnframt öflugar til að leysa ýmsan vanda sem krefst þess að aðstæður hverju sinni séu skoðaðar.

Skoðum kóðabút \ref{lst:lykkjur-while-endalaust} til þess að sjá hvernig má auðveldlega lenda í vandræðum við gerð slíkra lykkja og hvernig uppsetning þeirra lítur út.


\begin{lstlisting}[caption=while lykkja sem keyrir að eilífu, label=lst:lykkjur-while-endalaust]
while(True):
	# inndreginn kóði sem tilheyrir lykkjunni - stef lykkjunnar
	pass
print('kemst ekki hingað því lykkjan er enn að keyra')
\end{lstlisting}
\lstset{style=uttak}
\begin{lstlisting}
# ef lykkjan okkar gerði eitthvað væri þessi bútur troðfullur
\end{lstlisting}
\lstset{style=venjulegt}
Lykkjan í kóðabút \ref{lst:lykkjur-while-endalaust} keyrir að eilífu vegna þess að skilyrðið fyrir henni er \texttt{True} og ekkert breytir því í stefi hennar.
Hægt er að stöðva vélina handvirkt í þessum aðstæðum\footnote{Í Jupyter Notebooks er það gert með Kernel -> Restart Kernel}.
En nú viljum við að þetta gerist ekki aftur svo við notum break lykilorðið.


\begin{lstlisting}[caption=while lykkja sem keyrir ekki að eilífu en hún gerir ekkert, label=lst:lykkjur-while-break]
while(True):
	# nú ætlum við að reyna að komast út
	break
print('vei við komumst út, en hvað kostaði það?')
\end{lstlisting}
\lstset{style=uttak}
\begin{lstlisting}
vei við komumst út, en hvað kostaði það?
\end{lstlisting}
\lstset{style=venjulegt}

Við komumst út úr lykkjunni, hún keyrði einu sinni og hætti strax keyrslu, ekki mjög gagnleg lykkja en hún keyrði allavega ekki að eilífu.
Annað sem við þurfum líka að hugsa um er að skilyrðið okkar sé alveg örugglega rétt skilgreint, að við séum að ná að fanga þær aðstæður sem við vildum halda í.
Skoðum næsta kóðabút þar sem skilyrðið mun aldrei verða satt og því mun stef lykkjunnar aldrei keyrast og breytan sem er þar skilgreind aldrei fá stað í minni, sem veldur villu þegar á að nota breytuna á eftir lykkjunni. 

\begin{lstlisting}[caption=while lykkja sem keyrir aldrei, label=lst:lykkjur-while-false]
while(3 < 2):
	print('þetta mun aldrei prentast því að stef lykkjunnar mun aldrei keyrast')
	x = 5
print(x)
\end{lstlisting}
\lstset{style=uttak}
\begin{lstlisting}
NameError                                 Traceback (most recent call last)
<ipython-input-21-8ba2a8c60ab2> in <module>
2         print('þetta mun aldrei prentast því að stef lykkjunnar mun aldrei keyrast')
3         x = 5
----> 4 print(x)

NameError: name 'x' is not defined
\end{lstlisting}
\lstset{style=venjulegt}

Takið sérstaklega eftir því hvað villuskilaboðin eru skýr í úttakinu á kóðabút \ref{lst:lykkjur-while-false}, að villan er nafnavilla, hún á sér stað í línu fjögur í kóðanum og að það er vegna þess að 'x' er ekki skilgreint þegar það er notað í línu 4.

Skoðum nú einhverja gagnlega lykkju.
Segjum að við skuldum 8.000 krónur og við ætlum að borga inn á skuldina okkar 1.000 krónur í einu.
Við viljum að sjálfsögðu hætta að borga þegar við skuldum ekkert lengur og auðvitað viljum við að skuldin okkar lækki.

\begin{lstlisting}[caption=while lykkja sem eitthvað vit er í, label=lst:lykkjur-while-skuld]
skuld = 8000
innborgun = 1000
while(skuld > 0):
	skuld = skuld - innborgun
	print('nú er skuldin', skuld)
\end{lstlisting}
\lstset{style=uttak}
\begin{lstlisting}
nú er skuldin 7000
nú er skuldin 6000
nú er skuldin 5000
nú er skuldin 4000
nú er skuldin 3000
nú er skuldin 2000
nú er skuldin 1000
nú er skuldin 0
\end{lstlisting}
\lstset{style=venjulegt}

Nú þegar við höfum skoðað haldbært dæmi um eitthvað sem vit er í skulum við skoða óhlutbundið dæmi þar sem við erum að vinna með hugmyndina um að slökkva ljósin í stofunni ef allir eru farnir.

\begin{lstlisting}
while(True):
	if(fjöldi nemenda er 0):
		slökkva ljós
		break
		
	fjöldi nemenda talinn aftur
\end{lstlisting}
\lstset{style=venjulegt}

Þetta er ekki alvöru Python kóði, heldur \emph{sauðakóði} (e. psudocode) sem kemur þó merkingunni til skila, að aðalatriðið er að vera í sífellu að skoða það hvort að engir nemendur séu eftir og telja þá aftur.
Þar kemur while lykkjan sterk inn, að við viljum gera eitthvað á meðan eitthvað ástand varir.
Takið eftir að talning nemenda fer fram inni í lykkjunni, ef sá hluti yrði færður einum inndrætti innar væri það ekki lengur hluti af stefi lykkjunnar og keyrðist þegar henni væri lokið (en henni myndi aldrei ljúka því að aldrei yrði komist inn í skilyrðissetninguna því að ekkert breytir fjölda nemenda innan lykkjunnar).

Hugsum okkur nú að nota segð fyrir eitthvað flóknara skilyrði en við sáum í kóðabút \ref{lst:lykkjur-while-false}.
Eins og við sáum á myndum \ref{fig:flæðirit} og \ref{fig:flæðirit-neitun} þá skiptir máli hvernig við orðum skilyrðin okkar, eins og setningin ,,á meðan það er óuppvaskaður diskur við hliðina á vaskinum eða við eldhúsborðið þá ætla ég að vaska upp'' hvernig yrði hún forrituð sem skilyrði inn í while lykkju?
Athugum að þarna erum við með rökvirkjann \emph{eða} og því þarf annað hvort að vera skítugur diskur við vaskinn eða á borðinu.
\begin{lstlisting}[caption=while lykkja óhlutbundin til að sýna rökvirkja, label=lst:lykkjur-while-or]
while(það er skítugur diskur við vaskinn eða það er skítugur diskur við borðið):
	vaska upp disk
print(allir diskar eru hreinir)
\end{lstlisting}
\lstset{style=venjulegt}

Hér sjáum við eitt sem vefst fyrir mörgum, það er að skilyrðið í línu 1 virðist vera óþarflega nákvæmt, til hvers að taka fram skítugur diskur tvisvar?
Það er vegna þess að segðinni ,,það er skítugur diskur við vaskinn'' er hægt að svara með já eða nei og sömuleiðis ,,það er skítugur diskur við borðið''.
En ef skilyrðið okkar hefði einungis verið ,,það er skítugur diskur við vaskinn eða borðið'' þá lendir vélin í því að fá í hendurnar segð sem hægt er að svara hægra megin við eða og svo ,,borðið'' hinu megin.
Hvernig á að svara ,,eða borðið''?
Það er ekki hægt, því að það er ekki skiljanleg spurning.
Því þarf að muna að hafa alltaf heila skýra segð sem hægt er að meta sem sanna eða ósanna.

Tökum dæmi um rökvirkjanotkun í skilyrði í lykkju í kóðabút.
Þar sem við viljum vita hvort að við séum með líkamshita á eðlilegu bili, eftir að hafa mælt það einu sinni í upphafi og svo mælum við reglulega eftir það.

\begin{lstlisting}[caption=while lykkja með og rökvirkjanum, label=lst:lykkjur-while-and]
hiti = 37.0
while(hiti < 37.6 and hiti > 36.0):
	# mælum hitann með þessari óvísindalegu aðferð
	hiti = hiti + 0.5
	print(hiti)
\end{lstlisting}
\lstset{style=uttak}
\begin{lstlisting}
37.5
38.0
\end{lstlisting}
\lstset{style=venjulegt}

Við sjáum að skilyrðið í kóðabút \ref{lst:lykkjur-while-and} er ekki með \emph{eða} heldur \emph{og}, það sem er verið að spyrja er ,,er hitinn á milli talnanna 36.0 og 37.6?''.
Þannig að fyrst er spurt er hitinn lægri en 37.6, svo er spurt hvort hann sé hærri en 36.0 og ef svarið við báðum spurningum er já þá hlýtur hitinn að vera á milli þessara talna. 

Annað sem má gera við while lykkjur er að koma fram við þær sem skilyrðissetningu sem má fá else klausu aftan við sig sem keyrist þegar skilyrði lykkjunnar verður ósatt.

\begin{lstlisting}[caption=Að nota else með while, label=lst:lykkjur-while-else]
x = 5
while(x > 1):
	print("talan er", x, "sem er stærra en 1")
	x -= 1 
else:
	print("nú er talan orðin 1 því 1 er ekki stærri en 1 -->", x)
	
\end{lstlisting}
\lstset{style=uttak}
\begin{lstlisting}
talan er 5 sem er stærra en 1
talan er 4 sem er stærra en 1
talan er 3 sem er stærra en 1
talan er 2 sem er stærra en 1
nú er talan orðin 1 því 1 er ekki stærri en 1 --> 1
\end{lstlisting}
\lstset{style=venjulegt}



%-------------------------------
\newpage
\section{Æfingar}
\begin{exercise}\label{lyk1}
	Búið til lykkju sem keyrir 100 sinnum og prentar út númer keyrslunnar.
\end{exercise}
\setboolean{firstanswerofthechapter}{true}
\begin{Answer}[ref={lyk1}]
Þetta er mjög svipað kóðabút \ref{lst:lykkjur-range-kynnt}, síðustu lykkjunni.
\begin{lstlisting}
for x in range(100)
	print(x)\end{lstlisting}
\end{Answer}
\setboolean{firstanswerofthechapter}{false}

\begin{exercise}\label{lyk2}
Búið til lista sem inniheldur einungis tölur, lykkjið í gegnum allan listann og leggið saman tölurnar.
Prentið út summu listans að keyrslu lokinni.
\end{exercise}
\begin{Answer}[ref={lyk2}]
Athugum hér að til þess að geta haldið utan um summu þurfum við að skilgreina breytu áður en við gerum lykkjuna, sömuleiðis listann af tölunum.
Talnalistinn getur verið með hvaða tölum sem er, heiltölum eða fleytitölum.
Summuna verðum við að skilgreina og þar sem við höfum ekki séð neina tölu þá skilgreinum við hana sem núll í upphafi.
Svo lykkjum við í gegnum listann okkar og notum breytuna sem hleypur í gegnum listann til að bæta við summuna.
Þegar lykkjan er búin þá erum við ekki lengur í sama inndrætti og þá ætlum við að prenta út summuna okkar, þá prentast hún bara einu sinni.
 
\begin{lstlisting}
listi = [1,2,3,4,5,6,7,8,9,10]
summa = 0
for tala in listi:
	summa = summa + tala
print(summa)\end{lstlisting}
\end{Answer}

\begin{exercise}\label{lyk3}
Þetta er sama æfing og \ref{lyk2} nema í stað þess að búa til ykkar eigin talnalista eigið þið að finna summu talna frá 0 upp að 1000.
Prentið svo út summuna þegar keyrslu lykkjunnar lýkur.
\end{exercise}
\begin{Answer}[ref={lyk3}]
Svo við gerum það sama og áður, við þurfum summu breytu áður en við förum inn í lykkjuna en við notum \texttt{range()} fallið.
	
\begin{lstlisting}
summa = 0
for tala in range(1000):
	summa = summa + tala
print(summa)\end{lstlisting}
\end{Answer}

\begin{exercise}\label{lyk4}
Síðasta talnaæfingin með for-lykkjur.
Nú ætlið þið að prenta allar þær tölur sem eru á bilinu 0-100 sem eru með þversummu\footnote{https://is.wikipedia.org/wiki/\%C3\%9Eversumma} (e. transverse sum) hærri en sex.
Þar sem þessi æfing er töluvert flóknari en aðrar verður hún leyst í skrefum og hægt er að kíkja á svörin til að fá fyrst vísbendingu.
\end{exercise}
\begin{Answer}[ref={lyk4}]
Til þess að geta skoðað þversummu þurfum við að skoða hvern tölustaf fyrir sig og þá þurfum við að kasta í streng og skoða hvert stak í honum.
Þetta veldur því að við þurfum lykkju innan í lykkju.
Einnig þurfum við að kasta á milli taga og nota samanburð.
Hér er vísbending.
\begin{lstlisting}
for tala in range(100):
	talna_strengur = ?
	?
	for stak in talna_strengur:
		? += ?
	if(? > 6):
		?
\end{lstlisting}

Áður en þið skoðið lausnina skulið þið gera heiðarlega tilraun til að fylla inn fyrir spurningarmerkin.
Hér kemur langur texti um hvað þarf að gerast svo að þið sjáið ekki lausnina alveg strax.

Það sem þarf að setja inn er að gera \texttt{tala} að streng svo hægt sé að rúlla í gegnum hana í innri lykkjunni.

Áður en innri lykkjan er keyrð þarf að skilgreina breytu sem á að halda utan um summuna, ástæðan fyrir því að það gerist í ytri lykkjunni en ekki utan hennar eins og áður er vegna þess að við viljum fá nýja summu sem er skilgreind sem 0 fyrir hverja einustu tölu í ytri lykkjunni.

Þá erum við komin með breytu sem heldur utan um summu og getum ferðast í gegnum vísa talna strengs, það sem við þurfum þá að gera er kasta hverju staki fyrir sig í tölu svo að við getum notað það í útreikningi.
Þá viljum við leggja það saman við summuna og uppfæra summuna með þessari samlagningu.

Þegar við erum búin með innri lykkjuna erum við komin með þversummu fyrir einhverja eina tölu úr talnalistanum sem ytri lykkjan er að skoða.
Þá viljum við spyrja er þessi þversumma stærri en 6?

Ef svo er þá viljum við prenta út töluna sem hafði þessa þversummu.

\begin{lstlisting}
for tala in range(100):
	talna_strengur = str(tala)
	tversumma = 0
	for stak in talna_strengur:
		tversumma += int(stak)
	if(tversumma > 6):
		print(tala)\end{lstlisting}
\end{Answer}

\begin{exercise}\label{lyk5}
Síðasta for-lykkju æfingin.
Búið til lista með nokkrum strengjum, prentið út alla þá strengi sem innihalda táknið \texttt{a}.
Ábending, athugið að skoða kóðabút \ref{lst:lykkjur-in}.
\end{exercise}
\begin{Answer}[ref={lyk5}]
Athugið hér að nafnalistinn er ólíkur ykkar, þessi listi var gerður til að sýna að ANNA og Albert innihalda ekki táknið sem beðið var um og komast því ekki í gegn.
Ef óskað væri eftir að hafa það tákn líka væri hægt að rifja upp \texttt{.lower()} úr kafla \ref{k:strengir} eða hvernig eigi að nota rökvirkja í segð\footnote{þá í stað if "a" in stak væri if "a" in stak.lower(), eða með rökvirkja if "a" in stak or "A" in stak}.
	
\begin{lstlisting}
listi = ["Halldóra", "ANNA", "Sigurður", "Albert", "Jóna", "Valborg", "Unnur", "Pétur", "Unnar"]
for stak in listi:
	if "a" in stak:
		print(stak)\end{lstlisting}
\end{Answer}

\begin{exercise}\label{lyk6}
Búið til while-lykkju sem keyrir alltaf en er brotin í fyrstu keyrslu.
\end{exercise}
\begin{Answer}[ref={lyk6}]
Hér er beðið um að láta skilyrði lykkjunnar vera \texttt{True} og nota lykilorðið \texttt{break}
	
\begin{lstlisting}
while(True):
	break\end{lstlisting}
\end{Answer}

\begin{exercise}\label{lyk7}
Búið til breytu sem inniheldur einhverja tölu sem er á bilinu 0-10.
Búið svo til while-lykkju sem keyrir á meðan sú tala er lægri en 20.
Innan lykkjunnar ætlið þið að prenta út töluna og hækka hana svo um 1.
\end{exercise}
\begin{Answer}[ref={lyk7}]
Hér þarf að athuga inndráttinn á öllum aðgerðunum og að upphaflega lykkjuskilyrðið sé rétt skilgreint.
\begin{lstlisting}
tala = 3
while(tala < 20):
	print(tala)
	tala += 1\end{lstlisting}
\end{Answer}

\begin{exercise}\label{lyk8}
Búið til lista sem inniheldur nokkra strengi, en nokkur stök eru strengurinn \texttt{"popp"} þ.e. hann kemur nokkrum sinnum fyrir víðsvegar um listann.
Nú ætlið þið að búa til while-lykkju sem keyrir á meðan orðið \texttt{popp} er enn í listanum (eitt og sér til að einfala málin).
Það sem þið gerið svo innan í lykkjunni er að fjarlægja orðið \texttt{popp} úr listanum og prenta út breytta útgáfu.
Rifjið upp \texttt{.pop()} aðferðina úr kafla \ref{k:listar} ásamt \texttt{.index()} úr kafla \ref{k:strengir} eða flettið upp notkun á \texttt{.remove()} á netinu.
\end{exercise}
\begin{Answer}[ref={lyk8}]
Hér eru báðar útgáfur sýndar af þeim tillögum sem voru nefndar, en eins og með svo margt annað í forritun þá tókst ykkur mögulega að gera þetta öðruvísi.
Aðalmarkmiðið var að geta notað lykilorðið \texttt{in} í skilyrði lykkjunnar.
	\begin{lstlisting}
listi = ["nammi", "popp", "ávextir", "popp", "grænmeti", "popp", "hunang", "popp", "brauð"]
while("popp" in listi):
	visir = listi.index("popp")
	listi.pop(visir)
	print(listi)
		
#eða
while("popp" in listi):
	listi.remove("popp")
	print(listi)\end{lstlisting}
\end{Answer}

\chapterimage{chapters9.png} % Chapter heading image

\chapter{N-dir}\index{Ndir}\label{k:ndir}
Nú ætlum við að kynnast nýrri týpu, hún heitir \textbf{n-d} (lesist ennd) eða n-und (e. tuple) \footnote{\href{http://stæ.is/os}{Skoðið endilega orðasafn stærðfræðifélagssins}}.
Lykilorð þessarar týpu er \textbf{tuple}.
Nafnið er komið frá hugmyndinni um tvenndir og þrenndir, nema við vitum ekki hversu mörg stök er verið að hópa saman, þau gætu verið af n fjölda svo við köllum týpuna n-d eða n-und (þá frá tvíund og þríund).
Líklega eina orðið í íslensku sem inniheldur ekki sérhljóða.
Hér eftir verður týpan kölluð \emph{nd}, þó margir noti n-und.
 
Hún líkist listum að því leitinu til að margar af sömu aðgerðum sem má gera á lista má gera á ndir.
Hún líkist strengjum því að hún er óbreytanleg.
Það má ekki bæta við, breyta eða taka út stök eftir að ndin er skilgreind.

Ástæður fyrir því að nota ndir í stað lista er sú að það getur verið hagkvæmara, ndir nota ekki eins mikið minni, eða okkur er umhugað um gagnaheilindi, og svo sjáum við í kafla \ref{k:föll} um hvernig megi fá eina nd í stað margra skilagilda.

\section{Skilgreining}\index{Skilgreining n-da}
Við skilgreinum nd með svigum.
Athugið að hingað til höfum við notað sviga til að aðgreina segðir og það er vandmeðfarið að átta sig á því hvenær er sviginn stærðfræðilegur (þ.e. einungis fyrir forritarann til að aðgreina samhengi) og hins vegar skilgreining á gögnum af týpunni nd.
Aðgreiningin verður augljós þegar við áttum okkur á því að til þess að skilgreina nd þá þurfum við, líkt og með lista, að aðgreina stökin innan ndinnar með kommum.
Sjáum í kóðabút \ref{lst:ndir-kynntar} hvernig má skilgreina ndir og hvernig svigar gera það ekki nema við notum kommur.
\todo{vísa í kóðabút úr segðum eða tölum þar sem segðir eru aðgreindar með svigum til útskýringar, mögulega er todo að búa það til}
Þar sjáum við einnig að það eru einungis tvær aðferðir til fyrir týpuna, \texttt{.count()} og \texttt{.index()}.
Hvernig má það vera að týpan líkist listum þegar það eru bara til tvær aðferðir?
Var ekki verið að taka fram að það mætti gera margt það sama?
Jú, aðgerðir og aðferðir er ekki það sama.
Við getum ítrað í gegnum nd, við getum skeytt einni nd aftan við aðra (fáum þá nýja nd en breytum henni ekki), við getum náð í hluta úr ndinni (með hornklofum eins og hlutstrengi eða hluta úr lista).

Skoðum nú í kóðabút \ref{lst:ndir-kynntar} hvernig svigar geta annars vegar verið til að afmarka stærfræðilegan forgang og hins vegar til að skilgreina nýju týpuna okkar.
Tökum sérstaklega eftir notkuninni á innbyggða fallinu \texttt{type()} sem auðveldar okkur að skilja hvenær nd verður til.

\begin{lstlisting}[caption=Ndir skilgreindar, label=lst:ndir-kynntar]
a = (3+4)*2
print("a er af taginu", type(a))

b = (1)
print("b er af taginu", type(b))

c = () # þetta verður tóm nd
d = (1,) # þetta verður nd sem inniheldur eitt stak, athugið kommunotkunina
e = (1, 1, 2, 2, 5) # þetta verður nd sem inniheldur 5 stök
\end{lstlisting}
\lstset{style=uttak}
\begin{lstlisting}
a er af taginu <class 'int'>
b er af taginu <class 'int'>
c er <class 'tuple'> d er <class 'tuple'> og e er <class 'tuple'>
\end{lstlisting}
\lstset{style=venjulegt}

Nú gerum við ráð fyrir að eiga ennþá ndirnar \texttt{c,d} og \texttt{e} úr kóðabút \ref{lst:ndir-kynntar}, skoðum þá aðferðirnar tvær sem eru innbyggðar í kóðabút \ref{lst:ndir-adfr}, ásamt því hvað gerist þegar við skeytum einni nd aftan við aðra, hvernig ítrun með for lykkju er lík því sem við þekkjum með lista og að lokum hvernig megi sækja hlut-nd.

\begin{lstlisting}[caption=Ndir aðgerðir og aðferðir , label=lst:ndir-adfr]
print(d.index(1))
print(e.count(1)) 
print(e + d)
print()

for tala in e:
	print(tala)

# Athugum að við megum ekki breyta nd, svo eftirfarandi kóði veldur villu
# c[4] = 3

print(e[1:3])
\end{lstlisting}
\lstset{style=uttak}
\begin{lstlisting}
0
2
(1, 1, 2, 2, 5, 1)

1
1
2
2
5
(1, 2)
\end{lstlisting}
\lstset{style=venjulegt}



\section{Notkun}\index{Notkun nda}
Þar sem ndir eru óbreytanlegar er gagnlegt að nota þær til að halda utan um ástand sem við viljum ekki að sé hróflað við.
Segjum að það séu ákveðin tengsl á milli tveggja gilda og við viljum halda heilindum þeirra þá væri gott að nota nd.
Við getum líka notaðað þær til að spara minni þegar við þurfum litla lista sem þarf bara að nota tímabundið og breytast ekki.
Einnig geta þær nýst til að halda utan um breytur sem á svo að nota hverja í sínu lagi seinna.
Tökum eftir að vissulega megi útfæra fyrri þrjú, af þessum fjórum atriðum nefndum, með listum.

Skoðum kóðabút \ref{lst:nd-notkun} þar sem við sjáum dæmi um nd sem við viljum að haldist óbreytt þó hún sé skoðuð en við viljum getað úthlutað hverju staki í einhverja breytu (e. unpack) til að nota og skoða án þess að það hafi áhrif á ndina.
Við viljum ekki keyra aðferðir á borð við \texttt{.sort()} á ndina því að þá breytist hún, við viljum heldur ekki geta haft áhrif á einstaka sætisvísi (skoðið hvaða villu fæst við þá aðgerð með því að keyra línu 10 í kóðabút \ref{lst:ndir-adfr}), það sem við viljum er létt gagnagrind sem passar upp á gögnin.


\begin{lstlisting}[caption=Ndir notaðar fyrir það sem þær eru gagnlegar, label=lst:nd-notkun]
notanda_upplysingar = ("valborg", "rosalega gott lykilorð", "netfang@internet.is")

notandanafn = notanda_upplysingar[0]
lykilord = notenda_upplysingar[1]
netfang = notenda_upplysingar[2]

notandanafn, lykilord, netfang = notenda_upplysingar

print(notandanafn)
notandanafn = notandanafn.upper()
print(notenda_upplysingar)
\end{lstlisting}
\lstset{style=uttak}
\begin{lstlisting}
valborg
('valborg', 'rosalega gott lykilorð', 'netfang@internet.is')
\end{lstlisting}
\lstset{style=venjulegt}

Línur 3-5 og lína 7 eru jafngildar í kóðabút \ref{lst:nd-notkun}.
Þessi ,,afpökkun'' er læsileg og þægileg leið til að vinna með nd, við sjáum það svo betur þegar við skoðum skilagildi í kafla \ref{uk:skilagildi} hversu mikilvægt er að kunna á þetta.
Takið einnig eftir í úttakinu og þrátt fyrir að breytan \texttt{notandanafn} hafi verið uppfærð þá hafði það engin áhrif á ndina.
Reynið nú að uppfæra það gildi í ndinni, reynið að setja í staðinn fyrir fremsta stakið þessa nýju breytu og sjáið hvaða villu þið fáið.

Að sjálfsögðu er markmiðið okkar ekki enn sem komið er orðið að því að skrifa kóða í sem fæstum línum mögulegum, en það sem við viljum þó geta gert er að gera kóðann okkar eins læsilegan og mögulegt er með því að nota þær aðgerðir sem Python býður upp á.
Jafnvel þó að eini ávinningurinn er að við sjálf skiljum kóðann ennþá þegar við skoðum hann seinna.

%-------------------------------
\newpage
\section{Æfingar}
\begin{exercise}\label{nd1}
Búið til nd sem inniheldur 3 stök og setjið svo aftasta stakið í breytu.
\end{exercise}
\setboolean{firstanswerofthechapter}{true}
\begin{Answer}[ref={nd1}]
Hér vitum við að það eru þrjú stök í ndinni, svo að aftasta stakið hefur sætisnúmerið 2.
\begin{lstlisting}
nd = (1,2,3)
tala = nd[2]\end{lstlisting}
\end{Answer}
\setboolean{firstanswerofthechapter}{false}

\begin{exercise}\label{nd2}
Búið til nd sem inniheldur eingöngu tölur, ítrið í gegnum ndina og prentið út þær tölur sem eru stærri en 100.
\end{exercise}
\begin{Answer}[ref={nd2}]
Rifjum upp að til að ítra í gegnum hlut af gefinni stærð er best að nota for lykkju.
\begin{lstlisting}
talna_nd = (1,34,432,324,999,1,2,3,1,3,55,664,10000)
for tala in talna_nd:
	if tala > 100:
		print(tala)\end{lstlisting}
\end{Answer}

\begin{exercise}\label{nd3}
Búið til nd sem inniheldur tvær tölur, úthlutið svo þeim tveimur tölum í tvær breytur með afpökkun.
Geymið svo útkomuna úr því hvort að fremri talan sé stærri en sú seinni og búið til nýja nd þar sem útkomunni er skeytt aftan við upphaflegu ndina.
\end{exercise}
\begin{Answer}[ref={nd3}]
Við erum beðin um að búa til nd eð tveimur stökum, svo erum við beðin um að geyma útkomu sem þýðir að við þurfum breytu.
Sú breyta á að innihalda svarið við hvort að fyrri talan sé stærri en sú seinni svo það er sanngildi.
Að því loknu erum við beðin um að nota samskeytingu en útkoman er ekki nd svo við þurfum að setja hana í nd til að geta beytt samskeytingu.
	\begin{lstlisting}
nd = (1,2)
tala1, tala2 = nd
svar = tala1 > tala2
ny_nd = nd + (svar,)\end{lstlisting}
\end{Answer}

\begin{exercise}\label{nd4}
Búið til tóma nd.
Búið til lykkju sem keyrir fjórum sinnum og í hvert sinn spyr hún notandann um uppáhaldslitinn sinn.
Í hvert sinn sem notandinn er búinn að svara skal endurskilgreina ndina sem það sem hún var áður að viðskeyttu nýja svarinu.
Þegar lykkjan hefur lokið keyrslu sinni skulið þið prenta út ndina.
\end{exercise}
\begin{Answer}[ref={nd4}]
Rifjum upp  \texttt{input()} fallið úr kafla \ref{k:segðir}, einnig \texttt{range()} fallið úr kafla \ref{k:lykkjur}.
\begin{lstlisting}
nd = ()
for i in range(4):
	svar = input("hver er uppáhalds liturinn þinn?")
	nd = nd + (svar,)
print(nd)\end{lstlisting}
\end{Answer}

\chapterimage{chapters6.png} % Chapter heading image

\chapter{Orðabækur}\index{Orðabækur}\label{k:orðabækur}
Ný týpa sem við ætlum nú að fást við heitir \textbf{orðabók} (e. dictionary)  og lykilorðið hennar er \textbf{dict}.
Orðabók er orð sem hentar fyrir þýðingu á týpunni í Python en hún er einnig þekkt sem hakkatafla (e. hash table / hash map) í öðrum forritunarmálum.
Til þess að búa til orðabók eru notaðir slaufusvigar \{\}.

Orðabækur eru gagnagrindur eins og listar og ndir, það er þær geyma fyrir okkur gögn af einhverjum týpum.
Orðabækur eru þó frábrugnar listum að því leytinu til að þær eru \textit{óraðaðar}, sem þýðir að þær hafa enga sætisvísa sem hægt er að nota.\footnote{Í Python >3.6 eru þær raðaðar.
Stök eru sett inn í ákveðinni röð og helst sú ,,röðun“ þegar orðabókin er notuð.
Fyrir það voru stökin aðgengileg með handahófskenndri röð fyrir minnisbestun.}
Við getum því ekki sótt gögn í orðabækur með því að vita \textit{hvar} þau eru við þurfum að vita \textit{hver} þau eru.

Þetta er vegna þess að orðabækur eru skipulagðar sem lykla og gildis pör, við finnum þau gildi sem við viljum með því að vita hvaða lykill gengur að þeim.
Þetta er ekki ósvipað því að horfa á lyklakippurnar okkar.
Lyklarnir eru allir ólíkir.
Við getum alltaf reitt okkur á það að sama hvar einhver ákveðinn lykill er þá gengur hann alltaf að sama lásnum, svo ef við þekkjum lyklana okkar getum við auðveldlega náð í þann sem við viljum til þess að opna þann lás sem við viljum hverju sinni.

Lyklarnir verða því að vera ólíkir hver öðrum, annars gætum við ekki þekkt þá í sundur og tveir eins lyklar gætu ekki gengið að tveimur mismunandi lásum.
Lyklar verða því að vera aðgreinanlegir.

Orðabókin er mjög öflugt fyrirbæri og því þess virði að kynna sér vel hvernig þessi týpa virkar.
Einnig skoðum við í þessum kafla hvernig má ítra í gegnum orðabækur og hvers vegna það var ágætt að vera búin að skoða ndir áður en við komum að þessari mikilvægu týpu.

%\begin{wrapfigure}{i}{0.21\textwidth} %i o r l 
	\begin{center}
		\includegraphics{doodles31-17.png}
	\end{center}
%\end{wrapfigure}
%---- Athuga hvort þetta eigi heima einhversstaðar
% Við skoðum betur hvernig við getum gengið úr skugga um aðgreinanleika og hvað það þýðir.
%---
\section{Orðabækur skilgreindar og notaðar}\index{Orðabækur skilgreindar og notaðar}
Eins og kom fram í inngangi er gögnum í orðabókum skipt niður á lyklana sem ganga að þeim (sjá ítarefni um aðgreinanleika um hvað má vera lykill og hvað aðgreinanleiki þýðir).
Sjáum fyrir okkur lyklakippuna okkar þar sem við erum með stóran ASSA lykil að útidyrahurðinni, lítinn lykil með svörtu plasti að hjólalásnum, kassalaga lykil að útidyrahurðinni hennar ömmu og einn pínulítinn lykil að geymslunni.
Við eigum auðvelt með að halda utan um þetta litla lyklasafn og við vitum að hverju allir lyklarnir ganga.
En ef við værum nú með 10.000 lykla?
Skoðum kóðabút \ref{lst:dict-kynnt} til að sjá hvernig má búa til orðabók sem heldur utan um lyklakippuna sem var lýst hér að ofan.
Takið eftir að einungis er unnið með strengi innan orðabókarinnar.

\begin{lstlisting}[caption=Orðabók kynnt með lyklakippusamlíkingu, label=lst:dict-kynnt]
tom_ordabok = {}
kippa = {'ASSA': 'útidyrahurðin heima ', 'lítill svartur': 'hjólið', 'kassalaga': 'heima hjá ömmu', 'pínulítill': 'geymslulykillinn'}

print(kippa['lítill svartur'])
\end{lstlisting}
\lstset{style=uttak}
\begin{lstlisting}
hjólið
\end{lstlisting}
\lstset{style=venjulegt}
 \begin{wrapfigure}{o}{0.29\textwidth} %i o r l 
	\begin{center}
		\includegraphics{doodles31-06.png}
	\end{center}
\end{wrapfigure}
\phantom{}

Takið eftir hvernig stökin eru aðgreind, með kommum alveg eins og áður, nema núna eru stökin tvenndir sem hanga saman með tvípunkti.
Við sjáum einnig að til þess að nálgast gögn notum við hornklofa eins og áður en við gerum það ekki með sætisvísi heldur gerum við það með lyklinum sem við viljum finna gögnin að.

Ef lykill er heiltala náum við vissulega í gögnin á þeim lykli með því að nota heiltölu innan hornklofanna (sjá línu 3 í kóðabút \ref{lst:dict-kynnt2}).
Í kóðabút \ref{lst:dict-kynnt} sjáum við hvernig á að búa til tóma bók í fyrstu línu, við gerum ekkert frekar með þessa orðabók í þessum kóðabút en í kóðabút \ref{lst:dict-kynnt2} sjáum við hvernig má setja stök (pör af lyklum og gildum) inn í orðabók eftir að hún er skilgreind.
Í kóðabút \ref{lst:dict-kynnt3} sjáum við hvaða aðferðir eru til á þessa týpu.

\begin{lstlisting}[caption=Gögnum bætt við og þau tekin út, label=lst:dict-kynnt2]
ordabok2 = {1: 'gildi á lykli 1 sem er heiltala', 8: 'gildi sem er á lykli 8', 5: 'takið eftir að pörin eru aðgreind með kommu'}

print(ordabok2[5])

ordabok2[1] = 'nýtt gildi' 
ordabok2['nýr lykill'] = 'nýtt gildi'
print(ordabok2)
\end{lstlisting}
\lstset{style=uttak}
\begin{lstlisting}
takið eftir að pörin eru aðgreind með kommu
{1: 'nýtt gildi', 8: 'gildi sem er á lykli 8', 5: 'takið eftir að pörin eru aðgreind með kommu', 'nýr lykill': 'nýtt gildi'}
\end{lstlisting}
\lstset{style=venjulegt}

Í kóðabút \ref{lst:dict-kynnt2} sjáum við hvernig heiltölur geta verið lyklarnir í orðabókinni en við höfum þó aðeins verið að vinna með strengi sem gildi.
Í raun eru skorður á því hvað geta verið lyklar en ekki hvað geta verið gildi, við getum geymt hvað sem er sem gildi.
Við sjáum í kóðabút \ref{lst:dict-kynnt3} þegar listar eru notaðir sem gildi.

\begin{itarefni}
\textbf{Aðgreinanleiki}\\
Í Python er til \texttt{hash()} fall sem skilar ,,hakki“ af því sem því er gefið sem viðfang.
Við sáum í kafla \ref{k:tolur} að 1 og 1.0 er hægt að reikna með þó þær séu af mismunandi tagi og í kafla \ref{k:segðir} sáum við að 1 og 1.0 er jafngilt.
Það sem \texttt{hash()} gerir er að skila heiltölugildi fyrir viðfangið og þegar við viljum athuga hvort að eitthvað sé jafngilt með rökvirkjanum == erum við að spyrja hvort fallið skili mismunandi heilum tölum fyrir það sem er sitthvorum megin við rökvirkjann.
Í tilfellinu 1, 1.0 og True eru þau ekki aðgreinanleg og því ekki hægt að nota þau sem þrjá mismunandi lykla í sömu orðabókinni.
Þetta er ástæðan fyrir því að orðabók er oft kölluð hakkatafla.

Prófið ykkur áfram með \texttt{hash()} fallið og sjáið hvaða gögn má hakka, af hvaða týpu eru gögnin?
\end{itarefni}

Nú höfum við séð grunnvirknina við það að búa til orðabók og ná í gögn á lykil, en hvaða aðferðir eru til á þær?
Þær eru ekki margar og þess virði að taka nokkrar fyrir sérstaklega vegna þess hve gagnlegar þær eru strax fyrir byrjendur.

\begin{itemize}
\item[] \texttt{.get()} leyfir okkur að athuga hvort lykill sé til í orðabók án þess að valda villu, og ef við viljum skila stöðluðu gildi ef lykillinn fannst ekki.
\item[] \texttt{.popitem()} fjarlægir það stak sem síðast var sett inn (fjarlægir eitthvað stak í Python <3.6).
\item[] \texttt{.pop()} fjarlægir nákvæmlega það stak sem við viljum með því að gefa upp lykil.
\item[] \texttt{.items(), .keys()} og \texttt{.values()} skila okkur ítranlegum hlut af því sem við viljum geta unnið með, \texttt{items} eru lykla- og gildispör sem ndir, \texttt{keys} skilar bara lyklunum og \texttt{values} einungis gildunum.
\end{itemize}

Skoðum aðeins nánar hvernig \texttt{items(), keys()} og \texttt{values()} virka því að við viljum geta ítrað í gegnum orðabækur.

\begin{lstlisting}[caption=Aðferðir á orðabækur, label=lst:dict-kynnt3]
ordabok3 = {1: [1,2,3,4,5], 2: ["strengir", "í", "lista"], "þrír": [-1,-2,-3]}
print("gildin:", ordabok3.values())
print("lyklarnir", ordabok3.keys())
print("ndir af pörum", ordabok3.items())
\end{lstlisting}
\lstset{style=uttak}
\begin{lstlisting}
gildin: dict_values([[1, 2, 3, 4, 5], ['strengir', 'í', 'lista'], [-1, -2, -3]])
lyklarnir dict_keys([1, 2, 'þrír'])
ndir af pörum dict_items([(1, [1, 2, 3, 4, 5]), (2, ['strengir', 'í', 'lista']), ('þrír', [-1, -2, -3])])
\end{lstlisting}
\lstset{style=venjulegt}

Höfum ekki óþarfa áhyggjur af úttakinu þar sem stendur dict\_ eitthvað.
Það sem við þurfum að átta okkur á er að við fáum nokkurs konar lista (e. view) í hvert sinn og að stökin í listunum fást upp úr orðabókinni okkar með útreiknanlegum hætti.

Skoðum nú í næsta undirkafla hvernig má vinna með þetta.

\section{Ítrað í gegnum orðabækur}\index{Ítrað í gegnum orðabækur}
 \begin{wrapfigure}{r}{0.21\textwidth} %i o r l 
	\begin{center}
		\includegraphics{doodles31-27.png}
	\end{center}
\end{wrapfigure}

Nú höfum við séð for-lykkjur í kafla \ref{k:lykkjur} og hvernig má lykkja í gegnum lista í kóðabút \ref{lst:lykkjur-for-listi}.
Nú þurfum við hins vegar að fara yfir hvernig í ósköpunum á eiginlega að skoða stak í orðabók á kerfisbundinn máta þegar eitt stak er bæði lykill og gildi.

Útfærsla for-lykkja fyrir orðabókur í Python er þannig að ef kallað er í orðabók með for-lykkju er hlaupandi breytan að fara í gegnum lykla orðabókarinnar.
Hins vegar er auðveldlega hægt að ítra yfir lykla og gildi eða einungis gildin með því að kalla í aðferðirnar sem teknar voru fyrir í lok síðasta undirkafla.

Nú eru nöfnin á þessum aðferðum ágætlega lýsandi:

\begin{itemize}
	\item \texttt{.keys()}: við fáum í hendurnar ítranlegan hlut sem inniheldur alla lyklana.
	\item \texttt{.values()}: við fáum í hendurnar ítranlegan hlut sem inniheldur öll gildin.
	\item \texttt{.items()}: við fáum í hendurnar ítranlegan hlut sem inniheldur lista af tvenndum (nd með tveimur stökum) þar sem fyrra stakið er alltaf lykillinn og seinna stakið er alltaf gildi hans.
\end{itemize}

Til þess að sækja það sem við viljum skoða þurfum við því að nota þá aðferð á orðabókina okkar sem okkur hentar hverju sinni.
Ef við vildum til dæmis halda utan um bókasafnið okkar með orðabók og vinna með þær upplýsingar úr bókasafninu sem henta hverju sinni gætum við gert það eins og kemur fram í kóðabút \ref{lst:dict-bokasafn}.
Við viljum að höfundur sé lykillinn og að gildið sé listi af bókum sem við eigum eftir þann höfund.
Svo viljum við geta prentað út nöfn þeirra höfunda ásamt upplýsingum um hversu oft þeir koma fyrir í safninu.

\begin{lstlisting}[caption=Ítrað í gegnum orðabók, label=lst:dict-bokasafn]
bokasafn = {'Beazley': ['Python Essential Reference'],'Halldór Laxness': ['Íslandsklulkka', 'Salka Valka], 'Auður Haralds': ['Hlustið þér á Mozart', 'Læknamafían', 'Hvunndagshetjan'] }

for hofundur in bokasafn:
	if len(bokasafn[hofundur]) > 5:
		print('Á bókasafninu eru til fleiri en fimm bækur eftir höfundinn', hofundur)
	elif(len(bokasafn[hofundur]) > 2):
		print('Á bókasafninu eru til fleiri en tvær bækur en þó innan við sex, eftir höfundinn', hofundur)
	elif(len(bokasafn[hofundur]) > 1):
		print('Á bókasafninu eru til tvær bækur eftir höfundinn', hofundur)
	elif(len(bokasafn[hofundur]) > 0):
		print('Á bókasafninu er til ein bók eftir höfundinn', hofundur)
	else:
		print('Á bókasafninu er ekki til nein bók eftir höfundinn', hofundur)
\end{lstlisting}
\lstset{style=uttak}
\begin{lstlisting}
Á bókasafninu er til ein bók eftir höfundinn Beazley
Á bókasafninu eru til tvær bækur eftir höfundinn Halldór Laxness
Á bókasafninu eru til fleiri en tvær bækur en þó innan við sex, eftir höfundinn Auður Haralds
\end{lstlisting}
\lstset{style=venjulegt}

Í kóðabút \ref{lst:dict-bokasafn} var engum aðferðum beitt svo að við fengum það sem er staðlað að vinna með, einungis lyklana.
Takið eftir að þegar kallað er í \texttt{bokasafn[hofundur]} er verið að biðja um gildið sem tilheyrir þessum tiltekna höfundi, breytan \texttt{hofundur} er hlaupandi breytan sem rúllar í gegnum lyklana úr orðabókinni.
Við sjáum á úttakinu að fyrsta stakið sem \texttt{hofundur} fær úthlutað er Beazley og \texttt{bokasafn['Beazley']} er metið sem \texttt{['Python Essential Reference']} og svo er kallað á \texttt{len} (innbyggt fall sem skilar okkur lengd hluta eða fjölda sætisvísa) sem segir okkur að það sé 1 stak í listanum sem er gildið á lyklinum.
Þá rúllum við í næsta hluta skilyrðissetningarinnar því að 1 er vissulega ekki stærra en 5, við fáum sanngildi þegar spurt er hvort að 1 sé stærra en 0 og því fáum við úttakið: Á bókasafninu er til ein bók eftir höfundinn Beazley.
Svo gerist það sama aftur fyrir næsta höfund í röð lykla.

Í næsta kóðabút skoðum við svo hvernig á að fara að því að skoða bara bókalistana burtséð frá því hverjir höfundarnir eru, þetta gerum við með sömu \texttt{bokasafn} breytunni.
Við gerum ráð fyrir að hún sé enn aðgengileg í kóðabút \ref{lst:dict-bokasafn2}.
Markmiðið þar er ekki að skoða bækur eftir höfundum heldur prenta út nöfnin á öllum bókum sem eru nægilega löng.
Þar sem gildin eru listar af bókum þá getum við ítrað í gegnum hvern fyrir sig og þá hreiðrað aðra for lykkju inn í þá lykkju sem sér um að ítra í gegnum bókasafnið okkar.

%\begin{wrapfigure}{i}{0.21\textwidth} %i o r l 
	\begin{center}
		\includegraphics{doodles31-20.png}
	\end{center}
%\end{wrapfigure}
		
\begin{lstlisting}[caption=Ítrun í gegnum orðabækur með .values(), label=lst:dict-bokasafn2]
for bokalisti in bokasafn.values():
	for bok in bokalisti:
		if(len(bok) > 15):
			print(bok)
\end{lstlisting}
\lstset{style=uttak}
\begin{lstlisting}
Python Essential Reference
Hlustið þér á Mozart
\end{lstlisting}
\lstset{style=venjulegt}

Nú er ekki ýkja frábært að vera með hreiðraðar for lykkjur, þær eru gífurlega tímafrekar og það sem þær gera væri oft hægt að leysa á betri máta.
En eins og fram hefur komið áður erum við að reyna að átta okkur á því hvernig hlutir virka, við erum ekki að reyna að besta (e. optimize).

Prófið ykkur áfram með kóðann í kóðabút \ref{lst:dict-bokasafn2}, sjáið hvar breytunar eru aðgengilegar með því að prenta þær út og sjáið hvað breyturnar innihalda hverju sinni með útprentunum.
Prófið einnig að breyta til, og sjá hvort þið áttið ykkur á því hvað kemur út.

Í næsta kóðabút sjáum við svo hvað við gerum til að geta unnið með bæði lykil og gildi.
Notkun á \texttt{.items()} hefði mögulega sparað okkur smá hausverk í kóðabút \ref{lst:dict-bokasafn} og gert þann kóða læsilegri.
Tökum eftir að þar sem \texttt{.items()} skilar nd þá getum við annað hvort notað eitt breytuheiti til að taka við allri ndinni eða við getum úthlutað hverju staki úr ndinni í sína eigin breytu.
Það er gert í línu 1 í kóðabút \ref{lst:dict-bokasafn3}, við munum að aðferðin skilar nd þar sem lykillinn kemur fyrst og svo kemur gildið.
Því er breytan \texttt{hofundur} á undan í röðinni, breyturnar eru svo aðgreindar með kommu og þá kemur \texttt{bokalisti}.
Aftur gerum við ráð fyrir að sama bókasafnið sé okkur aðgengilegt.

\begin{lstlisting}[caption=Ítrun í gegnum orðabækur með .items(), label=lst:dict-bokasafn3]
for hofundur, bokalisti in bokasafn.items():
	# ef bókalistinn er ákveðið langur þá langar okkur að prenta út nafnið á höfundinum
	if(len(bokalisti) > 5):
		print(hofundur, "er mjög vinsæll höfundur")
	elif(len(bokalisti) > 2):
		print(hofundur, "er frekar vinsæll höfundur")
	elif(len(bokalisti) > 1):
		print(hofundur, "gæti verið vinsælli")
	elif(len(bokalisti) == 1):
		print(hofundur, "er vissulega til staðar")
	else:
		print(hofundur, "á ekki tilkall til einnar bókar í þessu bókasafni")
\end{lstlisting}
\lstset{style=uttak}
\begin{lstlisting}
Beazley er vissulega til staðar
Halldór Laxness gæti verið vinsælli
Auður Haralds er frekar vinsæll höfundur
\end{lstlisting}
\lstset{style=venjulegt}

Þetta eru mjög einföld dæmi en þau sýna það helsta sem þarf til þess að geta gert frekari tilraunir og leyst hin ýmsu verkefni.
Eins og áður þá næst árangur í forritun með því að gera tilraunir.

Skoðum nú næst kóðabút \ref{lst:dict-dict} þar sem rennt er í gegnum orðabók þar sem gildin eru orðabækur, takið sérstaklega eftir breytunni \texttt{upplysingar} og hvað hún gerir mikið fyrir okkur.
Við sjáum einnig í línu 15 að þar er innri lykkja sem er einungis keyrð fyrir þá höfunda sem eru með nógu háa meðaleinkunn.

\begin{lstlisting}[caption=Orðabók sem inniheldur orðabók sem gildi, label=lst:dict-dict]
itarlegt_bokasafn = {"Beazley": {'lesnar': 1, 'olesnar': 0, 'medaleinkunn': 5,'baekur': ['Python Essential Reference 4th ed'], 'besta bok': 'Python Essential Reference 5th ed' },
	'Halldór Laxness': {'lesnar': 1, 'olesnar': 1, 'medaleinkunn': 3,'baekur': ['Íslandsklulkka', 'Salka Valka'], 'besta bok': 'Vefarinn mikli frá Kasmír'}, 
	'Auður Haralds': {'lesnar': 3, 'olesnar': 0, 'medaleinkunn': 4,'baekur': ['Hlustið þér á Mozart', 'Læknamafían', 'Hvunndagshetjan'], 'besta bok': "Hvunndagshetjan"}
}
for hofundur, upplysingar in itarlegt_bokasafn.items():
	print(hofundur)
	if(upplysingar['olesnar'] > 0):
		print('Þú átt eftir að lesa einhverja af eftirfarandi bókum', upplysingar['baekur'])
	if len(upplysingar['baekur']) < 2:
		print('Það virðist vanta fleiri bækur eftir', hofundur)
	if upplysingar['besta bok'] not in upplysingar['baekur']:
		print('Þig vantar bestu bókina eftir höfundinn', hofundur, "sem er", upplysingar['besta bok'])
	if(upplysingar['medaleinkunn'] > 3):
		print(hofundur, 'er í miklu uppáhaldi og þú átt eftirfarandi bækur eftir viðkomandi:')
		for bok in upplysingar['baekur']:
			print (bok)
	print()
\end{lstlisting}
\lstset{style=uttak}
\begin{lstlisting}
Beazley
Það virðist vanta fleiri bækur eftir Beazley
Þig vantar bestu bókina eftir höfundinn Beazley sem er Python Essential Reference 5th ed
Beazley er í miklu uppáhaldi og þú átt eftirfarandi bækur eftir viðkomandi:
Python Essential Reference 4th ed

Halldór Laxness
Þú átt eftir að lesa einhverja af eftirfarandi bókum ['Íslandsklulkka', 'Salka Valka']
Þig vantar bestu bókina eftir höfundinn Halldór Laxness sem er Vefarinn mikli frá Kasmír

Auður Haralds
Auður Haralds er í miklu uppáhaldi og þú átt eftirfarandi bækur eftir viðkomandi:
Hlustið þér á Mozart
Læknamafían
Hvunndagshetjan
\end{lstlisting}
\lstset{style=venjulegt}

Í þessum síðasta kóðabút er nóg um að vera sem ætti að vera gott veganesti í æfingar þessa kafla.

%-------------------------------
\newpage
\section{Æfingar}
\begin{exercise}\label{dic1}
Búið til tóma orðabók, bætið svo við lykli og gildi.
\end{exercise}
\setboolean{firstanswerofthechapter}{true}
\begin{Answer}[ref={dic1}]
Hér er aðalatriðið að geta bætt við eftir skilgreiningu en ekki bara að geta búið til orðabók sem er skilgreind frá upphafi með einhverju lykla og gildis pari.
\begin{lstlisting}
bok = {}
bok['nýr lykill'] = 'Halló Heimur!'
\end{lstlisting}
\end{Answer}
\setboolean{firstanswerofthechapter}{false}

\begin{exercise}\label{dic2}
Búið til tóma orðabók.
Skrifið svo for-lykkju þannig að þið bætið tölum frá 0 og upp að n (að eigin vali) sem lyklum og sömu tölum í öðru veldi sem gildi, t.d ef n er 5 þá liti orðabókin svona út:
{0: 0, 1: 1, 2: 4, 3: 9, 4: 16}
\end{exercise}
\begin{Answer}[ref={dic2}]
Hér þarf að muna eftir range fallinu til að auðvelda okkur vinnuna annars er þetta sama verkefni og æfing \ref{dic1}.
\begin{lstlisting}
bok = {}
for i in range(5):
	bok[i] = i**2)\end{lstlisting}
\end{Answer}

\begin{exercise}\label{dic3}
Búið til orðabók með þremur stökum, fjarlægið einhver tvö þeirra.
\end{exercise}
\begin{Answer}[ref={dic3}]
Það eru til allavega tvær leiðir til að fjarlægja stak úr orðabók og fyrst við erum beðin um að fjarlægja tvö skulum við nota báðar aðferðirnar.
	\begin{lstlisting}
bok = {1: "þetta skal tekið", 2: "þetta skal vera", 3: "hiklaust fjarlægt"}
bok.pop(1)
bok.popitem()\end{lstlisting}
\end{Answer}

\begin{exercise}\label{dic4}
	Búið til orðabók með þremur stökum þar sem gildin eru listar af tölum, ítrið í gegnum orðabókina og prentið út það stak sem er minnst af öllum (einungis eina tölu) ef það er þó minna en talan 0, annars skal prenta út töluna 0.
	Athugið að nota \texttt{min()} fallið.
\end{exercise}
\begin{Answer}[ref={dic4}]
Hér þurfum við að athuga að áður en við förum að skoða hvað sé minnsta gildið þá þurfum að skilgreina breytu sem við viljum vera að bera saman við, og þarf sem við ætlum að prenta út 0 ef við finnum enga tölu lægri en það þá skilgreinum við breytuna sem við notum til samanburðar sem 0.
	\begin{lstlisting}
bok = {'lykill1': [3,4,2,4,2], 2: [-90, 2,3,1], "þrír": [-3,1000]}
minnsta_gildi = 0
for gildi in bok.values():
	if min(gildi) < minnsta_gildi:
		minnsta_gildi = min(gildi)
print(minnsta_gildi)\end{lstlisting}
\end{Answer}

\begin{exercise}\label{dic5}
Búið til orðabók þar sem gildin eru listar af strengjum.
Ítrið í gegnum orðabókina og bætið 'x' aftan við alla strengi í listunum sem eru í gildum  orðabókarinnar.
Athugið hér að nota \texttt{type()} fallið.
\end{exercise}
\begin{Answer}[ref={dic5}]
Það sem við þurfum að gera hér er að athuga sérstaklega hvort að við séum í raun að vinna með streng, við munum að lykilorðið fyrir streng er \textbf{str} og við athugum hvort að týpan af því sem við erum með í höndunum (hvert stak fyrir sig) sé strengur því að við megum ekki bæta strengnum x aftan við hvað sem er.
Einnig þurfum við að setja það inn aftur í staðinn, þessi leið er ekki fullkomin til þess en skoðið þennan kóða og áttið ykkur á hvað er um að vera áður en þið skoðið hina útfærsluna.
	\begin{lstlisting}
bok = {0: ["hér", "eru nokkrar týpur", 1, 2, 3, True], 1: ["líka hér", "nokkur tög", {1: ['Þetta telst ekki með', 'því þetta er innan orðabókar']}]}
for lykill, gildi in bok.items():
	for stak in gildi:
		if type(stak) is str:
			stadur = gildi.index(stak)
			gildi[stadur] = stak + 'x'\end{lstlisting}
Athugið að eitt auka hérx, ef því er bætt við fyrir ofan sést hve illa kóðinn grípur sum tifelli.
Hér er ekki verið að nota sætisnúmer með uppflettingu á stakinu heldur með númeri keyrslunnar.
\begin{lstlisting}
bok = {0: ["hér", "hérx" ,"eru nokkrar týpur", 1, 2, 3, True], 1: ["líka hér", "nokkur tög", {1: ['Þetta telst ekki með', 'því þetta er innan orðabókar']}]}
for lykill, gildi in bok.items():
	for i in range(len(gildi)):
		if type(gildi[i]) == str:
			gildi[i] = gildi[i] + 'x'\end{lstlisting}
\end{Answer}

\begin{exercise}\label{dic6}
	\textbf{Krefjandi æfing}\\
	Búið til orðabók sem inniheldur spurningar sem lykla og rétt svör við spurningunum sem gildi. 
	Semjið að minnsta kosti 5 spurningar, svörin þurfa að vera rétt eða rangt (til að einfalda hlutina töluvert).
	Búið til breytu sem heldur utan um stig notandans sem byrja í 0, hækkið þessa tölu þegar notandinn svarar rétt en breytið henni ekki annars. 
	Búið til lykkju sem fer í gegnum allar spurningarnar og spyr notandann að þeim.
	Þegar notandinn hefur svarað athugið þið hvort svarið sé rétt.
	Ef það er rétt hækkið þið einkunnina og látið notandann vita að svarið hafi verið rétt.
	Ef það er rangt látiði notandann vita að svarið var rangt.
	
	Athugið að gott er að staðla svar notandans t.d. með \texttt{.lower()} eða annarri sambærilegri aðferð. 
	Þegar spurningarnar eru búnar þá prentið þið út einkunn notandans og ef öll svörin voru rétt þá prentið þið 'og þú fékkst hæstu einkunn'.
\end{exercise}
\begin{Answer}[ref={dic6}]
	Þar sem þessi æfing er krefjandi eru hér eingöngu vísbendingar til að leysa hana.
	
	Athugum hér að við þurfum orðabók þar sem lyklar eru td. "er sólin blá" og gildi lykilsins væri "rangt", þegar hún er tilbúin þá getum við rúllað í gegnum hana og prentað út lyklana.
	Á eftir því að hafa prentað út lykil viljum við setja input skipun fyrir notandann, svarið ætlum við ekki að geyma neitt frekar en bara til að athuga hvort að það sé það sama og svarið (gildið á lyklinum sem við vorum að skoða).
	Þá förum við í reiknikúnstirnar að hækka ef rétt.
	
	Þegar allar spurningarnar eru þá búnar er loka einkunn komin, við getum borið einkunnina saman við fjölda lykla eða lengdina á bókinni ef það er það sama þá prentum við út auka textann um að þetta fór á besta veg með hæstu einkunn.
\end{Answer}

\begin{exercise}\label{dic7}
\textbf{Krefjandi æfing}\\
Búið til orðabók sem heldur utan um sveitarfélög á höfuðborgarsvæðinu sem lykla og íbúafjölda þeirra sem gildi.
Spyrjið notandann um tvö mismunandi sveitarfélög á höfuðborgarsvæðinu, gefið notandanum upp hversu mikill fjöldi býr þar samanlagt.
\end{exercise}
\begin{Answer}[ref={dic7}]
Þar sem þessi æfing er krefjandi eru hér eingöngu vísbendingar til að leysa hana.

Fyrir það fyrsta er að orðabókin með sveitarfélögunum sé stöðluð, það er rvk fyrir Reykjavík og þá hfj fyrir Hafnarfjörð en ekki á víxl.
Því næst er að athuga að spyrja notandann með input() fallinu og vegna þess að beðið er um mismunandi sveitarfélög þá verður bæði að athuga hvort að sveitarfélagið sem notandinn gaf upp sé til og það þarf einnig að athuga hvort að það sé það sama og viðkomandi gaf upp síðast.
Til þess þarf þá að nota while lykkju.
Þegar tvö mismunandi sveitarfélög eru komin er þá hægðarleikur að leggja saman gildin og prenta út nöfnin á þeim ásamt samanlögðum fjölda íbúa.
\end{Answer}




\chapterimage{chapter_head_2.pdf} % Chapter heading image

\chapter{Mengi}\index{Orðabækur}\label{k:sett}
Mengi (eða sett) eru týpa sem geymir óraðað safn af gögnum án tvítekninga, þau eru ein af fjórum innbyggðum gagnagrindum í Python (listar, ndir, orðabækur eru hinar) og geta þau geymt gögn af hvaða týpu sem er.
Lykilorðið þeirra er set.
Mengi eru skilgreind með slaufusvigum og eru stök þeirra aðgreind með kommum, ólíkt orðabókum þá eru engin lykla og gildispör sem hanga saman með tvípunkti og því ruglast vélinn ekki á þessum tveimur týpum.
Eins og orðabækur eru óraðaðar, þá er ekki hægt að nota vísa til þess að segja hvar eitthvað stak er í mengi.

Mengi þessi eru eins og mengi sem við könnumst við í stærðfræði, þar sem hvert stak kemur þó aðeins fyrir einu sinni.
Við getum framkvæmt ýmsar stærðfræðilegar aðgerðir á þau ásamt hefðbundnum aðgerðum til að bæta við eða fjarlægja stök, hins vegar er ekki hægt að breyta staki sem er nú þegar komið í mengið.

Skoðum kóðabút \ref{lst:set-kynnt} til þess að sjá hvernig mengi eru skilgreind og hvernig megi nota lykilorðið til að búa til mengi fyrir okkur úr gögnum.

\begin{lstlisting}[caption=Mengi skilgreind, label=lst:set-kynnt]
# Fyrsta mengið okkar inniheldur nokkrar tölur
mengid_mitt = {1,2,3,4}
print(mengid_mitt)
# úttakið verður 
# {1, 2, 3, 4}

# en til þess að búa til tómt mengi þarf að nota lykilorðið
tomt_mengi = set()

# því að þetta er tóm orðabók:
ekki_mengi = {}
\end{lstlisting}

\section{Tvítekning}\index{Tvítekning}
Tvítekning í mengjum er ekki leyfileg og því ágætt að nota mengi til þess að fjarlægja tvítekningar úr gögnunum okkar.
Ef við tökum fyrir orðið 'halló' og gerum mengi úr því með set('halló') þá fengjum við mengi sem innihéldi 'h', 'a', 'l', og 'ó'.
Stafurinn 'l' kemur tvisvar fyrir í strengnum en hann kemur einu sinni fyrir í menginu af strengnum.
Sjáum kóðabút \ref{lst:set-duplicate} hvernig við fáum ekki út tvítekningar sama hvernig við reynum.
Takið eftir í línu 10 þar sem stafirnir koma í einhverri röð, sú röð er ekki heilög þar sem þetta er óraðað gagnatag og þessi röðun verður ekki endilega eins við aðra keyrslu.

\begin{lstlisting}[caption=Mengi skilgreind, label=lst:set-duplicate]
# Skilgreinum mengi með endurtekningum
mengid_mitt = {1,2,3,4, 1, 2, 3, 4}
print(mengid_mitt)
# úttakið verður 
# {1, 2, 3, 4}

# notum lykilorðið til að búa til mengi úr streng
print(set("Valborg Sturludóttir vinsamlegast"))
# úttakið verður
# {'b', 'i', 'ó', 'n', 'l', 'a', 'g', 'd', 'o', 'V', ' ', 'e', 'S', 'r', 's', 'u', 't', 'm', 'v'}
\end{lstlisting}

\section{Aðgerðir}\index{Aðgerðir}
Aðgerðir sem hægt er að gera á set er að bæta við staki, \textbf{add()}, fjarlægja stak, \textbf{remove()} og uppfæra mengið með mörgum stökum, \textbf{update()}.
Engin þessara aðferða gerir okkur kleyft að eiga tvö eins stök í menginu.
Tvítekning er ekki liðin, sama hvernig við reynum að komast framhjá henni.

Stærðfræðilegar aðgerðir sem hægt er að gera með mengi er að finna sniðmengi eða sammengi tveggja mengja, eða mengi sem inniheldur einungis stök sem eru ekki í báðum mengjunum sem verið er að sameina. Allt þrennt af þessu er hægt að gera með innbyggðum föllum sem taka við tveimur mengjum og skila einu mengi til baka eða aðferðum á mengi sem breyta þá menginu sem aðferðin var kölluð á með tilliti til mengisins í viðfanginu.

Þetta reynist vel við að vinna með gögn eins og símaskrár eða tölvupóstföng því við viljum ekki tvítekningar og þegar á að sameina símaskrár eða tölvupóstföng með ákveðnum reglum er gott að vita að hægt sé að beita þessari týpu.

\chapterimage{chapter_head_2.pdf} % Chapter heading image

\chapter{Föll}\index{Föll}\label{k:föll}
Föll (e. functions), eins og lykkjur, eru kóðabútar sem má keyra oft.
Þau líkjast hins vegar frekar skilgreiningum eða uppskriftum frekar en lykkjum þar sem það þarf að \emph{nota} þau til þess að þau geri eitthvað ekki bara búa þau til.

Ágætar samlíkingar fyrir föll eru til dæmis stærðfræðilegar skilreiningar, uppskriftir eða réttir á matseðli.

\begin{itemize}
	\item Stærðfræðilega skilgreining á hring er eftirfarandi: ,,Hringur er safn punkta í gefinni fjarlægð frá ákveðinni miðju'' sem þýðir að til þess að búa til hring þarf einhverja miðju og teikna svo punkta í einhverjum tilteknum radíus frá þeirri miðju.
	Skilgreining þessi réttir okkur ekki hring með miðju í punkti (0,0) og radíus 3 þegar við setjum hana fram.
	En vegna þess að við eigum hana getum við notað hana til þess að búa til alla þá hringi sem okkur henta.
	\item Uppskrift í matreiðslubók er ákveðin runa af aðgerðum sem þarf að framkvæma, eins og kom fram í kafla \ref{uk:keyra-koda} um hnetusmjörssamlokuna, það að skrifa niður röð aðgerðanna og það sem þarf til er ekki jafngilt því að framkvæma aðgerðirnar og enda með matinn í höndunum
	\item Réttur á matseðli er skilreindur á ákveðinn hátt, með ákveðnu meðlæti og þess háttar.
	Við gætum þó viljað gera breytingu á þessari skilgreiningu til þess að fá mat sem er okkur meira að skapi en það sem kokknum datt í hug.
	Það gerum við með því að biðja um skilgreininguna á réttinum nema með breytingum.
\end{itemize}

Höfum þessar samlíkingar í huga þegar kemur að því að skrifa og beita föllum, því það hjálpar að átta sig á því strax að þegar við skilgreinum föll þá búum við til uppskrift sem hægt er að fylgja án þess þó að biðja vélina um að fylgja þeim.

Við höfum séð föll áður, eins og: 

\begin{itemize}
	\item print("fyrsta viðfangið", "næsta viðfang sem fallið tekur", "print er sér á báti, því það tekur við svo mörgum viðföngum")
	\item len("breyta sem ég vil vita lengdina á")
	\item range(0,50)
	\item type("breyta sem viðfang sem ég vil vita týpuna á")
\end{itemize}

Þetta eru innbyggð föll\footnote{\href{https://docs.python.org/3/library/functions.html}{Python skjölunin} inniheldur lista og upplýsingar um öll innbyggð föll}, það að þau séu innbyggð þýðir að við getum notað þau án þess að ná í einhvern annan kóða (sjá kafla \ref{k:import}) og eru þau flest svo gagnleg að ákveðið var að gefa notendum auðveldara aðgengi að þeim.

Við höfum þó séð fleiri föll sem eru aðferðir (e. methods), munurinn liggur í því að aðferð er hengd aftan á hlut með punkti og er fall sem keyrir á þann hlut en fall keyrir þegar kallað er í það og óþarfi er að hengja það við eitthvað annað.
Allar aðferðir eru föll, ekki öll föll eru aðferðir.

Til dæmis sáum við aðferðirnar .capitalize() á strengi og .sort() á lista.

Föll (e.function) og aðferðir (e. methods) eru aðgreinanlegar að þessu leiti, annað er skilgreint og virkar eins og það á að gera fyrir þau gögn sem þau eiga að virka á en aðferðir er eitthvað sem er fast við hlut og eiga að verka á þann hlut.

Við sjáum svo í öðrum hluta bókarinnar hvernig við skilgreinum aðferðir.

\section{Tilgangur falla}\index{Tilgangur falla}
Eins og með allt annað sem við lærum er gott að vita hvers vegna við erum á annað borð að læra um það.
Ástæðan fyrir því að við viljum læra um föll er að þau eru eitthvað það öflugasta sem við beitum í forritun, skoðum eftirfarandi lista til að skilja hvers megnug þau eru:

\begin{itemize}
	\item Við getum endurnýtt föll í stað þess að skrifa upp sama kóðann á bakvið þau á mörgum stöðum.
	\item Við getum gert föll aðgengileg út fyrir skjalið þar sem við skilgreindum þau.
	\item Við getum unnið með inntak frá notanda á skilvirkan máta.
	\item Föll halda utan um einhverja tiltekna virkni sem við viljum hafa aðgang að og eru skilvirk leið til að afmarka virkni.
\end{itemize}

Við höfum hingað til ekki fengist við meira en að átta okkur á grunnvirkni í forritun með hjálp Python en nú erum við komin á þann stað að við getum farið að leysa flókin vandamál.


\section{Að skilgreina föll}
Til þess að skrifa föll þurfum við að læra nýtt lykilorð, \textbf{def}.
Það stendur fyrir \emph{define} eða að skilgreina, þar sem við erum með því að búa til ákveðna skilgreiningu sem vélin getur svo notað til þess að framkvæma aðgerðir.

Það næsta sem þarf er að búa til nafn á fallið, nafnið er það sem vð notum til þess að beita fallinu okkar eftir að hafa skilgreint það.
Alveg eins og með aðrar breytur þá megum við ekki nota föll fyrr en búið er að skilgreina þau.

Þegar það er komið getum við byrjað að forrita virkni fallsins okkar, allt sem er a.m.k. einum inndrætti innar en def lykilorðið tilheyrir fallinu okkar.

Í kóðabút \ref{lst:foll-skilgreining} sjáum við hvernig á að búa til skilgreininguna og svo hvernig á að beita fallinu sem við bjuggum til, við ætlum að skoða minni úr forritun\footnote{Venjulega það fyrsta sem gert er í nýju forritunarmáli er að prenta út á staðalúttak "Halló Heimur!", þetta er skemmtileg hefð sem óþarfi er að gera miklar breytingar á.} og prenta út einfalda setningu.

Takið eftir að fallið er skilgreint og svo er það sem er inndregið undir því það sem fallið gerir, fyrir neðan skilgreininguna (ekki lengur inndregið undir henni) er svo kallað í fallið með því að skrifa nafnið á því og tóma sviga fyrir aftan nafnið.
Svigarnir eru nauðsynlegir því að þannig segjum við vélinni að við séum að kalla í fall.

\begin{lstlisting}[caption=Föll skilgreind, label=lst:foll-skilgreining]
def prentunarfall():
	print("Halló Heimur!")

prentunarfall()
\end{lstlisting}
\lstset{style=uttak}
\begin{lstlisting}
Halló Heimur!
\end{lstlisting}
\lstset{style=venjulegt}

Ástæðan fyrir því að svigarnir eru tómir í línu 4 er vegna þess að þeir eru tómir í línu 1 í skilgreiningunni.
Prófið að setja eitthvað á annan staðinn en ekki hinn og keyra svo kóðann.

\section{Viðföng}
Nú höfum við séð að hægt er að búa til skilgreiningar á föllum, en tökum eftir að í kóðabút \ref{lst:foll-skilgreining} þá eru tómir svigar fyrir aftan nafnið á fallinu.
Þessir svigar eru ekki þarna að ástæðulausu og þeir eru ekki tómir í þessum kóðabút að ástæðulausu heldur.

Það sem fer inn í svigana eru svo kölluð \emph{viðföng} (e. arguments), viðföngin skiptast í tvær tegundir \textbf{stöðubundin} (e. postitional) og \textbf{sjálfgefin} (e. named).
Hægt er að nota bæði í bland og eina viðmiðið er að gefa kost á þeim viðföngum sem notandinn ætti að fá eitthvað um að segja.

\begin{lstlisting}[caption=Föll með viðföngum, label=lst:foll-vidfong]
def prentunarfall(vidfang):
	print("Halló", vidfang, "!")
	
prentunarfall("Valborg")
\end{lstlisting}
\lstset{style=uttak}
\begin{lstlisting}
Halló Valborg!
\end{lstlisting}
\lstset{style=venjulegt}

Annað sem mikilvægt er að átta sig á er að viðföng fá breytuheiti sem eru aðeins aðgengileg innan fallsins en ekki utan þess.
Eins og breytan \texttt{vidfang} er aldrei formlega skilgreind svo við sjáum, allt í einu er hún bara komin þarna inn í svigann í línu 1 og strax notuð í línu 2.
Það er í rauninni ekki keyrsluröðin, það sem gerist er að vélin fær skilgreininguna á \texttt{prentunarfall} og að þeirri skilgreiningu fylgir einhver staðhaldari sem mun seinna fá eitthvað gildi.
Sem er það sem gerist í línu 4, við gefum breytunni gildið \texttt{"Valborg"}.

Við ætlum þá að skoða nýtt hugtak áður en við skoðum stöðubundin og sjálfgefin viðföng til þess að átta okkur á hvar þessi staðhaldari er í raun til.

\subsection{Gildissvið}\label{uk:gildissvið}
Gildissvið (e. scope) skiptis í staðvært (e. local) og víðvært (e. global). 
Það sem gildissvið þýðir er hvar eitthvað sé aðgengilegt.
Ef við búum til Jupyter vinnubók eða .py skjal þar sem við skilgreinum breytuna x er sú breyta ekki aðgengileg í öðru skjali.
Hins vegar ef við búum til breytu í vinnubók í einhverri sellu, með engum inndrætti, er sú breyta hluti af viðværu gildissviði og aðgengileg öllum sellum og allri virkni sem við viljum beita þessari breytu í.
En þegar við skilgreinum föll þá förum við inn á staðvært gildissvið sem þýðir að þegar kallað er í breytu byrjar vélin á að skoða hvort að breytan sé skilgreind innan þess sviðs, ef ekki þá notar hún víðværa gildissviðið.
En ef okkur langar að nota breytu sem var skilgreind innan einhvers falls (einhvers staðværs gildissviðs) þá höfum við ekki aðgang að henni í hinu víðværa gildisssviði.

Þetta virðist ótrúlega flókið í svona tæknilegu og löngu máli en skoðið kóðabút \ref{lst:foll-gildissvid} og skoðið hvaða x er verið að vísa í hverju sinni.
Það x sem er skilgreint í línu eitt er hluti af víðværu gildissviði og það sem er skilgreint innan fallsins í línu 3 er hluti af staðværu gildissviði og þetta eru því tvö mismunandi x sem hafa ekki áhrif hvort á annað.

\begin{lstlisting}[caption=Gildissvið, label=lst:foll-gildissvid]
x = 20
def gildissvids_prufa(x):
	x = x + 20
	print(x)

gildissvids_prufa(10)

def prufa_tvo():
	print(x)

prufa_tvo()
\end{lstlisting}
\lstset{style=uttak}
\begin{lstlisting}
30
20
\end{lstlisting}
\lstset{style=venjulegt}

Það má líta á þetta eins og að horfa á gosbrunn, þar sem víðværa gildissviðið er þar sem vatnið kemur upp efst í brunninum og svo fellur það niður í staðværa gildissviðið sem er neðri hluti brunnsins.
Allt sem er til í efri hlutanum getur neðri hlutinn fengið en það sem er í neðri hlutanum fer ekki upp (í þessari samlíkingu ætlum við að horfa framhjá innri virkni gosbrunnsins og sjá bara fyrir okkur hvernig er að horfa á fallegan gosbrunn sem er á tveimur eða fleiri hæðum).

Ástæða þess að það er mikilvægt að nefna gilssvið að svo stöddu er vegna þess að byrjendur vilja oft ruglast á breytunotkun með þessum hætti og halda að viðföng séu skilreindar breytur sem hægt sé að láta fallið hafa aftur, þegar það er líkara hugmyndinni um breytuna en ekki breytan sjálf.

Þetta skýrist þegar við skoðum stöðubundin viðföng.

\subsection{Stöðubundin viðföng}
Stöðubundin viðföng (e. postitional arguements) fá nafn og röðun þegar þau eru sett í skilgreiningu á falli.
Nöfnin á þeim lúta sömu lögmálum og nafnavenjur sem við höfum séð á breytum og er best að hafa nöfnin lýsandi fyrir virkni þeirra.

Þegar við köllum í fall sem skilgreint er með einu viðfangi og við gefum því engin viðföng þá fáum við villu, villumeldingin sem við fáum upp segir vantar eitt stöðubundið viðfang.
Í kóðabút \ref{lst:foll-villa-vidfang} sjáum við þessa villu og í kóðabút \ref{lst:foll-stodubundin-kynning} sjáum við hvernig má komast hjá þessari villu og þá hvernig viðföngin eru í raun stöðubundin, svo loks í kóðabút \ref{lst:foll-stodubundin-betra} sjáum við hvernig breytuheitin geta hjalpað okkur við notkun viðfanganna.

\begin{lstlisting}[caption=Villumelding fyrir ranga notkun á viðföngum, label=lst:foll-villa-vidfang]
def fall_sem_tekur_vid_streng(strengur):
	print(strengur.capitalize())
fall_sem_tekur_vid_streng()
\end{lstlisting}
\lstset{style=uttak}
\begin{lstlisting}
---------------------------------------------------------------------------
TypeError                                 Traceback (most recent call last)
<ipython-input-128-0e947c4208c2> in <module>
1 def fall_sem_tekur_vid_streng(strengur):
2         print(strengur.capitalize())
----> 3 fall_sem_tekur_vid_streng()

TypeError: fall_sem_tekur_vid_streng() missing 1 required positional argument: 'strengur'
\end{lstlisting}
\lstset{style=venjulegt}

Hér sjáum við að breytuheitið okkar hjálpar okkur við að sjá hvað það er sem við eigum að setja inn, það er einhvers konar stengur.
Það er ekki breytan strengur því að þá fengjum við nafnavillu (við eigum enga breytu sem heitir strengur).

Við þurfum þá að kalla í fallið og setja eitthvað inn í svigann þegar við köllum í það, eins og 'Halló Heimur!' sem myndi þá prenta út "Halló heimur!".

Nú viljum við skoða röðunina á viðföngunum ef fallið tekur við nokkrum.
Athugum svo að það skiptir máli í hvað röð viðföngin eru sett inn, í línum 4-6 í kóðabút \ref{lst:foll-stodubundin-kynning} er verið að setja inn fyrir \texttt{a}, \texttt{b} og \texttt{c} en aldrei í sömu röð og því er úttakið alltaf mismunandi.

\begin{lstlisting}[caption=Stöðubundin viðföng kynnt, label=lst:foll-stodubundin-kynning]
def fall(a,b,c):
	print(a**b/c)

fall(1,2,3)
fall(2,3,1)
fall(3,1,2)
\end{lstlisting}
\lstset{style=uttak}
\begin{lstlisting}
0.3333333333333333
8.0
1.5
\end{lstlisting}
\lstset{style=venjulegt}

Athugum að í kóðabút \ref{lst:foll-stodubundin-kynning} þá er hvorki nafnið á fallinu né viðfanga þess sérlega lýsandi.
Nafnið á fallinu gefur ekki til kynna hvað það gerir og nöfnin á viðföngunum segja ekkert til um hvernig þau verða notuð eða af hvaða týpu þau eiga að vera.
Þetta fall væri mögulega nothæft fyrir okkur sjálf, en um leið og annað fólk á að fara að nota kóðann okkar þá er eins gott að venja sig af því að nota svona ógegnsæjar nafnavenjur.
Betra væri, fyrir þessa tilteknu formúlu að finna eitthvað nafn á hana eða nefna fallið eftir nákvæmri virkni formúlunnar og nefna svo viðföngin eftir því hvar þau eru sett inn í formúluna.

Hingað til hafa öll föllin okkar verið að prenta út, við viljum geta gert eitthvað annað en það.
Við sjáum í kafla \ref{uk:skilagildi} hvað hægt er að gera annað en að prenta bara út.

Lítum á kóðabút \ref{lst:foll-stodubundin-betra} til að sjá hvernig betur mætti fara með skilgreininguna úr kóðabút \ref{lst:foll-stodubundin-kynning}.
Takið eftir að úttakið er að sjálfsögðu það sama því að kallað er í fallið með sömu tölum, það eina sem breyttist er að nú vitum við hvað er að fara inn fyrir hvað í útreikningunum.

\begin{lstlisting}[caption=Stöðubundin viðföng með skýrari breytuheitum, label=lst:foll-stodubundin-betra]
def hefja_i_veldi_og_deila(grunntala,veldisvisir,deiling):	
	print(grunntala**veldisvisir/deiling)

hefja_i_veldi_og_deila(1,2,3)
hefja_i_veldi_og_deila(2,3,1)
hefja_i_veldi_og_deila(3,1,2)
\end{lstlisting}
\lstset{style=uttak}
\begin{lstlisting}
0.3333333333333333
8.0
1.5
\end{lstlisting}
\lstset{style=venjulegt}

Ef við rifjum upp hugtakið gildissvið úr fyrri undirkafla og skoðum hvernig það á við um kóðabút \ref{lst:foll-stodubundin-betra} þá sjáum við að fallið heitir \texttt{hefja\_i\_veldi\_og\_deila} og það tekur við þremur viðföngum sem verða að vera sett inn í þeirri röð sem fallið kallar eftir þeim, fyrst grunntöluna svo töluna sem á að nota sem veldisvísi og svo loks töluna sem á að deila með.

Nú er við hæfi að taka fram hvers konar byrjenda mistök eru algeng hérna og ef meiri skilningur værir fyrir hendi á hvernig gildissvið virka myndu þessi mistök eiga sér sjaldnar stað.
Það er að breyturnar grunntala, veldisvisir og deiling eru hluti af staðværu gildissviði þessa falls, neðri hluta brunnsins sem efri hlutinn getur ekki sótt vatn úr.
Þess vegna getum við ekki kallað í fallið svona: \texttt{hefja\_i\_veldi\_og\_deila(grunntala,veldisvisir,deiling)} (það er skrifað inn nöfnin á viðföngunum eins og við eigum þau til sem breytur) því að þá erum við að biðja víðværa gildissviðið (efri hluta brunnsins) um að finna hjá sér einhverjar breytur sem heita þetta til þess að setja inn í staðinn fyrir þessar skilgreiningar.

Ef við horfum aftur til skilgreiningar hrings og myndum búa til fall sem lýsir því hvernig eigi að teikna hring með ákveðna miðju og tiltekinn radíus.
Nú ef okkur langar til að fá einhvern hring í hendurnar getum við ekki sagt við vélina láttu okkur hafa hring með radíus \texttt{radius} nema að sú breyta hafi þegar fengið gildið sem við ætluðum að nota.
Vélin, ef við manngerum hana örlítið, myndi þá segja ,,það er það sem þú átt að segja mér, hvað radíus er, ég veit ekkert hvað það er!''
Til þess að nota fallið þurfum við að gefa annað hvort upp gögn af týpunni sem um var beðið eða breytu sem er aðgengileg utan fallsins (annað hvort úr víðværu gildissviði eða víðara staðværu gildissviði) sem inniheldur gögn af týpunni sem um var beðið.

Tölvan reynir ekki að hafa vit fyrir okkur, hún lagar ekki inntakið þegar við setjum það í augljóslega ranga röð.
Hún annað hvort vinnur með vitlausa inntakið okkar og við fáum í hausinn eitthvað úttak sem við skiljum ekki eða við fáum villu.
Kíkið á kafla \ref{k:villur} til þess skoða hvernig megi taka á því þegar upp koma villur en við viljum að kóðinn okkar haldi samt áfram að keyra.

\subsection{Sjálfgefin viðfögn}\label{uk:föll-sjálfgefin}
Ef við viljum vera viss um að við getum unnið með eitthvað viðfang án þess að neyða notandann til þess að gefa okkur það getum við notað sjálfgefin viðföng (e. named arguments, default arguments), einnig kölluð nefnd viðföng.
Þá skilgreinum við fall og búum til skilgreiningu á viðföngunum þar sem við tökum þau fram.
Skoðum þetta í kóðabút \ref{lst:foll-sjalfgefin} þar sjáum við hvernig megi nota bæði stöðubundin og sjálfgefin saman og hvernig megi kalla í föll sem eru með bæði.

Þegar bæði er notað saman í bland þá þarf að setja stöðubundnu viðfögnin fremst og svo á eftir þeim koma sjálfgefnu viðföngin.

Athugum að bæði er í lagi að nota sjálfgefin viðföng sem stöðubundin, en þá verðum við líka að setja þau inn í réttri röð, og hins vegar að skrifa inn nafnið á viðfanginu og skilgreina það sem eitthvað (hvort sem það séu einhver gögn sem við setjum beint inn eða notum breytu), einnig er í lagi að sleppa því að taka þau fram.

\begin{lstlisting}[caption=Sjálfgefin viðföng kynnt, label=lst:foll-sjalfgefin]
def hefja_i_veldi_og_deila(grunntala = 1,veldisvisir = 1,deiling = 1):	
	print(grunntala**veldisvisir/deiling)
	
hefja_i_veldi_og_deila()
hefja_i_veldi_og_deila(2, 2, 2)
hefja_i_veldi_og_deila(deiling = 2)
hefja_i_veldi_og_deila(deiling = 2, veldisvisir = 1, grunntala = 4)

def hefja_i_veldi_og_deila_2(grunntala,veldisvisir,deiling = 1):	
	print(grunntala**veldisvisir/deiling)

print()
hefja_i_veldi_og_deila_2(1,2) 
hefja_i_veldi_og_deila_2(1, 2, 2) 
hefja_i_veldi_og_deila_2(1, 2, deiling = 4) 

print()
tala = 2
veldi = 2
deila = 2
hefja_i_veldi_og_deila_2(tala, veldi, deiling = deila)
\end{lstlisting}
\lstset{style=uttak}
\begin{lstlisting}
1.0
2.0
0.5
2.0

1.0
0.5
0.25

2.0
\end{lstlisting}
\lstset{style=venjulegt}

Skoðið hér hvernig er kallað í fallið \texttt{hefja\_i\_veldi\_og\_deila} í línum 4-7 í kóðabút \ref{lst:foll-sjalfgefin} og berið það saman við hvernig var kallað í föllin í kóðabútum \ref{lst:foll-villa-vidfang} og \ref{lst:foll-stodubundin-betra}.
Þar þurfti alltaf að setja inn öll viðföng annars fékkst villa, en vegna þess að í  skilgreiningunni kom fram að sjálfgefin gildi eru ákveðnar tölur sem hægt er að nota í útreikningum þá lendir vélin ekki í vandræðum þegar einhver viðföng vantar.

Að sjálfsögðu megum við ekki setja inn of mörg viðföng, þá lendum við í vandræðum, prófið ykkur áfram með það.

Takið líka eftir í línu 7 er kallað í viðföngin í ,,rangri'' röð með því að nota nöfnin á þeim.
Prófið að gera það með línu 21, að setja breyturnar inn fyrir viðföngin í annarri röð.
Takið eftir því að stöðubundnu viðföngin verða að koma á undan.

Í línu 5 og 14 er kallað eins í föllin, þá er komið fram við sjálfgefnu viðföngin eins og stöðubundin.
Þannig að sett er inn fyrir viðföngin í þeirri röð sem þau eru skilgreind í línum 1 og 9.

Eins og hefur komið fram er nauðsynlegt að gera tilraunir og prófanir til að ná árangri og skilningi á efninu.


\section{Skilagildi}\label{uk:skilagildi}
Nú höfum við séð hvernig á að búa til föll, við höfum séð hvernig á að láta föllin vinna með sjálfgefið inntak og það næsta sem við viljum skoða er hvernig á að láta föllin okkar skila gildum þannig að hægt sé að fá útkomu úr þeim sem megi nota áfram.
Við höfum séð hvernig aðferðir á strengi skila oft til okkar öðrum streng sem byggir á strengnum sem við notuðum aðferðina á (sjá kóðabút \ref{lst:str-adferdir}), til þess að geta notað úttkomuna þá getum við búið til breytu sem grípur það sem aðferðin \textit{skilar}.
Í kóðabútum \ref{lst:foll-stodubundin-betra} og \ref{lst:foll-sjalfgefin} þá er fallið hefja\_i\_veldi skilgreint, en aldrei er hægt að vinna eitthvað með úttkomuna úr útreikningnum, það sem er gert er annað hvort ekkert eða úttkoman er prentuð.
Það dugar okkur ekki ef við þurfum að nota útreikninginn að þurfa að horfa á það sem vélin skrifar út og skrifa það handvirkt inn sjálf.
Það sem við viljum geta sagt er ,,hey reiknaðu þetta út og notaðu það svo hér, mér er alveg sama hvað það er því að ég treysti því að þú hafir gert það rétt.''
Því að við manngerum tölvuna að sjálfsögðu, hvað gæti farið úrskeiðis?

Til þess að geta nýtt þessa virkni þurfum við að læra nýtt lykilorð sem er \textbf{return} sem þýðir skila, mjög gagnsætt og gott lykilorð (eins og flest lykilorðin sem við höfum séð hingað til, að mati höfundar).
Það sem lykilorðið gerir er svipað break lykilorðinu, þegar vélin kemur að línu þar sem return kemur fram þá gerir fallið ekkert meira en að skila því sem beðið er um og vélin heldur áfram í næstu línu frá því kallað var í fallið.
Sjáum í kóðabút \ref{lst:foll-skilagildi}

\begin{lstlisting}[caption=Hvernig á að láta fall skila gildi með return skipuninni, label=lst:foll-skilagildi]
def hefja_i_veldi_og_deila(grunntala,veldisvisir,deiling = deiling):	
	return grunntala**veldisvisir/deiling
	
# nú höfum við séð mýgrút leiða til að kalla í fallið svo við ætlum að einbeita okkur að því hvernig megi nota það sem fallið skilar í stað þess hvernig eigi að kalla í það með mismunandi viðföngum.
# Athugum að  hefja_i_veldi_og_deila(2, 2, 2) skilar okkur 2

tala = fall(2,2,2)
print(tala)
# hér verður úttakið 2

utreikningur = fall(2,2,2) * 4
print(utreikning)
# hér verður úttakið 8

breyta = utreikningur * tala
print(breyta)
# hér verður úttakið 16
\end{lstlisting}
 
Nú höfum við skilað einu gildi sem er heiltala eða fleytitala og unnið með hana með þeim aðgerðum og reiknivirkjum sem okkur langaði til að prófa.
Hægt er að skila gögnum af hvaða týpu sem er, og jafnvel fleiru en einu gildi í einu, og það af mismunandi týpum.

Sjáum í kóðabút \ref{lst:foll-ndir-skilagildi} hvernig hægt er að skila mörgum gildum og setja þau í breytu eða breytur.

\begin{lstlisting}[caption=Hvernig á að skila mörgum gildum, label=lst:foll-ndir-skilagildi]
def skilum_morgum_gildum(strengur, tala, listi):
	# Þetta fall tekur við streng, tölu og lista
	# Fallið skilar tveimur tölum og listanum aftur óbreyttum
	# Fyrri talan er hversu oft strengurinn í viðfanginu kom fyrir í listanum
	# Seinni talan er hversu margar tölur í listanum eru stærri en talan í viðfanginu
	
	skilatala = 0
	strengja_talning = 0
	for x in listi:
		# x stendur það stak sem verið er skoða úr listanum listi
		if x == strengur:
			strengja_talning +=1
		if(type(x) == int or type(x) == float ):
			if (tala < x):
				skilatala += 1
	return strengja_talning, skilatala,  listi
	
# hér gerist ekkert nema að við köllum í fallið
skilum_morgum_gildum("halló", 2, ["halló", "bless", 11, 6])
# Þetta skilar okkur úttakinu
(1, 2, ["halló", "bless", 11, 6])

# En ef okkur langar ekki að fá út n-d þá getum við gert dálitla töfra 
talning_strengs, staerri_tolur, listinn = skilum_morgum_gildum("halló", 2, ["halló", "bless", 11, 6])

# Nú fáum við ekkert úttak en breyturnar innihalda 1, 2 og ["halló", "bless", 11, 6] í þessari röð

# Við getum ekki notað fleiri en eina breytu nema að fjöldi þeirra passi við fjölda skilagilda:
a, b = skilum_morgum_gildum("halló", 2, ["halló", "bless", 11, 6])

# þetta veldur villu því annað hvort er í boði að hafa eina breytu sem inniheldur n-d eða þrjár breytur sem taka hver við sínu skilagildi.
\end{lstlisting}

\section{Lokun}

Föll mega innihalda önnur föll, athugum það sem við fórum yfir í kafla \ref{uk:gildissvið}, þessi föll geta verið gagnleg til að útfæra útreikning sem er svo notaður oft innan fallsins.
Sjáum dæmi í kóðabút \ref{lst:foll-innri-foll}, athugið sérstaklega gildissviðið því að í innra fallinu er vísað í viðfang sem heitir strengur og það er líka vísað í viðfang í ytra fallinu sem heitir strengur, en vegna þess að gildissviðið krefst þess að fyrst er athugað staðvært hvernig breytan er skilgreind þá skiptir ekki máli að breyturnar heita það sama.

\begin{lstlisting}[caption=Innri föll kynnt, label=lst:foll-innri-foll]
def breyta_strengjum(strengur):
	# þetta fall tekur við streng og skilar honum eftir að hafa breytt honum einhvern veginn
	
	def fremsti_stafur_er_nuna_aftastur(strengur):
		# þetta fall tekur við streng og lætur fremsta stafinn í honum aftast og aftasta stafinn fremst og skilar breytingunni 
		# ef strengurinn er eitt eða færra stafabil gerir fallið ekkert við strenginn og skilar honum óbreyttum
		if(len(strengur > 2)):
			fremst = strengur[0]
			aftast = strengur[-1]
			strengur = aftast + strengur[1:-1] + fremst
			return strengur
		else:
			return strengur
			
	# nú er skilgreiningin á innra fallinu búin og við erum stödd í ytra fallinu
	# nú getum við kallað í innra fallið
	
	skilastrengur = fremsti_stafur_er_nuna_aftastur(strengur)
	return skilastrengur
	
# Nú getum við kallað í ytra fallið
strengur = breyta_strengjum("halló")
# strengur inniheldur núna
# "óllah"

# En ytra fallið gerði samt eiginlega ekki neitt, hvers vegna erum við að nota það en ekki innra fallið?
strengur =  fremsti_stafur_er_nuna_aftastur("halló")

# hér fáum við villu því að það er ekkert fall til, í því gildissviði sem við erum stödd, sem heitir þessu nafni sem við höfum aðgang að, þess vegna verðum við að nota ytra fallið.
		
\end{lstlisting}

Nú höfum við séð hvernig megi skilgreina innri föll og það sem við ætlum að skoða næst er að það má skila föllum.
Skipunin return er þá notuð alveg eins og ef við værum að skila einu gildi, eða fleirum.
Sjáum í kóðabút \ref{lst:foll-lokun} hvernig við skilum falli og hvernig á að nota skilagildið sem inniheldur fallið.
Við sjáum í seinni hluta þessarar bókar, í umfjöllun um klasa, hvernig megi framkvæma sömu virkni en með því að sleppa klösum þá er vinnslutíminn umtalsvert minni.
Svo ef það skiptir máli að gera eitthvað hratt sem má leysa með lokun þá ætti frekar að beita henni heldur en klösum.

\begin{lstlisting}[caption=Lokun kynnt, label=lst:foll-lokun]
def prentunarfall(strengur):
	# prentunarfall tekur við einu viðfangi
	# prentunarfall skilar falli sem prentar þann streng ásamt viðbót
	def prentum():
		# prentum tekur ekki við neinu en notar viðfangið úr ytra fallinu
		# prentum skilar strengnum úr viðfanginu ásamt viðskeytingunni 'hér er viðbót af akademískri ástæðu'
		return str(strengur) + " hér er viðbót af akademískri ástæðu"
	return prentum

a = prentunarfall('halló heimur')
print(a) # skilar okkur að a sé fall sem sé geymt í einhverju minnishólfi
print(a()) # nú köllum við í fallið a sem skilar okkur úttakinu 'halló heimur hér er viðbót af akademískri ástæðu'

# Við megum svo nota prentunarfallið okkar aftur með einhverju öðru viðfangi
b = prentunarfall('nýr strengur')
print(b()) # skilar okkur úttakinu 'nýr strengur hér er viðbót af akademískri ástæðu'

# Sjáum ögn haldbærari notkun

def niðurtalning(n):
	# niðurtalning tekur við tölu
	# niðurtalning skilar falli sem telur niður úr þeirri tölu
	
	def teljari():
		# nú lendum við í vandræðum með n, þar sem við höfum ekki aðgang að því hérna 
		# og við lögum það með því að nota lykilorðið nonlocal (eða óstaðvært) til að segja vélinni að leita annarsstaðar að gildi fyrir n
		nonlocal n
		while(n>-1):
			print(n)
			n -= 1
	
	return teljari
	
# nú getum við skilgreint fall sem telur niður frá 10 og notað það þegar okkur hentar
teljum_fra_tiu = niðurtalning(5)

# nú hentar okkur að nota það:
teljum_fra_fimm()
# það skilar okkur úttakinu
# 5
# 4
# 3
# 2
# 1
# 0
\end{lstlisting}

Tökum eftir að í skilgreiningum á a og b í kóðabút \ref{lst:foll-lokun} þá erum við að nota ákveðinn streng sem þau eiga að prenta út.
Þetta er á þessu stigi málsins eilítið óhlutstætt og ekki augljóst hvernig það nýtist okkur því dæmið í kóðabútnum er ekki sérlega nothæft fall.
Niðurtalningarfallið hinsvegar er að framkvæma einhverja virkni sem við viljum hafa aðgang að þegar okkur hentar.
Við notuðum breytuna teljum\_fra\_fimm til þess að geyma fyrir okkur að kalla í fallið niðurtalning() með viðfanginu 5, svo þegar okkur hentar þá getum við beitt fallinu sem telur niður fyrir okkur.
Athugum þó sérstaklega að gildissviðið sem innrafallið teljari() tilheyrir það hefur ekki aðgang að neinu staðværu n-i svo það skilar villu nema að við segjum því falli sérstaklega að nota ekki staðvært n heldur leita út fyrir gildissviðið með lykilorðinu \textbf{nonlocal}.
Við munum ekki nota það orð af neinu viti í seinni hluta bókarinnar en það er þess virði að taka það fram að svo stöddu að þetta orð sé til og hvað það gerir.


%----------------------------------------------------------------------------------------
%	PART
%----------------------------------------------------------------------------------------

\part{Seinni hluti-\\ Hlutbundin forritun}\label{Seinni hluti - hlutbundin forritun}


\chapterimage{chapter_head_1.pdf} % Chapter heading image

\chapterimage{chapter_head_2.pdf} % Chapter heading image

\chapter{Kóðasöfn}\index{Kóðasöfn}\label{k:import}
Kóðasafn (e. library) er endurnýtanlegur kóði, sem sem útfærir ákveðna virkni og hefur ákveðið samhengi.
Tilgangur þeirra er að spara forriturum vinnu við að útfæra ýmsa algenga virkni og reiða sig í staðinn á kóða sem er nú þegar til.
Hjálpar okkur við að vera ekki að finna upp hjólið í sífellu.
Ómögulegt er að ætla að forrita að einhverju viti án þess að nota kóðasöfn.

Við notum kóðasöfn með \textbf{import} skipuninni. 
Þegar import hefur verið sett inn einhvers staðar í skjal þá er óþarfi að setja það inn aftur, venjan er að öll import eru gerð efst í skjali burt séð frá því hvar í skjalinu þau eru notuð.
Það gerir kóðann læsilegri og undirbýr okkur við lestur á kóða hvað er að fara að gerast.
Sem dæmi ef efst í skjali stendur að kóðasöfnin math og random séu notuð þá vitum við strax að í þessum kóða sé verið að vinna með einhverja handahófskennd á stærðfræðilegan máta, en ef efst stæði að kóðasöfnin datetime og time væru notuð þá erum við líklega að skoða kóða sem er að vinna með tíma og dagsetningar, það ætti þá ekki að koma okkur á óvart að sjá dagsetningarvinnslu.

\section{Notkun kóðasafna}\index{Tilgangur kóðasafna}\label{uk:kóðasöfn-kynnt}
Eins og kom fram í kynningu þá viljum við geta einbeitt okkur að því að leysa okkar vandamál í stað þess að finna upp hjólið og því viljum við kynna okkur þau kóðasöfn sem eru í boði sem útfæra virkni sem við viljum beita.

Tilgangur þeirra er að létta okkur lífið og gera virkni aðgengilega.
Í næsta undirkafla verða tekin fyrir nokkur gagnleg kóðasöfn en við getum varla talað um tilgang og gagnsemi kóðasafna án þess að taka eitthvert þeirra fyrir.
Í inngangi voru kóðasöfnin time og random nefnd.
Skoðum þau aðeins núna, sjá kóðabút \ref{lst:kóðasöfn-kynnt} þar sem kóðasöfnin time og random eru tekin fyrir.
Þau bjóða bæði upp á aragrúa aðferða og eiginda sem er út fyrir efni þessarar bókar að ræða í þaula en þó þess virði að taka fyrir ákveðna virkni sem búist er við að nota í æfingum í lok kaflans.
\paragraph
Um kóðasafnið time:

\begin{itemize}
	\item time.time() skilar okkur hversu margar sekúndur eru síðan tímatal í tölvum hófst 1.jan 1970
	\item time.sleep() tekur við tölu og lætur vélina bíða það lengi áður en hún framkvæmdir aðgerðina í næstu línu fyrir neðan
	\item time.localtime() skilar okkur nd sem inniheldur í minnkandi röð hver tíminn er, frá ári niður í sekúndur, ásamt deginum í vikunni og árinu og síðasta er gildi sem tekur mið af \textbf{is}\textbf{d}aylight\textbf{s}avings\textbf{t}ime
\end{itemize}

\paragraph
Um kóðasafnið random:Þetta kóðasafn gerir forriturum auðveldara fyrir með því að gera handahófskennd (e. randomness) aðgengilega, það að geta gert hluti af handahófi er mjög mikilvægt í tölvunarfræði og forritun.

\begin{itemize}
	\item random.randint()
	\item random.random()
	\item random.choice()
	\item random.choices()
	\item random.randrange()
	\item random.shuffle()
\end{itemize}

\begin{lstlisting}[caption=Notkun Kóðasafna, label=lst:kóðasöfn-kynnt]
# til þess að nota kóðasafn þarf að beita import skipununni

import time

# nú er kóðasafnið time aðgengilegt með breytuheitinu time
\end{lstlisting}

\section{Nokkur gagnleg kóðasöfn}\index{Nokkur gagnleg kóðasöfn}\label{uk:kóðasöfn-gagnleg}
lorem ipsum

\chapterimage{chapter_head_2.pdf} % Chapter heading image

\chapter{Skjalavinnsla}\index{Skjalavinnsla}\label{k:skjalavinnsla}
Að vinna með skjöl án þess að hafa þau opin í ritli, eins og MS Word, LibreOffice Write, Notepad eða álíka, er mjög ákjósanlegt ef t.d. þarf að gera litla breytingu í stóru skjali eða einhverja breytingu í mjög mörgum skjölum (skilgreiningin á mjög mörgum er sveiginaleg, sumum finnst það vera að gera eitthvað oftar en þrisvar).
Segjum sem svo að við hefðum skrifað ritgerð og við yfirlestur á henni tækjum við eftir að við gerðum eina ákveðna villu alltaf.
Villan væri að við hefðum gleymt að setja stóran staf í upphafi allra setinganna okkar!
Ó nei, hvernig tókst okkur að gleyma þessu?
Það á eftir að taka óratíma að lesa yfir og laga hvern einasta staf því ritgerðin er 20 blaðsíður.
En vegna þess að við erum snjöll og kunnum að vinna með strengi þá getum við gert þetta með hjálp Python.

Hægt er að búa til skjöl, opna þau, lesa þau, skrifa inn í þau, yfirskrifa þau, loka þeim og henda þeim.
Þetta er mikilvægt vegna þess að við viljum geta sagt tölvunni að nálgast skjöl og gera eitthvað við þau, við viljum ekki þurfa að handstýra tölvunni að óþörfu, til dæmis með því að lesa sjálf yfir 20 blaðsíður í leit að litlum staf þar sem á að vera stór.
Til þess er tölvan.

\begin{lstlisting}[caption=Hér sjáum við hvernig má búa til skjöl, label=lst:skjalavinnsla-kynning]
skjal = open('skjal1.txt', mode = 'w+') 

skjal.write('hér kemur eitthvað sem við viljum setja inn í skjalið okkar')

skjal.close()
\end{lstlisting}

Nú getum við séð, á sama stað og þessi kóði er keyrður, í skrársafninu (e. filesystem) okkar skjal sem heitir \texttt{skjal1.txt} vegna þess að við völdum það nafn í línu 1 innan svigans, ástæðan fyrir því að þarna stendur \emph{.txt} er sú að sú skráarending er fyrir einföld textaskjöl\footnote{svipað eins og .doc sem við könnumst flest við}.
Þar kemur einnig fram \texttt{mode} sem er valið sem \texttt{w+}, sem leyfir okkur að lesa og skrifa eða yfirskrifa.
Við veljum máta sem hentar okkur hverju sinni, þeir sem eru í boði eru:
\begin{itemize}
	\item \textbf{r} : leyfir okkur aðeins að lesa (read).
	\item \textbf{w} : leyfir okkur aðeins að skrifa (write).
	\item \textbf{a} : leyfir okkur aðeins að bæta við aftast (append).
	\item \textbf{r+} : leyfir okkur að lesa og skrifa (read +).
	\item \textbf{w+} : leyfir okkur að lesa og skrifa og yfirskrifar (write +).
\end{itemize}

Við tökum einnig eftir í kóðabút \ref{lst:skjalavinnsla-kynning} að við notum þrjú föll, fallið \texttt{open()} er það sem býr til skjalið með w+ og nafninu sem við gefum því, þá er skjalið opið og aðgengilegt.
Þá köllum við í \texttt{write} sem skrifar inn í skjalið, sú skipun er í boði vegna þess að við notuðum w+ (er ekki möguleg fyrir r t.d.).
Við erum vissulega bara að gera einfalda hluti en þetta er til að sýna virknina í grunninnn ekki til að finna upp hjólið.
Svo að lokum sjáum við eitthvað áhugavert, það er \texttt{close()}.
Til hvers að gera það?
Hafið þið lent í því að við það að reyna að henda skjali í ruslið af tölvunni ykkar fáið þið villu um að það sé ekki hægt því að skjalið er opið einhversstaðar?
Það er það sem við erum að fyrirbyggja hérna með því að loka skjalinu þegar við erum búin að vinna með það.
Við erum í rauninni bara að ganga frá eftir okkur svo að það sé ekki eitthvað opið sem er fyrir.

\section{Unnið með skjöl}\index{Unnið með skjöl}\label{uk:skjalavinnsla-kynnt}
Til þess að geta unnið með skjöl þurfum við að geta vísað í þau, til þess notum við breytur.
Eins og breytan \texttt{skjal} í kóðabúr \ref{lst:skjalavinnsla-kynning}.
Breytan okkar stendur þá ekki fyrir einhverja grunntýpu í Python heldur er það vísun í heilt skjal á skráarsafninu okkar.

Í kóðabút \ref{lst:skjalavinnsla-open} koma fram nokkrar aðferðir sem Python býður upp á fyrir skjöl sem búið er að opna.
Til þess að sækja upplýsingar úr skrá þarf tölvan að lesa hana.
Við tökum eftir því að þar er eitthvað til sem heitir \textit{seek}, ástæðan fyrir því að við þurfum það er að tölvan les frá vinstri til hægri eins og henni er sagt en ef hún á að lesa eitthvað aftur frá byrjun eða öðrum stað þá þarf að segja henni að færa leshausinn sinn þangað.
Nú er hætta á því að enginn lesandi hafi nokkurn tímann séð segulband en hugmyndin þar er sú sama, tölvan sem les segulbandið getur bara lesið bandið sem er undir leshausnum og sér ekkert annað.
Ef tölvan á að lesa einhvern annan hluta af segulbandinu þarf að spóla fram eða til baka.
Sama er upp á teningnum hérna, við þurfum að stilla leshausinn fyrir framan það gildi sem við viljum lesa hverju sinni.
Ef vélin er búin að lesa skjalið er leshausinn kominn út í enda og við getum ekki lesið meira nema færa hann.

Þetta er svipað því eins og ef við settum puttann niður í bók og mættum bara lesa orðin fyrir ofan puttann og svo bara til hægri, þá til að lesa eitthvað aftur eða fara lengra inn í bókina þyrftum við að taka upp puttann og færa hann þangað.

Byrjum á að gera textaskjalið okkar aðeins bitastæðara með því að keyra eftirfarandi kóða í sér sellu í vinnubók\footnote{virkar aðeins í Jupyter Notebooks og er ekki sér Python fyrirbæri heldur sérstætt fyrir þessar vinnubækur, þetta leyfir okkur að sleppa við $\backslash$n eða newline character og æfingar tengdum því}:
\begin{verbatim}
%%writefile skjal1.txt
hér kemur eitthvað sem við viljum setja inn í skjalið okkar
hér er næsta lína í skjalinu
úps engin lína endar á punkti og engin lína hefst á stórum staf
En nú lagast það.
æ, það gleymdist stór stafur.
Og nú gleymdist punktur
\end{verbatim}

Skoðum svo hvað hægt er að gera við þennan texta\footnote{Þarna kemur fram annað viðfang sem heitir encoding, þarna er það valið sem utf-8 sem er það táknasafn sem nær yfir alla séríslenka stafi. Ef þið lendið í vandræðum við að íslenskir stafir eru í ruglinu er gott að vita til þess að stafakóðunin er þá mögulega önnur en utf-8. Þetta á alls ekki bara við um Python heldur er þetta alþjóðlegur staðall sem nýtist hvarvetna.}.

\begin{lstlisting}[caption=Hér sjáum við einfalda skjalavinnslu, label=lst:skjalavinnsla-open]
vinnuskjal = open('skjal1.txt', encoding = 'utf-8', mode = 'r')	

linurnar = vinnuskjal.readlines()
vinnuskjal.seek(150)
print(vinnuskjal.read())

for lina in linur:
	if lina[0].islower() and lina[-1] != ".":
		print("línan byrjar á litlum staf og endar ekki á punkti")
	elif lina[0].islower() or lina[-1] != ".":
		print("línan endar ekki á punkti eða byrjar á litlum staf")
	else:
		print(lina)

vinnuskjal.close()
\end{lstlisting}
\lstset{style=uttak}
\begin{lstlisting}
t á stórum staf
En nú lagast það.
æ, það gleymdist stór stafur.
Og nú gleymdist punktur

línan byrjar á litlum staf og endar ekki á punkti
línan byrjar á litlum staf og endar ekki á punkti
línan byrjar á litlum staf og endar ekki á punkti
línan endar ekki á punkti eða byrjar á litlum staf
línan byrjar á litlum staf og endar ekki á punkti
línan endar ekki á punkti eða byrjar á litlum staf
\end{lstlisting}
\lstset{style=venjulegt}

Í kóðabút \ref{lst:skjalavinnsla-open} eru teknar fyrir þær helstu aðferðir sem eru í boði fyrir lestur á skjali \texttt{read()} og \texttt{readlines()}, annað gefur okkur streng en hitt gefur okkur lista af línum sem við getum ítrað í gegnum.
Takið eftir því að leshausinn er settur á stað 150 og svo er útkoman prentuð, en það hefur ekki áhrif á \texttt{linur} því að sú breyta var skilgreind þegar leshausinn var á 0 og færðist út í enda við það að nota \texttt{readlines()}.
Prófið ykkur áfram með röðunina á kóðalínunum.
 
Það gæti verið vesen að þurfa að muna eftir því að loka skjalinu og því viljum við skoða annan möguleika með nýju lykilorði \textbf{with}, við höfum séð \textbf{as} sem býr til \textit{alias} eða annað heiti.
Sjáum kóðabút \ref{lst:skjalavinnsla-open-as}, þar sem öll vinnslan í skjalinu tilheyrir inndrætti undir \texttt{with} og \texttt{as}.
Þetta er eins og með föll, þegar við skrifum eitthvað í sama inndrætti og í línu 1 þá erum við komin út fyrir skjalavinnsluna okkar og skjalið er ekki lengur aðgengilegt því þá er verið að reyna að vinna með skrá sem er lokuð.
 
\begin{lstlisting}[caption=Hér sjáum við nýja leið til að opna skjal og loka því sjálfkrafa, label=lst:skjalavinnsla-open-as]
with open('skjal1.txt', encoding='utf-8') as test:
	efni = test.read()
	test.seek(0)
	efni_i_listum = test.readlines()

with open('skjal1.txt', encoding='utf-8', mode= 'w+') as blergh:
	efni = blergh.write('Ég yfirskrifaði allt og á nú skjal sem heitir það sama en inniheldur bara þetta')
	blergh.seek(0)
	efni_i_listum2 = blergh.readlines()

print(efni_i_listum)
print(efni_i_listum2)
\end{lstlisting}
\lstset{style=uttak}
\begin{lstlisting}
['hér kemur eitthvað sem við viljum setja inn í skjalið okkar\n', 'hér er næsta lína í skjalinu\n', 'úps engin lína endar á punkti og engin lína hefst á stórum staf\n', 'En nú lagast það.\n', 'æ, það gleymdist stór stafur.\n', 'Og nú gleymdist punktur\n']
['Ég yfirskrifaði allt og á nú skjal sem heitir það sama en inniheldur bara þetta']
\end{lstlisting}
\lstset{style=venjulegt}

Tökum eftir hér að við fáum villu við að reyna að vísa í breyturnar \texttt{test} og \texttt{blergh} en hinar breyturnar sem við búum til á meðan vinnslunni stendur eru enn aðgengilegar eins og sést í línum 12 og 13.

En þá skulum við skoða hvernig við eigum að fara að því að laga textann án þess að opna skjalið handvirkt og breyta táknunum sjálf.
Það sem við vitum er að fremsti stafur á alltaf að vera stór og aftasta táknið á alltaf að vera punktur.
Við getum svo breytt handvirkt þeim fáeinu setningum sem við viljum að endi á spurningarmerki eða upphrópunarmerki.

\begin{lstlisting}[caption=Leysum punkta og hástafa vandann okkar, label=lst:skjalavinnsla-lausn]
with open('skjal1.txt', encoding='utf-8', mode= 'r') as lausn:
	lausn.seek(0)
	linur = lausn.readlines()
	for lina in linur:
		i = linur.index(lina)
		lina = lina[0].upper() + lina[1:]
		if lina[-2] != ".":
			lina = lina[:-1] + "." + lina[-1]
		linur[i] = lina

with open('skjal1.txt', encoding = 'utf-8', mode = 'w+') as laga:
	laga.writelines(linur)
	laga.seek(0)
	print(laga.read())
\end{lstlisting}
\lstset{style=uttak}
\begin{lstlisting}
Hér kemur eitthvað sem við viljum setja inn í skjalið okkar.
Hér er næsta lína í skjalinu.
Úps engin lína endar á punkti og engin lína hefst á stórum staf.
En nú lagast það.
Æ, það gleymdist stór stafur.
Og nú gleymdist punktur.
\end{lstlisting}
\lstset{style=venjulegt}

Athugið sérstaklega notkun á -1 og -2, athugið hvað gerist ef þið breytið því, þetta er útaf því að síðasta táknið er ,,new line character'' eða $\backslash$n (þó þetta eru í raun tvö tákn þá er þetta saman eitt tákn).
Þó það hefði mögulega verið minni hausverkur að laga þessar fáeinu setningar og að læra rétta stafsetningu þá er pælingin hérna var að við vorum búin að gera þetta síðustu tuttugu blaðsíðurnar og við vildum alls ekki gera einhverja villu í yfirferðinni með því að gleyma punkti einu sinni því að það fór framhjá okkur.
Við ættum að temja okkur að láta tölvuna sjá um allt það sem við erum fær um að útskýra fyrir henni hvernig eigi að gera.

%-------------------------------
\newpage
\section{Æfingar}

\begin{exercise}\label{doc1}
Búiið til skjal með .txt endingu, setjið inn í það nokkrar mismunandi línur (annað hvort með \%\%writefile eða annarri leið og munið þá eftir $\backslash$n til þess að fá mismunandi línur).
Lykkjið svo í gegnum línurnar og prentið út þær sem byrja á sérhljóða.
\end{exercise}
\setboolean{firstanswerofthechapter}{true}
\begin{Answer}[ref={doc1}]

	\begin{lstlisting}
%%writefile doc1.txt
Þessi byrjar ekki á sérhljóða
En þessi gerir það
\end{lstlisting}
\begin{lstlisting}
with open('doc1.txt', mode = 'r') as doc1:
linur = doc1.readlines()
for lina in linur:
if(lina[0].lower() in 'aáeéiíoóuúyýæö'):
print(lina)\end{lstlisting}
\end{Answer}
\setboolean{firstanswerofthechapter}{false}


\begin{exercise}\label{doc2}
Búið til textaskjal sem inniheldur söngtexta úr einhverju lagi (þá er gott að nota \%\%writefile), lesið svo skjalið og geymið línurnar í lista.
Prentið svo eina línu af handahófi úr listanum.
\end{exercise}
\begin{Answer}[ref={doc2}]
	\begin{lstlisting}
%%writefile doc2.txt
Krummi svaf í klettagjá,
kaldri vetrarnóttu á,
::verður margt að meini::
Fyrr en dagur fagur rann,
freðið nefið dregur hann
::undan stórum steini.::

Allt er frosið úti gor,
ekkert fæst við ströndu mor
::svengd er metti mína.::
Ef að húsum heim ég fer
heimafrakkur bannar mér
::seppi' úr sorp að tína.::

Öll er þakin ísi jörð,
ekki séð á holtabörð
::fleygir fuglar geta.::
En þó leiti út um mó,
auða hvergi lítur tó;
::hvað á hrafn að éta.::

Á sér krummi ýfði stél,
einnig brýndi gogginn vel,
::flaug úr fjallagjótum::
Lítur yfir byggð og bú
á bænum fyrr en vakna hjú,
::veifar vængjum skjótum.::

Sálaður á síðu lá
sauður feitur garði hjá,
::fyrrum frár á velli.::
Krunk, krunk, nafnar, komið hér,
krunk, krunk, því oss búin er
::krás á köldu svelli.::		
\end{lstlisting}
	\begin{lstlisting}
import random 
with open('doc2.txt', mode = 'r') as doc2:
	linur = doc2.readlines()
	print(random.choice(linur))
\end{lstlisting}
\end{Answer}

\chapterimage{chapter_head_2.pdf} % Chapter heading image

\chapter{Klasar og hlutir}\index{Klasar og hlutir}\label{k:klasar}
Forritun snýst um að meðhöndla gögn, hingað til höfum við kynnst nokkrum gagnatögum (t.d. strengir og listar), þær gegna mismunandi hlutverkum og bjóða upp á mismunandi aðgerðir til að vinna með gögnin.
Þessar innbyggðu týpur duga þó ekki alltaf og því er mikilvægt að vita að þegar við forritum getum við smíðað okkar eigin.
Þannig getum við aðlagað týpurnar okkar að þeim gögnum sem forritið okkar meðhöndlar og útfært okkar eigin aðferðir á þær.
Klasar gera forriturum kleift að skilgreina sína eigin hluti í flestum hlutbundnum málum, Python er hlutbundið forritunarmál.
Til þess að læra á hvernig eigi að búa til klasa þarf að átta sig á til hvers þeir eru nytsamlegir.

Gagnlegt er að hugsa sér klasa sem skilgreiningu eða uppskrift alveg eins og föll.
Skilgreiningin ein og sér gerir ekki neitt, það er ekki fyrr en við búum okkur til ákveðna útgáfu sem við getum farið að vinna með hana.
Gott dæmi um það er skilgreiningin á rétti á matseðli á veitingastað, textinn á matseðlinum er eingöngu hvað er í boði en er ekki útgáfa af matnum sjálfum.

Klasar eru hlutir sem hugsaðir eru til þess að búa til eintök af og geyma þannig eitthvert ástand og mögulega hafa áhrif á það.
Hugmyndin er að eiga hlut eða \emph{tilvik}, eina tiltekna útgáfu, sem má framkvæma aðgerðir á og eitthvað ástand hlutarins breytist eftir því hvað var gert, þannig er hægt að búa til mörg eintök af sama klasanum og láta hvert tilvik verða fyrir mismunandi áhrifum\footnote{Athuga þarf sérstaklega gildissvið þegar klasar eru annarsvegar, gildissvið í Python geta verið ögn ruglingsleg en við munum ekki beita klösum á það sérhæfðan máta að við lendum í miklum vandræðum.
Hér er tilvalið að skoða ,,meta programming''}.

\section{Klasar skilgreindir}\index{Klasar skilgreindir}\label{uk:klasar-skilgreindir}

Klasar nota lykilorðið \textbf{class} og eru skilgreindir með því orði, allt sem tilheyrir klasanum er inndregið undir honum.

\begin{lstlisting}[caption=Klasar skilgreindir, label=lst:klasar-skilgreindir-tegund]
class Bíll:
	tegund = "Citroen"

fyrsti_billinn = Bíll()
print(fyrsti_billinn.tegund)
\end{lstlisting}
\lstset{style=uttak}
\begin{lstlisting}
Citroen
\end{lstlisting}
\lstset{style=venjulegt}

Hugsum okkur að við búum til skilgreiningu á bíl, hann þarf að vera af einhverri tegund, skoðum línur 1-2 í kóðabút \ref{lst:klasar-skilgreindir-tegund}.
Svo viljum við fá tilvik af skilgreiningunni í hendurnar (lína 3), þá búum okkur til breytu sem fær gildi eins og við höfum gert hundrað sinnum áður, nema núna er gildið sem breytan fær nafnið á klasanum okkar ásamt svigum eins og við séum að kalla í hann.
Prófið núna að búa til annað tilvik af klasanum \texttt{Bíll} án þess að nota svigana og prófið þá að prenta út það sem \texttt{type} skilar fyrir breyturnar tvær.

\begin{lstlisting}[caption=Klasar skilgreindir, label=lst:klasar-skilgreindir-subaru]
class Bíll:
	tegund = "Citroen"
	
fyrsti_billinn = Bíll()
print(fyrsti_billinn.tegund)
annar_bill = Bíll()
annar_bill.tegund = "Subaru"
print(annar_bill.tegund)
\end{lstlisting}
\lstset{style=uttak}
\begin{lstlisting}
Citroen
Subaru
\end{lstlisting}
\lstset{style=venjulegt}

Breytan \texttt{fyrsti\_bilinn} kemur ekki í veg fyrir það að við getum átt fleiri bíla, en hún heldur utan um ástandið á nákvæmlega þessum bíl okkar.
Segjum að við fáum okkur svo annan bíl, þá getum við búið til aðra breytu (lína 6) í kóðabút \ref{lst:klasar-skilgreindir-subaru} fyrir annað tilvik af klasanum.
Bílarnir eru, fyrir okkur, óaðgreinanlegir í línu 6\footnote{Þar sem ekki hefur verið útfærð \_\_eq\_\_ aðferðin þá er notast við id() fallið úr type klasanum sem klasinn okkar erfir frá bakvið tjöldin.
Við skoðum erfðir betur seinna í kaflanum.} en það breytist svo snarlega þegar við endurskilgreinum \emph{klasabreytuna}\footnote{Í öðrum hlutbundnum málum er venjulega talað um klasafasta en í Python er auðvelt að breyta þeim svo við hæfi að nota annað orð en klasa\textbf{fasti}} í línu 7, \texttt{tegund}.

Prófið nú að skipta um gildi á klasabreytunni \texttt{tegund} fyrir ykkar eigið tilvik af \texttt{Bíll}.

Þá skulum við skoða dálítið sérkennilegt fyrirbæri í Python, það er að við getum endurskilgreint klasabreyturnar okkar, sem hefur áhrif á öll tilvikin okkar.
Til að skoða það skulum við nota aftur kóðann úr kóðabút \ref{lst:klasar-skilgreindir-tegund}.

\begin{lstlisting}[caption=Klasar skilgreindir, label=lst:klasar-skilgreindir-tegund2]
class Bíll:
	tegund = "Citroen"
	
fyrsti_billinn = Bíll()
print(fyrsti_billinn.tegund)
Bíll.tegund = "Volvo"
print(fyrsti_billinn.tegund)
\end{lstlisting}
\lstset{style=uttak}
\begin{lstlisting}
Citroen
Volvo
\end{lstlisting}
\lstset{style=venjulegt}

Hér sjáum við hvernig tilvikið okkar, \texttt{fyrsti\_billinn}, af bílaklasanum breytist.
Í línu 5 er \texttt{tegund} "Citroen" en í línu 7 er það orðið að "Volvo".
Þetta gerist því að tilvikið okkar er af þessum klasa og hann breyttist í línu 6.
Við endurskilgreindum klasann og því breytast öll tilvik af honum í samræmi.

Nú eigum við tvær breytur sem við getum unnið með, kannski setja bensín á bílinn eða fylla á rúðuvökva og þá gerum við það við þá tilteknu breytu sem við ætlum að framkvæma þá aðgerð á.
En þessi skilgreining innihélt engar aðferðir, við sjáum það í hluta  \ref{uk:klasar-aðferðir}.

-----------------------------------------------------


Til þess að skilgreina klasa þarf einungis lykilorðið \textbf{class} og réttan inndrátt.
Í kóðabút \ref{lst:klasar-skilgreindir} sjáum við hvernig má búa til eins einfaldan klasa og mögulegt er og svo sjáum við í kóðabút \ref{lst:klasar-skilgreindir2} hvernig á að bæta við aðferð á klasa.

Við munum svo beita klösum á hnitmiðaðri máta með svo kölluðum \textit{töfra aðferð} (e. magic method, double underscore method, dunder method\footnote{þarna er orðunum double og under skeytt saman í dunder}) og skoða hvernig á að útbúa hlut með ákveðnum grunnupplýsingum.

\begin{lstlisting}[caption=Klasar skilgreindir, label=lst:klasar-skilgreindir2]
class Tala():
	x = 5
	def leggja_saman(self, x):
		print(self.x + x)

t = Tala()
t.leggja_saman(6)
\end{lstlisting}
\lstset{style=uttak}
\begin{lstlisting}
11
\end{lstlisting}
\lstset{style=venjulegt}

Tökum eftir hvernig breytan \texttt{t} er skilgreind í kóðabút \ref{lst:klasar-skilgreindir2}, hún er skilgreind eins og hvaða önnur breyta sem við höfum búið til áður.
En það sem kemur hinu megin við jafnaðarmerkið er eins og verið sé að kalla í fall.
Eina sem gefur til kynna að þetta sé ekki fall er að Klasi er með stórum staf.
Ef við gleymum að gera svigana þá fáum við ekki eintak af klasanum til að vinna með heldur fáum við nýja vísun á klasann sjálfan.
Það er við erum með nýtt nafn sem gerir það sama og breytan \texttt{Tala} gerir, annan vísi á \texttt{Tala} en ekki útgáfu til að vinna með.

Það er nafnavenja í Python að klasar séu nefndir með stórum staf, það auðveldar lestur fyrir mannfólk.

Þá sjáum við að í línu 7 er kallað í aðferðina \texttt{leggja\_saman}, hún tekur við einu viðfangi.
En ef við skoðum skilgreininguna á aðferðinni þá eru þar skilgreind tvö viðföng.
Fyrra viðfangið \texttt{self} er þarna notað fyrir klasann til að vita að það sé verið að tala um hann sjálfan, svo þarna inni eru tvö mismunandi x.
Fyrra x-ið er úr línu 2 og seinna x-ið er úr viðfanginu.
Þetta getur verið ruglandi en við munum sjá fleiri dæmi um þetta og vonandi verður þetta skýrara.

\section{Tilviksbreytur}\index{Tilveiksbreytur}\label{uk:klasar-tilviksbreytur}
Nú höfum við seð hvernig hægt er að búa til tilvik af klasa, en klasinn úr kóðabút \ref{lst:klasar-skilgreindir} er sérstaklega ber og gagnlítill.
En hvers eru klasar megnugir?

Athugum eftirfarandi samlíkingu áður en lengra er haldið.
Þegar við förum á veitingastað þá er okkur boðinn ákveðinn matseðill, við fáum að vita að það séu þrír réttir á matseðlinum (þrír klasar) og í þeim réttum eru ákveðin hráefni (tilviksbreytur) og þegar við pöntum okkur mat fáum við í hendurnar eitt tiltekið tilvik af skilgreiningunni á matseðlinum (tilvik af klasa).
Nú eru hráefnin kannski ekki okkur að skapi og við viljum fá að hafa áhrif á hvaða hráefni fara í réttinn okkar (okkar tiltekna tilvik) svo við gefum upp hvað við viljum fá (inntak) sem skilar sér í okkar tiltekna rétti (úttak).

Í þessari samlíkingu er matreiðslufólkið smiðurinn á bakvið klasann, í kóðabút \ref{lst:klasar-notkun} er aðferðin \_\_init\_\_ sá smiður.
Aðferðin smíðar fyrir okkur tilvik af klasanum með því inntaki sem hún fær.

\begin{lstlisting}[caption=Klasar skilgreindir með töfraaðferðinni \_\_init\_\_, label=lst:klasar-notkun]
class Samloka():
	# sjáum að hér er klasinn skilgreindur með svigum
	
	def __init__(self):
		# aðferðin tekur ekki við neinu og gerir ekkert
		pass
	
samlokan_min = Rettur()
# nú eigum við tilvik af klasanum Samloka í breytunni samlokan_min en við getum lítið gert við það, sjáum hvernig væri ef við hefðum einhver hráefni

class Samloka():
	def __init__(self, sosa, alegg):
		self.sosa = sosa
		self.alegg = alegg
		
samlokan_min = Samloka('bbq', ['skinka', 'ostur', 'paprika'])
# nú á ég tiltekna samloku sem hefur bbq sósu og þrjár áleggstegundir

# Athugum nú að matreiðslufólkið gæti boðið upp á einhverja ákveðna samloku
class Samloka_med_skinku():
	def __init__(self, sosa = "", alegg = ['skinka']):
		self.sosa = sosa
		self.alegg = alegg
		
skinku_samloka = Samloka_med_skinku('bbq')
# Nú eigum við tilvik af skinkusamloku sem er með bbq sósu og einu áleggi, skinku.
\end{lstlisting}

Samlíkingin okkar með samlokur á veitingastað er ágæt en nú skulum við skoða hvað er eiginlega í gangi í kóðabút \ref{lst:klasar-notkun}.
Fyrir það fyrsta er klasinn núna skilgreindur sem \textit{Samloka()} með svigum, það var ekki þannig í kóðabút \ref{lst:klasar-skilgreindir}.
Ástæðan er svipuð og í kafla \ref{k:segðir} þar sem mátti sleppa svigum utan um segðir fyrir skilyrðissetningar nema það væri þörf á þeim til útreiknings.
Klasar eiga möguleika á að \textbf{erfa} (e. inherit) frá öðrum klösum, við munum tala um það í undirkafla \ref{uk:klasar-erfðir}, og þeir erfa í grunninn allir frá klasanum \textit{Object}.
Það sem tómur svigi þýðir (eða að sleppa sviganum alfarið) er að klasi erfi ekki frá öðrum klasa.
Því er það upp á einstaklinginn komið að venja sig á að gera alltaf annað hvort, höfundur hefur vanið sig á tóma sviga en er það enginn heilagur sannleikur.

Næsta sem við þurfum að athuga er töfraaðferðin init og orðið \textit{self}.
Orðið self eitt og sér er ekki töfraorð, það má skipta því út fyrir eitthvað annnað, hins vegar hefur komist ákveðin venja á að nota það orð og gerir það kóða læsilegri að halda sig við það.
En hvað gerir orðið self?
Þetta orð er eins konar vísir fyrir klasann til að vita að það sér verið að nota skilgreiningar innan klasans sjálfs sem tilvikið hefur aðgang að.
Í klasanum Samloka eru viðföngin sosa (sem við búumst við að sé strengur án þess að athuga það neitt sérstaklega, sjá kafla \ref{k:villur} um hvernig megi taka á því) og alegg (sem við búumst við að sé listi af strengjum).
Ef notandinn gefur okkur ekkert inntak við gerð samlokunnar þá er ekki hægt að búa til tilvik af samlokunni, því klasinn býst við tveimur stöðubundnum viðföngum inn í aðferðina init og getur ekkert gert án þeirra nema skila villu (eins og klasinn er skilgreindur þarna).
Þegar við skilgreindum samlokan\_min þá sögðum við við klasasmiðinn (init) að við ætluðum að eiga aðgang að inntakinu okkar ('bbq' og ['skinka', 'ostur', 'paprika']).
Þannig að sosa inniheldur núna strenginn bbq fyrir þetta tiltekna tilvik af klasanum og þennan tiltekna lista af áleggstegundum.

Það þriðja og kannski það erfiðasta að skilja er að init aðferðin í klasanum Samloka\_med\_skinku tekur við nefndum viðföngum, eins og við sáum í kafla \ref{uk:föll-sjálfgefin}, sem hafa einhver tiltekin gildi nú þegar skilgreind.
Sem þýðir að við getum búið til einhverja óbreytta, staðlaða, sjálfgefna skinku samloku.
Við þurfum ekki að gefa neitt upp til þess að fá tilvikið í hendurnar, hins vegar ef okkur langar til þess að fá samloku með einhverri sósu og einhverju öðru áleggi þurfum við að gefa það upp og við getum gert það alveg eins og þegar við notum föll með sjálfgefnum/nefndum viðföngum.

\section{Aðferðir}\index{Aðferðir}\label{uk:klasar-aðferðir}
Við þekkjum aðferðir, við höfum séð þær notaðar á týpurnar sem við þekkjum, eins og .capitalize() á strengi, .sort() á lista og .get("x", "y") á orðabækur.
\todo{vera viss um að kynna .get í dict kafla}
Aðferðir eru í raun föll sem eru skilgrein inni í klösum og verka á hlutinn sem klasinn skilgreinir.

Nú ætlum við að skilgreina okkar eigin aðferðir á hlutina okkar.
Við ætlum að skoða aðferðir með tilliti til rafbíla.
Það sem við viljum geta gert þegar við búum til tilvik af rafbíl er að segja hvaða tegund hann hefur, hvaða árgerð hann er af, hversu mikla drægni hann hefur á 100km, hversu margar kílówatt stundir rafhlaðan er og hversu marga kílómetra er búið að aka bílnum.

\begin{lstlisting}[caption=Klasa aðferðir á rafbílaklasa, label=lst:klasar-aðferðir1]

class Rafbill():
	def __init__(self, tegund, model, draegni = 16.7, kws = 40, akstur = 0):
		self.tegund = tegund     
		self.argerd = argerd        
		self.eydsla = draegni/100     # hversu mörgum kw stundum bíllinn eyðir á 1 km
		self.kws = kws               # hversu mikil hleðsla kemst fyrir
		self.akstur = akstur         # km sem hafa verið eknir

	def breyta_tegund(self, ny_tegund):
		# kom í ljós að bíllinn var vitlaust skráður og það þarf að endurskoða gildið tegund
		self.tegund = ny_tegund

	def breyta_model(self, nytt_model):
		# kom í ljós að módelið var vitlaust skráð, og við lögum það
		self.model = nytt_model

	def keyra_km(self, km):
		# við aukum við keyrða kílómetra og við minnkum hleðsluna sem um nemur eyðslu á kílómetra bílsins
		self.akstur += km
		self.kws -= self.eydsla * km  

	def hlada_bilinn(self, kw):
		# Nú viljum við auka við hleðsluna í rafhlöðunni okkkar
		self.kws += kw
		
billinn_minn = Rafbill('Rafio', 2021) # við stillum bílinn í upphafi sem bara staðlaðan rafbíl frá fyrirtækinu Rafio.
billinn_minn.keyra_km(500)
print(billinn_minn.akstur) # skilar úttakinu 500
billinn_minn.hlada_bilinn(900)
print(billinn_minn.kws) # skilar úttakinu 856.5
\end{lstlisting}

Við viljum að það að aka bílnum ákveðna kílómetra hafi áhrif á stöðu rafhlöðunnar.
Við viljum líka geta hlaðið bílinn.
En eins og sést í kóðabút \ref{lst:klasar-aðferðir1} þá er hægt að hlaða bílinn endalaust og það er hægt að keyra hann endalaust líka.
Við settum engin takmörk á það hvað má keyra marga kílómetra, við höldum bara áfram að lækka hleðsluna og við leyfðum okkur svo að hlaða bílinn langt umfram það hversu margar kílówattstundir komast fyrir í rafhlöðunni.
Einnig er galli á þessum klasa að engin leið er til að halda utan um hvert er hámark hleðslu rafhlöðunnar.

En þetta dugar til að sýna fram á hvernig aðferðir eru skilgreindar, hvernig á að kalla í þær, hvernig þær hafa áhrif á tilveiksbreyturnar okkar og svo hvernig má kalla í tilviksbreyturnar til að sjá áhrifin.

Aðferðir þurfa þó ekki endilega að hafa áhrif á tilvikið okkar heldur geta skilað okkur til baka einhverri niðurstöðu, eins og flestar aðferðir á strengi (því við munum að strengir eru óbreytanlegir).

Engin aðferðanna í þessum klasa skilaði nokkurri niðurstöðu.

Tökum nú nýtt dæmi þar sem við skoðum ímyndað lestarkerfi á Íslandi.
Í þessu dæmi höldum við utan um tvennt með klösum, annars vegar lestarstöðvar sem hafa nöfn og eru í ákveðinni fjarlægð frá upphafsstöðinni á leiðinni sinni og hins vegar lestar sem eru á ákveðinni leið og eru staddar á ákveðinni stöð.
Í kóðabút \ref{lst:klasar-aðferðir-lestar} sjáum við hvernig aðferðir geta skilað einhverju án þess að hafa áhrif á tilviksbreytur og við sjáum einnig að smiðurinn init tekur bara við tveimur breytum frá notanda en skilgreinir þrjár tilviksbreytur, þetta er vegna þess að klasinn býður notandanum ekki að hafa áhrif á þessa breytu við smíð klasans.
Notandinn verður því að fá í hendurnar við grunnstillingu lest sem hefur ekki ferðast neitt.

\begin{lstlisting}[caption=Aðferðir kynntar með lestarkerfi, label=lst:klasar-aðferðir-lestar]
class Stod():
	def __init__(self, nafn, fjarlaegd):
		self.nafn = nafn
		self.fjarlaegd = fjarlaegd
		self.farnir_km = 0
	
	
class Lest():
	def __init__(self, leid, byrjunar_stod):
		self.leid = leid
		self.nuverandi_stod = byrjunar_stod
	
	def fara_til_numer(self, numer):
		# Þessi aðferð tekur við sætisnúmeri í leidalistanum self.leid
		#
		# Hún á að skila fjarlægðinni sem þarf að fara frá núverandi stöð að stöðinni í því sætisnúmeri
		return abs(self.leid[numer].fjarlaegd - self.nuverandi_stod.fjarlaegd)
	
	def fara_til_stod(self, stod):
		# Þessi aðferð tekur við hlut af týpunni Stod
		#
		# Hún á að skila fjarlægðinni sem þarf að fara frá núverandi stöð til að komast á stöðina í inntakinu
		return abs(stod.fjarlaegd - self.nuverandi_stod.fjarlaegd)
	
	def fara_til_stodvarnafn(self, stodvarnafn):
		# Þessi aðferð tekur við streng sem er stöðvarnafn
		#
		# Hún á að skila fjarlægðinni frá núverandi stöð að fjarlægðinni að stöðinni með nafnið í inntakinu
		for stod in self.leid:
			if(stod.nafn == stodvarnafn):
				return abs(stod.fjarlaegd - self.nuverandi_stod.fjarlaegd)
	
    # Það sem við viljum gera núna er að geta uppfært núverandi stöð á lestinni okkar
	# Og við viljum þá uppfæra hversu marga km hún hefur ferðast
	def ny_nuverandi_stod(self, stod):
		# aðferðin tekur við hlut af týpunni Stod
		#
		# Það sem aðferðin gerir er að uppfæra tilviksbreytuna nuverandi_stod sem inntaksstodina
		# og setja í tilviksbreytuna farnir_km hversu langt lestin þurfti að ferðast til að komast þangað
		#
		# Aðferðin á að skila km sem voru farnir til að komast þangað
		
		km = self.fara_til_stod(stod)
		self.farnir_km += km
		self.nuverandi_stod = stod
		
		return self.farnir_km
		
# Þetta eru lestarstöðvar
# Stöðvarnar hafa nöfn og fjarlægð frá aðalbrautarstöðinni í Reykjavík
reykjavik = Stod("Reykjavík", 0)
borgarnes = Stod("Borgarnes", 76)
akureyri = Stod("Akureyri", 388)
egilsstadir = Stod('Egilsstaðir', 636)

# leið 1, hún fer frá Reykjavík til Egilsstaða, með tveimur stoppum á milli
leid1 = [reykjavik, borgarnes, akureyri, egilsstadir]

# lest1 er ferðast þessa tilteknu leið og hún byrjar ferð sína í Reykjavík
lest1 = Lest(leid1, reykjavik)

lest1.fara_til_numer(3)
lest1.fara_til_stod(egilsstadir)
lest1.fara_til_stodvarnafn('Egilsstaðir')
# allar þessar aðferðir skila okkur tölunni 636

hofn = Stod('Höfn í Hornafirði', 820)
lest1.ny_stod_a_leid(hofn)
# skilar okkur lista af hlutum af týpunni Stod sem er nú nýja leið lestarinnar okkar lest1
\end{lstlisting}

\section{Töfra aðferðir}\index{Töfra aðferðir}\label{uk:klasar-töfra-aðferðir}
Nú höfum við séð hvernig á að skilgreina okkar eigin aðferðir á klasa.
Og við höfum verið að nota eina töfraaðferð til þess að smíða klasana okkar, init.
En það er til mýgrútur af töfraaðferðum sem við getum nýtt okkur til þess að gera klasana okkar nothæfari.
Í þessum kafla verða nokkrar slíkar teknar fyrir en alls ekki allar.
Við munum að töfraaðferðir (e. magic methods, double underscore methods, dunder methods) eru aðferðir sem eru með tveimur undirstrikum fyrir framan sig og aftan og gegna því hlutverki að útfæra innbyggða virkni.

Helst ber að nefna \_\_str\_\_ aðferðina, sem nemendur vilja oftast geta beitt strax og skilja ekki hvers vegna print skilar einhverju furðulegu.
Hingað til höfum við ekki verið að beita innbyggða fallinu print á klasana okkar í kóðabútum því að hún gerir ekkert skilmerkilegt ennþá.
Til þess að hún geri það þurfum við að útfæra töfraaðferðina str.
Það sem sú aðferð þarf að gera er að skila streng.
Nú er það upp á okkur komið hvað okkur finnst vera nógu merkilegar upplýsingar til þess að setja í strenginn sem á að prenta.
Hingað til þegar við beitum print fallinu höfum við verið að skoða úttak sem er af einhverri týpu sem við þekkjum, heiltölur eða strengir til dæmis.
En nú þegar við erum með okkar eigin klasa/hluti viljum við kannski fá einvherjar tilteknar upplýsingar í ákveðinni röð.

Skoðum kóðabút \ref{lst:klasar-str} þar sem við skilgreinum klasa sem heldur utan um rafbílinn okkar aftur, hins vegar ætlum við að sleppa aðferðunum á bílinn og bæta við nokkrum klasaföstum.
Klasafastar eru skilgreindir efst í klasa og er nafnavenjan með þá að nota eingöngu hástafi.
Það sem klasafastar gera fyrir okkur er að halda utan um breytur sem við viljum að séu aðgengilegar allsstaðar í klasanum, við viljum ekki endilega að þær séu hluti af inntaki frá notanda við smíð klasans og þeir gera yfirferð og prófun klasans auðveldari.
Með auðveldari prófunum er átt við að gildi séu ekki harðkóðuð víðsvegar og erfitt að skipta þeim út (eins og ef nota ætti ákveðna námundun á pí) heldur eru þau skilgreind á einum stað og auðvelt að átta sig á notkun þeirra (ef breytuheitin eru skýr).

\begin{lstlisting}[caption=Töfraaðferðin \_\_str\_\_, label=lst:klasar-str]
class Leikur():
	HAMARKS_LIF  = 100
	LAGMARKS_LIF = 0
	HAMARKS_PENINGUR = 9999
	LAGMARKS_PENINGUR = -9999
	
	def __init__(self, nafn, lif, peningur):
		self.nafn = nafn
		if(lif > self.HAMARKS_LIF or lif < self.LAGMARKS_LIF):
			# líf er utan þess sem er leyfilegt
			self.lif = 100
		else:
			self.lif = lif
		if(peningur > self.HAMARKS_PENINGUR or peningur < self.LAGMARKS_PENINGUR):
			# peningar er utan þess sem er leyfilegt
			self.peningur = 0
		else:
			self.peningur = peningur
	
	def __str__(self):
		return "Persónan heitir {} og á {} gullpeninga og hefur {} í líf".format(self.nafn, self.lif, self.peningur)

valborg = Leikur('Valborg', 200, 90)
print(valborg)
# skilar úttakinu "Persónan heitir Valborg og á 100 gullpeninga og hefur 90 í líf"
\end{lstlisting}

Ef þessarar str töfraaðferðar nyti ekki við þá væri úttakið á þessa leið \textit{<\_\_main\_\_.Leikur object at *minnissvæði*}.
Einnig er nýtt í þessum kóðabút að við vinnum eitthvað með inntakið frá notandanum áður en við stillum tilviksbreyturnar.
Þetta er ekki gert á nógu tryggan máta og við munum sjá í kafla \ref{k:villur} hvernig má meðhöndla inntak frá notanda þannig að vafalaust sé um réttinntak að ræða.
En við ætlum enn sem komið er að skoða hlutina á einfaldan og brothættan máta því við erum að kynnast svo mörgu nýju og óþarfi að gera allt kórrétt frá upphafi, mikilvægara er að byggja upp skilning hægt og rólega.

Það sem töfraaðferðirnar gera er að gera okkur kleyft að beita innbyggðum föllum eins og print og len á tilvik af klösunum okkar, og að beita hinum ýmsu virkjum (reikni-, samanburðar- og rökvikjum) milli tilvika eða annara gilda.




\section{Erfðir}\index{erfðir}\label{uk:klasar-erfðir}
Klasarnir okkar hafa hingað til verið skilgreindir með tómum svigum sem segir vélinni að þeir erfi ekki frá neinum klasa nema \textit{object} sem gerði það að verkum að við gátum útfært töfraaðferðir.

Nú ætlum við að skoða í kóðabút \ref{lst:klasar-erfðir} hvernig á að búa til \textbf{yfirklasa} (e. superclass) og \textbf{undirklasa} (e. subclass).
Við skoðum dæmi þar sem prentari er tekinn fyrir, það sem hann þarf að kunna að gera er að prenta út streng, segja til um blekhlutfallið sitt og minnka blekið um eitt prósentustig.
Þetta er alfarið æfing og því ekki endilega mjög raunhæft dæmi, en þar sem við erum að reyna að átta okkur á því hvað erfðir eru þá ætlum við að gera ráð fyrir því að við viljum að allir prentararnir okkar byrji með 100\% af bleki og hafi möguleikann á að lækka það.
Hins vegar er það ekki útrætt hvernig eigi að fara að því að prenta út og því ætlum við að útfæra sérstaka prentara sem eru eins og grunnprentarinn okkar (með tilliti til bleks) en meðhöndlar prentun á annan máta.

Undirliggjandi ástæður fyrir því að við myndum vilja gera þetta er sú að við viljum að einhver grunn virkni sé til staðar og sé aðgengileg, en það er einhver tiltekin virkni sem við viljum að sé öðruvísi.
Tökum sem dæmi klasa sem vinnur talar við gagnasafn og fær til baka helling af gögnum, vinnur gögnin einhvern veginn fyrir okkur og skilar þeim til okkar sem streng.
En við viljum kannski hafa þann möguleika að í stað þess að fá streng þá sendir klasinn gögnin sem tölvupóst eða býr til skjal á tölvunni sem geymir þau.
Þá myndum við nota erfðir fyrir þá tilteknu notkun.

\begin{lstlisting}[caption=Erfðir kynntar með klasanum Prentari, label=lst:klasar-erfðir]

class Prentari():
	BLEK = 100
	
	def prentun(self, strengur):
		print(strengur)
	
	def minnka_blek(self):
		self.BLEK -= 1
	
	def stada_bleks(self):
		print(self.BLEK)

p1 = Prentari()
p1.prentun('Valborg')
p1.minnka_blek()
print(p1.BLEK)

# úttakið verður
# Valborg
# 99

import random
class HandahofsPrentari(Prentari):
	def prentun(self, strengur):
		handahof = random.randint(1,5)
			for i in range(handahof):
				print(strengur)

p2 = HandahofsPrentari()
p2.prentun('Forritun')
p2.minnka_blek()
p2.stada_bleks()
# úttakið verður (handahófskennt)
# Forritun
# Forritun
# Forritun
# 99


class InntaksPrentari(Prentari):
	def prentun(self):
		strengur = input('hvað viltu prenta? ')
		fjoldi = int(input('hversu oft viltu prenta það? '))
		for i in range(fjoldi):
			print(strengur)

p3 = InntaksPrentari()
p3.prentun()
#> hvað viltu prenta? Tölva
#> hversu oft viltu prenta það? 3
print(p3.BLEK)
# úttakið verður
# Tölva
# Tölva
# Tölva
# 100
\end{lstlisting}

Í kóðabút \ref{lst:klasar-erfðir} er einungis verið að yfirskrifa aðferðina prentun því að það er aðferðin sem við vildum að væri með einhverjum sértækum hætti.
Við vildum ekki bara prenta út einu sinni heldur fá notandann til að segja okkur hversu oft og hvað á að prenta, eða geta gert það handahófskennt oft.

\subsubsection{Fjölmótun}
Tengt erfðum er þess virði að nefna fjölmótun (e. polymorphism) í Python.
Því það er fráburgðið t.d. C++ og Java.

Fjölmótun í Python virkar þannig að klasar þurfa ekki að erfa frá öðrum klösum til að haga sér eins og þeir.
Þetta er vegna þess að þegar vélin athugar hvort að einhver hlutur eigi einhver tiltekin eigindi skoðar hún klasann og þá klasa sem hann erfir frá (í röð) og skilar þeirri útgáfu af eigindinu sem finnst.

Til dæmis HandahofsPrentari og eigindið stada\_bleks(), þá er fyrst athugað innan klasans HandahofsPrentari og svo Prentari hvernig eigi að nota stada\_bleks.
Hins vegar ef við værum að vinna með eitthvað sem við vildum að hegðaði sér eins og prentari án þess að spá í öllu sem prentaraklasinn er hugsaður fyrir gætum við búið til hlut sem útfærir bara aðferðina stada\_bleks og erfir ekki frá neinum.
Hlutinn myndum við kannski kalla Blekathugun, og það sem aðferðin stada\_bleks gerir í þeim klasa er að skrifa stöðu bleksins, á einhverju tæki sem vill notfæra sér þessa aðferð, í tölvupóst.

Ef við tökum praktískara dæmi þá er hægt að sjá fyrir sér klasa sem sér um að vinna með gögn og til þess að geta sent gögnin frá þessum klasa á ákveðinn máta má láta hann fá hlut í hendurnar sem útfærir \textit{write} aðferð.
Klasinn sem útfærir write aðferðina þarf ekkert að gera annað en að útfæra þessa einu aðferð á einhvern ákveðinn máta og þá er hægt að fullvissa sig um að gögnin hafi verið skrifuð á þann máta.

Svo ef við viljum eiga nokkra mismunandi klasa sem allir kunna mismunandi write aðferðir þá þurfum við bara að ganga úr skugga um að gangavinnsluklasinn okkar fékk write fallið sem við vildum nota úr viðeigandi klasa.

Þetta er kallað \textbf{duck typing} og bjóða ekki öll forritunarmál upp á það.
Hugtakið kemur úr frasanum ,,if it looks lika a duck, quacks like a duck and walks like a duck, it's a duck''.
Hugmyndin er að klasinn sem útfærir einungis aðferðina write fyrir okkur er alveg jafn mikil önd eins og kóðasafnið \textit{os} sem sér um að vinna með skrársafnið og skrifa í skjöl.

Ef við höldum áfram með dæmið um klasana sem útfæra write, þá gæti einn þeirra skrifað í skjal á tölvu úti í þýskalandi, einn sendir skjalið í tölvupósti og einn lætur talgervil lesa það upp í strætó leið 14.
Upphaflegi gagnaklasinn veit ekkert um það heldur treystir bara á að fá einhvern hlut í hendurnar sem kann þessa aðferð sama hvernig hún er útfærð.

\chapterimage{chapter_head_2.pdf} % Chapter heading image

\chapter{Villur og villumeðhöndlun}\index{Villur og villumeðhöndlun}\label{k:villur}
Hingað til þegar við fáum villur í kóðann okkar hefur hann hreinlega hætt keyrslu og við þurft að laga eitthvað.
Í kafla \ref{uk:tolur-villur} sáum við upptalningu á þeim helstu villum sem við getum lent í.
Það sem við viljum hinsvegar geta gert er að bregðast við villum til þess að forritin okkar haldi áfram keyrslu þrátt fyrir að eitthvað hafi farið úrskeiðis.
Við viljum geta sagt vélinni að reyna að gera eitthvað og ef henni tekst það ekki því að það myndi valda villu þá viljum við geta gert eitthvað annað og haldið áfram.

\section{Algengar villur}\index{Algengar villur}\label{uk:villur-algengar}
Byrjum á að rifja upp algengar villur og bætum nokkrum við:

\begin{itemize}
	\item Nafna villa - NameError, nafn á breytu var notað sem er ekki skilgreint
	\item Inndráttar villa - IndentationError, röngum inndrætti beitt
	\item Málskipunar villa - SyntaxError, rangt tákn notað eða tákn notað vitlaust
	\item Týpu villa - TypeError, týpan styður ekki aðgerðina sem er verið að framkvæma
	\item Vísis villa - IndexError, verið er að nota sætisvísi sem er ekki til í hlutnum
	\item Gildis villa - ValueError, verið er að nota gildi sem er ekki til
	\item Eiginda villa - AttributeError, verið er að nota eigindi sem hluturinn á ekki til
	\item Lykla villa - KeyError, verið er að ná í lykil sem er ekki til
	\item Endurkvæmnis villa - RecursionError, þegar búið er að ná hámarks leyfilegri endurkvæmni án niðurstöðu
	\item Staðvær nafna villa - UnboundLocalError, þegar verið er að vísa í staðvært breytuheiti en það hefur ekki verið skilgreint á þeim stað í gildissviðinu
	\item Inntaks/úttaks villa - IOError, þegar villa kom upp við meðhöndlun inntaks eða úttaks.
\end{itemize}

Ástæðan fyrir því að nefna nákvæmlega þessar villur en ekki allar sem eru skráðar í skjölun (e. documention) Python forritunarmálsins er vegna þess að þessar villur eru líklegri en aðrar til að koma upp hjá byrjendum og við viljum geta tekið á þeim.
Inndráttarvillur og málskipunarvillur er þó ekki hægt að grípa í keyrslutíma því að þær eru gripnar áður en keyrsla á sér stað og kóðinn hreinlega keyrir ekki neitt.
Það er ágætt að hafa í huga að kóðinn okkar þarf að vera réttur og rétt upp settur til þess að geta keyrt yfirhöfuð.

Hinar villurnar viljum við kannski geta gripið og meðhöndlað svo að við getum haldið áfram með það sem við vorum að gera, við viljum ekki að notandinn sé allt í einu læstur úti eða forritið hætti alfarið keyrslu ef eitthvað minniháttar kemur upp eins og að inntakið frá notanda var ekki af réttri týpu eða ekki var hægt að kasta því í rétta týpu.

\section{Að grípa villur}\index{Að grípa villur}\label{uk:villur-grípa}
Til þess að grípa villur og meðhöndla þær þurfum við nokkur ný lykilorð.
Þau eru \textbf{try}, \textbf{except}, \textbf{else}, \textbf{finally} eða \textit{reyna}, \textit{nema}, \textit{annars}, \textit{að lokum}.
Við höfum séð else áður og það virkar nokkuð svipað í þessari stöðu.
Það sem try gerir er það sem við viljum reyna á, það sem við höldum að muni valda villu.
Við viljum geta reynt að keyra kóðann, til dæmis kalla á einhverja vefþjónustu eða kasta inntaki frá notanda, án þess að forritið hætti.
Hins vegar ef að kóðinn sem við reyndum að keyra veldur villu þá getum við gripið hana með except klausu, þannig að við ætlum að reyna að keyra kóða nema ef það virkar ekki þá viljum gera eitthvað annað.
Annars (else) ef það virkaði að keyra kóðann þá getum við gert eitthvað vitandi að það mun ekki valda villu.
Svo að lokum getum við gert eitthvað burt séð frá því hvort það olli villu eða ekki, finally klausan mun alltaf keyrast.

Flæðiritið fyrir þessa hugmynd er nokkuð svipað skilyrðissetningum með if elif og else.
Það kemur ein try setning, á eftir henni koma eins margar except setningar og við viljum (þar sem ver og ein er þá að taka á einhverri tiltekinni villu), þá má koma ein else setning og að lokum má koma ein finally setning.
Hún keyrist sama hvað og er notuð til þess að framkvæma þá virkni sem verður að eiga sér stað, eins og til dæmis að loka skjali sem verið er að vinna í.

Ástæða þess að það er gagnlegt að vita hvað villurnar heita er að except klausurnar okkar geta gripið ákveðnar villur og ef sú tiltekna villa kom upp getum við tekið á nákvæmlega því tilfelli.
Sjáum í kóðabút \ref{lst:villur-grip} hvernig á að beita þessum nýju lykilorðum og hvernig uppsetningin á þeim þarf að vera.

\begin{lstlisting}[caption=Hvernig á að beita try - except - else, label=lst:villur-grip-kynnt]
tala = input('veldu tölu ')
try:
	tala = int(tala)
except:
	print('þú gafst ekki upp neitt sem mátti túlka sem tölu')
	tala = 0 # notum þá bara eitthvað annað gildi
	
print('talan sem þú ert með er', tala)
# hér er tekið á því tilfelli að ekki gekk að kasta inntakinu og breytan er samt skilgreind.

# hvaða villu erum við samt að grípa?
tala = input('veldu tölu')
try:
	int(tala)
except TypeError:
	print('ekki gekk að kasta í tölu útaf týpuvillu')
except ValueError:
	print('ekki gekk að kasta í tölu útaf gildisvillu')
except AttributeError:
	print('ekki gekk að kasta í tölu útaf eigindavillu')
except:
	print('ekki gekk að kasta útaf einhverri annarri villu sem ekki er reynt að grípa sérstaklega)
else:
	print('það gekk bara víst að kasta í tölu')
	
> veldu tölu strengur # inntakið verður strengur
# úttakið verður
# ekki gekk að kasta útaf gildisvillu

# Hvernig virkar finally?
try:
	int('strengur')
except TypeError:
	print('hér er tekið á villu sem á sér ekki stað')
except:
	print('hér er tekið á öllum öðrum villum, ef þessari klausu er sleppt munum við ekki grípa neina villu því þetta er vissulega ekki týpuvilla')
else:
	print('það er ljóst að við förum ekki hingað inn því kóðinn veldur villu')
finally:
	print('við förum alltaf hér inn sama hvað, hvort sem try virkaði eða ekki, jú nema við gleymdum að grípa villuna og forritið hætti keyrslu')
	
# úttakið verður:
# hér er tekið á öllum öðrum villum, ef þessari klausu er sleppt munum við ekki grípa neina villu því þetta er vissulega ekki týpuvilla
# við förum alltaf hér inn sama hvað, hvort sem try virkaði eða ekki, jú nema við gleymdum að grípa villuna og forritið hætti keyrslu
\end{lstlisting}

Þegar við erum að reyna að grípa svona margar villur eins og í kóðabút \ref{lst:villur-grip-kynnt} er það vegna þess að við erum ekki viss hvað það er sem mun fara úrskeiðis, try klausan okkar er tiltölulega einföld og því fátt sem kemur til greina til að fara úrskeiðis, en við gætum verið að reyna á margt í einu og því gagnlegt að vita af því að við getum verið að grípa margar villur á einu bretti.
Þó er líklega best að hafa try klausurnar hnitmiðaðar en ekki öll framkvæmdin í forritunu okkar, svona ef ské kynni að eitthvað gæti farið úrskeiðis einhvers staðar.

Annar möguleiki sem við viljum geta reynt á er að hreiðra klausurnar okkar þannig að ef við reyndum eitthvað sem gekk ekki viljum við grípa það en reyna eitthvað annað.
Þar kemur einnig sterkt inn að vita hvað villurnar okkar heita svo að við getum reynt eitthvað ákveðið byggt á því hvaða villu við fengum.
Í kóðabút \ref{lst:villur-grip-hreiðrað} sjáum við hvernig hægt er að halda áfram við að reyna að kasta inntaki þegar það gekk ekki við fyrstu tilraun.

\begin{lstlisting}[caption=Hvernig á má hreiðra try - except - else, label=lst:villur-grip-hreiðrað]
# Við viljum kannski reyna eitthvað byggt á því að hafa reynt eitthvað annað sem olli villu.

try:
	tala = input('skrifaðu tölustaf ')
	tala = int(tala)
except:
	# grípum allar villur sem gætu komið upp með einni klausu en reynum þá eitthvað annað
	# kannski skrifaði notandinn tölustaf en gerði punkt eða setti eitthvað tákn fyrir aftan?
	try:
		tala = int(tala[0]) # við vitum að tala er strengur því input skilar alltaf streng og við sækjum fremstu táknið
	except:
		print('þú skrifaðir ekki tölu sem hægt var að skilja')
		tala = 0 # hér tryggjum við að geta notað breytuna án þess að valda nafnavillu 

print('talan var', tala)

> skrifaðu tölustaf 9.
# úttakið verður 
# talan var 9
\end{lstlisting}

\subsection{Að meðhöndla eigin villur}\index{Að meðhöndla eigin villur}\label{uk:villur-raise}
ræða við mér vitrara fólk 


\chapterimage{chapter_head_2.pdf} % Chapter heading image

\chapter{Reiknirit}\index{Reiknitir}\label{k:reiknirit}
Reiknirit (e. algorythm) er forritsbútur sem sinnir sérhæfðum útreikningi.
Dularfyllra er það ekki.
Reiknirit sinna því ákveðnum tilgangi og eru þau oft í grunninn stærðfræðlegs eðlis.

Dæmi um reiknirit sem við höfum séð áður í þessari bók væri útfærsla á fjarlægð milli lesta og að setja nýja lestarstöð inn á leið lestar í kóðabút \ref{lst:klasar-aðferðir-lestar}.

Ástæðan fyrir því að nauðsyn þykir að kynna reiknirit í bók sem þessari er að ef nemendur hafa áhuga á að kynna sér tölvunarfræði í framhaldssnámi er gott að hafa fengið nasasjón af því hvað felst í að beita reikniritum og útfæra þau.
Margir nemendur hefja nám í tölvunarfræði með ýmsar forhugmyndir sem eiga sér sumar ekki stoð í raunveruleikanum.
Þessi kafli og sá næsti fjalla um þau atriði sem leikmenn átta sig ekki endilega á að séu stór hluti af tölvunarfræðum og hugbúnaðarþróun.
Stærðfræði og samvinna.
Þessi kafli er um stærðfræðilegu hliðina og næsti um samvinnuna.

\comment{
\section{Reiknirit sem við höfum séð}\index{Reiknirit sem við höfum séð}\label{uk:reiknirit-okkar}
omg omg omg
\begin{lstlisting}[caption=Við höfum séð eftirfarandi reiknirit, label=lst:reiknirit-okkar]
# kóði
\end{lstlisting}
}

\section{Þekkt reiknirit}\index{Þekkt reiknirit}\label{uk:reiknirit-þekkt}
Það sem við ætlum að skoða í þessum kafla eru tvö ákveðin reiknirit, bæði mjög þekkt og svo hugmyndin um endurkvæmni.


\subsection{Endurkvæmni}\index{Endurkvæmni}\label{uk:reiknirit-endurkvæmni}
Endurkvæmni (e. recursion) er sú virkni forrits að vísa í sig sjálft.
Þið þekkið eflaust listaverk sem virka eins og skynvillur, þar sem manneskja getur labbað í hring upp stiga en endað á sama stað því stiginn fer í raun í hring.
Eða þið hafið séð ykkur sjálf í spegli þar sem var annar spegill fyrir aftan og þið sáuð ótal spegilmyndir raðast af ykkur.

Í forritun heitir það endurkvæmni þegar við látum forrit vísa í sig sjálft, nota sinn eigin kóða.
Gott dæmi um hvernig megi beita endurkvæmni til að fá skilmerkilega niðurstöðu er að útfæra fall sem reiknar fyrir okkur einhverja gildi í fibonacci röðinni.
En áður en við gerum það skulum við skoða enn einfaldara dæmi þar sem við erum með fall sem kallar í sig sjálft og gerir ekkert annað en það.
Í kóðabút \ref{lst:reiknirit-endurkvæmni1} sjáum við einfalda útgáfu af endurkvæmni, þar sem hugmyndin er í raun kynnt án þess að fallið sé neitt gagnlegt.
Það eina sem fallið gerir er að athuga hvort að talan sé stærri en núll og ef hún er það þá kallar fallið í sig sjálft með einu lægra gildi, annars ef talan er ekki stærri en núll prentast ,,þú kannt á endurkvæmni''.
Hveru oft ætli það prentist ef við setjum inn töluna 5?

\begin{lstlisting}[caption=Endurkvæmni - einfalt, label=lst:reiknirit-endurkvæmni1]
def endurkvæmt_fall(tala):
	if(tala > 0):
		# á meðan talan er hærri en 0 þá köllum við aftur í fallið
		# en við köllum í það af gildi einu lægra
		return endurkvæmt_fall(tala-1)
	else:
		# hér erum við komin niður í 0 og prentum eftirfarandi texta
		print('við kunnum endurkvæmni')
		
endurkvæmt_fall(5)
# úttakið verður
# 'við kunnum endurkvæmni'
\end{lstlisting}

Við sjáum í kóðabút \ref{lst:reiknirit-endurkvæmni1} að þó að við kölluðum í fallið með tölunni 5 þá fengum við bara einu sinni út strenginn ,,við kunnum endurkvæmni''.
Það er vegna þess að við kölluðum í fallið fyrir fimm og það sem fallið gerir fyrir okkur er að klára endurkvæmnina fyrir það kall, það verða ekki til fjögur önnur köll.
Heldur verður fimm að fjórum sem skilar okkur svo niðurstöðunni fyrir þrjá og svo koll af kolli þar til við erum komin niður í núll og þá hættir reikniritið keyrslu.
Í þessu tilfelli skilar það engu til baka upp fallakallið en klárast engu að síður þarna í línu 8 þegar það kemst þangað.

Það sem þetta reiknirit okkar gerir ekki er að skila einhverri niðurstöðu til baka, en nú skulum við skoða tvö þannig reiknirit sem eru endurkvæm í kóðabút \ref{lst:reiknirit-endurkvæmni2}.
Annað þeirra reiknar summu af einhverri tölu og öllum jákvæðum tölum lægri en henni, svo talan 5 gæfi útkomuna 5 + 4 + 3 + 2 + 1 sem er 15 og fyrir 100 væri það 100 + 99 + ... + 1 sem gæfi 5050.
Hitt reiknar n-tu töluna í fibonacci rununni.
Endurkvæmnin er því fólgin í því að upphaflega fallakallið með summa(100) skilar okkur 100 + summa(99), sem skilar okkur 100 + 99 + summa(98) og svo koll af kolli þar til talan 1 fæst, og við fáum útreikninginn 100 + 99 + ... + 1 sem skilar okkur 5050.

Endurkvæmnin í fib(4) er fólgin í því að skoða alltaf tvær summur í einu en þær skila sér í sama fallakallið.
Svo fib(4) skila því sem fib(2) + fib(3) skila, þau köll skila okkur annars vegar fib(0) + fib(1) og hins vegar fib(1) + fib(2).
Fyrri hlutinn sem fékkst úr fib(2), sem varð að fib(0) + fib(1) skilar sér sem 0 + 1 = 1.
Og seinni hlutinn er fib(1) + fib(2) og fib(1) skilar 1 og við vitum að fib(2) mun skila 1 svo þar stendur 1 + 1 = 2 svo við fáum upp til baka að fib(4) = 1 + 2 = 3.

\begin{lstlisting}[caption=Endurkvæmni - þar sem við skilum gildum upp keðjuna, label=lst:reiknirit-endurkvæmni2]
def summa(n):
	if n <= 1:
		return 1
	else:
		return n + f(n-1)

summa(100)
# úttakið verður 5050
	
def fib(n):
	if n <= 1:
		return n
	else:
		return fib(n-2) + fib(n-1)

fib(4)
# fjórða talan í fionacci röðinni er 3
\end{lstlisting}

\subsection{Helmingunarleit}\index{Helmingunarleit}\label{uk:reiknirit-helmingunarleit}
Byrjum á að skoða eitthvert þekktasta reiknirit sem til er. 
Helmingunarleit að tölu á bili. 
Hugsum okkur að við séum með raðaðan lista af tölum og við viljum finna eina tiltekna tölu. 
Ef við ættum að skoða hverja einustu tölu í listanum til að finna hana þá tæki það mjög langan tíma.
Eða allavega fyrir okkur sem manneskjur, en allur tímasparnaður er góður.
Því að aðgerðin ,,að skoða spil'' kostar einhvern tíma og því færri þannig aðgerðir sem við getum gert því hraðara er reikniritið okkar.

Tölvunarfræðingar eru mjög uppteknir af því hvað aðgerðir og reikningar taka mikinn tíma.
Þetta er kallað tímaflækja (e. time complexity) og er tölvunarfræðingum mjög hugleikin.
Tímaflækja helmingunarleitar er sérstaklega lág eða $log_2{n}$ vegna þess að reikniritið helmingar alltaf vandamálið (e. problem space).

Skoðum nú tvær útgáfur af reikniritinu í kóðabút \ref{lst:reiknirit-helm-for} er while-lykkju beitt til þess að finna gildið x í listanum listi, í kóðabút \ref{lst:reiknirit-helm-end} er gefið upp listinn sem á að leita að x í og þau sætisnúmer í listanum sem á að leita á milli.

Hugmyndin er sú sama í báðum útfærslum, að verið sé að skoða í fyrstu allan listann sem er raðaður (mjög mikilvægt, annars virkar þetta engan veginn) og miðju gildið er skoðað, ef gildið sem við leitum að er stærra en miðjugildið þá eigum við að vera að leita þeim megin við miðjugildið sem stærri gildi eru.
Svo gerum við þetta endurtekið, helmingum alltaf vandamálið þar til við höfum annað hvort fundið gildið sem við leitum að eða bilið sem við erum að leita á inniheldur engin stök.


\begin{lstlisting}[caption=Helmingunarleit að tölu í röðuðum lista með lykkju, label=lst:reiknirit-helm-for]
def helmingunarleit_med_lykkju(listi, x):
	# Þetta fall tekur við lista sem er raðaður í vaxandi röð
	# og gildi sem á að leita að í listanum
	# Þetta er gert með lykkju og án endurkvæmni
	
	# Fallið skilar sætisvísinum sem gildið er í eða -1 ef gildið er ekki í listanum
	
	# minnsti og hæsti vísirinn í listanum
	minnsti = 0
	haesti = len(listi)-1

	while minnsti <= haesti:
		# Þetta keyrist á meðan minnsti er enn minni eða jafn og haesti, það er við erum enn með bil til að leita á

		midjan = int((minnsti + haesti)/2)
		if listi[midjan] == x:
			# við fundum gildið og skilum vísinum sem það er í
			return midjan
		if listi[midjan] > x:
			haesti = midjan - 1
		else:
			minnsti = midjan + 1
	
	# lykkjan hætti keyrslu svo minnsti vísirinn er orðinn hærri en sá hæsti og þá vitum við að talan er ekki í listanum 
	# við skilum því tölunni -1 til að segja að það var enginn vísir sem x var í
	return -1
\end{lstlisting}

Nú höfum við séð þessa lykkju útfærslu sem keyrir á meðan við höfum bil til að leita á en nú skulum við skoða hvernig megi gera þetta endurkvæmt.
\todo{útlit}

\begin{lstlisting}[caption=Helmingunarleit að tölu í röðuðum lista með endurkvæmni, label=lst:reiknirit-helm-end]
def helmingunarleit_med_endurkvaemni(listi, vinstri, haegri, x): 
	# Endurkvæmt fall sem skilar sætisvísinum sem x finnst í í listanum listi
	# Ef x er ekki að finna í listanum listi þá fæst -1
	
	# Athugum hvort að vinstri sé enn vinstra megin
	if haegri >= vinstri: 
	
		# Stillum miðjuna
		midjan = int((vinstri + haegri)/2)
		#midjan = math.floor(vinstri + (haegri - vinstri)/2)
		
		# Athugum hvort gildið sé í miðjustakinu
		if listi[midjan] == x: 
			return midjan 
		
		# Ef gildið er lægra en miðjan þá þurfum við að leita frá vinstri að miðju
		elif listi[midjan] > x: 
			return helmingunarleit_med_endurkvaemni(listi, vinstri, midjan-1, x) 
	
		# Annars hlýtur gildið að vera frá miðju að hægra gildi
		else: 
			return helmingunarleit_med_endurkvaemni(listi, midjan+1, haegri, x) 
	
	else: 
		# Nú er hægra orðið minna en vinstra svo að
		# við erum búin að leita af okkur allan grun
		# gildið er ekki í listanum
		return -1
\end{lstlisting}

\subsection{Bubble sort}\index{Bubble sort}\label{uk:reiknirit-bubble}
Nú þegar við höfum leyst það hvernig á að leita að tölu á bili ef bilið er raðað, þá er mikilvægt að skoða hvernig er eiginlega hægt að raða?

Við höfum séð að listar, innbyggða gagnatagið í Python, hefur innbyggðu aðferðina sort().
Það sem sort gerði var að breyta lista fyrir okkur þannig að hann væri raðaður í vaxandi röð.

Nú viljum við hinsvegar átta okkur á því hvernig við getum útfært reiknirit sem raðar fyrir okkur stökum í lista.
Í kóðabút \ref{lst:reiknirit-bubble} sjáum við útfærslu á bubble sort.
Útfærslan felst í tveimur hreiðruðum for-lykkjum sem er yfirleitt ekki góðs viti þegar kemur að tímaflækju, enda er hægt að gera ráð fyrir að tíminn sem það tekur að keyra bubble sort sé $n^2$, sem segir kannski ekkert fyrir óþjálfað auga en hægt er að treysta því að það er ekki ákjósanlegt.

Reikniritið virkar í grunninn þannig að það tekur við lista sem á að raða, það rúllar í gegnum listann frá upphafi og út í enda og ýti stærsta stakinu út í enda.
Þegar það er búið að rúlla einu sinni í gegnum listann er stærsta stakið komið út í enda og það stak er ekki skoðað aftur heldur álitið á sínum stað.
Þá er aftur rúllað í gegnum listann og stakið sem er þá stærst fer út í enda vinstra megin við stakið sem var stærst.

Þannig að ytri for lykkjan keyrir fyrir hvert stak í listanum, eða segir til um hversu oft þurfi að finna stærsta stakið, og innri lykkjan sér um samanburðinn og skiptingarnar.
Tökum sérstakleg eftir þar að við getum horft á næsta stak hægra megin þegar við erum að bera saman og það er vegna þess að við hættum fyrir framan aftasta stakið hverju sinni í innri lykkjunni.
Ef við myndum bara beita breytunni vinstri\_hlid sem n-i þá myndum við fá vísisvillu því við værum að vísa út fyrir listann okkar en vegna þess að við lækkum okkur um 1 þar að auki þá er möguleiki að skoða listi[j+1] sem er næsta stak hægra megin við það stak sem við erum stödd í listi[j]
\todo{útlit}

\begin{lstlisting}[caption=Bubble sort reikniritið, label=lst:reiknirit-bubble]
def bubblesort(listi):
	# n er þá fjöldi staka í listanum
	n = len(listi)
	
	# Förum í gegnum öll stökin
	for i in range(n):
		# vinstri hliðin er óröðuð 
		# í fyrstu ítrun er i 0 og vinstri_hlid er því jöfn n-1
		# sem er síðasta stakið í listanum
		# svo verður vinstri_hlid alltaf minni og minni
		# Því síðustu i stökin eru komin á sinn stað
		vinstri_hlid = n-i-1
	
		for j in range(0, vinstri_hlid):
			vinstra = listi[j]
			haegra = listi[j+1]
			# Hér förum við frá 0 upp í n-i-1
			# af því að viljum byrja úti í vinstri enda 
			# og við viljum geta skoðað næsta stak fyrir aftan
			# 
			# Svo skiptum við á stakinu við stakið hægra megin
			# ef stakið er stærra en það sem er hægra megin
			if vinstra > haegra:
				listi[j] = haegra
				listi[j+1] = vinstra
			
	return listi
\end{lstlisting}



\chapterimage{chapters17.png} % Chapter heading image

\chapter{Hugbúnaðarþróun}\index{Hugbúnaðarþróun}\label{k:dev}
Eins og fram kom í síðasta kafla þá snýst endirinn á þessari bók um framhaldið fyrir nema sem vilja leggja land undir fót í tölvunarfræðum.
Þessi kafli er þó meira í anda hugbúnaðarverkfræði en tölvunarfræði.

Það kemur nefnilega mörgum á óvart hvað góð samskipti eru stór hluti af því að þróa hugbúnað, forhugmyndir um fólk sem situr eitt við tölvuna sína og gerir eitthvað alveg af sjálfdáðum eru sterkar.

Hugmynd þarf einhvern veginn að verða að veruleika og það má vel vera að hugmynd að góðum og nothæfum hugbúnaði hafi sprottið upp hjá einum einstaklingi en flestur hugbúnaður sem við notum í dag, eins og forrit á símum, eru yfirleitt ekki hönnuð, forrituð, prófuð og markaðssett af einni manneskju.
Þess vegna er þess virði að taka fyrir stuttan kafla um hvað felst í hugbúnaðarþróun.

Áður en lengra er haldið er þó rétt að taka fram að það er engin ein ,,rétt“ leið til að þróa hugbúnað, fólk verður að prófa sig áfram til þess að finna hvað hentar.

\section{Útgáfustjórnun}\index{Útgáfustjórnun}
Fyrir það fyrsta er útgáfustjórnun (e. version control) nauðsynleg.
Ekki bara mikilvæg, nauðsynleg.
Hellingur af lausnum er til, opinn hugbúnaður sem og lokaður, sem sinnir þessu mikilvæga hlutverki.

Útgáfustjórnun kannast kannski sum við sem hafa notfært sér Word úr Office pakkanum, að geta rúllað til baka í einhverja útgáfu af tilteknu skjali þegar það skemmist eða gögn tapast skyndilega.
Það er í raun allt og sumt.
Að geyma kóðann á bakvið hugbúnaðinn þar sem allir eiga að hafa aðgang hafa viðeigandi aðgang.\footnote{Viðeigandi aðgangur gætu verið t.d. skrifréttindi og lesréttindi, að þau sem þurfa ekki að geta breytt neinu geta bara lesið og þau sem eiga að geta gert breytingar hafa réttindi til að skrifa.}

Burt séð frá aðgangsmálum þá er útgáfustjórnun falin í því að gera litlar breytingar því stór jafnt sem lítil kerfi geta verið brothætt og því mikilvægt að geyma þær breytingar sem við gerum í litlum skrefum svo það sé hægt að snúa við og hætta notkun einhverra breytinga sem komu í ljós að hafa valdið villum.
Einnig gerir útgáfustjórnun kóðarýni (e. code review) auðveldari.
Kóðarýni er mikivægur hluti af hugbúnaðarþróun, þar sem einhver er fenginn til að fara yfir kóða frá öðrum og finna hvað mætti betur fara.

Þessi bók var skrifuð með aðstoð git útgáfustjórnunartólsins og gitbub hýsingaraðilans.
Fleiri góð tól eru til eins og Bitbucket, SVN, Mercurial og önnur sem eru innbökuð í hugbúnaðarþróunartól (e. IDE eða integrated development environment) eins og VisualStudio.
Aðalatriðið í vali á tóli fyrir útgáfustjórnunina er að það henti öllum þeim sem eiga að koma að þróuninni, passi við þau stýrikerfi sem fólk notar, og best er ef fyrri reynsla er góð.

\section{Stefnur og straumar}\index{Stefnur og straumar}
Alls konar hugmyndafræði hefur legið til grundvallar við gerð hugbúnaðar og stórra kerfa.
Til eru nokkuð stórar stefnur innan hugbúnaðarþróunar og er ein vinsælasta hugmyndafræðin kvik þróun eða agile með sinn eiginn undirkafla.

En það þýðir ekki að það séu ekki til fleiri aðferðafræðir og í öllum þeim er hornsteinninn teymisvinna forritara.

Þegar hugbúnaður var fyrst þróaður á 20. öldinni þá voru verkfræðingar og stærðfræðingar í fararbroddi.
Því þarf ekki að koma á óvart að verkfræðileg nálgun varð vinsæl stefna í þróun hugbúnaðar, sú sem mest var beitt heitir fossalíkanið (e. waterfall model).
Fossalíkanið byggir á því að komast að því hverjar þarfir og skorður eru á verkefninu, hanna út frá því vöru og prófa hana svo, afurðin er fullbúin vara.
Þessi hugmyndafræði virkar fyrir hin ýmsu verkefni þar sem hægt er að vita skorður og þarfir á mjög hnitmiðaðan, skýran og óyggjandi máta.
Eftir því sem notendur fengu meira vægi þá þurfti að gera breytingar á því hvernig hugbúnaður var þróaður og fékk önnur hugmyndafræði að ryðja sér rúms, samfelld þróun (e. continuous development) þar sem stöðugt var verið að líta til baka og gera breytingar (þetta mun hljóma afskaplega svipað agile en það eru þó einhverjir megin drættir sem eru ólíkir). 

Eftir því sem fleiri fóru að þróa hugbúnað sem leið á öldina því fleiri hugmyndafræðilegir ágreiningar komu í ljós.
Önnur hugmyndafræði sem naut mikilla vinsælda var prófunarþróun (e. test driven development eða TDD) sem er ennþá við lýði í dag.
Hún snýst um að skorður og þarfir má prófa stöðugt og prófanirnar sýna að hugbúnaðurinn standist þær kröfur sem hann eigi að standast.
Gallinn við þá nálgun er að prófanirnar eru mögulega ekki nógu yfirgripsmiklar og eitthvað lendir á milli og gleymist.

\section{Kvik þróun - agile}\index{Kvik þróun - agile}
Kvik þróun fær sér kafla ekki til að setja þessa hugmyndafræði á einhvern stall heldur því hún er svo ótrúlega vinsæl.
Kvik þróun hefur verið notuð bæði sem haldbær þróunaraðferð en einnig sem tískuorð (e. buzz word).

Agile snýst um samfelldan þróunarferil þar sem litlar breytingar eru teknar inn á afmörkuðu tímabili, og þessi tímabil, kölluð sprettir (e. sprints), eru endurtekin þar til hugbúnaðinum er skilað (nema auðvitað honum sé viðhaldið).
Afurðin er því enn í vinnslu þó henni sé sleppt í hendur notenda.

Grunnurinn byggir á fjórum hornsteinum:
\begin{itemize}
	\item Einstaklingar ofar tólum.
	\item Hugbúnaður sem virkar ofar yfirgripsmikilli skjölun.
	\item Samskipti ofar samningum.
	\item Viðbregðni ofar því að fylgja plani.
\end{itemize}

Ásamt þessum hornsteinum eru tólf grunngildi sem snúa að því hvernig eigi að vinna sem teymi, líta til baka og gera endurbætur.

Þessar grunnhugmyndir eru nokkuð opnar sem hefur leitt til þess að til eru nokkuð mörg afbrigði (e. flavor) af Agile hugmyndafræðinni.
Öll afbrigðin eiga það sameiginlegt að taka hornsteinana frekar heilaga en leggja áherslur hver á sín grunngildi eða túlka þau mismunandi.

Helst af afbrigðum ber að nefna SCRUM, XP og Kanban.
SCRUM (sem er ekki stytting á neinu, en er þó alltaf skrifað í hástöfum) þar sem mikil áhersla er lögð á hin ýmsu hlutverk þróunarferlisins.
XP stendur fyrir ,,extreme programming“ og byggist á því að fólk vinni mjög náið saman, helst tvö eða fleiri á einni tölvu.
Kanban er ákveðið vinnuflæðirit sem má nýta með SCRUM og XP en einnig sem bara frjálsleg útfærlsa við Agile hugmyndafræðina.

Ástæða fyrir því að Agile er stundum notað sem tískuorð er vegna þess að þessi hugmyndafræði er það vinsæl að flest fyrirtæki vilja segjast hafa tileinkað sér hana.
Þau eru það þó ekki öll, það er engin vottun til fyrir slíkt og eru dæmi um fyrirtæki sem segjast vera kvik en það tekur heilt ár fyrir starfsfólk að skrá sig úr mötuneytisáskrift.

\section{Að lokum}\label{Að lokum}

Nú þar sem við höfum náð góðum tökum á grunninum í forritun í Pyhton og skoðað við hverju má búast ef farið er lengra er hægt að líta til annarra mála eða frekari notkun Python fyrir ákveðin verkefni.
Þá er gott að skoða pip, PyPI, anaconda, og virtualenv.

Mikið er um góð námskeið og kennsluefni á netinu, eins og á udemy, khan academy, codewars, codecombat og svo er ýmislegt að finna á youtube.

Ekki hika við að gera mistök, þannig verðið þið betri í því sem þið takið ykkur fyrir hendur. 

	\begin{center}
		\includegraphics[scale=5.9]{doodles31-02.png}
	\end{center}


\part{Lausnir verkefna}

\chapterimage{chapters7.png} % Chapter heading image

\chapter{Lausnir verkefna}\index{Lausnir verkefna}\label{Lausnir}

\shipoutAnswer


%----------------------------------------------------------------------------------------
%	INDEX
%----------------------------------------------------------------------------------------

\cleardoublepage % Make sure the index starts on an odd (right side) page
\phantomsection
\setlength{\columnsep}{0.75cm} % Space between the 2 columns of the index
\addcontentsline{toc}{chapter}{\textcolor{ocre}{Index}} % Add an Index heading to the table of contents
%\printindex % Output the index

%----------------------------------------------------------------------------------------

\end{document}
