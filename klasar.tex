\chapterimage{chapters13.png} % Chapter heading image

\chapter{Klasar og hlutir}\index{Klasar og hlutir}\label{k:klasar}
Forritun snýst um að meðhöndla gögn, hingað til höfum við kynnst nokkrum gagnatögum (t.d. strengir og listar), þær gegna mismunandi hlutverkum og bjóða upp á mismunandi aðgerðir til að vinna með gögnin.
Þessar innbyggðu týpur duga þó ekki alltaf og því er mikilvægt að vita að þegar við forritum getum við smíðað okkar eigin.
Þannig getum við aðlagað týpurnar okkar að þeim gögnum sem forritið okkar meðhöndlar og útfært okkar eigin aðferðir á þær.
Klasar gera forriturum kleift að skilgreina sína eigin hluti í flestum hlutbundnum málum, Python er hlutbundið forritunarmál.
Til þess að læra á hvernig eigi að búa til klasa þarf að átta sig á til hvers þeir eru nytsamlegir.

Gagnlegt er að hugsa sér klasa sem skilgreiningu eða uppskrift alveg eins og föll.
Skilgreiningin ein og sér gerir ekki neitt, það er ekki fyrr en við búum okkur til ákveðna útgáfu sem við getum farið að vinna með hana.
Gott dæmi um það er skilgreiningin á rétti á matseðli á veitingastað, textinn á matseðlinum er eingöngu hvað er í boði en er ekki útgáfa af matnum sjálfum.

Klasar eru hlutir sem hugsaðir eru til þess að búa til eintök af og geyma þannig eitthvert ástand og mögulega hafa áhrif á það.
Hugmyndin er að eiga hlut eða \emph{tilvik}, eina tiltekna útgáfu, sem má framkvæma aðgerðir á og eitthvað ástand hlutarins breytist eftir því hvað var gert, þannig er hægt að búa til mörg eintök af sama klasanum og láta hvert tilvik verða fyrir mismunandi áhrifum\footnote{Athuga þarf sérstaklega gildissvið þegar klasar eru annarsvegar, gildissvið í Python geta verið ögn ruglingsleg en við munum ekki beita klösum á það sérhæfðan máta að við lendum í miklum vandræðum.
Hér er tilvalið að skoða ,,meta programming''}.

\section{Klasar skilgreindir}\index{Klasar skilgreindir}\label{uk:klasar-skilgreindir}

Klasar nota lykilorðið \textbf{class} og eru skilgreindir með því orði, allt sem tilheyrir klasanum er inndregið undir honum.
Nöfn klasa eru með stórum staf í Python og flestum hlutbundnum forritunarmálum.

\begin{lstlisting}[caption=Klasinn Bíll skilgreindur, label=lst:klasar-skilgreindir-tegund]
class Bíll:
	tegund = "Citroen"

fyrsti_billinn = Bíll()
print(fyrsti_billinn.tegund)
\end{lstlisting}
\lstset{style=uttak}
\begin{lstlisting}
Citroen
\end{lstlisting}
\lstset{style=venjulegt}

Hugsum okkur að við búum til skilgreiningu á bíl, hann þarf að vera af einhverri tegund, skoðum línur 1-2 í kóðabút \ref{lst:klasar-skilgreindir-tegund}.
Svo viljum við fá tilvik af skilgreiningunni í hendurnar (lína 4), þá búum við til breytu sem fær gildi eins og við höfum gert hundrað sinnum áður, nema núna er gildið sem breytan fær nafnið á klasanum okkar ásamt svigum eins og við séum að kalla í hann.
Prófið núna að búa til annað tilvik af klasanum \texttt{Bíll} án þess að nota svigana og prófið þá að prenta út það sem \texttt{type} skilar fyrir breyturnar tvær.

\begin{lstlisting}[caption=Klasinn Bíll skilgreindur og tvö tilvik búin til, label=lst:klasar-skilgreindir-subaru]
class Bíll:
	tegund = "Citroen"
	
fyrsti_billinn = Bíll()
print(fyrsti_billinn.tegund)
annar_bill = Bíll()
annar_bill.tegund = "Subaru"
print(annar_bill.tegund)
\end{lstlisting}
\lstset{style=uttak}
\begin{lstlisting}
Citroen
Subaru
\end{lstlisting}
\lstset{style=venjulegt}

Breytan \texttt{fyrsti\_bilinn} kemur ekki í veg fyrir það að við getum átt fleiri bíla, en hún heldur utan um ástandið á nákvæmlega þessum bíl okkar.
Segjum að við fáum okkur svo annan bíl, þá getum við búið til aðra breytu (lína 6) í kóðabút \ref{lst:klasar-skilgreindir-subaru} fyrir annað tilvik af klasanum.
Bílarnir eru, fyrir okkur, óaðgreinanlegir í línu 6\footnote{Þar sem ekki hefur verið útfærð \_\_eq\_\_ aðferðin þá er notast við id() fallið úr type klasanum sem klasinn okkar erfir frá bakvið tjöldin.
Við skoðum erfðir betur seinna í kaflanum.} en það breytist svo snarlega þegar við endurskilgreinum \emph{klasabreytuna}\footnote{Í öðrum hlutbundnum málum er venjulega talað um klasafasta en í Python er auðvelt að breyta þeim svo við hæfi að nota annað orð en klasa\textbf{fasti}} í línu 7, \texttt{tegund}.

Prófið nú að skipta um gildi á klasabreytunni \texttt{tegund} fyrir ykkar eigið tilvik af \texttt{Bíll}.

Þá skulum við skoða dálítið sérkennilegt fyrirbæri í Python, það er að við getum endurskilgreint klasabreyturnar okkar, sem hefur áhrif á öll tilvikin okkar.
Til að skoða það skulum við nota aftur kóðann úr kóðabút \ref{lst:klasar-skilgreindir-tegund}.

\begin{lstlisting}[caption=Endurskilgreining á því sem klasinn býður upp á, label=lst:klasar-skilgreindir-tegund2]
class Bíll:
	tegund = "Citroen"
	
fyrsti_billinn = Bíll()
print(fyrsti_billinn.tegund)
Bíll.tegund = "Volvo"
print(fyrsti_billinn.tegund)
\end{lstlisting}
\lstset{style=uttak}
\begin{lstlisting}
Citroen
Volvo
\end{lstlisting}
\lstset{style=venjulegt}

Hér sjáum við hvernig tilvikið okkar, \texttt{fyrsti\_billinn}, af bílaklasanum breytist.
Í línu 5 er \texttt{tegund} "Citroen" en í línu 7 er það orðið að "Volvo".
Þetta gerist því að tilvikið okkar er af þessum klasa og klasinn breyttist í línu 6.
Við endurskilgreindum klasann og því breytast öll tilvik af honum í samræmi.

Nú eigum við tvær breytur sem við getum unnið með, kannski setja bensín á bílinn eða fylla á rúðuvökva og þá gerum við það við þá tilteknu breytu sem við ætlum að framkvæma þá aðgerð á.
En þessi skilgreining innihélt engar aðferðir, við sjáum það í hluta  \ref{uk:klasar-aðferðir}.

\comment{
	
	--------------

Við munum svo beita klösum á hnitmiðaðri máta með svo kölluðum \textit{töfra aðferð} (e. magic method, double underscore method, dunder method\footnote{þarna er orðunum double og under skeytt saman í dunder}) og skoða hvernig á að útbúa hlut með ákveðnum grunnupplýsingum.

Áður en við höldum svo langt skulum við byrja á að skoða orðið \texttt{self} sem er ekki frátekið lykilorð heldur er það hefð og venja að nota það orð til að segja klasanum að nú sé hann að nota sig sjálfan (sjá endurkvæmni í kafla \ref{k:reiknirit})\footnote{Skipta má út orðinu self fyrir hvað sem er, en það þarf þá að halda samhengi, best er að venja sig á nota self svo kóðinn verði læsilegri}.

-----------

}

\comment{
	
------------------------
	
Við sáum þetta í \ref{lst:klasar-self} þegar búið að er að setja fall inn í klasann, við munum hvernig föll eru skilgreind úr kafla \ref{k:föll}, hvernig þau vinna með viðföng og hvernig á að kalla í þau.

\begin{lstlisting}[caption=Klasar , label=lst:klasar-self]
class Tala():
	x = 5
	def leggja_saman(self, x):
		print(self.x + x)

talan_min = Tala()
talan_min.leggja_saman(6)
\end{lstlisting}
\lstset{style=uttak}
\begin{lstlisting}
11
\end{lstlisting}
\lstset{style=venjulegt}

Við sjáum að fallið tekur við tveimur viðföngum \texttt{self} og \texttt{x}.


Tökum eftir hvernig breytan \texttt{t} er skilgreind í kóðabút \ref{lst:klasar-skilgreindir2}, hún er skilgreind eins og hvaða önnur breyta sem við höfum búið til áður.
En það sem kemur hinu megin við jafnaðarmerkið er eins og verið sé að kalla í fall.
Eina sem gefur til kynna að þetta sé ekki fall er að Klasi er með stórum staf.
Ef við gleymum að gera svigana þá fáum við ekki eintak af klasanum til að vinna með heldur fáum við nýja vísun á klasann sjálfan.
Það er við erum með nýtt nafn sem gerir það sama og breytan \texttt{Tala} gerir, annan vísi á \texttt{Tala} en ekki útgáfu til að vinna með.

Það er nafnavenja í Python að klasar séu nefndir með stórum staf, það auðveldar lestur fyrir mannfólk.

Þá sjáum við að í línu 7 er kallað í aðferðina \texttt{leggja\_saman}, hún tekur við einu viðfangi.
En ef við skoðum skilgreininguna á aðferðinni þá eru þar skilgreind tvö viðföng.
Fyrra viðfangið \texttt{self} er þarna notað fyrir klasann til að vita að það sé verið að tala um hann sjálfan, svo þarna inni eru tvö mismunandi x.
Fyrra x-ið er úr línu 2 og seinna x-ið er úr viðfanginu.
Þetta getur verið ruglandi en við munum sjá fleiri dæmi um þetta og vonandi verður þetta skýrara.


----------------
}



\section{Tilviksbreytur}\index{Tilveiksbreytur}\label{uk:klasar-tilviksbreytur}
Nú höfum við seð hvernig hægt er að búa til tilvik af klasa, en klasinn úr kóðabút \ref{lst:klasar-skilgreindir} er sérstaklega ber og gagnlítill.
En hvers eru klasar megnugir?

Athugum eftirfarandi samlíkingu áður en lengra er haldið.
Þegar við förum á veitingastað þá er okkur boðinn ákveðinn matseðill, við fáum að vita að það séu þrír réttir á matseðlinum (þrír klasar) og í þeim réttum eru ákveðin hráefni (tilviksbreytur) og þegar við pöntum okkur mat fáum við í hendurnar eitt tiltekið tilvik af skilgreiningunni á matseðlinum (tilvik af klasa).
Nú eru hráefnin kannski ekki okkur að skapi og við viljum fá að hafa áhrif á hvaða hráefni fara í réttinn okkar (okkar tiltekna tilvik) svo við gefum upp hvað við viljum fá (inntak) sem skilar sér í okkar tiltekna rétti (úttak).

Í þessari samlíkingu er matreiðslufólkið smiðurinn á bakvið klasann, í kóðabút \ref{lst:klasar-notkun} er aðferðin \texttt{\_\_init\_\_} sá smiður.
Aðferðin smíðar fyrir okkur tilvik af klasanum með því inntaki sem hún fær.

\begin{lstlisting}[caption=Klasar skilgreindir með töfraaðferðinni \_\_init\_\_, label=lst:klasar-notkun]
class Samloka():
	def __init__(self, sosa, alegg):
		self.sosa = sosa
		self.alegg = alegg
		
samlokan_min = Samloka('bbq', ['skinka', 'ostur', 'paprika'])

class Samloka_med_skinku():
	def __init__(self, sosa = "", alegg = ['skinka']):
		self.sosa = sosa
		self.alegg = alegg
		
skinku_samloka = Samloka_med_skinku('bbq')
\end{lstlisting}

Í fyrri klasanum, \texttt{Samloka}, verðum við að gefa upp inntak fyrir \texttt{sosa} og \texttt{alegg} þegar við búum okkur til hlut því annars fáum við villu en í seinni klasanum, \texttt{Samloka\_med\_skinku} þá eru tilviksbreyturnar með sjálfgefin gildi.
Þetta rímar ágætlega við raunheiminn, þar sem við verðum að tilgreina hvað við meinum með ,,samloka'' en ,,skinkusamloka'' er mun afmarkaðra.

Samlíkingin okkar með samlokur á veitingastað er ágæt en nú skulum við skoða hvað er eiginlega í gangi í kóðabút \ref{lst:klasar-notkun}.
Fyrir það fyrsta er klasinn núna skilgreindur sem \texttt{Samloka()} með svigum, það var ekki þannig í kóðabútum \ref{lst:klasar-skilgreindir-tegund} - \ref{lst:klasar-skilgreindir-tegund2}.

\begin{itarefni}
\textbf{Svigar eða ekki svigar?}\\
Ástæðan fyrir því að svigar eru valkvæmir er svipuð og í kafla \ref{k:segðir} þar sem mátti sleppa svigum utan um segðir fyrir skilyrðissetningar nema það væri þörf á þeim til útreiknings.
Klasar eiga möguleika á að \textbf{erfa} (e. inherit) frá öðrum klösum, við munum tala um það í undirkafla \ref{uk:klasar-erfðir}, og þeir erfa í grunninn allir frá klasanum \textit{Object}.
Það sem tómur svigi þýðir (eða að sleppa sviganum alfarið) er að klasi erfi ekki frá öðrum klasa.
Því er það upp á einstaklinginn komið að venja sig á að gera alltaf annað hvort, höfundur hefur vanið sig á tóma sviga en er það enginn heilagur sannleikur.
\end{itarefni}

Næsta sem við þurfum að athuga er töfraaðferðin\footnote{töfraaðferðir (e. magic methods, double underscore methods, dunder methods (þarna er orðunum double og under skeytt saman í dunder)) er hópur aðferða sem Python býður upp sem staðlað viðmót sem gerir forriturum kleift að nýta grunnvirkni eins og samanburður með samanburðarvirkja.} \texttt{\_\_init\_\_} og orðið \emph{self}.
Orðið self eitt og sér er ekki lykilorð, það má skipta því út fyrir eitthvað annnað, hins vegar hefur komist ákveðin venja á að nota það orð og gerir það kóða læsilegri að halda sig við það.
En hvað gerir orðið self?
Þetta orð er breyta sem inniheldur tilviki af klasanum sjálfum, í okkar tilfelli er það \texttt{samlokan\_min} í línu 6, þetta skýrist kannski þegar við skoðum endurkvæmni í kafla \ref{k:reiknirit}.

Í klasanum \texttt{Samloka} eru viðföngin \texttt{sosa} (sem við búumst við að sé strengur án þess að athuga það neitt sérstaklega, sjá kafla \ref{k:villur} um hvernig megi taka á því) og \texttt{alegg} (sem við búumst við að sé listi af strengjum).
Ef notandinn gefur okkur ekkert inntak við gerð samlokunnar þá er ekki hægt að búa til tilvik af samlokunni, því \texttt{\_\_init\_\_} aðferðin, \emph{smiðurinn}, býst við tveimur stöðubundnum viðföngum og getur ekkert gert án þeirra nema skila villu að svo stöddu.
Þegar við skilgreindum \texttt{samlokan\_min} þá sögðum við við smiðinn að við ætluðum að eiga eitt stykki samloku með bbq sósu, skinku, osti og papriku.
Þannig að \texttt{sosa} inniheldur núna strenginn \texttt{bbq} fyrir þetta tiltekna tilvik af klasanum og \texttt{alegg} þennan tiltekna lista af áleggstegundum.

Það þriðja og kannski það erfiðasta að skilja er að \texttt{init} aðferðin í klasanum Samloka\_med\_skinku tekur við nefndum viðföngum, eins og við sáum í kafla \ref{uk:föll-sjálfgefin}, sem hafa einhver tiltekin gildi nú þegar skilgreind.
Sem þýðir að við getum búið til einhverja óbreytta, staðlaða, sjálfgefna skinku samloku.
Við þurfum ekki að gefa neitt upp til þess að fá tilvikið í hendurnar, hins vegar ef okkur langar til þess að fá samloku með einhverri sósu og einhverju öðru áleggi þurfum við að gefa það upp og við getum gert það alveg eins og þegar við notum föll með sjálfgefnum/nefndum viðföngum.

\section{Aðferðir}\index{Aðferðir}\label{uk:klasar-aðferðir}
Við þekkjum aðferðir, við höfum séð þær notaðar á týpurnar sem við þekkjum, eins og .capitalize() á strengi, .sort() á lista og .get("x", "y") á orðabækur.
Aðferðir eru í raun föll sem eru skilgreind inni í klösum og verka á hlutinn sem klasinn skilgreinir (til upprifjunar sjá kafla \ref{uk:strengjaaðferðir}).

Nú ætlum við að skilgreina okkar eigin aðferðir á hlutina okkar.
Við ætlum að skoða aðferðir með tilliti til rafbíla.
Það sem við viljum geta gert þegar við búum til tilvik af rafbíl er að segja hvaða tegund hann hefur, hvaða árgerð hann er af, hversu mikla drægni hann hefur á 100km, hversu margar kílówatt stundir rafhlaðan er og hversu marga kílómetra er búið að aka bílnum.

\begin{lstlisting}[caption=Klasa aðferðir á rafbílaklasa, label=lst:klasar-aðferðir1]
class Rafbill():
	def __init__(self, tegund, model, draegni = 16.7, kws = 40, akstur = 0):
		self.tegund = tegund     
		self.argerd = argerd        
		self.eydsla = draegni/100     # hversu mörgum kw stundum bíllinn eyðir á 1 km
		self.kws = kws               # hversu mikil hleðsla kemst fyrir
		self.akstur = akstur         # km sem hafa verið eknir

	def keyra_km(self, km):
		self.akstur += km
		self.kws -= self.eydsla * km  

	def hlada_bilinn(self, kw):
		self.kws += kw
\end{lstlisting}

Við viljum að það að aka bílnum ákveðna kílómetra hafi áhrif á stöðu rafhlöðunnar.
Við viljum líka geta hlaðið bílinn.
En eins og sést í kóðabút \ref{lst:klasar-aðferðir1} þá er hægt að hlaða bílinn endalaust og það er hægt að keyra hann endalaust líka.
Við settum engin takmörk á það hvað má keyra marga kílómetra, við höldum bara áfram að lækka hleðsluna og við leyfðum okkur svo að hlaða bílinn langt umfram það hversu margar kílówattstundir komast fyrir í rafhlöðunni.
Einnig er galli á þessum klasa að engin leið er til að halda utan um hvert er hámark hleðslu rafhlöðunnar.

En þetta dugar til að sýna fram á hvernig aðferðir eru skilgreindar, hvernig á að kalla í þær, hvernig þær hafa áhrif á tilveiksbreyturnar okkar og svo hvernig má kalla í tilviksbreyturnar til að sjá áhrifin.
Takið sérstaklega eftir því í línu 14 að \texttt{kw} og \texttt{self.kw} er ekki það sama, eins og kom fram áður þá er self að vísa í tilvik af klasanum og klasi af þessu tagi hefur \texttt{kws} sem tilviksbreytu.
Hér er þá verið að vísa í þær kw-stundir sem tilvikið bjó yfir þegar kallað var í aðferðina en seinna viðfangið, staka \texttt{kws} er fengið frá notandanum sem kallaði í aðferðina og í núna er það bara hugmynd að einhver muni á endanum gera það, svo sjáum við í kóðabút \ref{lst:klasar-aðferðir1-2} hvernig það er gert.

Aðferðir þurfa þó ekki endilega að hafa áhrif á tilvikið okkar heldur geta skilað okkur til baka einhverri niðurstöðu, eins og flestar aðferðir á strengi (því við munum að strengir eru óbreytanlegir).

Engin aðferðanna í þessum klasa skilaði nokkurri niðurstöðu.
En skoðum þó hvernig megi nota aðferðirnar á eitthvað tiltekið tilvik.

\begin{lstlisting}[caption=Tilvik af rafbílaklasanum búið til og notað, label=lst:klasar-aðferðir1-2]
rafbill = Rafbill('Rafio', 2021) 
rafbill.keyra_km(500)
print(rafbill.akstur) 
rafbill.hlada_bilinn(900)
print(rafbill.kws)
rafbill.tegund = "Oiarf"
print('nú er bíllinn af tegundinni', rafbill.tegund)
rafbill.keyra_km(500)
print('nú er bíllinn búinn að keyra', rafbill.akstur)
\end{lstlisting}
\lstset{style=uttak}
\begin{lstlisting}
500
856.5
nú er bíllinn af tegundinni Oiarf
nú er bíllinn búinn að keyra 1000
\end{lstlisting}
\lstset{style=venjulegt}

Hérna er kallað í báðar aðferðirnar sem við skilgreindum í kóðabút \ref{lst:klasar-aðferðir1} með ákveðnu inntaki.
Takið einnig eftir að \texttt{rafbill} og \texttt{Rafbill} er ekki það sama, í Python skiptir máli hvort eru notaðir hástafir eða lágstafir. 

Tökum nú nýtt dæmi þar sem við skoðum ímyndað lestarkerfi á Íslandi.
Í þessu dæmi höldum við utan um tvennt með klösum, annars vegar lestarstöðvar sem hafa nöfn og eru í ákveðinni fjarlægð frá upphafsstöðinni á leiðinni sinni og hins vegar lestar sem eru á ákveðinni leið og eru staddar á ákveðinni stöð.
Í kóðabút \ref{lst:klasar-aðferðir-lestar} sjáum við hvernig aðferðir geta skilað einhverju án þess að hafa áhrif á tilviksbreytur og við sjáum einnig að smiðurinn init tekur bara við tveimur breytum frá notanda en skilgreinir þrjár tilviksbreytur, þetta er vegna þess að klasinn býður notandanum ekki að hafa áhrif á þessa breytu við smíð klasans.
Notandinn verður því að fá í hendurnar við grunnstillingu lest sem hefur ekki ferðast neitt.

\begin{lstlisting}[caption=Aðferðir kynntar með lestarkerfi, label=lst:klasar-aðferðir-lestar]
class Stod():
	def __init__(self, nafn, fjarlaegd):
		self.nafn = nafn
		self.fjarlaegd = fjarlaegd
		
class Lest():
	def __init__(self, leid, byrjunar_stod):
		self.leid = leid
		self.nuverandi_stod = byrjunar_stod
		self.farnir_km = 0
	
	def fara_til_numer(self, numer):
		return abs(self.leid[numer].fjarlaegd - self.nuverandi_stod.fjarlaegd)
	
	def fara_til_stod(self, stod):
		return abs(stod.fjarlaegd - self.nuverandi_stod.fjarlaegd)
	
	def fara_til_stodvarnafn(self, stodvarnafn):
		for stod in self.leid:
			if(stod.nafn == stodvarnafn):
				return abs(stod.fjarlaegd - self.nuverandi_stod.fjarlaegd)
\end{lstlisting}

Skoðið hér að þrjár aðferðir eru til þess að segja til um fjarlægð lestar frá stöð og það er til þess fallið að sýna að oft eru margar leiðir til þess að leysa verkefni, sérstaklega þegar þau verða flóknari.
Fara til aðferðirnar taka við mismunandi inntaki en þær skila allar sömu niðurstöðu, það er hvað er langt á milli lestar og stöðvar.
Síðasta aðferðin er frábrugðin að því leitinu til að hún færir lestina, framkvæmir breytingu á ástandi lestarinnar.

\begin{lstlisting}[caption=Tilvik af lestum og stöðvum búin til og notuð, label=lst:klasar-aðferðir-lestar-2]
reykjavik = Stod("Reykjavík", 0)
borgarnes = Stod("Borgarnes", 76)
akureyri = Stod("Akureyri", 388)
egilsstadir = Stod('Egilsstaðir', 636)

leid1 = [reykjavik, borgarnes, akureyri, egilsstadir]

lest1 = Lest(leid1, reykjavik)

print(lest1.fara_til_numer(3))
print(lest1.fara_til_stod(egilsstadir))
print(lest1.fara_til_stodvarnafn('Egilsstaðir'))

print(lest1)
\end{lstlisting}
\lstset{style=uttak}
\begin{lstlisting}
636
636
636
<__main__.Lest object at 0x7f29f845fb50>
\end{lstlisting}
\lstset{style=venjulegt}

Það er gott að sjá að allar aðferðirnar skiluðu sömu niðurstöðinni þegar spurt var hve langt er frá Reykjavík til Egilsstaða en af hverju fengum við svona ljóta útprentun þegar við prentuðum út lestina okkar?
Sjáum hvernig við leysum það í næsta kafla.

\section{Töfra aðferðir}\index{Töfra aðferðir}\label{uk:klasar-töfra-aðferðir}
Nú höfum við séð hvernig á að skilgreina okkar eigin aðferðir á klasa.
Og við höfum verið að nota eina töfraaðferð til þess að smíða klasana okkar, init.
En það er til mýgrútur af töfraaðferðum sem við getum nýtt okkur til þess að gera klasana okkar nothæfari.
Í þessum kafla verða nokkrar slíkar teknar fyrir (en alls ekki allar).
Við munum að töfraaðferðir eru aðferðir sem eru með tveimur undirstrikum fyrir framan sig og aftan og gegna því hlutverki að útfæra innbyggða virkni.

Helst ber að nefna \texttt{\_\_str\_\_} aðferðina, sem nemendur vilja oftast geta beitt strax og skilja ekki hvers vegna print skilar einhverju furðulegu.
Hingað til höfum við ekki verið að beita innbyggða fallinu print á klasana okkar í kóðabútum því að hún gerir ekkert skilmerkilegt ennþá (eins og í úttaki kóðabúts \ref{lst:klasar-aðferðir-lestar-2}).
Til þess að hún geri það þurfum við að útfæra töfraaðferðina \texttt{\_\_str\_\_}.
Það sem sú aðferð þarf að gera er að skila streng.
Nú er það upp á okkur komið hvað okkur finnst vera nógu merkilegar upplýsingar til þess að setja í strenginn sem á að prenta.
Hingað til þegar við beitum print fallinu höfum við verið að skoða úttak sem er af einhverri týpu sem við þekkjum, heiltölur eða strengir til dæmis.
En nú þegar við erum með okkar eigin klasa/hluti viljum við kannski fá einvherjar tilteknar upplýsingar í ákveðinni röð.

Skoðum kóðabút \ref{lst:klasar-str} þar sem við skilgreinum klasa sem heldur utan um tölvuleikjapersónuna okkar aftur, hins vegar ætlum við að sleppa aðferðunum að sinni og bæta við nokkrum klasabreytum.
Klasabreytur eru skilgreindar efst í klasa og er nafnavenjan með þá að nota eingöngu hástafi.
Það sem klasabreytur gera fyrir okkur er að halda utan um breytur sem við viljum að séu aðgengilegar allsstaðar í klasanum, við viljum ekki endilega að þær séu hluti af inntaki frá notanda við smíð klasans og þeir gera yfirferð og prófun klasans auðveldari.
Með auðveldari prófunum er átt við að gildi séu ekki harðkóðuð víðsvegar og erfitt að skipta þeim út (eins og ef nota ætti ákveðna námundun á pí) heldur eru þau skilgreind á einum stað og auðvelt að átta sig á notkun þeirra (ef breytuheitin eru skýr).

\begin{lstlisting}[caption=Töfraaðferðin \_\_str\_\_, label=lst:klasar-str]
class Leikur():
	HAMARKS_LIF  = 100
	LAGMARKS_LIF = 0
	HAMARKS_PENINGUR = 9999
	LAGMARKS_PENINGUR = -9999
	
	def __init__(self, nafn, peningur, lif):
		self.nafn = nafn
		if(lif > self.HAMARKS_LIF or lif < self.LAGMARKS_LIF):
			self.lif = 100
		else:
			self.lif = lif
		if(peningur > self.HAMARKS_PENINGUR or peningur < self.LAGMARKS_PENINGUR):
			self.peningur = 0
		else:
			self.peningur = peningur
		
	def __str__(self):
		return "Persónan heitir {} og á {} gullpeninga og hefur {} í líf".format(self.nafn, self.peningur, self.lif)

valborg = Leikur('Valborg', 200, 90)
groblav = Leikur('Groblav', 1000000, -44444)
print(valborg)
print(groblav)
\end{lstlisting}
\lstset{style=uttak}
\begin{lstlisting}
Persónan heitir Valborg og á 200 gullpeninga og hefur 90 í líf
Persónan heitir Groblav og á 0 gullpeninga og hefur 100 í líf
\end{lstlisting}
\lstset{style=venjulegt}

Ef þessarar str töfraaðferðar nyti ekki við þá væri úttakið á þessa leið \textit{<\_\_main\_\_.Leikur object at *minnissvæði*}.
Einnig er nýtt í þessum kóðabút að við vinnum eitthvað með inntakið frá notandanum áður en við stillum tilviksbreyturnar.
Þetta er ekki gert á nógu tryggan máta og við munum sjá í kafla \ref{k:villur} hvernig má meðhöndla inntak frá notanda þannig að vafalaust sé um rétt inntak að ræða.
En við ætlum enn sem komið er að skoða hlutina á einfaldan og brothættan máta því við erum að kynnast svo mörgu nýju og óþarfi að gera allt kórrétt frá upphafi, mikilvægara er að byggja upp skilning.

Það sem töfraaðferðirnar gera er að gera okkur kleyft að beita innbyggðum föllum eins og print og len á tilvik af klösunum okkar, og að beita hinum ýmsu virkjum (reikni-, samanburðar- og rökvikjum) milli tilvika eða annara gilda.


\section{Erfðir}\index{erfðir}\label{uk:klasar-erfðir}
Klasarnir okkar hafa hingað til verið skilgreindir með tómum svigum sem segir vélinni að þeir erfi ekki frá neinum klasa nema \textit{object} sem gerði það að verkum að við gátum útfært töfraaðferðir.

Nú ætlum við að skoða í kóðabút \ref{lst:klasar-erfðir} hvernig á að búa til \textbf{yfirklasa} (e. superclass) og \textbf{undirklasa} (e. subclass).
Við skoðum dæmi þar sem prentari er tekinn fyrir, það sem hann þarf að kunna að gera er að prenta út streng, segja til um blekhlutfallið sitt og minnka blekið um eitt prósentustig.
Þetta er alfarið æfing og því ekki endilega mjög raunhæft dæmi, en þar sem við erum að reyna að átta okkur á því hvað erfðir eru þá ætlum við að gera ráð fyrir því að við viljum að allir prentararnir okkar byrji með 100\% af bleki og hafi möguleikann á að lækka það.
Hins vegar er það ekki útrætt hvernig eigi að fara að því að prenta út og því ætlum við að útfæra sérstaka prentara sem eru eins og grunnprentarinn okkar (með tilliti til bleks) en meðhöndlar prentun á annan máta.

Undirliggjandi ástæður fyrir því að við myndum vilja gera þetta er sú að við viljum að einhver grunn virkni sé til staðar og sé aðgengileg, en það er einhver tiltekin virkni sem við viljum að sé öðruvísi.
Tökum dæmi um spilastokk með 52 spilum og við viljum útfæra nýtt spil sem minnir á ólsen ólsen nema hvað við breytum reglunni að áttan breytir ekki hún lit heldur lætur einhvern annan spilara draga tvö spil.
Þá í staðinn fyrir að skrifa upp allar reglurnar í ólsen ólsen þá skrifum við bara ,,alveg eins og ólsen ólsen nema áttan er öðruvísi''.

\begin{lstlisting}[caption=Erfðir kynntar með klasanum Prentari, label=lst:klasar-erfðir]
class Prentari():
	BLEK = 100
	
	def prentun(self, strengur):
		print(strengur)
	
	def minnka_blek(self):
		self.BLEK -= 1
	
	def stada_bleks(self):
		print(self.BLEK)
		
import random
class HandahofsPrentari(Prentari):
	def prentun(self, strengur):
		handahof = random.randint(1,5)
			for i in range(handahof):
				print(strengur)

class InntaksPrentari(Prentari):
	def prentun(self):
		strengur = input('hvað viltu prenta? ')
		fjoldi = int(input('hversu oft viltu prenta það? '))
		for i in range(fjoldi):
			print(strengur)
\end{lstlisting}

Í kóðabút \ref{lst:klasar-erfðir} er einungis verið að yfirskrifa aðferðina prentun því að það er aðferðin sem við vildum að væri með einhverjum sértækum hætti.
Við vildum ekki bara prenta út einu sinni heldur fá notandann til að segja okkur hversu oft og hvað á að prenta, eða geta gert það handahófskennt oft.

\begin{lstlisting}[caption=Prentaraklasarnir notaðir, label=lst:klasar-erfðir-2]
p1 = Prentari()
p1.prentun('Fyrsti klasinn')
p1.minnka_blek()
print(p1.BLEK)
p2 = HandahofsPrentari()
p2.prentun('Handahóf')
p2.minnka_blek()
p2.stada_bleks()
print(p2.BLEK)
p3 = InntaksPrentari()
p3.prentun()
print(p3.BLEK)
\end{lstlisting}
\lstset{style=uttak}
\begin{lstlisting}
Fyrsti klasinn
99
Handahóf
Handahóf
Handahóf
Handahóf
99
99
hvað viltu prenta? Inntak
hversu oft viltu prenta það? 2
Inntak
Inntak
100
\end{lstlisting}
\lstset{style=venjulegt}


\subsubsection{Fjölmótun}
Tengt erfðum er þess virði að nefna \emph{fjölmótun} (e. polymorphism) í Python.
Því það er fráburgðið t.d. C++ og Java.
Fjölmótun er sá eiginleiki að tagið sem inntakið okkar bindur okkur ekki lengur, notandinn á að geta sett inn ólíkar týpur án þess að skemma nokkuð.

Fjölmótun í Python virkar þannig að klasar þurfa ekki að erfa frá öðrum klösum til að haga sér eins og þeir.
Þetta er vegna þess að þegar vélin athugar hvort að einhver hlutur eigi einhver tiltekin eigindi skoðar hún klasann og þá klasa sem hann erfir frá (í röð) og skilar þeirri útgáfu af eigindinu sem finnst.

Til dæmis HandahofsPrentari og eigindið stada\_bleks(), þá er fyrst athugað innan klasans HandahofsPrentari og svo Prentari hvernig eigi að nota stada\_bleks.
Hins vegar ef við værum að vinna með eitthvað sem við vildum að hegðaði sér eins og prentari án þess að spá í öllu sem prentaraklasinn er hugsaður fyrir gætum við búið til hlut sem útfærir bara aðferðina stada\_bleks og erfir ekki frá neinum.
Hlutinn myndum við kannski kalla Blekathugun, og það sem aðferðin stada\_bleks gerir í þeim klasa er að skrifa stöðu bleksins, á einhverju tæki sem vill notfæra sér þessa aðferð, í tölvupóst.

Ef við tökum praktískara dæmi þá er hægt að sjá fyrir sér klasa sem sér um að vinna með gögn og til þess að geta sent gögnin frá þessum klasa á ákveðinn máta má láta hann fá hlut í hendurnar sem útfærir \textit{write} aðferð.
Klasinn sem útfærir write aðferðina þarf ekkert að gera annað en að útfæra þessa einu aðferð á einhvern ákveðinn máta og þá er hægt að fullvissa sig um að gögnin hafi verið skrifuð á þann máta.

Svo ef við viljum eiga nokkra mismunandi klasa sem allir kunna mismunandi write aðferðir þá þurfum við bara að ganga úr skugga um að gangavinnsluklasinn okkar fékk write aðferðin sem við vildum nota úr viðeigandi klasa.

Þetta er kallað \textbf{duck typing} og bjóða ekki öll forritunarmál upp á það.
Hugtakið kemur úr frasanum ,,if it looks lika a duck, quacks like a duck and walks like a duck, it's a duck''.
Hugmyndin er að klasinn sem útfærir einungis aðferðina write fyrir okkur er alveg jafn mikil önd eins og kóðasafnið \textit{os} sem sér um að vinna með skrársafnið og skrifa í skjöl.

Ef við höldum áfram með dæmið um klasana sem útfæra write, þá gæti einn þeirra skrifað í skjal á tölvu úti í þýskalandi, einn sendir skjalið í tölvupósti og einn lætur talgervil lesa það upp í strætó leið 14.
Upphaflegi gagnaklasinn veit ekkert um það heldur treystir bara á að fá einhvern hlut í hendurnar sem kann þessa aðferð sama hvernig hún er útfærð.

%-------------------------------
\newpage
\section{Æfingar}
\begin{exercise}\label{kla1}
Útfærið nýja aðferð í klasann \texttt{Lest} úr kóðabút \ref{lst:klasar-aðferðir-lestar} sem uppfærir núverandi stöðu lestarinnar.
Aðferðin tekur við hlut af taginu \texttt{Stod} og notar hann til að uppfæra hvar lestin er staðsett, og hversu langt hún hefur farið.

Afritið kóðann úr kóðabútnum, og bætið þessari nýju aðferð inn í klasann og prófið hana.
Til þess að prófa hana þurfið þið nýja stöð, athugið kóðabút \ref{lst:klasar-aðferðir-lestar-2} til að sjá hvernig það var gert.
Til dæmis væri hægt að setja Höfn í Hornafirði sem er í 820km fjarlægð frá Reykjavík ef farið er norður.
\end{exercise}
\setboolean{firstanswerofthechapter}{true}
\begin{Answer}[ref={kla1}]
Það sem þarf að athuga hér er að \texttt{Lest} á tilviksbreytuna \texttt{nuverandi\_stod} sem við viljum uppfæra en það má ekki gerast fyrr en við erum búin að reikna hversu langt er þangað.
Til þess að finna hvað er langt á milli getum við notað einhverja af þeim þremur aðferðum sem búið var að útfæra.
Einnig má ekki gleymast að uppfæra kílómetrastöðuna.
EFtirfarandi er kóðinn fyrir útfærsluna á aðferðinni en prófanir eru eftirlátnar lesanda.
	\begin{lstlisting}
def ny_nuverandi_stod(self, stod):
	km = self.fara_til_stod(stod)
	self.farnir_km += km
	self.nuverandi_stod = stod

	return self.farnir_km\end{lstlisting}
\end{Answer}
\setboolean{firstanswerofthechapter}{false}

\begin{exercise}\label{kla2}
Notið klasann \texttt{Leikur} úr kóðabút \ref{lst:klasar-str} til þess að útfærið töfraaðferðirnar \texttt{\_\_eq\_\_} og \texttt{\_\_lt\_\_}, sem eru til þess að geta notað samanburðarvirkjana == og <.
Búið svo til tvö eintök af klasanum og berið þau saman með þessum samanburðarvirkjum.

Skoðið vel kóðabútinn og sjáið hvernig \texttt{\_\_str\_\_} er útfærð, þar er einungis verið að skila streng.
Svo skilagildið úr þessum aðferðum sem við erum að vinna í að útfæra hér eru fengið með þeim samanburðarvirkjum sem þau snúast um.
\end{exercise}
\begin{Answer}[ref={kla2}]
	Hér eru margar leiðir til þess að leysa verkefnið, hvað er það sem gerir tvo leiki jafna?
	Hér er það útfært þannig að tveir leikir eru jafnir ef nöfnin eru þau sömu, sömuleiðis er útfærslan á < er þannig að einn leikur er strangt minni en annar ef þar er minni peningur.
	
	Þetta gæti líka verið gert með rökvirkjum, til þess að tveir leikir séu jafnir þurfa allar tilviksbreytur að vera eins, eða einhvern veginn allt öðruvísi.
	
	Prófanir eru eftirlátnar lesanda.
	\begin{lstlisting}
def __eq__(self, other):
	return self.nafn == other.nafn

def __lt__(self,other):
	return self.peningur < other.peningur\end{lstlisting}
\end{Answer}