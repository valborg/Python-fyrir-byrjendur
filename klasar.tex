\chapterimage{chapter_head_2.pdf} % Chapter heading image

\chapter{Klasar}\index{Klasar}\label{k:klasar}
Klasar eru hlutir sem hugsaðir eru til þess að geyma eitthvað ástand, halda utan um eitthvað og mögulega hafa áhrif á það.
Klasar eru það sem gera forritunarmál að hlutbundnum málum, Python er hlutbundið forritunarmál.
Til þess að læra á hvernig eigi að búa til klasa þarf að átta sig á til hvers þeir eru nytsamlegir.
Hugmyndin er að eiga eintak sem má framkvæma aðgerðir á og eitthvað ástand hlutarins breytist eftir því hvað var gert, þannig er hægt að búa til mörg eintök af sama hlutnum og láta hvert eintak verða fyrir mismunandi áhrifum.
Athuga þarf sérstaklega gildissvið þegar klasar eru annarsvegar, gildissvið í Python geta verið ögn ruglingsleg en við munum ekki beita klösum á það sérhæfðan máta að við lendum í miklum vandræðum.

Klasar nota lykilorðið \textbf{class} og eru skilgreindir með því orði, allt sem tilheyrir klasanum er inndregið undir honum.

Hugsum okkur að við búum til skilgreiningu á bíl, hann þarf að vera af einhverri tegund og hafa árgerð.
Svo viljum við fá eintak af skilgreiningunni í hendurnar og búum okkur til breytu sem heitir fyrsti\_billinn\_minn sem er Subaru 2002 módel.
Það kemur ekki í veg fyrir það að við getum átt fleiri bíla, en nú erum við með einhverja ákveðna breytu sem heldur utan um ástandið á nákvæmlega þessum bíl okkar.
Segjum að við fáum okkur svo annan bíl, þá getum við búið til aðra breytu sem inniheldur aðra tegund og annað módel, t.d. Citroen 2017.
Nú eigum tvær breytur sem við getum unnið með, kannski setja bensín á bílinn eða fylla á rúðuvökva og þá gerum við það við þá tilteknu breytu sem við ætlum að framkvæma þá aðgerð á.

\section{Klasar skilgreindir}\index{Klasar skilgreindir}\label{uk:klasar-skilgreindir}


\begin{lstlisting}[caption=Klasar skilgreindir, label=lst:klasar-skilgreindir]
# lorem ipsum
\end{lstlisting}

\section{Tilviksbreytur}\index{Tilveiksbreytur}\label{uk:klasar-tilviksbreytur}
class lorem ipsum

\section{Aðferðir}\index{Aðferðir}\label{uk:klasar-aðferðir}
class lorem ipsum

\section{Innbyggðar aðferðir}\index{Innbyggðar aðferðir}\label{uk:klasar-innbyggðar-aðferðir}
class lorem ipsum