\chapterimage{chapter_head_2.pdf} % Chapter heading image

\chapter{Klasar}\index{Klasar}\label{k:klasar}
Klasar eru hlutir sem hugsaðir eru til þess að geyma eitthvað ástand, halda utan um eitthvað og mögulega hafa áhrif á það.
Klasar eru það sem gera forritunarmál að hlutbundnum málum, Python er hlutbundið forritunarmál.
Til þess að læra á hvernig eigi að búa til klasa þarf að átta sig á til hvers þeir eru nytsamlegir.
Hugmyndin er að eiga eintak sem má framkvæma aðgerðir á og eitthvað ástand hlutarins breytist eftir því hvað var gert, þannig er hægt að búa til mörg eintök af sama hlutnum og láta hvert eintak verða fyrir mismunandi áhrifum.
Athuga þarf sérstaklega gildissvið þegar klasar eru annarsvegar, gildissvið í Python geta verið ögn ruglingsleg en við munum ekki beita klösum á það sérhæfðan máta að við lendum í miklum vandræðum.

Klasar nota lykilorðið \textbf{class} og eru skilgreindir með því orði, allt sem tilheyrir klasanum er inndregið undir honum.

Hugsum okkur að við búum til skilgreiningu á bíl, hann þarf að vera af einhverri tegund og hafa árgerð.
Svo viljum við fá eintak af skilgreiningunni í hendurnar og búum okkur til breytu sem heitir fyrsti\_billinn\_minn sem er Subaru 2002 módel.
Það kemur ekki í veg fyrir það að við getum átt fleiri bíla, en nú erum við með einhverja ákveðna breytu sem heldur utan um ástandið á nákvæmlega þessum bíl okkar.
Segjum að við fáum okkur svo annan bíl, þá getum við búið til aðra breytu sem inniheldur aðra tegund og annað módel, t.d. Citroen 2017.
Nú eigum tvær breytur sem við getum unnið með, kannski setja bensín á bílinn eða fylla á rúðuvökva og þá gerum við það við þá tilteknu breytu sem við ætlum að framkvæma þá aðgerð á.

\section{Klasar skilgreindir}\index{Klasar skilgreindir}\label{uk:klasar-skilgreindir}
Til þess að skilgreina klasa þarf einungis lykilorðið \textbf{class} og réttan inndrátt.
Í kóðabút \ref{lst:klasar-skilgreindir} sjáum við hvernig má búa til eins einfaldan klasa og mögulegt er og svo sjáum við í kóðabút \ref{lst:klasar-notkun} hvernig við beitum klösum á hnitmiðaðri máta með svo kallaðari \textit{töfraaðferð} (e. magic method, double underscore method, dunder method (samskeyting ))

\begin{lstlisting}[caption=Klasar skilgreindir, label=lst:klasar-skilgreindir]
# hér er klasi sem hefur eina staðværa breytu
# tökum eftir að breytuheitið fyrir klasann er Klasi með stóru K, því að nafnavenjan í Pyhton er að klasar heiti nöfnum með stórum stöfum fremst.
class Klasi:
	x = 5
	# Þetta er heill klasi.
	
# til þess að búa til eintak af klasanum þá búum við til breytu og köllum í klasann með nafni hans og setjum sviga.
eintak = Klasi()
print(eintak.x)
# úttakið verður fimm

eintak.x = 7
# nú er búið að endurskilgreina x fyrir þetta tiltekna eintak

annad_eintak = Klasi()
print(annad_eintak.x)
# úttakið verður fimm 
\end{lstlisting}

Tökum eftir hvernig breytan eintak er skilgreind í kóðabút \ref{lst:klasar-skilgreindir}, hún er skilgreind eins og hvaða önnur breyta sem við höfum búið til áður.
En það sem kemur hinu megin við jafnaðarmerkið er eins og verið sé að kalla í fall.
Eina sem gefur til kynna að þetta sé ekki fall er að Klasi er með stórum staf.
Ef við gleymum að gera svigana þá fáum við ekki eintak af klasanum til að vinna með heldur fáum við nýja vísun á klasann.
Það er við erum með nýtt nafn sem gerir það sama og breytan Klasi gerir.

En við viljum fá eintak af klasanum sem við getum gert eitthvað við, þessi klasi er þó ekkert stórbrotinn svo það er ekki mikið sem við getum gert.
Innan klasans er ein breyta skilgreind, það er x sem er jafngilt 5.
Eintakið okkar sem er geymt í breytunni eintak hefur aðgang að þessu x-i og það sem gerist eftir skilreininguna á eintak er að við sækjum eigindið (e. attribute) x á eintakið af klasanum og fáum að vita að það sé fimm.
Svo leyfum við okkur að breyta því, við endurskilgreinum það fyrir þetta tiltekna eintak af klasanum.
Eins og kom fram í inngangi þessa kafla þá getum við litið svo á að þetta sé eins og einn tiltekinn bíll og við vorum að aðlaga bensínstöðuna á honum.
Sem sést að hefur engin áhrif á breytuna annad\_eintak því að það fær bara grunnstillinguna á klasanum til sín og hefur ekkert að gera með hitt eintakið okkar.

\begin{lstlisting}[caption=Klasar skilgreindir með töfraaðferðinni \_\_init\_\_, label=lst:klasar-notkun]

\end{lstlisting}

\section{Tilviksbreytur}\index{Tilveiksbreytur}\label{uk:klasar-tilviksbreytur}
class lorem ipsum

\section{Aðferðir}\index{Aðferðir}\label{uk:klasar-aðferðir}
class lorem ipsum

\section{Innbyggðar aðferðir}\index{Innbyggðar aðferðir}\label{uk:klasar-innbyggðar-aðferðir}
class lorem ipsum