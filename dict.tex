\chapterimage{chapter_head_2.pdf} % Chapter heading image

\chapter{Orðabækur}\index{Orðabækur}\label{k:orðabækur}
Ný týpa sem vil ætlum að nú að fást við heitir \textbf{orðabók} (e. dictionary)  og lykilorðið hennar er \textbf{dict}.
Orðabók er orð sem hentar fyrir þýðingu á týpunni í Python en hún er einnig þekkt sem hakkatafla (e. hash table / hash map) í öðrum forritunarmálum.
Til þess að búa til orðabók eru notaðir slaufusvigar \{\}.

Orðabækur eru gagnagrindur eins og listar og ndir, það er þær geyma fyrir okkur gögn af einhverjum týpum.
Orðabækur eru þó frábrugnar listum að því leitinu til að þær eru \textit{óraðaðar}, sem þýðir að þær hafa enga sætisvísa sem hægt er að nota\footnote{Í Python >3.6 eru þær ekki alveg óraðar, stök eru sett inn í ákveðinni röð og helst sú ,,röðun'' þegar orðabókin er notuð fyrir það voru stökin aðgenileg með handahófskenndri röð fyrir minnisbestun.}.
Við getum því ekki sótt gögn í orðabækur með því að vita \textit{hvar} þau eru við þurfum að vita \textit{hver} þau eru.

Þetta er vegna þess að orðabækur eru skipulagðar sem lykla og gildis pör, við finnum þau gildi sem við viljum með því að vita hvaða lykill gengur að þeim.
Þetta er ekki ósvipað því að horfa á lyklakippurnar okkar.
Lyklarnir eru allir ólíkir.
Við getum alltaf reitt okkur á það að sama hvar einhver ákveðinn lykill er þá gengur hann alltaf að sama lásnum, svo ef við þekkjum lyklana okkar getum við auðveldlega náð í þann sem við viljum til þess að opna þann lás sem við viljum hverju sinni.

Lyklarnir verða því að vera ólíkir hverjum öðrum, annars gætum við ekki þekkt þá í sundur og tveir eins lyklar gætu ekki gengið að tveimur mismunandi lásum.
Svo lyklar verða að vera aðgreinanlegir.

Orðabókin er mjög öflugt fyrirbæri, og því þess virði að kynna sér vel hvernig þessi týpa virkar.
Einnig skoðum við í þessum kafla hvernig má ítra í gegnum orðabækur og hvers vegna það var ágætt að vera búin að skoða ndir áður en við komum að þessari mikilvægu týpu.
%---- Athuga hvort þetta eigi heima einhversstaðar
% Við skoðum betur hvernig við getum gengið úr skugga um aðgreinanleika og hvað það þýðir.
%---
\section{Orðabækur skilgreindar og notaðar}\index{Orðabækur skilgreindar og notaðar}
Eins og kom fram í inngangi er gögnum í orðabókum skipt niður á lyklana sem ganga að þeim (sjá ítarefni \todo{vísa í ítarefnið} um hvað má vera lykill og hvað aðgreinanleiki þýðir).
Sjáum fyrir okkur lyklakippuna okkar þar sem við erum með stóran ASSA lykil að útidyrahurðinni, lítinn lykil með svörtu plasti að hjólalásnum, kassalaga lykil að útidyrahurðinni hennar ömmu og einn pínulítinn lykil að geymslunni.
Við eigum auðvelt með að halda utan um þetta litla lyklasafn og við vitum að hverju allir lyklarnir ganga.
En ef við værum nú með 10.000 lykla?
Skoðum kóðabút ?? til að sjá hvernig megi búa til orðabók sem heldur utan um lyklakippuna sem var lýst hér að ofan.
Takið eftir að einungis er unnið með strengi innan orðabókarinnar.

\begin{lstlisting}[caption=Orðabók kynnt með lyklakippusamlíkingu, label=lst:dict-kynnt]
tom_ordabok = {}
kippa = {'ASSA': 'útidyrahurðin heima ', 'lítill svartur': 'hjólið', 'kassalaga': 'heima hjá ömmu', 'pínulítill': 'geymslulykillinn'}

print(kippa['lítill svartur'])
\end{lstlisting}
\lstset{style=uttak}
\begin{lstlisting}
hjólið
\end{lstlisting}
\lstset{style=venjulegt}

Takið eftir hvernig stökin eru aðgreind, með kommum alveg eins og áður.
Nema núna eru stökin tvenndir sem hanga saman með tvípunkti.
Við sjáum einnig að til þess að nálgast gögn þá notum við hornklofa eins og áður en við gerum það ekki með sætisvísi heldur gerum við það með lyklinum sem við viljum finna gögnin að.
En ef lykill er heiltala þá náum við vissulega í gögnin á þeim lykli með því að nota heiltölu innan hornklofanna (sjá línu 4 í kóðabút \ref{lst:dict-kynnt2} \todo{viðhalda línuref})
Í kóðabút \ref{lst:dict-kynnt} þá sjáum við hvernig á að búa til tóma bók í fyrstu línu, við gerum ekkert frekar með þessa orðabók í þessum kóðabút en í kóðabút \ref{lst:dict-kynnt2} sjáum við hvernig má setja stök (lykla og gildis pör) inn í orðabók eftir að hún er skilgreind og við sjáum í kóðabút \ref{lst:dict-kynnt3} hvaða aðferðir eru til á þessa týpu.

\begin{lstlisting}[caption=Gögnum bætt við og þau tekin út, label=lst:dict-kynnt2]
ordabok2 = {1: 'gildi á lykli 1 sem er heiltala', 8: 'gildi sem er á lykli 8', 5: 'takið eftir að pörin eru aðgreind með kommu'}

print(ordabok2[5])

ordabok2[1] = 'nýtt gildi' 
ordabok2['nýr lykill'] = 'nýtt gildi'
print(ordabok2)
\end{lstlisting}
\lstset{style=uttak}
\begin{lstlisting}
takið eftir að pörin eru aðgreind með kommu
{1: 'nýtt gildi', 8: 'gildi sem er á lykli 8', 5: 'takið eftir að pörin eru aðgreind með kommu', 'nýr lykill': 'nýtt gildi'}
\end{lstlisting}
\lstset{style=venjulegt}

Í kóðabút \ref{lst:dict-kynnt2} sjáum við hvernig heilartölur geta verið lyklarnir í orðabókinni en við höfum þó aðeins verið að vinna með strengi sem gildi.
Í raun eru skorður á því hvað geta verið lyklar en ekki hvað geta verið gildi, við getum geymt hvað sem er sem gildi og við sjáum í kóðabút \ref{lst:dict-itrad} þegar við notum lista sem gildi.

\begin{itarefni}
\textbf{Aðgreinanleiki}\\
Í Python er til \texttt{hash()} fall sem skilar ,,hakki'' af því sem því er gefið sem viðfang.
Við sáum í kafla \ref{k:tolur} að 1 og 1.0 var hægt að reikna með þó þær væru af mismunandi tagi og í kafla \ref{k:segðir} sáum við að 1 og 1.0 var jafngilt.
Það sem \texttt{hash()} gerir er að skila heiltölugildi fyrir viðfangið og þegar við viljum athuga hvort að eitthvað sé jafngilt með rökvirkjanum == erum við að spyrja hvort að fallið skili mismunandi heilum tölum fyrir það sem er sitthvoru megin við rökvirkjann.
Í tilfellinu 1, 1.0 og True þá eru þau ekki aðgreinanleg og því ekki hægt að nota þau sem þrjá mismunandi lykla í sömu orðabókinni.

Prófið ykkur áfram með \texttt{hash()} fallið og sjáið hvaða tögum gögnin eru sem þið megið hakka. 
\end{itarefni}

Nú höfum við séð grunnvirknina við það að búa til orðabók og ná í gögn á lykil en hvaða aðferðir eru til á þær?
Þær eru ekki margar og þess virði að taka nokkrar fyrir sérstaklega vegna þess hve gagnlegar þær eru strax fyrir byrjendur.

\begin{itemize}
\item[] \texttt{get()} leyfir okkur að athuga hvort að lykill sé til í orðabók án þess að valda villu, og ef við viljum skila stöðlu gildi ef lykillinn fannst ekki.
\item[] \texttt{pop()} fjarlægir það stak sem síðast var sett inn (fjarlægir eitthvað stak í Python <3.6)
\item[] \texttt{popitem()} fjarlægir nákvæmlega það stak sem við viljum með því að við gefum upp lykil
\item[] \texttt{items(), keys()} og \texttt{values()} skilar okkur ítranlegum hlut af því sem við viljum geta unnið með, items eru lykla og gildis pör sem ndir, keys eru bara lyklarnir og values eru gildin.
\end{itemize}

Skoðum aðeins nánar hvernig \texttt{items(), keys()} og \texttt{values()} virka því að við viljum geta ítrað í gegnum orðabækur.

\begin{lstlisting}[caption=Aðferðir á orðabækur, label=lst:dict-kynnt3]
ordabok3 = {1: [1,2,3,4,5], 2: ["strengir", "í", "lista"], "þrír": [-1,-2,-3]}
print("gildin:", ordabok3.values())
print("lyklarnir", ordabok3.keys())
print("ndir af pörum", ordabok3.items())
\end{lstlisting}
\lstset{style=uttak}
\begin{lstlisting}
gildin: dict_values([[1, 2, 3, 4, 5], ['strengir', 'í', 'lista'], [-1, -2, -3]])
lyklarnir dict_keys([1, 2, 'þrír'])
ndir af pörum dict_items([(1, [1, 2, 3, 4, 5]), (2, ['strengir', 'í', 'lista']), ('þrír', [-1, -2, -3])])
\end{lstlisting}
\lstset{style=venjulegt}

Höfum ekki óþarfa áhyggjur af úttakinu þar sem stendur dict\_ eitthvað.
Það sem við þurfum að átta okkur á er að við fáum lista í hvert sinn og að stökin í listunum fást upp úr orðabókinni okkar með útreiknanlegum hætti.

Skoðum nú í næsta undirkafla hvernig megi vinna með þetta.

\comment{

Lyklarnir þurfa að vera aðgreinanlegir, hvað þýðir það?
Skoðum innbyggða fallið hash() til þess að átta okkur á því hvað má nota sem lykil, sjá kóðabút \ref{lst:dict-hash}.
Það sem hash() fallið gerir er að skila okkur einu tilteknu heiltölu gildi, tveir hlutir sem eru álitnir jafngildir fá sömu heiltöluna úr hash() fallinu.
Ekki er hægt að kalla í hash() af öllum týpum, því sumar týpur eru óhakkanlegar.

Í kóðabút \ref{lst:dict-kynntar} sjáum við hvernig á að skilgreina orðabók, lykla og gildispör og hvernig á að aðgreina stök.
Stak í orðabók er eitt lykla og gildispar.
Lyklar þurfa að vera hakkanlegir og því geta listar og orðabækur ekki verið lyklar en hvað sem er má vera gildi, eins og sést í kóðabút \ref{lst:dict-hash}.
Lyklar og gildi geta verið breytur, en þá eins og alltaf þegar við notum breytur þurfum við að vera búin að skilgreina breytuna áður en við notum hana.

\begin{lstlisting}[caption=Skoðum hash() fallið til að skilja aðgreinanleika gagna, label=lst:dict-hash]
a = [1,2,3] # a er listi
hash(a) # skilar villu 

b = 12 
hash(b) # skilar 12

c = 12.2
hash(c) # skilar stórri heilli tölu.

d = 12.0
hash(d) # skilar 12 

# svo hash skilar okkur tölu eftir því hvernig má túlka gögn sem eina heila tölu ef það er mögulegt, svo b og d eru óaðgreinanleg og því má ekki nota bæði sem lykla í sömu orðabók.
\end{lstlisting}

Ástæðan fyrir því að við þurfum að skilja þetta er vegna þess að við þurfum að átta okkur á því hvað má setja sem lykil og hvers vegna við getum kannski ekki notað einhvern tiltekinn lykil.


}

\section{Ítrað í gegnum orðabækur}\index{Ítrað í gegnum orðabækur}

Nú höfum við séð for lykkjur í kafla \ref{k:lykkjur} og hvernig mátti lykkja í gegnum lista í kóðabút \ref{lst:lykkjur-for}.
Nú hins vegar þurfum við að fara yfir hvernig í ósköpunum á eiginlega að skoða stak í orðabók á kerfisbundinn máta þegar eitt stak er bæði lykill og gildi.

Þetta er útfærsluatrðið sem ákveðið var að yrði gert þannig að þegar óskað er eftir að ítra í gegnum orðabók eru lyklarnir hennar eingöngu teknir fyrir.
Hins vegar er hægt að gera bæði lykla og gildi eða einungis gildin aðgengileg með því að kalla í aðferðirnar .items() og .values(), að rúlla í gegnum orðabók án þess að taka fram einhverja aðferð er eins og að hafa kallað í aðferðina .keys().
Nú eru nöfnin á þessum aðferðum ágætlega lýsandi:

\begin{itemize}
	\item ordabok.keys() við fáum í hendurnar ítranlegan hlut sem inniheldur alla lykla í lista úr breytunni ordabok (e. view)
	\item ordabok.values() við fáum í hendurnar ítranlegan hlut sem inniheldur öll gildi í lista úr breytunni ordabok (e. view)
	\item ordabok.items() við fáum í hendurnar ítranlegan hlut sem inniheldur lista af tvenndum (nd með tveimur stökum) úr breytunni ordabok (e. view).
\end{itemize}

Þannig að til þess að sækja það sem við viljum skoða þurfum við að nota þá aðferð á orðabókina okkar sem okkur hentar hverju sinni.
Ef við vildum til dæmis halda utan um bókasafnið okkar með orðabók og vinna með þær upplýsingar úr bókasafninu sem henta hverju sinni gætum við gert það eins og kemur fram í kóðabút \ref{lst:dict-bokasafn}.

\begin{lstlisting}[caption=Skoðum hvernig megi ítra í gegnum orðabækur, label=lst:dict-bokasafn]
# Skilgreinum orðabók sem heldur utan um bókasafnið okkar, þar sem lykill er höfundur og gildi er listi af bókum sem við eigum eftir þann höfund:
bokasafn = {'Beazley': ['Python Essential Reference'],'Halldór Laxness': ['Íslandsklulkka', 'Salka Valka], 'Auður Haralds': ['Hlustið þér á Mozart', 'Læknamafían', 'Hvunndagshetjan'] }

# til þess að vinna með alla þá höfunda sem við eigum til getum við gert eftirfarandi:

for hofundur in bokasafn:
	
	# við vitum að bokasafn[hofundur] gefur okkur gildi þess höfundar, sem er í okkar tilfelli alltaf listi
	
	# athugum nú hvað við eigum margar bækur eftir höfundana með því að skoða lengd listans
	if len(bokasafn[hofundur]) > 5:
		print('Á bókasafninu eru til fleiri en fimm bækur eftir höfundinn', hofundur)
	elif(len(bokasafn[hofundur]) > 2):
		print('Á bókasafninu eru til fleiri en tvær bækur en þó innan við sex, eftir höfundinn', hofundur)
	elif(len(bokasafn[hofundur]) > 1):
		print('Á bókasafninu eru til tvær bækur eftir höfundinn', hofundur)
	elif(len(bokasafn[hofundur]) > 0):
		print('Á bókasafninu er til ein bók eftir höfundinn', hofundur)
	else:
		print('Á bókasafninu er ekki til nein bók eftir höfundinn', hofundur)
		
		
# við getum einnig gert þetta:

for hofundur in bokasafn.keys():
	# nú er hofundur breytan sem hleypur í gegnum ítranlega hlutinn sem .keys() skilar alveg jafngild breytunni hofundur í lykkjunni fyrir ofan
	# svo vid getum enn gert bokasafn[hofundur]
	
	if hofundur < "Miðgildi":
		print('höfundurinn", hofundur, "er framarlega í stafrófinu")
	else:
		print("höfundurinn", hofundur, "er aftarlega í stafrófinu")
		
# en ef við viljum einungis skoða gildin, það er listana sem innihalda bækurnar sjálfar og okkur er sama um höfundana

for bokalisti in bokasafn.values():
	# bokalisti er breyta sem inniheldur lista
	# svo við getum ítrað í gegnum hann
	for bok in bokalisti:
		# ef hún er með langan titil viljum við prenta hana út:
		if(len(bok) > 20):
			print(bok)
			
# Og ef við viljum skoða bæði í einu án þess að þurfa að sækja gildið á lykilin sjálf með orðabók[lykill]:

for hofundur, bokalisti in bokasafn.items():
	# ef bókalistinn er ákveðið langur þá langar okkur að prenta út nafnið á höfundinum
	if(len(bokalisti) > 5):
		print(hofundur, "er mjög vinsæll höfundur")
	elif(len(bokalisti) > 2):
		print(hofundur, "er frekar vinsæll höfundur")
	elif(len(bokalisti) > 1):
		print(hofundur, "gæti verið vinsælli")
	elif(len(bokalisti) == 1):
		print(hofundur, "er vissulega til staðar")
	else:
		print(hofundur, "á ekki tiltall til einnar bókar í þessu bókasafni")
\end{lstlisting}

Nú gera lykkjurnar í línu 6 (engin aðferð, bara lykkjað í gegnum bókasafnið eins og Python gerir á sjálgefinn máta) og þessi í línu 46 (.items() aðferðin sem fær par sem látið er í tvær breytur hofundur og bokalisti) nokkurn veginn það sama en við tökum eftir að sú seinni er aðeins læsilegri því að breytan bokalisti er nokkuð lýsandi fyrir það hvað hún inniheldur á meðan bokasafn[hofundur] gæti verið strengur eða nd eða eitthvað allt annað (eins og enn önnur orðabók).